\article{Introduction}
\label{introduction}

Welcome to the official Haxe 3 Tutorials book! This document provides in-depth tutorials on various Haxe-related aspects. I hate writing introductions!

\section*{About this document}
\label{introduction-about-this-document}

This document is a collection of independent articles about Haxe. The idea is to have several community members contribute and maintain documentation on interesting Haxe projects for others to read and, ideally, replicate in order to get (more) familiar with Haxe. These articles are grouped by difficulty, roughly following these guidelines:

\begin{description}
	\item[Beginner:] People who are new to Haxe and are not yet very familiar with its concepts.
	\item[Intermediate:] People who have been using Haxe for a while but would like to extend their horizon.
	\item[Advanced:] People who have been using Haxe for quite a while and are interested in advanced concepts and gory details.
\end{description}

Unlike the Haxe 3 Manual, this is not a strictly technical document. This should be reflected in the general tone which may be very relaxed and casual. We also encourage authors to break up text with screenshots, diagrams or other images.

\section*{Information for authors}
\label{introduction-information-for-authors}

Each article should be placed in a directory under /HaxeTutorials. The directory name should start with the difficulty level (Beginner, Intermediate, Advanced) and not contain any spaces (we recommend using dashes instead). Within this directory the author has free reign, but we recommend to have a single entry-point .tex file so it can easily be included in HaxeTutorials.tex.

Each article has to start with three self-explanatory commands:

\begin{itemize}
	\item {\textbackslash}article\{name\}
	\item {\textbackslash}label\{label\}
	\item {\textbackslash}maintainer\{author name\}
\end{itemize}

Note that the label is what is used for the URL of the article so it should not contain any spaces.

The document can be structured using these commands:

\begin{itemize}
	\item {\textbackslash}section*\{name\}
	\item {\textbackslash}subsection*\{name\}
	\item {\textbackslash}paragraph\{name\}
\end{itemize}

The asterisk is necessary to suppress section numbering. Furthermore, both section and subsection require a label so a URL can be determined.

Finally, if you want to write an article but cannot be arsed to use .tex, just write it in any other format and we'll port it.