\chapter{コンパイラの機能}
\label{cr-features}
\state{NoContent}

\section{ビルトインのコンパイラメタデータ}
\label{cr-metadata}

Haxe 3.0以降では、\expr{haxe --help-metas}を実行することで定義済みのコンパイラメタデータのリストを得ることができます。

\begin{center}
\begin{tabular}{| l | l | l |}
	\hline
	\multicolumn{3}{|c|}{グローバルメタデータ} \\ \hline
	メタデータ &  説明  &  プラットフォーム \\ \hline
	@:abi & ABI呼び出し規約を使う  & cpp \\
	@:abstract &  基底クラスの実装を\tref{抽象型}{types-abstract}として使います。  &  cs  java \\
	@:access \_(Target path)\_  &   型またはフィールドへのプライベートなアクセスを強制する。詳しくは\tref{アクセス制御}{lf-access-control}  &  all \\
	@:allow \_(Target path)\_  &   型またはフィールドからのプライベートなアクセスを許可する。詳しくは\tref{アクセス制御}{lf-access-control}  &  all \\
	@:analyzer & 静的アナライザを設定する  &  all \\
	@:annotation  &  Annotation (\expr{@interface}) definitions on \expr{-java-lib} imports will be annotated with this metadata. Haxeでコンパイルした型には影響ありません   &  java \\
	@:arrayAccess  &  抽象型への\tref{配列アクセス}{types-abstract-array-access}を許可する  &  all \\
	@:autoBuild \_(Build macro call)\_  &   \expr{@:build}メタデータをその型のすべての子クラス、実装クラスに反映する。詳しくは\tref{autobuildマクロ}{macro-auto-build}  &  all \\
	@:bind  &  SWFライブラリで定義されているクラスを上書きします  &  flash \\
	@:bitmap \_(Bitmap file path)\_  &  与えられたビットマップデータをクラスに埋め込む(\expr{flash.display.BitmapData}の継承が必要)   &  flash \\
	@:bridgeProperties  &  クラスのすべてのHaxeプロパティに、ネイティブのプロパティのブリッジを用意する  &  cs \\
	@:build \_(Build macro call)\_  &   マクロからクラスまたは列挙型を構築する。詳しくは\tref{型構築}{macro-type-building}  &  all \\
	@:buildXml  &  Build.xmlに挿入するxmlデータを指定する。  &  cpp \\
	@:callable  &  抽象型で基底型の関数呼び出しのアクセスをできるようにする。  &  all \\
	@:classCode  &  クラスにプラットフォームネイティブのコードを挿入する  &  cs  java \\
	@:commutative  &  抽象型の2項演算子の各項を交換可能にする  &  all \\
	@:compilerGenerated  &  コンパイラから生成されたフィールドをマークする。ユーザーは使用すべきでは無い。  &  cs  java \\
	@:coreApi &  このクラスをコアAPIのクラスとして認識する。(APIチェックを強制する)  &  all \\
	@:coreType  &  抽象型を\tref{コアタイプ}{types-abstract-core-type}として認識する。そのため、型には実装があってはいけない。  &  all \\
	@:cppFileCode  &  C++の出力ファイルに挿入するコード  &  cpp \\
	@:cppInclude  &  C++の出力ファイルに含めるファイル  &  cpp \\
	@:cppNamespaceCode  &    &  cpp \\
	@:dce  &  \expr{-dce full}が指定されていない場合でも、\tref{デッドコード削除}{cr-dce}を強制する  &  all \\
	@:debug  &  \expr{-debug}が指定されていない場合でも、出力するSWFにデバッグ情報を強制する   &  flash \\
	@:decl   &     &  cpp \\
	@:defParam  &    &  all \\
	@:delegate  &  \expr{-net-lib}のデリゲートで自動的に追加される   &  cs \\
	@:depend  &     &  cpp \\
	@:deprecated   &  \expr{-java-lib}の\expr{@Deprecated}で修飾されているフィールドで自動的に追加される。Haxeでコンパイルされる型には何の影響も与えない。  &  java \\
	@:event  &  \expr{-net-lib}のイベントに自動的に追加される。Haxeでコンパイルされる型には何の影響も与えない。   &  cs \\
	@:enum  &  抽象型の定義につけることで有限の値のセットを定義する。詳しくは\tref{抽象型列挙体}{types-abstract-enum}  &  all \\
	@:expose \_(?Name=Class path)\_  &  クラスを\expr{window}オブジェクトか、node.jsの場合は\expr{exports}で使用可能にする。詳しくは\tref{HaxeのクラスをJavaScriptに露出させる}{target-javascript-expose} &  js \\
	@:extern  &  フィールドを\expr{extern}としてマークする。結果として出力されなくなる。  &  all \\
	@:fakeEnum \_(Type name)\_  &  列挙型を指定した型の集合としてあつかう。  &  all \\
	@:file(File path)  &  SWFターゲットに指定したバイナリファイルを含めてそのクラスと関連づける。(\expr{flash.utils.ByteArray}の継承が必要)  &  flash \\
	@:final  &  クラスが継承されるのを妨害する  &  all \\
	@:font \_(TTF path Range String)\_  &  指定したTrueTypeフォントをクラスに埋め込む(\expr{flash.text.Font}の継承が必要)  &  flash \\
	@:forward \_(List of field names)\_  & 基底型から\tref{フィールドアクセスの繰り上げ}{types-abstract-forward}を行う。  &  all \\
	@:from   &  抽象型のフィールドで、その型からのキャスト操作をその関数で定義する。詳しくは\tref{暗黙のキャスト}{types-abstract-implicit-casts}。  &  all \\
	@:functionCode  &     &  cpp \\
	@:functionTailCode  &    &  cpp \\
	@:generic &  クラスまたはクラスフィールドを\tref{ジェネリック}{type-system-generic}としてマークして、すべてのパラメータの組み合わせについて出力を行う。  &  all \\
	@:genericBuild  &  型のインスタンスを指定したマクロを使って生成する。   &  all \\
	@:getter \_(Class field name)\_  &  指定したフィールドにネイティブのゲッター関数を生成する。   &  flash \\
	@:hack   &  \expr{@:final}のマークがされているクラスの継承を許可する。  &  all \\
	@:headerClassCode  &  そのヘッダーで、生成されたクラスにコードを挿入する。  &  cpp \\
	@:headerCode   &  生成されたヘッダファイルにコードを挿入する。  &  cpp \\
	@:headerNamespaceCode  &    &  cpp \\
	@:hxGen  &  Haxeによって生成されたexternクラスに付く  &  cs  java \\
	@:ifFeature \_(Feature name)\_  &  指定された機能がコンパイルに含まれていた場合に、フィールドを\tref{DCE}{cr-dce}から保護する。  &  all \\
	@:include &     &  cpp \\
	@:initPackage  &    &  all \\
	@:internal  &  クラスやフィールドに\expr{internal}アクセスの修飾をする。  &  cs  java \\
	@:isVar  &  物理的フィールドが不要なプロパティに対して、物理的フィールドを強制する。  &  all \\
	@:javaCanonical \_(Output type package,Output type name)\_ &  Javaターゲットで型の正規パスを指定するのに使われる。  &  java \\
	@:jsRequire  &  その\expr{extern}に必要なJavaScriptモジュールを出力する。  &  js \\
	@:keep   &  \tref{DCE}{cr-dce}から、フィールドや型を保護する。  &  all \\
	@:keepInit  &  クラスからすべてのフィールドが削除された場合でも、\tref{DCE}{cr-dce}から保護する。  &  all \\
	@:keepSub &  すべての実装クラス、子孫クラスに\expr{@:keep}メタデータを継承する。  &  all \\
	@:macro  &  \_(deprecated)\_  &  all \\
	@:mergeBlock  &  修飾したブロックを現在のスコープにマージする。 &  all \\
	@:meta   &  内部的にクラスフィールドがメタデータフィールドであるとマークするのに使われる。  &  all \\
	@:multiType \_(Relevant type parameters)\_  &  抽象型でその\expr{@:to}関数の中からthisの型を選択する。  &  all \\
	@:native \_(Output type path)\_  &  クラスと列挙型のパスを出力の過程で書き換える。  &  all \\
	@:nativeChildren  &  型のそのすべての子を\expr{extern}であるかのように扱う。(プラットフォームネイティブ)  &  cs java \\
	@:nativeGen  &  型を\expr{extern}であるかのように扱う。(プラットフォームネイティブ)  &  cs  java \\
	@:nativeProperty  &  動的な使用がされる場合であっても、ネイティブのプロパティを使う。  &  cpp \\
	@:noCompletion  &  そのフィールドをコンパイラの\tref{補完}{cr-completion}の候補に含めない。  &  all \\
	@:noDebug &  \expr{-debug}が設定されている場合でもSWFにデバッグ情報を含めない。   &  flash \\
	@:noDoc  &  型がドキュメント出力に含まれないようにする。  &  all \\
	@:noImportGlobal  &  \expr{import Class.*}で静的フィールドが含まれるのを妨害します。  &  all \\
	@:noPrivateAccess  &  指定した式からの、プライベートアクセスを禁止します。  &  all \\
	@:noStack &     &  cpp \\
	@:noUsing &  フィールドが\expr{using}で使用されるのを防ぎます。  &  all \\
	@:nonVirtual &  関数をvirtualでないと宣言する。  &  cpp \\
	@:notNull &  抽象型が\tref{\expr{null}値}{types-nullability}を許容しないことを宣言する。  &  all \\
	@:ns  &  SWFのジェネレータが名前空間を扱うために内部的に使う。   &  flash \\
	@:op \_(The operation)\_  &  抽象型のフィールドを\tref{演算子オーバーロード}{types-abstract-operator-overloading}として定義する。  &  all \\
	@:optional  &  構造体のフィールドをオプションのものとしてマークする。詳しくは\tref{オプション引数}{types-nullability-optional-arguments}  &  all \\
	@:overload \_(Function specification)\_  &  フィールドが異なる引数の型で呼び出しされるのを許可する。対象の関数は式を持てない。  &  all \\
	@:privateAccess  &  指定された式からのあらゆるプライベートアクセスを許可する。  &  all \\
	@:property  &  プロパティフィールドをネイティブのC\#プロパティにコンパイルされるようにマークする。   &  cs \\
	@:protected  &  クラスフィールドを\expr{protected}としてマークする。  &  all \\
	@:public  &  クラスフィールドを\expr{public}としてマークする。 &  all \\
	@:publicFields  &  クラスのすべてのクラスフィールドを\expr{public}にする。  &  all \\
	@:pythonImport  &  \expr{extern}クラス対してpythonのインポート文を生成する。  &  python \\
	@:readOnly  &  ネイティブの\expr{readonly}キーワードを付けたフィールドを出力する。   &  cs \\
	@:remove  &  インターフェースをすべての実装クラスから出力前に削除する。  &  all \\
	@:require \_(Compiler flag to check)\_  &  特定の\tref{コンパイラフラグ}{lf-condition-compilation}が指定されている場合のみアクセスを許可する。  &  all \\
	@:rtti   &  実行時型情報を追加する。詳しくは\tref{RTTI}{cr-rtti}。  &  all \\
	@:runtime  &    &  all \\
	@:runtimeValue  &  抽象型を実行時の値としてマークする。  &  all \\
	@:selfCall  &  メソッド呼び出しをオブジェクトそのものの呼び出しに変換する。  &  js \\
	@:setter \_(Class field name)\_  &  指定したフィールドのネイティブのセッター関数を生成する。   &  flash \\
	@:sound \_(File path)\_  &  \_.wav\_または\_.mp3\_ファイルをSWFを埋め込んでクラスに紐づける。(\expr{flash.media.Sound}の継承が必要)  &  flash \\
	@:sourceFile  &  外部クラスのソースコードのファイル名。  &  cpp \\
	@:strict  &  C\#のネイティブ属性、Javaのメタデータを宣言するのに使う。型チェックがされる。  &  cs java \\
	@:struct  &  クラスを構造体としてマークする。   &  cs \\
	@:structAccess  &  \expr{extern}クラスをポインタ('->')ではなく構造体アクセス('.')を使うようにマークする。  &  cpp \\
	@:suppressWarnings  &  出力されるJavaクラスにSuppressWarningsの修飾を行う。  &  java \\
	@:throws \_(Type as String)\_  &  \expr{throws}宣言を出力される関数に追加する。   &  java \\
	@:to  &  抽象型のフィールドで、その型へのキャスト操作をその関数で定義する。詳しくは\tref{暗黙のキャスト}{types-abstract-implicit-casts}。  & all \\
	@:transient  &  クラスフィールドに\expr{transient}フラグを追加する。  &  java \\
	@:unbound  &  コンパイラの内部で無制限のグローバル変数を表すのに使う。  &  all \\
	@:unifyMinDynamic  &  型の集合を\type{Dynamic}に単一化するのを許可する。  &  all \\
	@:unreflective  &    &  cpp \\
	@:unsafe  &  クラスまたはメソッドでC\#の\expr{unsafe}フラグを宣言する。   &  cs \\
	@:usage  &    &  all \\
	@:value  &  フィールドや関数の引数のデフォルト値を記録するのに使う。  &  all \\
	@:void  &  C++ネイティブの'void'を戻り値に使う。  &  cpp \\
	@:volatile  &    &  cs  java \\
\end{tabular}
\end{center}

\section{デッドコード削除}
\label{cr-dce}

デッドコード削除(Dead Code Elimination、\emph{DCE})は、未使用のコードを出力から取り除くコンパイラ機能です。型付けの後に、DCEの始点(多くの場合はmainメソッド)から再帰的にたどっていきどのフィールドと型が使用されているかを決定します。これにより使用済みのフィールドはマークされ、マークされていないフィールドはクラスから取り除かれます。

DCEには3つのモードがあり、コマンドラインからの呼び出し時に指定します。

\begin{description}
	\item[-dce std:] Haxeの標準ライブラリのクラスのみがDCEの影響を受けます。これがすべてのターゲットでのデフォルト値です。
	\item[-dce no:] DCEされません。
	\item[-dce full:] すべてのクラスがDCEの影響を受けます。
\end{description}
DCEのアルゴリズムは型付けされたコードではうまく働きますが、\tref{Dynamic}{types-dynamic}や\tref{リフレクション}{std-reflection}を使っていると失敗する場合があります。そういった場合は、以下のメタデータを使ったクラスやフィールドの明示的な修飾が必要かもしれません。

\begin{description}
	\item[\expr{@:keep}:] クラスに使用するとすべてのフィールドがDCEの対象から除外されます。フィールドに使用するとそのフィールドがDCEの対象になりません。
	\item[\expr{@:keepSub}:] クラスに使用すると、その子孫クラスすべてを\expr{@:keep}で修飾したのと同様の動作をします。
	\item[\expr{@:keepInit}:] 通常、クラスはすべてのフィールドがDCEによって削除されると(あるいは最初から空だと)出力から削除されます。このメタデータを使うと、空のクラスが保持されます。
\end{description}

ソースコードを編集するのではなくコマンドラインからクラスを\expr{@:keep}としてマークしたい場合、コンパイラマクロの\expr{--macro keep('type dot path')}を使うことでそれが可能です。このマクロについて詳しくは\href{http://api.haxe.org/haxe/macro/Compiler.html#keep}{haxe.macro.Compiler.keep API}をご覧ください。パッケージをマークするとそのモジュールやサブタイプがDCEから保護されて、コンパイルに含まれます。

コンパイラは現在のモードに応じて、自動的に\expr{dce}フラグの値を\expr{"std"}、\expr{"no"}、\expr{"full"}のいずれかに設定します。このフラグは\tref{条件付きコンパイル}{lf-condition-compilation}で使用できます。

\trivia{DCEの書き直し}{
DCEは元々Haxe 2.07で実装されましたが、その実装では関数は明示的に型付けされているときに使用しているという判定がされていました。このせいでいくつか機能で問題がありました。とくにインタフェースで深刻で、型安全性を確かめるためにはすべてのクラスフィールドを型付けする必要がありました。このせいでDCEは完全に破たんし、Haxe 2.10での書き直しにつながりました。}

\trivia{DCEとtry.haxe.org}{\url{http://try.haxe.org}のサイトが公開されたとき、\type{JavaScript}ターゲットのDCEは大きく改善されました。\target{JavaScript}の出力コードに対する反応はさまざまでしたが、これにより削除されるコードの選択がより細かく行われるようになりました。}


\section{コンパイラサービス}
\label{cr-completion}
\state{NoContent}

\subsection{概要}
\label{cr-completion-overview}

Haxeの豊富な\tref{型システム}{type-system}は、IDEやエディタが正確な補完情報を提供することを難しくしています。\tref{型推論}{type-system-type-inference}と\tref{マクロ}{macro}については、必要な挙動を再現するのに相当な量の仕事が必要です。これがHaxeのコンパイラが、サードパーティーのソフトウェアが使うためのビルトインの補完モードを備えている理由です。

すべての補完は\ic{--display file@position[@mode]}のコンパイラ引数を使うことで開始されます。この引数は以下を必要とします。

\begin{description}
	\item[file:] 補完のためチェックを行うファイルです。これは.hxファイルの絶対パスまたは相対パスです。クラスパスやライブラリではありません。
	\item[position:] 補完のためのチェックを行う与えられたファイルのバイト位置です(文字位置ではありません)。
	\item[mode:] 使用する補完モードです(後述)。
\end{description}

補完モードの詳細については以下の通りです。

\begin{description}
	\item[\tref{フィールドアクセス}{cr-completion-field-access}:] その型のアクセス可能なフィールドのリストを提供します。
	\item[\tref{関数の引数}{cr-completion-call-argument}:] 呼び出そうとしている関数の型を取得します。
	\item[\tref{型のパス}{cr-completion-type-path}:] 子パッケージ、サブタイプ、静的フィールドをリストアップします。
	\item[\tref{使用状況}{cr-completion-usage}:] コンパイルされるファイルのすべて中から指定された型またはフィールド、変数の出現位置をリストアップします。(\ic{"usage"}モード)
	\item[\tref{定義位置}{cr-completion-position}:] 指定された型またはフィールド、変数の定義位置を取得します。(\ic{"position"}モード)
	\item[\tref{トップレベル}{cr-completion-top-level}:] 指定した位置で使用可能なすべての識別子(mode: \ic{"toplevel"})
\end{description}

Haxeのコンパイラは非常に速いので、多くの場合は通常のコンパイラの呼び出しを使っても問題がありません。しかし、より大きなHaxeプロジェクトのために\tref{サーバーモード}{cr-completion-server}を用意してあります。これにより、ファイルに実際に変更があった場合や依存関係の更新がされたときだけ再コンパイルがされます。

\paragraph{インターフェースについての注意事項}
\label{cr-completion-interface-notes}

\begin{itemize}
	\item ファイルの目的の位置にパイプライン\ic{|}の文字を置くことで、\ic{position}引数を0に設定することができます。これは、バイト数のカウント無しでIDEが何をすべきなのかをテストしたり実演したりするのに、便利です。この章のサンプルはこの機能を使っていきます。この機能は\ic{|}が不正な構文になる位置に置かれている場合のみ動作します。例えば、ドットの直後(\ic{.|})や小かっこの直後(\ic{(|})です。
	\item 出力はHTMLエスケープされます。つまり、\ic{\&}、\ic{<}、\ic{>}はそれぞれ\ic{\&amp;}、\ic{\&lt;}、\ic{\&gt;}に変換されます。
	\item ドキュメントの出力は、ソースコード内のものと同じように改行とタブ文字を含みます。
	\item 補完モードの実行ではコンパイラはエラーを表示せずににエラーからの復帰を試みます。補完中に致命的なエラーが発生した場合、Haxeのコンパイラは補完結果の代わりにエラーメッセージを出力します。XMLではあらゆる出力は致命的なエラーメッセージであると判断できます。
\end{itemize}

\subsection{フィールドアクセス補完}
\label{cr-completion-field-access}

フィールドの補完はドット\ic{.}文字の後から開始されて、その型で利用可能なフィールドをリストアップします。コンパイラは補完の位置までのすべての構文解析と型付けを行い、関連する情報を標準エラー出力に出力します。

\begin{lstlisting}
class Main {
  public static function main() {
    trace("Hello".|
  }
}
\end{lstlisting}

このファイルをMain.hxとして保存すると、補完は\ic{haxe --display Main.hx@0}のコマンドを使って呼び出せます。その出力は以下のようなものでしょう(いくつかの情報を可読性のために削ったりフォーマットをかけたりしています)。

\begin{lstlisting}
<list>
<i n="length">
  <t>Int</t>
  <d>
    The number of characters in `this` String.
  </d>
</i>
<i n="charAt">
  <t>index : Int -&gt; String</t>
  <d>
    Returns the character at position `index` of `this` String.
    If `index` is negative or exceeds `this.length`, the empty String
    "" is returned.
  </d>
</i>
<i n="charCodeAt">
  <t>index : Int -&gt; Null&lt;Int&gt;</t>
  <d>
    Returns the character code at position `index` of `this` String.
    If `index` is negative or exceeds `this.length`, null is returned.
    To obtain the character code of a single character, "x".code can
    be used instead to inline the character code at compile time.
    Note that this only works on String literals of length 1.
  </d>
</i>
</list>
\end{lstlisting}

このXMLの構造は以下の通りです。

\begin{itemize}
	\item ドキュメントの\ic{list}ノードがいくつかの\ic{i}ノードを含み、このそれぞれが1つのフィールドを表現しています。
	\item \ic{n}属性はフィールドの名前です。
	\item \ic{t}ノードはフィールドの型です。
	\item \ic{d}ノードはフィールドのドキュメントです。
\end{itemize}

\since{3.2.0}

\ic{-D display-details}をつけてコンパイルすると、各フィールドに\ic{var}と\ic{method}のいずれかの\ic{k}属性が付きます。これにより、関数型の変数フィールドとメソッドフィールドを区別できます。

\subsection{呼び出し引数の補完}
\label{cr-completion-call-argument}

呼び出し引数の補完は小かっこ\ic{(}の後から開始されて、呼び出しをしようとしている関数の型を返します。これはコンストラクタの呼び出しを含むすべての関数呼び出しで使用可能です。

\begin{lstlisting}
class Main {
  public static function main() {
    trace("Hello".split(|
  }
}
\end{lstlisting}

このファイルをMain.hxとして保存すると、補完は\ic{haxe --display Main.hx@0}のコマンドを使って呼び出せます。その出力は以下のようなものになります。

\begin{lstlisting}
<type>
delimiter : String -&gt; Array&lt;String&gt;
</type>
\end{lstlisting}

IDEはここから、呼び出す関数が\ic{delimiter}という\type{String}型の引数が1つあって\type{Array<String>}を返すということを読み取れます。

\trivia{出力構造の問題}{私たちは現在のフォーマットはほんの少しのうっとおしい自前の構文解析が必要になることを認めます。特に関数については、将来的にはより構造化された出力を提供するようになるかもしれません。}

\subsection{型のパスの補完}
\label{cr-completion-type-path}

型のパスの補完は\tref{import宣言}{type-system-import}、\tref{using宣言}{lf-static-extension}あるいはあらゆる位置での型の記述で発生します。そしてこれは以降の3種類に分けることができます。

\paragraph{パッケージの補完}

以下はhaxeパッケージに属する子パッケージとモジュールのすべてをリストアップします。

\begin{lstlisting}
import haxe.|
\end{lstlisting}

\begin{lstlisting}
<list>
<i n="CallStack"><t></t><d></d></i>
<i n="Constraints"><t></t><d></d></i>
<i n="DynamicAccess"><t></t><d></d></i>
<i n="EnumFlags"><t></t><d></d></i>
<i n="EnumTools"><t></t><d></d></i>
<i n="Http"><t></t><d></d></i>
<i n="Int32"><t></t><d></d></i>
<i n="Int64"><t></t><d></d></i>
<i n="Json"><t></t><d></d></i>
<i n="Log"><t></t><d></d></i>
<i n="PosInfos"><t></t><d></d></i>
<i n="Resource"><t></t><d></d></i>
<i n="Serializer"><t></t><d></d></i>
<i n="Template"><t></t><d></d></i>
<i n="Timer"><t></t><d></d></i>
<i n="Ucs2"><t></t><d></d></i>
<i n="Unserializer"><t></t><d></d></i>
<i n="Utf8"><t></t><d></d></i>
<i n="crypto"><t></t><d></d></i>
<i n="ds"><t></t><d></d></i>
<i n="extern"><t></t><d></d></i>
<i n="format"><t></t><d></d></i>
<i n="io"><t></t><d></d></i>
<i n="macro"><t></t><d></d></i>
<i n="remoting"><t></t><d></d></i>
<i n="rtti"><t></t><d></d></i>
<i n="unit"><t></t><d></d></i>
<i n="web"><t></t><d></d></i>
<i n="xml"><t></t><d></d></i>
<i n="zip"><t></t><d></d></i>
</list>
\end{lstlisting}


\paragraph{モジュールのインポートの補完}

以下は、\type{haxe.Unserializer}モジュールの\tref{サブタイプ}{type-system-module-sub-types}と、\type{haxe.Unserializer}の\expr{public static}なフィールド(これらもインポート可能なので)のすべてをリストアップします。

\begin{lstlisting}
import haxe.Unserializer.|
\end{lstlisting}

\begin{lstlisting}
<list>
<i n="DEFAULT_RESOLVER">
  <t>haxe.TypeResolver</t>
  <d>
    This value can be set to use custom type resolvers.

    A type resolver finds a Class or Enum instance from a given String.
    By default, the haxe Type Api is used.

    A type resolver must provide two methods:

    1. resolveClass(name:String):Class&lt;Dynamic&gt; is called to
      determine a Class from a class name
    2. resolveEnum(name:String):Enum&lt;Dynamic&gt; is called to
      determine an Enum from an enum name

    This value is applied when a new Unserializer instance is created.
    Changing it afterwards has no effect on previously created
    instances.
  </d>
</i>
<i n="run">
  <t>v : String -&gt; Dynamic</t>
  <d>
    Unserializes `v` and returns the according value.

    This is a convenience function for creating a new instance of
    Unserializer with `v` as buffer and calling its unserialize()
    method once.
  </d>
</i>
<i n="TypeResolver"><t></t><d></d></i>
<i n="Unserializer"><t></t><d></d></i>
</list>
\end{lstlisting}


\begin{lstlisting}
using haxe.Unserializer.|
\end{lstlisting}


\paragraph{その他のモジュールの補完}

以下は、\type{haxe.Unserializer}のすべての\tref{サブタイプ}{type-system-module-sub-types}をリストアップします。

\begin{lstlisting}
using haxe.Unserializer.|
\end{lstlisting}

\begin{lstlisting}
class Main {
  static public function main() {
    var x:haxe.Unserializer.|
  }
}
\end{lstlisting}

\begin{lstlisting}
<list>
<i n="TypeResolver"><t></t><d></d></i>
<i n="Unserializer"><t></t><d></d></i>
</list>
\end{lstlisting}


\subsection{使用状況の補完}
\label{cr-completion-usage}
\since{3.2.0}

使用状況の補完は\ic{"usage"}モードの引数を使うことで使用可能です(詳しくは\Fullref{cr-completion-overview})。ローカル変数を使って実演してみますが、フィールドと型についても同じように動作することに気をつけてください。

\begin{lstlisting}
class Main {
  public static function main() {
    var a = 1;
    var b = a + 1;
    trace(a);
    a.|
  }
}
\end{lstlisting}

このファイルをMain.hxとして保存すると、補完は\ic{haxe --display Main.hx@0@usage}のコマンドを使って呼び出せます。この出力は以下のようなものになります。

\begin{lstlisting}
<list>
<pos>main.hx:4: characters 9-10</pos>
<pos>main.hx:5: characters 7-8</pos>
<pos>main.hx:6: characters 1-2</pos>
</list>
\end{lstlisting}


\subsection{定義位置の補完}
\label{cr-completion-position}
\since{3.2.0}

定義位置の補完は\ic{"position"}モードの引数を使うことで使用可能です(詳しくは\Fullref{cr-completion-overview})。フィールドを使って実演しますが、ローカル変数と型でも同じように動作することに気をつけてください。

\begin{lstlisting}
class Main {
  static public function main() {
    "foo".split.|
}
\end{lstlisting}

このファイルをMain.hxとして保存すると、補完は\ic{haxe --display Main.hx@0@position}のコマンドを使って呼び出せます。この出力は以下のようなものになります。

\begin{lstlisting}
<list>
<pos>std/string.hx:124: characters 1-54</pos>
</list>
\end{lstlisting}

\trivia{ターゲットの特定の省略による影響}{このサンプルでは標準のString.hxが取得されましたが、実際の実装はありません。これはどのターゲットとも特定しなかったためであり、補完モードではそれでも構いません。例えば\ic{-neko neko.n}のコマンドラインが含められた場合、結果として取得される位置は代わりに\ic{std/neko/_std/string.hx:84: lines 84-98.}となるでしょう。}

\subsection{トップレベルの補完}
\label{cr-completion-top-level}
\since{3.2.0}

トップレベルの補完は、与えられた補完位置での使用可能なHaxeコンパイラが知るかぎりのすべての識別子を表示します。この補完機能だけはその効果を実演するために、実際の位置を引数であたえてやる必要があります。

\begin{lstlisting}
class Main {
  static public function main() {
    var a = 1;
  }
}

enum MyEnum {
  MyConstructor1;
  MyConstructor2(s:String);
}
\end{lstlisting}

このファイルをMain.hxとして保存すると、補完は\ic{haxe --display Main.hx@63@toplevel}のコマンドを使って呼び出せます。その出力は以下のようなものになります(簡潔さのためにいくつかの要素を削っています)。

\begin{lstlisting}
<il>
<i k="local" t="Int">a</i>
<i k="static" t="Void -&gt; Unknown&lt;0&gt;">main</i>
<i k="enum" t="MyEnum">MyConstructor1</i>
<i k="enum" t="s : String -&gt; MyEnum">MyConstructor2</i>
<i k="package">sys</i>
<i k="package">haxe</i>
<i k="type" p="Int">Int</i>
<i k="type" p="Float">Float</i>
<i k="type" p="MyEnum">MyEnum</i>
<i k="type" p="Main">Main</i>
</il>
\end{lstlisting}

XMLの構造は各要素の\ic{k}属性によります。すべての場合で\ic{i}のノードはその値として名前を持ちます。

\begin{description}
	\item[\ic{local}, \ic{member}, \ic{static}, \ic{enum}, \ic{global}:] \ic{t}属性にその変数やフィールドの型を持ちます。
	\item[\ic{global}, \ic{type}:] \ic{p}属性にその型やフィールドが属するモジュールのパスを持ちます。
\end{description}

\subsection{補完サーバー}
\label{cr-completion-server}

コンパイルと補完を最速で行いたいのであれば、\ic{--wait}のコマンドラインパラメータでHaxeの補完サーバーを立ち上げることができます。また、\ic{-v}でサーバーがログを出力するようになります。以下が例です。

\begin{lstlisting}
haxe -v --wait 6000
\end{lstlisting}

こうすることで、Haxeサーバーに接続して、コマンドラインパラメータを送って、NULL文字を送ると、レスポンスの読み取りができます(補完が成功の場合も失敗の場合も)。

\ic{--connect}のコマンドラインのパラメータを使うことで、Haxeはコンパイルコマンドを直接実行するのではなくサーバーに送るようになります。

\begin{lstlisting}
haxe --connect 6000 myproject.hxml
\end{lstlisting}

はじめに\ic{--cwd}のパラメータを使うことで、Haxeサーバーの現在の作業ディレクトリを変更することができることに気をつけてください。多くの場合クラスパスとそのほかのファイルはプロジェクトからの相対パスで指定されます。

\paragraph{動作の詳細}

コンパイルサーバーは以下の内容をキャッシュします。

\begin{description}
	\item[構文解析したファイル] ファイルは編集があせれたときが解析エラーになったときのみ再度構文解析が行われます。
	\item[Haxelibの呼び出し] 前回のHaxelibの呼び出し結果は再度利用されます(補完時のみです。コンパイルには関係ありません)
	\item[型付けされたモジュール] モジュールのコンパイルはコンパイル成功した場合にキャッシュされて、その依存関係が更新されるまで補完とコンパイルに再利用されます。
\end{description}

コマンドラインで\ic{--times}を追加することで、コンパイラが使用した正確な時間を取得して、コンパイルサーバーがどのような影響を与えたかを知ることができます。

\paragraph{プロトコル}
次のHaxe/Nekoの例からわかるとおり、サーバーのポートに単純につないで、1行づつコマンドを送って、NULL文字で終了します。その後に結果を読み取ります。

マクロやその他のコマンドはエラー以外のログを出力することができます。コマンドラインからの実行の場合は、標準出力と標準エラー出力にプリントされるものの違いがありますが、ソケットモードの場合は違います。この2つを区別するために、ログメッセージ(エラーではない)は\ic{\\x01}の文字から始まり、そのメッセージのすべての改行文字は\ic{\\x01}で置き換えられます。

警告やそのほかのメッセージはエラーと考えられますが、致命的なものではありません。致命的なエラーが起こると、\ic{\\x02}から始まる1行のメッセージが送られます。

以下にサーバーへ接続して、このプロトコルに従った処理を行うコードがあります。

\begin{lstlisting}
class Test {
    static function main() {
		var newline = "\textbackslash\ n";
        var s = new neko.net.Socket();
        s.connect(new neko.net.Host("127.0.0.1"),6000);
        s.write("--cwd /my/project" + newline);
        s.write("myproject.hxml" + newline);
        s.write("\textbackslash\ 000");
		
        var hasError = false;
        for (line in s.read().split(newline))
		{
            switch (line.charCodeAt(0)) {
				case 0x01: 
					neko.Lib.print(line.substr(1).split("\textbackslash\ x01").join(newline));
				case 0x02: 
					hasError = true;
				default: 
					neko.io.File.stderr().writeString(line + newline);
            }
		}
        if (hasError) neko.Sys.exit(1);
    }
}
\end{lstlisting}

\paragraph{マクロの影響}

コンパイルサーバーは\tref{マクロの実行}{macro}に副作用を与えます。

\section{リソース}
\label{cr-resources}
\flag{fold}{true}

Haxeは単純なリソース埋め込みのシステムを提供しています。これによりファイルをコンパイル後のアプリケーションに直接埋め込むことができます。

この方法は画像や音楽のような巨大なファイルの埋め込みには適していないかもしれませんが、設定やXMLのようなより小さなデータを埋め込むのにはとても便利です。

\subsection{リソースの埋め込み}
\label{cr-resources-embed}

以下のように、\emph{-resource}のコンパイラ引数をつかって外部ファイルの埋め込みができます。

\todo{what to use for listing of non-haxe code like hxml?}
\begin{lstlisting}
-resource hello_message.txt@welcome
\end{lstlisting}

\emph{@}マークの後の文字列は\emph{リソースの識別子}です。コードからリソースを取得するのに使います。省略された場合(\emph{@}マークごと)、ファイル名がリソース識別子として使われます。

\subsection{テキストリソースを取得する}
\label{cr-resources-getString}

埋め込んだリソースを取得するには、\type{haxe.Resource}の\emph{getString}の静的メソッドにリソース識別子を渡して事項します。

\haxe{assets/ResourceGetString.hx}

上記のコードは先ほどの\emph{welcome}を識別子として使って\emph{hello_message.txt}ファイルの内容を表示します。

\subsection{バイナリリソースを取得する}
\label{cr-resources-getBytes}

巨大バイナリファイルをアプリケーションに埋め込むのは推奨されないものの、バイナリデータの埋め込みは便利です。埋め込んだリソースは\type{haxe.Resource}の\emph{getBytes}の静的メソッドを使うことでバイナリとして取得できます。

\haxe{assets/ResourceGetBytes.hx}

\emph{getBytes}メソッドの戻り値の型は、データの各バイトにアクセスできる\type{haxe.io.Bytes}です。

\subsection{実装の詳細}
\label{cr-resources-impl}

ターゲットのプラットフォームにリソースの埋め込み機能があればそれを使います。その他の場合、独自の実装を持ちます。

\begin{itemize}
\item \emph{Flash} リソースは\type{ByteArray}として定義されて埋め込まれる。
\item \emph{C\#} コンパイルされたアセンブリに含まれる。
\item \emph{Java} JARファイル内にパッケージされる。
\item \emph{C++} グローバルなバイト列の定数として記録される。
\item \emph{JavaScript} Haxeシリアル化フォーマットに従ってシリアル化されて\type{haxe.Resource}の静的フィールドに記録される。
\item \emph{Neko} 文字列として\type{haxe.Resource}クラスの静的フィールドに記録される。
\end{itemize}


\section{実行時型情報(RTTI)}
\label{cr-rtti}

Haxeコンパイラは\expr{:rtti}メタデータで修飾されたクラス、あるいはその子孫クラスに対して実行時型情報(RTTI)を生成します。

\since{3.2.0}

\type{haxe.rtti.Rtti}型がRTTIについての処理を簡単にするために導入されました。現在では、情報の取得はとても簡単です。

\haxe{assets/RttiUsage.hx}

\subsection{RTTIの構造}
\label{cr-rtti-structure}

\paragraph{一般的な型情報}

\begin{description}
	\item[path:] \tref{型のパス}{define-type-path}。
	\item[module:] その型を含んでいる\tref{モジュール}{define-module}のパス。
	\item[file:] その型を含む.hxのファイルのフルパス。例えば\tref{マクロ}{macro}から定義された場合など、そのようなファイルがない場合\expr{null}になる場合がある。
	\item[params:] その型が持つ\tref{型パラメータ}{type-system-type-parameters}を表す文字列の配列。Haxe3.2.0では、\tref{制約}{type-system-type-parameter-constraints}についての情報を持ちません。
	\item[doc:] その型のドキュメント。この情報は\expr{-D use_rtti_doc}の\tref{コンパイラフラグ}{define-compiler-flag}を付けた場合のみ使用できます。フラグがないか、ドキュメントがない場合は\expr{null}です。
	\item[isPrivate:] その型が\tref{private}{define-private-type}かどうか。
	\item[platforms:] その型が利用可能なターゲットのリスト。
	\item[meta:] その型につけられているメタデータ。
\end{description}
	
\paragraph{クラスの型情報}
\label{cr-rtti-class-type-information}

\begin{description}
	\item[isExtern:] クラスが\tref{extern}{lf-externs}かどうか。
	\item[isInterface:] \tref{インターフェース}{types-interfaces}かどうか。
	\item[superClass:] その親クラスの型のパスと型パラメータのリスト。
	\item[interfaces:] そのクラスのインターフェースの型のパスと型パラメータのリストのリスト。
	\item[fields:] \Fullref{cr-rtti-class-field-information}に記載されている、\tref{クラスフィールド}{class-field}のリスト
	\item[statics:] \Fullref{cr-rtti-class-field-information}に記載されている、静的クラスフィールドのリスト。
	\item[tdynamic:] そのクラスに\tref{動的に実装}{types-dynamic-implemented}された型、あるいは型が存在しない場合は\expr{null}
\end{description}

\paragraph{列挙型の型情報}

\begin{description}
	\item[isExtern:] その列挙型が\tref{extern}{lf-externs}かどうか。
	\item[constructors:] その列挙型コンストラクタのリスト。
\end{description}

\paragraph{抽象型の型情報}

\begin{description}
	\item[to:] 定義されている\tref{暗黙のtoキャスト}{types-abstract-implicit-casts}の配列。
	\item[from:] 定義されている\tref{暗黙のtoキャスト}{types-abstract-implicit-casts}の配列。
	\item[impl:] 実装しているクラスの\tref{クラスの型情報}{cr-rtti-class-type-information}。
	\item[athis:] その抽象型の\tref{基底型}{define-underlying-type}。
\end{description}
	
	
\paragraph{クラスフィールド情報}
\label{cr-rtti-class-field-information}

\begin{description}
	\item[name:] フィールドの名前。
	\item[type:] フィールドの型。
	\item[isPublic:] フィールドが\tref{public}{class-field-visibility}かどうか。
	\item[isOverride:] フィールドが別のフィールドの\tref{オーバーライド}{class-field-override}かどうか。
	\item[doc:] フィールドのドキュメント。この情報は\expr{-D use_rtti_doc}の\tref{コンパイラフラグ}{define-compiler-flag}を付けた場合のみ使用できます。フラグがないか、ドキュメントがない場合は\expr{null}です。
	\item[get:] フィールドの\tref{読み込みアクセスの挙動}{define-read-access}。
	\item[set:] フィールドの\tref{書き込みアクセスの挙動}{define-write-access}。
	\item[params:] そのフィールドが持つ\tref{型パラメータ}{type-system-type-parameters}を表す文字列の配列。Haxe3.2.0では、\tref{制約}{type-system-type-parameter-constraints}についての情報を持ちません。
	\item[platforms:] フィールドが使用可能なターゲットのリスト。
	\item[meta:] フィールドに付けられているメタデータ。
	\item[line:] フィールドが定義されている行番号。この情報はフィールドが式を持つ場合のみ使用可能です。その他の場合は\expr{null}です
	\item[overloads:] このフィールドに対する利用可能なオーバーロードのリスト。存在しなければ\expr{null}です。
\end{description}

\paragraph{列挙型コンストラクタ情報}
\label{cr-rtti-enum-constructor-information}

\begin{description}
	\item[name:] コンストラクタ名。
	\item[args:] 引数のリスト。存在しなければ\expr{null}です。
	\item[doc:] コンストラクタのドキュメント。この情報は\expr{-D use_rtti_doc}の\tref{コンパイラフラグ}{define-compiler-flag}を付けた場合のみ使用できます。フラグがないか、ドキュメントがない場合は\expr{null}です。
	\item[platforms:] コンストラクタが使用可能なターゲットのリスト。
	\item[meta:] コンストラクタに付けられているメタデータ。
\end{description}
