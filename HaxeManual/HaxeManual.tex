\documentclass{../haxe}

%for JapaneseLocalize
\usepackage{xltxtra}
\setmainfont{IPAPMincho}
\setsansfont{IPAPGothic}
\setmonofont{IPAGothic}
\XeTeXlinebreaklocale "ja"

\XeTeXlinebreakskip=0em plus 0.8em minus 0.07em
\XeTeXlinebreakpenalty=0

\setlength{\textwidth}{17cm}
\setlength{\textheight}{24cm}
\setlength{\leftmargin}{-1cm}
\setlength{\topmargin}{-2cm}
\setlength{\oddsidemargin}{0cm}
\setlength{\evensidemargin}{0cm}

% todo-related
\usepackage[left=4.7cm, right=2cm, top=2cm, bottom=4.2cm]{geometry}
\usepackage[draft]{todonotes}
\reversemarginpar

% title (TODO: move this to class file once it looks good)

\renewcommand{\maketitle}{
   \begin{titlepage}
     \setcounter{page}{-1}
			\begin{center}
				~\\[3cm]
				\includegraphics[scale=1.25]{../assets/logo.pdf}~\\[1cm]
				{\huge \bfseries Haxe 3 マニュアル}\\[7cm]
				Haxe Foundation\\
				April 12, 2015\\
				(訳 : \today)
			\end{center}
   \end{titlepage}
}


\input{../tikz}

% Conventions:

% run-time, compile-time
% Haxe, Haxelib (unless we are talking about the command itself)
% Haxe Standard Library, Haxe Compiler
% object-oriented

% code example width for ebooks: 47

\begin{document}
\title{Haxe 3 マニュアル}
\author{Haxe Foundation}
\date{\today}
\maketitle


\clearpage
\todototoc
\listoftodos
\clearpage

\clearpage
\tableofcontents
\clearpage

\chapter{導入}
\label{introduction}
\state{NoContent}

\section{Haxeって何?}
\label{introduction-what-is-haxe}

\todo{Could we have a big Haxe logo in the First Manual Page (Introduction) under the menu (a bit like a book cover ?) It looks a bit empty now and is a landing page for "Manual"}

Haxeはオープンソースの高級プログラミング言語とコンパイラで構成されており、ECMAScript\footnote{http://www.ecma-international.org/publications/standards/Ecma-327.htm}を元にした構文で書いて、さまざまなターゲットの言語へとコンパイルすることを可能です。適度な抽象化を行うため、1つのコードベースから複数のターゲットへコンパイルすることも可能です。

Haxeは強く型付けされている一方で、必要に応じて型付けを弱めることも可能です。型情報を活用すれば、ターゲットの言語では実行時にしか発見できないようなエラーをコンパイル時に検出することができます。さらに型情報は、ターゲットへの変換時に最適化や堅牢なコードを生成するためにも使用されます。

現在、Haxeには9つのターゲット言語があり、さまざまな用途に利用できます。

\begin{center}
\begin{tabular}{| l | l | l |}
	\hline
	名前 & 出力形式 & 主な用途 \\ \hline
	JavaScript & ソースコード & ブラウザ, デスクトップ, モバイル, サーバー \\
	Neko & バイトコード & デスクトップ, サーバー \\
	PHP & ソースコード & サーバー \\
	Python & ソースコード & デスクトップ, サーバー \\
	C++ & ソースコード & デスクトップ, モバイル, サーバー \\
	ActionScript 3 & ソースコード & ブラウザ, デスクトップ, モバイル \\
	Flash & バイトコード & ブラウザ, デスクトップ, モバイル \\ 
	Java & ソースコード & デスクトップ, サーバー \\
	C\# & ソースコード & デスクトップ, モバイル, サーバー \\ \hline
\end{tabular}
\end{center}

この\Fullref{introduction}の残りでは、Haxeのプログラムがどのようなものなのか、Haxeはが2005年に生まれてからどのように進化してきたのか、を概要でお送りします。

\Fullref{types}では、Haxeの7種類の異なる型についてとそれらがどう関わりあっているのかについて紹介します。型に関する話は、\Fullref{type-system}へと続き、\emph{単一化(Unification)}、\emph{型パラメータ}、\emph{型推論}についての解説がされます。

\Fullref{class-field}では、Haxeのクラスの構造に関する全てをあつかいます。加えて、\emph{プロパティ}、\emph{インラインフィールド}、\emph{ジェネリック関数}についてもあつかいます。

\Fullref{expression}では、\emph{式}を使用して実際にいくつかの動作をさせる方法をお見せします。

\Fullref{lf}では、\emph{パターンマッチング}、\emph{文字列補間}、\emph{デッドコード削除}のようなHaxeの詳細の機能について記述しています。ここで、Haxeの言語リファレンスは終わりです。

そして、Haxeのコンパイラリファレンスへと続きます。まずは\Fullref{compiler-usage}で基本的な内容を、そして、\Fullref{cr-features}で高度な機能をあつかいます。最後に\Fullref{macro}で、ありふれたタスクを\emph{Haxeマクロ}がどのように単純かするのかを見ながら、刺激的なマクロの世界に挑んでいきます。

次の\Fullref{std}のでは、Haxeの標準ライブラリに含まれる主要な型や概念を一つ一つ見ていきます。そして、\Fullref{haxelib}でHaxeのパッケージマネージャであるHaxelibについて学びます。

Haxeは様々なターゲット間の差を吸収してくれますが、場合によってはターゲットを直接的にあつかうことが重要になります。これが、\Fullref{target-details}の話題です。

\section{このドキュメントについて}
\label{introduction-about-this-document}

このドキュメントは、Haxe3の公式マニュアル(の非公式日本語訳)です。そのため、初心者向けののチュートリアルではなく、プログラミングは教えません。しかし、項目は大まかに前から順番に読めるように並べてあり、前に出てきた項目と、次に出てくる項目との関連づけがされています。先の項目で後の項目でててくる情報に触れた方が説明しやすい場所では、先にその情報に触れています。そのような場面ではリンクがされています。リンク先は、ほとんどの場合で先に読むべき内容ではありません。

このドキュメントでは、理論的な要素を実物としてつなげるために、たくさんのHaxeのソースコードを使います。これらのコードのほとんどはmain関数を含む完全なコードでありそのままコンパイルが可能ですが、いくつかはそうではなくコードの重要な部分の抜き出しです。

ソースコードは以下のように示されます:

\begin{lstlisting}
Haxe code here
\end{lstlisting}

時々、Haxeがどのようなコードを出力をするかを見せるため、ターゲットの\target{JavaScript}などのコードも示します。

さらに、このドキュメントではいくつかの単語の定義を行います。定義は主に、新しい型やHaxe特有の単語を紹介するときに行われます。私たちが紹介するすべての新しい内容に対して定義をするわけではありません(例えば、クラスの定義など)。

定義は以下のように示されます。

\define{定義の名前}{define-definition}{定義の説明}

また、いくつかの場所には\emph{トリビア}欄を用意してます。トリビア欄では、Haxeの開発過程でどうしてそのような決定がなされたのか、なぜその機能が過去のHaxeのバージョンから変更されたのかなど非公開の情報をお届けします。この情報は一般的には重要ではない、些細な内容なので読み飛ばしても構いません。

\trivia{トリビアについて}{これはトリビアです}

\subsection{著者と貢献者}
\label{introduction-authors-and-contributions}

このドキュメントの大半の内容は、Haxe Foundationで働くSimon Krajewskiによって書かれました。そして、このドキュメントの貢献者である以下の方々に感謝の意を表します。

\begin{itemize}
	\item Dan Korostelev: 追加の内容と編集
	\item Caleb Harper: 追加の内容と編集
	\item Josefiene Pertosa: 編集
	\item Miha Lunar: 編集
	\item Nicolas Cannasse: Haxe創始者
\end{itemize}

\subsection{License}
\label{introduction-license}

The Haxe Manual by \href{http://haxe.org/foundation}{Haxe Foundation} is licensed under a \href{http://creativecommons.org/licenses/by/4.0/}{Creative Commons Attribution 4.0 International License}.

Based on a work at \href{https://github.com/HaxeFoundation/HaxeManual}{https://github.com/HaxeFoundation/HaxeManual}.

\section{Hello World}
\label{introduction-hello-world}

次のプログラムはコンパイルして実行をすると``Hello World''と表示します:

\haxe{assets/HelloWorld.hx}

上記のコードは、\ic{HelloWorld.hx}という名前で保存して、\ic{haxe -main HelloWorld --interp}というコマンドでHaxeコンパイラを呼び出すと実際に動作させることが可能です。これで\ic{HelloWorld.hx:3: Hello world}という出力がされるはずです。このことから以下のいくつかのことを学ぶことができます。

\todo{This generates the following output: too many 'this'. You may like a passive sentence: the following output will be generated...though this is to be avoided, generally}

\begin{itemize}
	\item Haxeのコードは\ic{.hx}という拡張子で保存する。
	\item Haxeのコンパイラはコマンドラインツールであり、\ic{-main HelloWorld}や\ic{--interp}のようなパラメータをつけて呼び出すことができる。
	\item Haxeのプログラムにはクラスがあり(\type{HelloWorld}、大文字から始まる)、クラスには関数がある(\expr{main}、小文字からはじまる)。 
	\item Haxeのmainクラスをふくむファイルは、そのクラス名と同じ名前です(この場合だと、\type{HelloWorld.hx})。
\end{itemize}

\section{歴史}
\label{introduction-haxe-history}
\state{Reviewed}

Haxeのプロジェクトは、2005年10月22日にフランスの開発者の\emph{Nicolas Cannasse}によって、オープンソースのActionScript2コンパイラである\emph{MTASC}(Motion-Twin Action Script Compiler)と、Motion-Twinの社内言語であり、実験的に型推論をオブジェクト指向に取り入れた\emph{MTypes}の後継として始まりました。Nicolasのプログラミング言語の設計に対する長年の情熱と、\emph{Motion-Twin}でゲーム開発者として働くことで異なる技術が混ざり合う機会を得たことが、まったく新しい言語の作成に結び付いたのです。

そのころのつづりは\emph{haXe}で、2006年の2月にAVMのバイトコードとNicolas自身が作成した\emph{Neko}バーチャルマシン\footnote{http://nekovm.org}への出力をサポートするベータ版がリリースされました。

この日からHaxeプロジェクトのリーダーであり続けるNicolas Cannasseは明確なビジョンをもってHaxeの設計を続け、そして2006年5月のHaxe1.0のリリースに導きました。この最初のメジャーリリースから\target{Javascript}のコード生成をサポートの始まり、型推論や構造的部分型などの現在のHaxeの機能のいくつかはすでにこのころからありました。

Haxe1では、2年間いくつかのマイナーリリースを行い、2006年8月に\target{Flash AVM2}ターゲットと\emph{haxelib}ツール、2007年3月に\target{ActionScript3}ターゲットを追加しました。この時期は安定性の改善に強く焦点が当てられ、その結果、数回のマイナーリリースが行われました。

Haxe2.0は2008年7月にリリースされました。\emph{Franco Ponticelli}の好意により、このリリースには\target{PHP}ターゲットが含まれました。同様に、\emph{Hugh Sanderson}の貢献により、2009年7月のHaxe2.04リリースで\target{C++}ターゲットが追加されました。

Haxe1と同じように、以降の数か月で安定性のためのリリースを行いました。そして2011年1月、\emph{macros}をサポートするHaxe2.07がリリースされました。このころに、\emph{Bruno Garcia}が\target{JavaScript}ターゲットのメンテナとしてチームに加わり、 2.08と2.09のリリースで劇的な改善が行われました。

2.09のリリース後、\emph{Simon Krajewski}がチームに加わり、Haxe3の出発に向けて働き始めました。さらに\emph{Cau\^{e} Waneck}の\target{Java}と\target{C\#}のターゲットがHaxeのビルドに取り込まれました。そしてHaxe2は次で最後のリリースとなることが決まり、2012年1月にHaxe2.10がリリースされました。

2012年の終盤、Haxe3にスイッチを切り替えて、Haxeコンパイラチームは、新しく設立された\emph{Haxe Foundation}\footnote{http://haxe-foundation.org}の援助を受けながら、次のメジャーバージョンに向かっていきました。そして、Haxe3は2013年の5月にリリースされました。


\part{言語リファレンス}
\chapter{型}
\label{types}

Haxeコンパイラは豊かな型システムを持っており、これがコンパイル時に型エラーを検出することを手助けします。型エラーとは、文字列による割り算や、整数のフィールドへのアクセス、不十分な(あるいは多すぎる)引数での関数呼び出し、といった型が不正である演算のことです。

いくつかの言語では、この安全性を得るためには各構文での明示的な型の宣言が強いられるので、コストがかかります。

\begin{lstlisting}
var myButton:MySpecialButton = new MySpecialButton(); // AS3
MySpecialButton* myButton = new MySpecialButton(); // C++ 
\end{lstlisting}

一方、Haxeではコンパイラが型を\emph{推論}できるため、この明示的な型注釈は必要ではありません。

\begin{lstlisting}
var myButton = new MySpecialButton(); // Haxe
\end{lstlisting}

型推論の詳細については\Fullref{type-system-type-inference}で説明します。今のところは、上のコードの変数\expr{myButton}は\type{MySpecialButton}の\emph{クラスインスタンス}とわかると言っておけば十分でしょう。

Haxeの型システムは、以下の7つの型を認識します。

\begin{description}
 \item[\emph{クラスインスタンス}:] クラスかインスタンスのオブジェクト
 \item[\emph{列挙インスタンス}:] Haxeの列挙型(enum)の値
 \item[\emph{構造体}:] 匿名の構造体。例えば、連想配列。
 \item[\emph{関数}:] 引数と戻り値1つの型の複合型。
 \item[\emph{ダイナミック}:] あらゆる型に一致する、なんでも型。
 \item[\emph{抽象(abstract)}:] 実行時には別の型となる、コンパイル時の型。
 \item[\emph{単態(monomorph)}:] 後で別の型が付けられる未知(Unknown)の型。
\end{description}

ここからの節で、それぞれの型のグループとこれらがどうかかわっているのかについて解説していきます。

\define{複合型(Compound Type)}{define-compound-type}{
複合型というのは、従属する型を持つ型です。\tref{型パラメータ}{type-system-type-parameters}を持つ型や、\tref{関数}{types-function}型がこれに当たります。
}

\section{基本型}
\label{types-basic-types}

\emph{基本型}は\type{Bool}と\type{Float}と\type{Int}です。文法上、これらの値は以下のように簡単に識別可能です。

\begin{itemize}
	\item \expr{true}と\expr{false}は\type{Bool}。
	\item \expr{1}、\expr{0}、\expr{-1}、\expr{0xFF0000}は\type{Int}。
	\item \expr{1.0}、\expr{0.0}、\expr{-1.0}、\expr{1e10}は\type{Float}。
\end{itemize}

Haxeでは基本型は\tref{クラス}{types-class-instance}ではありません。これらは\tref{抽象型}{types-abstract}として実装されており、以降の項で解説するコンパイラ内部の演算処理に結び付けられています。

\subsection{数値型}
\label{types-numeric-types}

\define[Type]{Float}{define-float}{IEEEの64bit倍精度浮動小数点数を表します。}

\define[Type]{Int}{define-int}{整数を表します。}

\type{Int}は\type{Float}が期待されるすべての場所で使用することが可能です (IntはFloatへの代入が可能で、Floatとして表現可能です)。ですが、逆はできません。 \type{Float}から\type{Int}への代入は精度を失ってしまう場合があり、信頼できません。

\subsection{オーバーフロー}
\label{types-overflow}

パフォーマンスのためにHaxeコンパイラはオーバーフローに対する挙動を矯正しません。オーバーフローに対する挙動は、ターゲットのプラットフォームが責任を持ちます。各プラットフォームごとのオーバーフローの挙動を以下にまとめています。

\begin{description}
	\item[C++, Java, C\#, Neko, Flash:] 一般的な挙動をもつ32Bit符号付き整数。
	\item[PHP, JS, Flash 8:] ネイティブの\emph{Int}型を持たない。Floatの上限(2\textsuperscript{52})を超えた場合に精度を失う。
\end{description}

代替手段として、プラットフォームごとの追加の計算を行う代わりに、正しいオーバーフローの挙動を持つ\emph{haxe.Int32}と\emph{haxe.Int64}クラスが用意されています。

\subsection{数値の演算子}
\label{types-numeric-operators}

\todo{make sure the types are right for inc, dec, negate, and bitwise negate}
\todo{While introducing the different operations, we should include that information as well, including how they differ with the "C" standard, see http://haxe.org/manual/operators}

\begin{center}
\begin{tabular}{| l | l | l | l | l |}
	\hline
	\multicolumn{5}{|c|}{算術演算} \\ \hline
	演算子 & 演算 & 引数1 & 引数2 & 戻り値 \\ \hline
	\expr{++} & 1増加 & \type{Int} & なし & \type{Int}\\
	& & \type{Float} & なし & \type{Float}\\
	\expr{--} & 1減少 & \type{Int} & なし & \type{Int}\\
	& & \type{Float} & なし & \type{Float}\\
	\expr{+} & 加算 & \type{Float} & \type{Float} & \type{Float} \\
	& & \type{Float} & \type{Int} & \type{Float} \\
	& & \type{Int} & \type{Float} & \type{Float} \\
	& & \type{Int} & \type{Int} & \type{Int} \\
	\expr{-} & 減算 & \type{Float} & \type{Float} & \type{Float} \\
	& & \type{Float} & \type{Int} & \type{Float} \\
	& & \type{Int} & \type{Float} & \type{Float} \\
	& & \type{Int} & \type{Int} & \type{Int} \\
	\expr{*} & 乗算 & \type{Float} & \type{Float} & \type{Float} \\
	& & \type{Float} & \type{Int} & \type{Float} \\
	& & \type{Int} & \type{Float} & \type{Float} \\
	& & \type{Int} & \type{Int} & \type{Int} \\	
	\expr{/} & 除算 & \type{Float} & \type{Float} & \type{Float} \\
	& & \type{Float} & \type{Int} & \type{Float} \\
	& & \type{Int} & \type{Float} & \type{Float} \\
	& & \type{Int} & \type{Int} & \type{Float} \\
	\expr{\%} & 剰余 & \type{Float} & \type{Float} & \type{Float} \\
	& & \type{Float} & \type{Int} & \type{Float} \\
	& & \type{Int} & \type{Float} & \type{Float} \\
	& & \type{Int} & \type{Int} & \type{Int} \\	 \hline
	\multicolumn{5}{|c|}{比較演算} \\ \hline
	演算子 & 演算 & 引数1 & 引数2 & 戻り値 \\ \hline
	\expr{==} & 等価 & \type{Float/Int} & \type{Float/Int} & \type{Bool} \\
	\expr{!=} & 不等価 & \type{Float/Int} & \type{Float/Int} & \type{Bool} \\
	\expr{<} & より小さい & \type{Float/Int} & \type{Float/Int} & \type{Bool} \\
	\expr{<=} & より小さいか等しい & \type{Float/Int} & \type{Float/Int} & \type{Bool} \\
	\expr{>} & より大きい & \type{Float/Int} & \type{Float/Int} & \type{Bool} \\
	\expr{>=} & より大きいか等しい & \type{Float/Int} & \type{Float/Int} & \type{Bool} \\ \hline
	\multicolumn{5}{|c|}{ビット演算} \\ \hline
	演算子 & 演算 & 引数1 & 引数2 & 戻り値 \\ \hline
	\expr{\textasciitilde} & ビット単位の否定(NOT) & \type{Int} & なし & \type{Int} \\	
	\expr{\&} & ビット単位の論理積(AND) & \type{Int} & \type{Int} & \type{Int} \\	
	\expr{|} & ビット単位の論理和(OR) & \type{Int} & \type{Int} & \type{Int} \\	
	\expr{\^} & ビット単位の排他的論理和(XOR) & \type{Int} & \type{Int} & \type{Int} \\	
	\expr{<<} & 左シフト & \type{Int} & \type{Int} & \type{Int} \\
	\expr{>>} & 右シフト & \type{Int} & \type{Int} & \type{Int} \\
	\expr{>>>} & 符号なしの右シフト & \type{Int} & \type{Int} & \type{Int} \\ \hline
\end{tabular}
\end{center}

\subsection{Bool(真偽値)}
\label{types-bool}

\define[Type]{Bool}{define-bool}{真(\emph{true})または、偽(\emph{false})のどちらかになる値を表す。}

\type{Bool}型の値は、\tref{\expr{if}}{expression-if}や\tref{\expr{while}}{expression-while}のような\emph{条件文}によく表れます。以下の演算子は、\type{Bool}値を受け取って\type{Bool}値を返します。

\begin{itemize}
	\item \expr{\&\&} (and)
	\item \expr{||} (or)
	\item \expr{!} (not)
\end{itemize}

Haxeは、Bool値の2項演算は、実行時に左から右へ必要な分だけ評価することを保証します。例えば、\expr{A \&\& B}という式は、まず\expr{A}を評価して\expr{A}が\expr{true}だった場合のみ\expr{B}が評価されます。同じように、\expr{A || B}では\expr{A}が\expr{true}だった場合は、\expr{B}の値は意味を持たないので評価されません。

これは、以下のような場合に重要です。

\begin{lstlisting}
if (object != null && object.field == 1) {
  ...
}
\end{lstlisting}

\expr{object}が\expr{null}の場合に\expr{object.field}にアクセスするとランタイムエラーになりますが、事前に\expr{object != null}のチェックをすることでエラーから守ることができます。

\subsection{Void}
\label{types-void}

\define[Type]{Void}{define-void}{Voidは型が存在しないことを表します。特定の場面(主に関数)で値を持たないことを表現するのに使います。}

Voidは型システムにおける特殊な場合です。Voidは実際には型ではありません。Voidは特に関数の引数と戻り値で型が存在しないことを表現するのに使います。私たちはすでに最初の``Hello World''の例でVoidを使用しています。
\todo{please review, doubled content}

\haxe{assets/HelloWorld.hx}

関数型について詳しくは\Fullref{types-function}で解説しますが、ここで軽く予習をしておきましょう。上の例の\expr{main}関数は\type{Void->Void}型です。これは``引数は無く、戻り値も無い''という意味です。

Haxeでは、フィールドや変数に対してVoidを指定することはできません。以下のように書こうとするとエラーが発生します。
\todo{review please, sounds weird}

\begin{lstlisting}
// Arguments and variables of type Void
// are not allowed
var x:Void;
\end{lstlisting}



\section{Nullable(null許容型)}
\label{types-nullability}

\define{Nullable}{define-nullable}{Haxeでは、ある型が値として\expr{null}をとる場合に\emph{Nullable}(null許容型)であるとみなす。}

プログラミング言語は、Nullableについてなにか1つ明確な定義を持つのが一般的です。ですが、Haxeではターゲットとなる言語のもともとの挙動に従うことで妥協しています。ターゲット言語のうちのいくつかは全てがデフォルト値として\expr{null}をとり、その他は特定の型では\expr{null}を許容しません。つまり、以下の2種類の言語を区別しなくてはいけません。

\define{静的ターゲット}{define-static-target}{静的ターゲットでは、その言語自体が基本型が\expr{null}を許容しないような型システムを持っています。この性質は\target{Flash}、\target{C++}、\target{Java}、\target{C\#}ターゲットに当てはまります。}
\define{動的ターゲット}{define-dynamic-target}{動的ターゲットはもっと型に関して寛容で、基本型が\expr{null}を許容します。これは\target{JavaScript}と\target{PHP}、\target{Neko}、\target{Flash 6-8}ターゲットが当てはまります。}
\todo{for starters...please review}

\define{デフォルト値}{define-default-value}{
  基本型は、静的ターゲットではデフォルト値は以下になります。
  \begin{description}
		\item[\type{Int}:] \expr{0}。
		\item[\type{Float}:] \target{Flash}では\expr{NaN}。その他の静的ターゲットでは\expr{0.0}。
		\item[\type{Bool}:] \expr{false}。
	\end{description}
}

その結果、Haxeコンパイラは静的ターゲットでは基本型に対する\expr{null}を代入することはできません。\expr{null}を代入するためには、以下のように基本型を\type{Null$<$T$>$}で囲う必要があります。

\begin{lstlisting}
// error on static platforms
var a:Int = null;
var b:Null<Int> = null; // allowed
\end{lstlisting}

同じように、基本型は\type{Null$<$T$>$}で囲わなければ\expr{null}と比較することはできません。

\begin{lstlisting}
var a : Int = 0;
// error on static platforms
if( a == null ) { ... }
var b : Null<Int> = 0;
if( b != null ) { ... } // allowed
\end{lstlisting}

この制限は\tref{unification}{type-system-unification}が動作するすべての状況でかかります。

\define[Type]{\expr{Null<T>}}{define-null-t}{静的ターゲットでは、\type{Null<Int>}、\type{Null<Float>}、\type{Null<Bool>}の型で\expr{null}を許容することが可能になります。動的ターゲットでは\expr{Null<T>}に効果はありません。また、\expr{Null<T>}はその型が\expr{null}を持つことを表すドキュメントとしても使うことができます。}

nullの値は隠匿されます。つまり、\type{Null$<$T$>$}や\type{Dynamic}のnullの値を基本型に代入した場合には、デフォルト値が使用されます。

\begin{lstlisting}
var n : Null<Int> = null;
var a : Int = n;
trace(a); // 0 on static platforms
\end{lstlisting}



\subsection{オプション引数とnull許容}
\label{types-nullability-optional-arguments}

null許容について考える場合、オプション引数についても考慮しなくてはいけません。

特に、null許容ではない\emph{ネイティブ}のオプション引数と、それとは異なる、null許容であるHaxe特有のオプション引数があることです。この違いは以下のように、オプション引数にクエスチョンマークを付けることで作ります。

\begin{lstlisting}
// x is a native Int (not nullable)
function foo(x : Int = 0) {...}
// y is Null<Int> (nullable)
function bar( ?y : Int) {...}
// z is also Null<Int>
function opt( ?z : Int = -1) {...}
\end{lstlisting}
\todo{Is there a difference between \type{?y : Int} and \type{y : Null$<$Int$>$} or can you even do the latter? Some more explanation and examples with native optional and Haxe optional arguments and how they relate to nullability would be nice.}

\trivia{アーギュメント(Argument)とパラメータ(Parameter)}{他のプログラミング言語では、よく\emph{アーギュメント}と\emph{パラメータ}は同様の意味として使われます。Haxeでは、関数に関連する場合に\emph{アーギュメント}を、\Fullref{type-system-type-parameters}と関連する場合に\emph{パラメータ}を使います。}

\section{クラスインスタンス}
\label{types-class-instance}


多くのオブジェクト指向言語と同じように、Haxeでも大抵のプログラムではクラスが最も重要なデータ構造です。Haxeのすべてのクラスは、明示された名前と、クラスの配置されたパスと、0個以上のクラスフィールドを持ちます。ここではクラスの一般的な構造とその関わり合いについて焦点を当てていきます。クラスフィールドの詳細については後で\Fullref{class-field}の章で解説をします。
\todo{please review future tense}

以下のサンプルコードが、この節で学ぶ基本になります。

\haxe{assets/Point.hx}

意味的にはこれは不連続の2次元空間上の点を表現するものですが、このことはあまり重要ではありません。代わりにその構造に注目しましょう。

\begin{itemize}
	\item \expr{class}のキーワードは、クラスを宣言していることを示すものです。
	\item \type{Point}はクラス名です。\tref{型の識別子のルール}{define-identifier}に従っているものが使用できます。
	\item クラスフィールドは\expr{$\left\{\right\}$}で囲われます。
	\item \type{Int}型の\expr{x}と\expr{y}の2つの\emph{変数}フィールドと、
	\item クラスの\emph{コンストラクタ}となる特殊な\emph{関数}フィールド\expr{new}と、
	\item 通常の関数\expr{toString}でクラスフィールドが構成されています。
\end{itemize}

また、Haxeにはすべてのクラスと一致する特殊な型があります。

\define[Type]{\expr{Class$<$T$>$}}{define-class-t}{
この型はすべてのクラスの型と一致します。つまり、すべてのクラス(インスタンスではなくクラス)をこれに代入することができます。コンパイル時に、\type{Class<T>}は全てのクラスの型の共通の親の型となります。しかし、この関係性は生成されたコードに影響を与えません。

この型は、任意のクラスを要求するようなAPIで有用です。例えば、\tref{HaxeリフレクションAPI}{std-reflection}のいくつかのメソッドがこれに当てはまります。
}

\subsection{クラスのコンストラクタ}
\label{types-class-constructor}

クラスのインスタンスは、クラスのコンストラクタを呼び出すことで生成されます。この過程は一般的に\emph{インスタンス化}と呼ばれます。クラスインスタンスは、別名として\emph{オブジェクト}と呼ぶこともあります。ですが、クラス/クラスインスタンスと、列挙型/列挙型インスタンスという似た概念を区別するために、クラスインスタンスと呼ぶことが好まれます。

\begin{lstlisting}
var p = new Point(-1, 65);
\end{lstlisting}

この例で、変数\expr{p}に代入されたのが\type{Point}クラスのインスタンスです。\type{Point}のコンストラクタは\expr{-1}と\expr{65}の2つの引数を受け取り、これらをそれぞれインスタンスの\expr{x}と\expr{y}の変数に代入しています(\Fullref{types-class-instance}で、定義を確認してください)。\expr{new}の正確な意味については、後の\ref{expression-new}の節で再習します。現時点では、\expr{new}はクラスのコンストラクタを呼び、適切なオブジェクトを返すものと考えておきましょう。


\subsection{継承}
\label{types-class-inheritance}

クラスは他のクラスから継承ができます。Haxeでは、継承は\expr{extends}キーワードを使って行います。

\haxe{assets/Point3.hx}

この関係は、よく"BはAである(is-a)"の関係とよく言われます。つまり、すべての\type{Point3}クラスのインスタンスは、同時に\type{Point}のインスタンスである、ということです。\type{Point}は\type{Point3}の\emph{親クラス}であると言い、\type{Point3}は\type{Point}の\emph{子クラス}であると言います。1つのクラスはたくさんの子クラスを持つことができますが、親クラスは1つしか持つことができません。ただし、``クラスXの親クラス''というのは、直接の親クラスだけでなく、親クラスの親クラスや、そのまた親、また親のクラスなどを指すこともよくあります。

上記のクラスは\type{Point}コンストラクタによく似ていますが、2つの新しい構文が登場しています。

\begin{itemize}
	\item \expr{extends Point} は\type{Point}からの継承を意味します。
	\item \expr{super(x, y)} は親クラスのコンストラクタを呼び出します。この場合は\expr{Point.new}です。
\end{itemize}

上の例ではコンストラクタを定義していますが、子クラスで自分自身のコンストラクタを定義する必要はありません。ただし、コンストラクタを定義する場合\expr{super()}の呼び出しが必須になります。他のよくあるオブジェクト指向言語とは異なり、\expr{super()}はコンストラクタの最初である必要はなく、どこで呼び出しても構いません。

また、クラスはその親クラスの\tref{メソッド}{class-field-method}を\expr{override}キーワードを明示して記述することで上書きすることができます。その効果と制限について詳しくは\Fullref{class-field-overriding}であつかいます。


\subsection{Interfaces}
\label{types-interfaces}

An interface can be understood as the signature of a class because it describes the public fields of a class. Interfaces do not provide implementations but pure structural information:

\begin{lstlisting}
interface Printable {
	public function toString():String;
}
\end{lstlisting}
The syntax is similar to classes, with the following exceptions:

\begin{itemize}
	\item \expr{interface} keyword is used instead of \expr{class} keyword
	\item functions do not have any \tref{expressions}{expression}
	\item every field must have an explicit type
\end{itemize}
Interfaces, unlike \tref{structural subtyping}{type-system-structural-subtyping}, describe a \emph{static relation} between classes. A given class is only considered to be compatible to an interface if it explicitly states so:

\begin{lstlisting}
class Point implements Printable { }
\end{lstlisting}
Here, the \expr{implements} keyword denotes that \type{Point} has a "is-a" relationship to \type{Printable}, i.e. each instance of \type{Point} is also an instance of \type{Printable}. While a class may only have one parent class, it may implement multiple interfaces through multiple \expr{implements} keywords:

\begin{lstlisting}
class Point implements Printable
  implements Serializable
\end{lstlisting}

The compiler checks if the \expr{implements} assumption holds. That is, it makes sure the class actually does implement all the fields required by the interface. A field is considered implemented if the class or any of its parent classes provide an implementation.

Interface fields are not limited to methods. They can be variables and properties as well:

\haxe{assets/InterfaceWithVariables.hx}

\trivia{Implements Syntax}{Haxe versions prior to 3.0 required multiple \expr{implements} keywords to be separated by a comma. We decided to adhere to the de-facto standard of Java and got rid of the comma. This was one of the breaking changes between Haxe 2 and 3.}


\section{Enum Instance}
\label{types-enum-instance}

Haxe provides powerful enumeration (short: enum) types, which are actually an \emph{algebraic data type} (ADT)\footnote{\url{http://en.wikipedia.org/wiki/Algebraic_data_type}}. While they cannot have any \tref{expressions}{expression}, they are very useful for describing data structures:

\haxe{assets/Color.hx}
Semantically, this enum describes a color which is either red, green, blue or a specified RGB value. The syntactic structure is as follows:
\begin{itemize}
	\item The keyword \expr{enum} denotes that we are declaring an enum.
	\item \type{Color} is the name of the enum and could be anything conforming to the rules for \tref{type identifiers}{define-identifier}.
	\item Enclosed in curly braces \expr{$\left\{\right\}$} are the \emph{enum constructors},
	\item which are \expr{Red}, \expr{Green} and \expr{Blue} taking no arguments,
	\item as well as \expr{Rgb} taking three \type{Int} arguments named \expr{r}, \expr{g} and \expr{b}.
\end{itemize}
The Haxe type system provides a type which unifies with all enum types:

\define[Type]{\expr{Enum$<$T$>$}}{define-enum-t}{This type is compatible with all enum types. At compile-time, \type{Enum<T>} can bee seen as the common base type of all enum types. However, this relation is not reflected in generated code.} 
\todo{Same as in 2.2, what is \type{Enum$<$T$>$} syntax?}

\subsection{Enum Constructor}
\label{types-enum-constructor}

Similar to classes and their constructors, enums provide a way of instantiating them by using one of their constructors. However, unlike classes, enums provide multiple constructors which can easily be used through their name:

\begin{lstlisting}
var a = Red;
var b = Green;
var c = Rgb(255, 255, 0);
\end{lstlisting}
In this code the type of variables \expr{a}, \expr{b} and \expr{c} is \type{Color}. Variable \expr{c} is initialized using the \expr{Rgb} constructor with arguments.
\todo{list arguments}

All enum instances can be assigned to a special type named \type{EnumValue}.

\define[Type]{EnumValue}{define-enumvalue}{EnumValue is a special type which unifies with all enum instances. It is used by the Haxe Standard Library to provide certain operations for all enum instances and can be employed in user-code accordingly in cases where an API requires \emph{an} enum instance, but not a specific one.}

It is important to distinguish enum types and enum constructors, as this example demonstrates:

\haxe{assets/EnumUnification.hx}

If the commented line is uncommented, the program does not compile because \expr{Red} (an enum constructor) cannot be assigned to a variable of type \type{Enum<Color>} (an enum type). The relation is analogous to a class and its instance.

\trivia{Concrete type parameter for \type{Enum$<$T$>$}}{One of the reviewers of this manual was confused about the difference between \type{Color} and \type{Enum<Color>} in the example above. Indeed, using a concrete type parameter there is pointless and only serves the purpose of demonstration. Usually we would omit the type there and let \tref{type inference}{type-system-type-inference} deal with it.

However, the inferred type would be different from \type{Enum<Color>}. The compiler infers a pseudo-type which has the enum constructors as ``fields''. As of Haxe 3.2.0, it is not possible to express this type in syntax but also, it is never necessary to do so.}



\subsection{Using enums}
\label{types-enum-using}

Enums are a good choice if only a finite set of values should be allowed. The individual \tref{constructors}{types-enum-constructor} then represent the allowed variants and enable the compiler to check if all possible values are respected. This can be seen here:

\haxe{assets/Color2.hx}

After retrieving the value of \expr{color} by assigning the return value of \expr{getColor()} to it, a \tref{\expr{switch} expression}{expression-switch} is used to branch depending on the value. The first three cases \expr{Red}, \expr{Green} and \expr{Blue} are trivial and correspond to the constructors of \type{Color} that have no arguments. The final case \expr{Rgb(r, g, b)} shows how the argument values of a constructor can be extracted: they are available as local variables within the case body expression, just as if a \tref{\expr{var} expression}{expression-var} had been used.

Advanced information on using the \expr{switch} expression will be explored later in the section on \tref{pattern matching}{lf-pattern-matching}.


\section{Anonymous Structure}
\label{types-anonymous-structure}

Anonymous structures can be used to group data without explicitly creating a type. The following example creates a structure with two fields \expr{x} and \expr{name}, and initializes their values to \expr{12} and \expr{"foo"} respectively:

\haxe{assets/Structure.hx}
The general syntactic rules follow:

\begin{enumerate}
	\item A structure is enclosed in curly braces \expr{$\left\{\right\}$} and
	\item Has a \emph{comma-separated} list of key-value-pairs.
	\item A \emph{colon} separates the key, which must be a valid \tref{identifier}{define-identifier}, from the value.
	\item\label{valueanytype} The value can be any Haxe expression.
\end{enumerate}
Rule \ref{valueanytype} implies that structures can be nested and complex, e.g.:

\todo{please reformat}

\begin{lstlisting}
var user = {
  name : "Nicolas",
	age : 32,
	pos : [
	  { x : 0, y : 0 },
		{ x : 1, y : -1 }
  ],
};
\end{lstlisting}
Fields of structures, like classes, are accessed using a \emph{dot} (\expr{.}) like so:

\begin{lstlisting}
// get value of name, which is "Nicolas"
user.name;
// set value of age to 33
user.age = 33;
\end{lstlisting}
It is worth noting that using anonymous structures does not subvert the typing system. The compiler ensures that only available fields are accessed, which means the following program does not compile:

\begin{lstlisting}
class Test {
  static public function main() {
    var point = { x: 0.0, y: 12.0 };
    // { y : Float, x : Float } has no field z
    point.z;
  }
}
\end{lstlisting}
The error message indicates that the compiler knows the type of \expr{point}: It is a structure with fields \expr{x} and \expr{y} of type \type{Float}. Since it has no field \expr{z}, the access fails.
The type of \expr{point} is known through \tref{type inference}{type-system-type-inference}, which thankfully saves us from using explicit types for local variables. However, if \expr{point} was a field, explicit typing would be necessary:

\begin{lstlisting}
class Path {
    var start : { x : Int, y : Int };
    var target : { x : Int, y : Int };
    var current : { x : Int, y : Int };
}
\end{lstlisting}
To avoid this kind of redundant type declaration, especially for more complex structures, it is advised to use a \tref{typedef}{type-system-typedef}:

\begin{lstlisting}
typedef Point = { x : Int, y : Int }

class Path {
    var start : Point;
    var target : Point;
    var current : Point;
}
\end{lstlisting}


\subsection{JSON for Structure Values}
\label{types-structure-json}

It is also possible to use \emph{JavaScript Object Notation} for structures by using \emph{string literals} for the keys:

\begin{lstlisting}
var point = { "x" : 1, "y" : -5 };
\end{lstlisting}
While any string literal is allowed, the field is only considered part of the type if it is a valid \tref{Haxe identifier}{define-identifier}. Otherwise, Haxe syntax does not allow expressing access to such a field, and \tref{reflection}{std-reflection} has to be employed through the use of \expr{Reflect.field} and \expr{Reflect.setField}.

\subsection{Class Notation for Structure Types}
\label{types-structure-class-notation}

When defining a structure type, Haxe allows using the same syntax as described in \Fullref{class-field}. The following \tref{typedef}{type-system-typedef} declares a \type{Point} type with variable fields \expr{x} and \expr{y} of type \type{Int}:

\begin{lstlisting}
typedef Point = {
    var x : Int;
    var y : Int;
}
\end{lstlisting}

\subsection{Optional Fields}
\label{types-structure-optional-fields}

\todo{I don't really know how these work yet.}

\subsection{Impact on Performance}
\label{types-structure-performance}

Using structures and, by extension,\tref{structural subtyping}{type-system-structural-subtyping} has no impact on performance when compiling to \tref{dynamic targets}{define-dynamic-target}. However, on \tref{static targets}{define-static-target} a dynamic lookup has to be performed which is typically slower than a static field access.



\section{Function Type}
\label{types-function}

\todo{It seems a bit convoluted explanations. Should we maybe start by "decoding" the meaning of  Void -> Void, then Int -> Bool -> Float, then maybe have samples using \$type}

The function type, along with the \tref{monomorph}{types-monomorph}, is a type which is usually well-hidden from Haxe users, yet present everywhere. We can make it surface by using \expr{\$type}, a special Haxe identifier which outputs the type its expression has during compilation :

\haxe{assets/FunctionType.hx}

There is a strong resemblance between the declaration of function \expr{test} and the output of the first \expr{\$type} expression, yet also a subtle difference:

\begin{itemize}
	\item \emph{Function arguments} are separated by the special arrow token \expr{->} instead of commas, and
	\item the \emph{function return type} appears at the end after another \expr{->}.
\end{itemize}

In either notation it is obvious that the function \expr{test} accepts a first argument of type \type{Int}, a second argument of type \type{String} and returns a value of type \type{Bool}. If a call to this function, such as \expr{test(1, "foo")}, is made within the second \expr{\$type} expression, the Haxe typer checks if \expr{1} can be assigned to \type{Int} and if \expr{"foo"} can be assigned to \type{String}. The type of the call is then equal to the type of the value \expr{test} returns, which is \type{Bool}.

If a function type has other function types as argument or return type, parentheses can be used to group them correctly. For example, \type{Int -> (Int -> Void) -> Void} represents a function which has a first argument of type \type{Int}, a second argument of function type \type{Int -> Void} and a return of \type{Void}.



\subsection{Optional Arguments}
\label{types-function-optional-arguments}

Optional arguments are declared by prefixing an argument identifier with a question mark \expr{?}:

\haxe[label=assets/OptionalArguments.hx]{assets/OptionalArguments.hx}
Function \expr{test} has two optional arguments: \expr{i} of type \type{Int} and \expr{s} of \type{String}. This is directly reflected in the function type output by line 3. 
This example program calls \expr{test} four times and prints its return value.

\begin{enumerate}
	\item The first call is made without any arguments.
	\item The second call is made with a singular argument \expr{1}.
	\item The third call is made with two arguments \expr{1} and \expr{"foo"}.
	\item The fourth call is made with a singular argument \expr{"foo"}.
\end{enumerate}
The output shows that optional arguments which are omitted from the call have a value of \expr{null}. This implies that the type of these arguments must admit \expr{null} as value, which raises the question of its \tref{nullability}{types-nullability}. The Haxe Compiler ensures that optional basic type arguments are nullable by inferring their type as \type{Null<T>} when compiling to a \tref{static target}{define-static-target}.

While the first three calls are intuitive, the fourth one might come as a surprise: It is indeed allowed to skip optional arguments if the supplied value is assignable to a later argument.


\subsection{Default values}
\label{types-function-default-values}

Haxe allows default values for arguments by assigning a \emph{constant value} to them:

\haxe{assets/DefaultValues.hx}
This example is very similar to the one from \Fullref{types-function-optional-arguments}, with the only difference being that the values \expr{12} and \expr{"bar"} are assigned to the function arguments \expr{i} and \expr{s} respectively. The effect is that the default values are used instead of \expr{null} should an argument be omitted from the call.

%TODO: Default values do not imply nullability, even if the value is \expr{null}. 

Default values in Haxe are not part of the type and are not replaced at call-site (unless the function is \tref{inlined}{class-field-inline}, which can be considered as a more typical approach. On some targets the compiler may still pass \expr{null} for omitted argument values and generate code similar to this into the function:
\begin{lstlisting}
	static function test(i = 12, s = "bar") {
		if (i == null) i = 12;
		if (s == null) s = "bar";
		return "i: " +i + ", s: " +s;
	}
\end{lstlisting}
This should be considered in performance-critical code where a solution without default values may sometimes be more viable.




\section{Dynamic}
\label{types-dynamic}

While Haxe has a static type system, this type system can, in effect, be turned off by using the \type{Dynamic} type. A \emph{dynamic value} can be assigned to anything; and anything can be assigned to it. This has several drawbacks:

\begin{itemize}
	\item The compiler can no longer type-check assignments, function calls and other constructs where specific types are expected.
	\item Certain optimizations, in particular when compiling to static targets, can no longer be employed.
	\item Some common errors, e.g. a typo in a field access, can not be caught at compile-time and likely cause an error at runtime.
	\item \Fullref{cr-dce} cannot detect used fields if they are used through \type{Dynamic}.
\end{itemize}
It is very easy to come up with examples where the usage of \type{Dynamic} can cause problems at runtime. Consider compiling the following two lines to a static target:

\begin{lstlisting}
var d:Dynamic = 1;
d.foo;
\end{lstlisting}

Trying to run a compiled program in the Flash Player yields an error \texttt{Property foo not found on Number and there is no default value}. Without \type{Dynamic}, this would have been detected at compile-time.

\trivia{Dynamic Inference before Haxe 3}{The Haxe 3 compiler never infers a type to \type{Dynamic}, so users must be explicit about it. Previous Haxe versions used to infer arrays of mixed types, e.g. \expr{[1, true, "foo"]}, as \type{Array<Dynamic>}. We found that this behavior introduced too many type problems and thus removed it for Haxe 3.}

Use of \type{Dynamic} should be minimized as there are better options in many situations but sometimes it is just practical to use it. Parts of the Haxe \Fullref{std-reflection} API use it and it is sometimes the best option when dealing with custom data structures that are not known at compile-time.

\type{Dynamic} behaves in a special way when being \tref{unified}{type-system-unification} with a \tref{monomorph}{types-monomorph}. Monomorphs are never bound to \type{Dynamic} which can have surprising results in examples such as this:

\haxe{assets/DynamicInferenceIssue.hx}

Although the return type of \expr{Json.parse} is \type{Dynamic}, the type of local variable \expr{json} is not bound to it and remains a monomorph. It is then inferred as an \tref{anonymous structure}{types-anonymous-structure} upon the \expr{json.length} field access, which causes the following \expr{json[0]} array access to fail. In order to avoid this, the variable \expr{json} can be explicitly typed as \type{Dynamic} by using \expr{var json:Dynamic}.

\trivia{Dynamic in the Standard Library}{Dynamic was quite frequent in the Haxe Standard Library before Haxe 3. With the continuous improvements of the Haxe type system the occurences of Dynamic were reduced over the releases leading to Haxe 3.}

\subsection{Dynamic with Type Parameter}
\label{types-dynamic-with-type-parameter}

\type{Dynamic} is a special type because it allows explicit declaration with and without a \tref{type parameter}{type-system-type-parameters}. If such a type parameter is provided, the semantics described in \Fullref{types-dynamic} are constrained to all fields being compatible with the parameter type:

\begin{lstlisting}
var att : Dynamic<String> = xml.attributes;
// valid, value is a String
att.name = "Nicolas";
// dito (this documentation is quite old)
att.age = "26";
// error, value is not a String
att.income = 0;
\end{lstlisting}


\subsection{Implementing Dynamic}
\label{types-dynamic-implemented}

Classes can \tref{implement}{types-interfaces} \type{Dynamic} and \type{Dynamic$<$T$>$} which enables arbitrary field access. In the former case, fields can have any type, in the latter, they are constrained to be compatible with the parameter type:

\haxe{assets/ImplementsDynamic.hx}

Implementing \type{Dynamic} does not satisfy the requirements of other implemented interfaces. The expected fields still have to be implemented explicitly.

Classes that implement \type{Dynamic} (with or without type parameter) can also utilize a special method named \expr{resolve}. If a \tref{read access}{define-read-access} is made and the field in question does not exist, the \expr{resolve} method is called with the field name as argument:

\haxe{assets/DynamicResolve.hx}



\section{Abstract}
\label{types-abstract}

An abstract type is a type which is actually a different type at run-time. It is a compile-time feature which defines types ``over'' concrete types in order to modify or augment their behavior:

\haxe[firstline=1,lastline=5]{assets/MyAbstract.hx}
We can derive the following from this example:

\begin{itemize}
	\item The keyword \expr{abstract} denotes that we are declaring an abstract type.
	\item \type{Abstract} is the name of the abstract and could be anything conforming to the rules for type identifiers.
	\item Enclosed in parenthesis \expr{()} is the \emph{underlying type} \type{Int}.
	\item Enclosed in curly braces \expr{$\left\{\right\}$} are the fields,
	\item which are a constructor function \expr{new} accepting one argument \expr{i} of type \type{Int}.
\end{itemize}

\define{Underlying Type}{define-underlying-type}{The underlying type of an abstract is the type which is used to represent said abstract at runtime. It is usually a concrete (i.e. non-abstract) type but could be another abstract type as well.}

The syntax is reminiscent of classes and the semantics are indeed similar. In fact, everything in the ``body'' of an abstract (that is everything after the opening curly brace) is parsed as class fields. Abstracts may have \tref{method}{class-field-method} fields and non-\tref{physical}{define-physical-field} \tref{property}{class-field-property} fields.

Furthermore, abstracts can be instantiated and used just like classes:

\haxe[firstline=7,lastline=12]{assets/MyAbstract.hx}
As mentioned before, abstracts are a compile-time feature, so it is interesting to see what the above actually generates. A suitable target for this is \target{Javascript}, which tends to generate concise and clean code. Compiling the above (using \texttt{haxe -main MyAbstract -js myabstract.js}) shows this \target{Javascript} code:

\begin{lstlisting}
var a = 12;
console.log(a);
\end{lstlisting}
The abstract type \type{Abstract} completely disappeared from the output and all that is left is a value of its underlying type, \type{Int}. This is because the constructor of \type{Abstract} is inlined - something we shall learn about later in the section \Fullref{class-field-inline} - and its inlined expression assigns a value to \expr{this}. This might be surprising when thinking in terms of classes. However, it is precisely what we want to express in the context of abstracts. Any \emph{inlined member method} of an abstract can assign to \expr{this}, and thus modify the ``internal value''.


A good question at this point is ``What happens if a member function is not declared inline'' because the code obviously has to go somewhere. Haxe creates a private class, known to be the \emph{implementation class}, which has all the abstract member functions as static functions accepting an additional first argument \expr{this} of the underlying type. While technically this is an implementation detail, it can be used for \tref{selective functions}{types-abstract-selective-functions}.



\trivia{Basic Types and abstracts}{Before the advent of abstract types, all basic types were implemented as extern classes or enums. While this nicely took care of some aspects such as \type{Int} being a ``child class'' of \type{Float}, it caused issues elsewhere. For instance, with \type{Float} being an extern class, it would unify with the empty structure \expr{\{\}}, making it impossible to constrain a type to accepting only real objects.}




\subsection{Implicit Casts}
\label{types-abstract-implicit-casts}

Unlike classes, abstracts allow defining implicit casts. There are two kinds of implicit casts:

\begin{description}
	\item[Direct:] Allows direct casting of the abstract type to or from another type. This is defined by adding \expr{to} and \expr{from} rules to the abstract type and is only allowed for types which unify with the underlying type of the abstract.
	\item[Class field:] Allows casting via calls to special cast functions. These functions are defined using \expr{@:to} and \expr{@:from} metadata. This kind of cast is allowed for all types.
\end{description}
The following code example shows an example of \emph{direct} casting:

\haxe{assets/ImplicitCastDirect.hx}
We declare \type{MyAbstract} as being \expr{from Int} and \expr{to Int}, meaning it can be assigned from \type{Int} and assigned to \type{Int}. This is shown in lines 9 and 10, where we first assign the \type{Int} \expr{12} to variable \expr{a} of type \type{MyAbstract} (this works due to the \expr{from Int} declaration) and then that abstract back to variable \expr{b} of type \type{Int} (this works due to the \expr{to Int} declaration).

Class field casts have the same semantics, but are defined completely differently:

\haxe{assets/ImplicitCastField.hx}
By adding \expr{@:from} to a static function, that function qualifies as implicit cast function from its argument type to the abstract. These functions must return a value of the abstract type. They must also be declared \expr{static}.

Similarly, adding \expr{@:to} to a function qualifies it as implicit cast function from the abstract to its return type. These functions are typically member-functions but they can be made \expr{static} and then serve as \tref{selective function}{types-abstract-selective-functions}.

In the example the method \expr{fromString} allows the assignment of value \expr{"3"} to variable \expr{a} of type \type{MyAbstract} while the method \expr{toArray} allows assigning that abstract to variable \expr{b} of type \type{Array<Int>}.

When using this kind of cast, calls to the cast-functions are inserted where required. This becomes obvious when looking at the \target{Javascript} output:

\begin{lstlisting}
var a = _ImplicitCastField.MyAbstract_Impl_
  .fromString("3");
var b = _ImplicitCastField.MyAbstract_Impl_
  .toArray(a);
\end{lstlisting}
This can be further optimized by \tref{inlining}{class-field-inline} both cast functions, turning the output into the following:
\todo{please review your use of ``this'' and try to vary somewhat to avoid too much word repetition}

\begin{lstlisting}
var a = Std.parseInt("3");
var b = [a];
\end{lstlisting}
The \emph{selection algorithm} when assigning a type \expr{A} to a type \expr{B} with at least one of them being an abstract is simple:

\begin{enumerate}
	\item If \expr{A} is not an abstract, go to 3.
	\item If \expr{A} defines a \emph{to}-conversions that admits \expr{B}, go to 6.
	\item If \expr{B} is not an abstract, go to 5.
	\item If \expr{B} defines a \emph{from}-conversions that admits \expr{A}, go to 6.
	\item Stop, unification fails.
	\item Stop, unification succeeds.
\end{enumerate}

\begin{flowchart}{types-abstract-implicit-casts-selection-algorithm}{選択アルゴリズムのフローチャート}

\tikzset {
	level distance = 1.8cm
}

\tikzstyle{edgeBelow} = [ auto = left, outer sep = 0.2cm ]

\Tree
[.\node [decisionc] (dec1) {\expr{A}は抽象型};
\edge [edgeBelow] node {はい};
[.\node [decisionc] (dec2) {\expr{A}に\expr{B}への\expr{to}変換がある};
\edge [edgeBelow] node {いいえ};
[.\node [decisionc] (dec3) {\expr{B}は抽象型};
\edge [edgeBelow] node {はい};
[.\node [decisionc] (dec4) {\expr{B}に\expr{A}からの\expr{from}変換がある};
\edge [edgeBelow] node {いいえ};
[.\node [startstop, fill = red!70] (fail) {単一化失敗};
]]]]]


\node [startstop, fill = green!70, xshift = 3cm] (success) at (fail.east) {単一化成功};

\tikzstyle{altNode} = [above right, at start]
\tikzstyle{skipNode} = [above left, at start]
\tikzstyle{skipArrow} = [out = 195, in = 165, looseness = 1.6]
\coordinate (altAnchor) at (success.north);

\draw [flowchartArrow] (dec1.west) [skipArrow] to node [skipNode] {いいえ} (dec3.north west);
\draw [flowchartArrow] (dec3.south west) [skipArrow] to node [skipNode] {いいえ} (fail.west);
\draw [flowchartArrow] (dec2) -| (altAnchor) node [altNode] {はい};
\draw [flowchartArrow] (dec4) -| (altAnchor) node [altNode] {はい};

\end{flowchart}

By design, implicit casts are \emph{not transitive}, as the following example shows:

\haxe{assets/ImplicitTransitiveCast.hx}
While the individual casts from \type{A} to \type{B} and from \type{B} to \type{C} are allowed, a transitive cast from \type{A} to \type{C} is not. This is to avoid ambiguous cast-paths and retain a simple selection algorithm. 




\subsection{Operator Overloading}
\label{types-abstract-operator-overloading}

Abstracts allow overloading of unary and binary operators by adding the \expr{@:op} metadata to class fields:

\haxe{assets/AbstractOperatorOverload.hx}
By defining \expr{@:op(A * B)}, the function \expr{repeat} serves as operator method for the multiplication \expr{*} operator when the type of the left value is \type{MyAbstract} and the type of the right value is \type{Int}. The usage is shown in line 17, which turns into this when compiled to \target{Javascript}:

\begin{lstlisting}
console.log(_AbstractOperatorOverload.
  MyAbstract_Impl_.repeat(a,3));
\end{lstlisting}
Similar to \tref{implicit casts with class fields}{types-abstract-implicit-casts}, a call to the overload method is inserted where required.

The example \expr{repeat} function is not commutative: While \expr{MyAbstract * Int} works, \expr{Int * MyAbstract} does not. If this should be allowed as well, the \expr{@:commutative} metadata can be added. If it should work \emph{only} for \expr{Int * MyAbstract}, but not for \expr{MyAbstract * Int}, the overload method can be made static, accepting \type{Int} and \type{MyAbstract} as first and second type respectively.

Overloading unary operators is analogous:

\haxe{assets/AbstractUnopOverload.hx}
Both binary and unary operator overloads can return any type.

It is also possible to omit the method body of a \expr{@:op} function, but only if the underlying type of the abstract allows the operation in question and if the resulting type can be assigned back to the abstract.
\todo{please review for correctness}


\subsection{Array Access}
\label{types-abstract-array-access}

Array access describes the particular syntax traditionally used to access the value in an array at a certain offset. This is usually only allowed with arguments of type \type{Int}. Nevertheless, with abstracts it is possible to define custom array access methods. The \tref{Haxe Standard Library}{std} uses this in its \type{Map} type, where the following two methods can be found:
\todo{You have marked ``Map'' for some reason}

\begin{lstlisting}
@:arrayAccess public inline function
get(key:K) return this.get(key);
@:arrayAccess public inline function
arrayWrite(k:K, v:V):V {
	this.set(k, v);
	return v;
}
\end{lstlisting}
There are two kinds of array access methods:

\begin{itemize}
	\item If an \expr{@:arrayAccess} method accepts one argument, it is a getter.
	\item If an \expr{@:arrayAccess} method accepts two arguments, it is a setter.
\end{itemize}
The methods \expr{get} and \expr{arrayWrite} seen above then allow the following usage:

\haxe{assets/AbstractArrayAccess.hx}

At this point it should not be surprising to see that calls to the array access fields are inserted in the output:

\begin{lstlisting}
map.set("foo",1);
1;
console.log(map.get("foo"));
\end{lstlisting}


\subsection{Selective Functions}
\label{types-abstract-selective-functions}

Since the compiler promotes abstract member functions to static functions, it is possible to define static functions by hand and use them on an abstract instance. The semantics here are similar to those of \tref{static extensions}{lf-static-extension}, where the type of the first function argument determines for which types a function is defined:

\haxe{assets/SelectiveFunction.hx}
The method \expr{getString} of abstract \type{MyAbstract} is defined to accept a first argument of \type{MyAbstract$<$String$>$}. This causes it to be available on variable \expr{a} on line 14 (because the type of \expr{a} is \type{MyAbstract$<$String$>$}), but not on variable \expr{b} whose type is \type{MyAbstract$<$Int$>$}.

\trivia{Accidental Feature}{ Rather than having actually been designed, selective functions were discovered. After the idea was first mentioned, it required only minor adjustments in the compiler to make them work. Their discovery also lead to the introduction of multi-type abstracts, such as Map. }


\subsection{Enum abstracts}
\label{types-abstract-enum}
\since{3.1.0}

By adding the \expr{:enum} metadata to an abstract definition, that abstract can be used to define finite value sets:

\haxe{assets/AbstractEnum.hx}

The Haxe Compiler replaces all field access to the \type{HttpStatus} abstract with their values, as evident in the \target{Javascript} output:

\begin{lstlisting}
Main.main = function() {
	var status = 404;
	var msg = Main.printStatus(status);
};
Main.printStatus = function(status) {
	switch(status) {
	case 404:
		return "Not found";
	case 405:
		return "Method not allowed";
	}
};
\end{lstlisting}

This is similar to accessing \tref{variables declared as inline}{class-field-inline}, but has several advantages:

\begin{itemize}
	\item The typer can ensure that all values of the set are typed correctly.
	\item The pattern matcher checks for \tref{exhaustiveness}{lf-pattern-matching-exhaustiveness} when \tref{matching}{lf-pattern-matching} an enum abstract.
	\item Defining fields requires less syntax.
\end{itemize}


\subsection{Forwarding abstract fields}
\label{types-abstract-forward}
\since{3.1.0}

When wrapping an underlying type, it is sometimes desirable to ``keep'' parts of its functionality. Because writing forwarding functions by hand is cumbersome, Haxe allows adding the \expr{:forward} metadata to an abstract type:

\haxe{assets/AbstractExpose.hx}

The \type{MyArray} abstract in this example wraps \type{Array}. Its \expr{:forward} metadata has two arguments which correspond to the field names to be forwarded to the underlying type. In this example, the \expr{main} method instantiates \type{MyArray} and accesses its \expr{push} and \expr{pop} methods. The commented line demonstrates that the \expr{length} field is not available.

As usual we can look at the \target{Javascript} output to see how the code is being generated:

\begin{lstlisting}
Main.main = function() {
	var myArray = [];
	myArray.push(12);
	myArray.pop();
};
\end{lstlisting}

It is also possible to use \expr{:forward} without any arguments in order to forward all fields. Of course the Haxe Compiler still ensures that the field actually exists on the underlying type.

\trivia{Implemented as macro}{Both the \expr{:enum} and \expr{:forward} functionality were originally implemented using \tref{build macros}{macro-type-building}. While this worked nicely in non-macro code, it caused issues if these features were used from within macros. The implementation was subsequently moved to the compiler.}


\subsection{Core-type abstracts}
\label{types-abstract-core-type}

The Haxe Standard Library defines a set of basic types as core-type abstracts. They are identified by the \expr{:coreType} metadata and the lack of an underlying type declaration. These abstracts can still be understood to represent a different type. Still, that type is native to the Haxe target. 

Introducing custom core-type abstracts is rarely necessary in user code as it requires the Haxe target to be able to make sense of it. However, there could be interesting use-cases for authors of macros and new Haxe targets.

In contrast to opaque abstracts, core-type abstracts have the following properties:

\begin{itemize}
	\item They have no underlying type.
	\item They are considered nullable unless they are annotated with \expr{:notNull} metadata.
	\item They are allowed to declare \tref{array access}{types-abstract-array-access} functions without expressions.
	\item \tref{Operator overloading fields}{types-abstract-operator-overloading} that have no expression are not forced to adhere to the Haxe type semantics.
\end{itemize}



\section{Monomorph}
\label{types-monomorph}

A monomorph is a type which may, through \tref{unification}{type-system-unification}, morph into a different type later. We shall see details about this type when talking about \tref{type inference}{type-system-type-inference}.

\chapter{型システム}
\label{type-system}

私たちは\Fullref{types}の章でさまざまな種類の型について学んできました。ここからはそれらがお互いにどう関連しあっているかを見ていきます。まず、複雑な型に対して名前(別名)を与える仕組みである\tref{Typedef}{type-system-typedef}の紹介から簡単に始めます。typedefは特に、\tref{型パラメータ}{type-system-type-parameters}を持つ型で役に立ちます。

<<<<<<< HEAD:03-type-system.tex
任意の2つの型について、その上位の型のグループが矛盾しないかをチェックすることで多くの型安全性が得られます。これがコンパイラが試みる\emph{単一化}であり、\Fullref{type-system-unification}の節で詳しく説明します。
=======
A lot of type-safety is achieved by checking if two given types of the type groups above are compatible. Meaning, the compiler tries to perform \emph{unification} between them as detailed in \Fullref{type-system-unification}.
>>>>>>> english/master:HaxeManual/03-type-system.tex

すべての型は\emph{モジュール}に所属し、\emph{パス}を通して呼び出されます。\Fullref{type-system-modules-and-paths}では、これらに関連した仕組みについて詳しい説明を行います。

\section{typedef}
\label{type-system-typedef}

<<<<<<< HEAD:03-type-system.tex
typedefは\tref{匿名構造体}{types-anonymous-structure}の節で、すでに登場しています。そこでは複雑な構造体の型について名前を与えて簡潔にあつかう方法を見ています。この利用法はtypedefが一体なにに良いのかを的確に表しています。構造体の型に対して名前を与えるのは、typedefの主たる用途かもしれません。実際のところ、この用途が一般的すぎて、多くのHaxeユーザーがtypdefを構造体のためのものだと思ってしまっています。
=======
We briefly looked at typedefs while talking about \tref{anonymous structures}{types-anonymous-structure} and saw how we could shorten a complex \tref{structure type}{types-anonymous-structure} by giving it a name. This is precisely what typedefs are good for. Giving names to structure types might even be considered their primary use. In fact, it is so common that the distinction appears somewhat blurry and many Haxe users consider typedefs to actually \emph{be} the structure.
>>>>>>> english/master:HaxeManual/03-type-system.tex

typedefは他のあらゆる型に対して名前を与えることが可能です。

\begin{lstlisting}
typedef IA = Array<Int>;
\end{lstlisting}
<<<<<<< HEAD:03-type-system.tex

これにより\expr{Array$<$Int$>$}が使われる場所で、代わりに\expr{IA}を使うことが可能になります。この場合、はほんの数回のタイプ数しか減らせませんが、より複雑な複合型の場合は違います。これこそが、typedefと構造体が強く結びついて見える理由です。
=======
This enables us to use \expr{IA} in places where we would normally use \expr{Array$<$Int$>$}. While this saves only a few keystrokes in this particular case, it can make a much bigger difference for more complex, compound types. Again, this is why typedef and structures seem so connected:
>>>>>>> english/master:HaxeManual/03-type-system.tex

\begin{lstlisting}
typedef User = {
    var age : Int;
    var name : String;
}
\end{lstlisting}
<<<<<<< HEAD:03-type-system.tex

typedefはテキスト上の置き換えではなく、実は本物の型です。Haxe標準ライブラリの\type{Iterable}のように\tref{型パラメータ}{type-system-type-parameters}を持つことができます。
=======
A typedef is not a textual replacement but actually a real type. It can even have \tref{type parameters}{type-system-type-parameters} as the \type{Iterable} type from the Haxe Standard Library demonstrates:
>>>>>>> english/master:HaxeManual/03-type-system.tex

\begin{lstlisting}
typedef Iterable<T> = {
	function iterator() : Iterator<T>;
}
\end{lstlisting}

\subsection{拡張}
\label{type-system-extensions}

% TODO: move to structures? %
<<<<<<< HEAD:03-type-system.tex

拡張は、構造体が与えられた型のフィールドすべてと、加えていくつかのフィールドを持っていることを表すために使われます。
=======
Extensions are used to express that a structure has all the fields of a given type as well as some additional fields of its own:
>>>>>>> english/master:HaxeManual/03-type-system.tex

\haxe{assets/Extension.hx}
大なりの演算子を使うことで、追加のクラスフィールドを持つ\type{Iterable$<$T$>$}の拡張が作成されました。このケースでは、読み込み専用の\tref{プロパティ}{class-field-property} である\type{Int}型の\expr{length}が要求されます。 

\type{IterableWithLength$<$T$>$}に適合するためには、\type{Iterable$<$T$>$}にも適合してさらに読み込み専用の\type{Int}型のプロパティ\expr{length}を持ってなきゃいけません。例では、Arrayが割り当てられており、これはこれらの条件をすべて満たしています。

\since{3.1.0}
複数の構造体を拡張することもできます。

\haxe{assets/Extension2.hx}

\section{型パラメータ}
\label{type-system-type-parameters}

\tref{クラスフィールド}{class-field}や\tref{列挙型コンストラクタ}{types-enum-constructor}のように、Haxeではいくつかの型についてパラメータ化を行うことができます。型パラメータは山カッコ\expr{$<>$}内にカンマ区切りで記述することで、定義することができます。シンプルな例は、Haxe標準ライブラリの\type{Array}です。

\begin{lstlisting}
class Array<T> {
	function push(x : T) : Int;
}
\end{lstlisting}
\type{Array}のインスタンスが作られると、型パラメータ\type{T}は\tref{単相}{types-monomorph}となります。つまり、1度に1つの型であれば、あらゆる型を適用することができます。この適用は以下のどちらか方法で行います

\begin{description}
	\item[明示的に、]\expr{new Array$<$String$>$()}のように型を記述してコンストラクタを呼び出して適用する。
	\item[暗黙に]、\tref{型推論}{type-system-type-inference}で適用する。例えば、\expr{arrayInstance.push("foo")}を呼び出す。
\end{description}

型パラメータが付くクラスの定義の内部では、その型パラメータは不定の型です。\tref{制約}{type-system-type-parameter-constraints}が追加されない限り、コンパイラはその型パラメータはあらゆる型になりうるものと決めつけることになります。その結果、型パラメータの\tref{cast}{expression-cast}を使わなければ、その型のフィールドにアクセスできなくなります。また、\tref{ジェネリック}{type-system-generic}にして適切な制約をつけない限り、その型パラメータの型のインスタンスを新しく生成することもできません。

以下は、型パラメータが使用できる場所についての表です。

\begin{center}
\begin{tabular}{| l | l | l |}
	\hline
	パラメータが付く場所 & 型を適用する場所 & 備考 \\ \hline
	Class & インスタンス作成時 & メンバフィールドにアクセスする際に型を適用することもできる \\
	Enum & インスタンス作成時 & \\
	Enumコンストラクタ & インスタンス作成時 & \\
	関数 & 呼び出し時 & メソッドと名前付きのローカル関数で利用可能	\\
	構造体 & インスタンス作成時 & \\ \hline
\end{tabular}
\end{center}
<<<<<<< HEAD:03-type-system.tex

関数の型パラメータは呼び出し時に適用される、この型パラメータは(制約をつけない限り)あらゆる型を許容します。しかし、一回の呼び出しにつき適用は1つの型のみ可能です。このことは関数が複数の引数を持つ場合に役立ちます。

\haxe{assets/FunctionTypeParameter.hx}

\expr{equals}関数の\expr{expected}と\expr{actual}の引数両方が、\type{T}型になっています。これは\expr{equals}の呼び出しで2つの引数の型が同じでなければならないことを表しています。コンパイラは最初(両方の引数が\type{Int}型)と2つめ(両方の引数が\type{String}型)の呼び出しは認めていますが、3つ目の呼び出しはコンパイルエラーにします。

\trivia{式の構文内での型パラメータ}{なぜ、\expr{method<String>(x)}のようにメソッドに型パラメータをつけた呼び出しができないのか?という質問をよくいただきます。このときのエラーメッセージはあまり参考になりませんが、これには単純な理由があります。それは、このコードでは、\expr{<}と\expr{>}の両方が2項演算子として構文解析されて、\expr{(method < String) > (x)}と見なされるからです。}
=======
With function type parameters being bound upon invocation, such a type parameter (if unconstrained) accepts any type. However, only one type per invocation is accepted. This can be utilized if a function has multiple arguments:

\haxe{assets/FunctionTypeParameter.hx}

Both arguments \expr{expected} and \expr{actual} of the \expr{equals} function have type \type{T}. This implies that for each invocation of \expr{equals} the two arguments must be of the same type. The compiler admits the first call (both arguments being of \type{Int}) and the second call (both arguments being of \type{String}) but the third attempt causes a compiler error.

\trivia{Type parameters in expression syntax}{We often get the question why a method with type parameters cannot be called as \expr{method<String>(x)}. The error messages the compiler gives are not very helpful. However, there is a simple reason for that: The above code is parsed as if both \expr{<} and \expr{>} were binary operators, yielding \expr{(method < String) > (x)}.}
>>>>>>> english/master:HaxeManual/03-type-system.tex

\subsection{制約}
\label{type-system-type-parameter-constraints}

型パラメータは複数の型で制約を与えることができます。

\haxe{assets/Constraints.hx}

\expr{test}メソッドの型パラメータ\type{T}は、\type{Iterable$<$String$>$}と\type{Measurable}の型に制約されます。\type{Measurable}の方は、便宜上\tref{typedef}{type-system-typedef}を使って、\type{Int}型の読み込み専用\tref{プロパティ}{class-field-property}\expr{length}を要求しています。つまり、以下の条件を満たせば、これらの制約と矛盾しません。

\begin{itemize}
	\item \type{Iterable$<$String$>$}である
	\item かつ、\type{Int}型の\expr{length}を持つ
\end{itemize}
<<<<<<< HEAD:03-type-system.tex
=======
We can see that invoking \expr{test} with an empty array in line 7 and an \type{Array$<$String$>$} in line 8 works fine. This is because \type{Array} has both a \expr{length}-property and an \expr{iterator}-method. However, passing a \type{String} as argument in line 9 fails the constraint check because \type{String} is not compatible with \type{Iterable$<$T$>$}. 
>>>>>>> english/master:HaxeManual/03-type-system.tex

7行目では空の配列で、8行目では\type{Array$<$String$>$}で\expr{test}関数を呼び出すことができることを確認しました。しかし、10行目の\type{String}の引数では制約チェックで失敗しています。これは、\type{String}は\type{Iterable$<$T$>$}と矛盾するからです。

\section{ジェネリック}
\label{type-system-generic}

<<<<<<< HEAD:03-type-system.tex
大抵の場合、Haxeコンパイラは型パラメータが付けられていた場合でも、1つのクラスや関数を生成します。これにより自然な抽象化が行われて、ターゲット言語のコードジェネレータは出力先の型パラメータはあらゆる型になりえると思い込むことになります。つまり、生成されたコードで型チェックが働き、動作が邪魔されることがあります。
=======
Usually, the Haxe Compiler generates only a single class or function even if it has type parameters. This results in a natural abstraction where the code generator for the target language has to assume that a type parameter could be of any type. The generated code then might have to perform some type checks which can be detrimental to performance.
>>>>>>> english/master:HaxeManual/03-type-system.tex

クラスや関数は、\expr{:generic} \tref{メタデータ}{lf-metadata}で\emph{ジェネリック}属性をつけることで一般化することができます。これにより、コンパイラは型パラメータの組み合わせごとのクラスまたは関数を修飾された名前で書き出します。このような設計により\tref{静的ターゲット}{define-static-target}のパフォーマンスに直結するコード部位では、出力サイズの巨大化と引き換えに、速度を得られます。

\haxe{assets/GenericClass.hx}

<<<<<<< HEAD:03-type-system.tex
あまり使わない明示的な\type{MyArray<String>}の型宣言があり、よく使う\tref{型推論}{type-system-type-inference}であつかっていますが、これが重要です。コンパイラは、コンストラクタの呼び出し時にジェネリッククラスの正確な型な型を知っている必要があります。この\target{JavaScript}出力は以下のような結果になります。
=======
It seems unusual to see the explicit type \type{MyArray<String>} here as we usually let \tref{type inference}{type-system-type-inference} deal with this. Nonetheless, it is indeed required in this case. The compiler has to know the exact type of a generic class upon construction. The \target{Javascript} output shows the result:
>>>>>>> english/master:HaxeManual/03-type-system.tex

\begin{lstlisting}
(function () { "use strict";
var Test = function() { };
Test.main = function() {
	var a = new MyValue_String("Hello");
	var b = new MyValue_Int(5);
};
var MyValue_Int = function(value) {
	this.value = value;
};
var MyValue_String = function(value) {
	this.value = value;
};
Test.main();
})();
\end{lstlisting}

\type{MyArray<String>}と\type{MyArray<Int>}は、それぞれ\type{MyArray_String}と\type{MyArray_Int}になっています。これはジェネリック関数でも同じです。

\haxe{assets/GenericFunction.hx}

\target{JavaScript}出力を見れば明白です。

\begin{lstlisting}
(function () { "use strict";
var Main = function() { }
Main.method_Int = function(t) {
}
Main.method_String = function(t) {
}
Main.main = function() {
	Main.method_String("foo");
	Main.method_Int(1);
}
Main.main();
})();
\end{lstlisting}


\subsection{ジェネリック型パラメータのコンストラクト}
\label{type-system-generic-type-parameter-construction}

\define{ジェネリック型パラメータ}{define-generic-type-parameter}{型パラメータを持っているクラスまたはメソッドがジェネリックであるとき、その型パラメータもジェネリックであるという。}

<<<<<<< HEAD:03-type-system.tex
普通の型パラメータでは、\expr{new T()}のようにその型をコンストラクトすることはできません。これは、Haxeが1つの関数を生成するために、そのコンストラクトが意味をなさないからです。しかし、型パラメータがジェネリックの場合は違います。これは、コンパイラはすべての型パラメータの組み合わせに対して別々の関数を生成しています。このため\expr{new T()}の\type{T}を実際の型に置き換えることができます。

\haxe{assets/GenericTypeParameter.hx}

ここでは、\type{T}の実際の型の決定は、\tref{トップダウンの推論}{type-system-top-down-inference}で行われることに注意してください。この方法での型パラメータのコンストラクトを行うには2つの必須事項があります。
=======
It is not possible to construct normal type parameters, e.g. \expr{new T()} is a compiler error. The reason for this is that Haxe generates only a single function and the construct makes no sense in that case. This is different when the type parameter is generic: Since we know that the compiler will generate a distinct function for each type parameter combination, it is possible to replace the \type{T} \expr{new T()} with the real type.

\haxe{assets/GenericTypeParameter.hx}

It should be noted that \tref{top-down inference}{type-system-top-down-inference} is used here to determine the actual type of \type{T}. There are two requirements for this kind of type parameter construction to work: The constructed type parameter must be
>>>>>>> english/master:HaxeManual/03-type-system.tex

\begin{enumerate}
	\item ジェネリックであること
	\item 明示的に、\tref{コンストラクタ}{types-class-constructor}を持つように\tref{制約}{type-system-type-parameter-constraints}されていること
\end{enumerate}

先ほどの例は、1つ目は\expr{make}が\expr{@:generic}メタデータを持っており、2つ目\type{T}が\type{Constructible}に制約されています。\type{String}と\type{haxe.Template}の両方とも1つ\type{String}の引数のコンストラクタを持つのでこの制約に当てはまります。確かにJavascript出力は予測通りのものになっています。

\begin{lstlisting}
var Main = function() { }
Main.__name__ = true;
Main.make_haxe_Template = function() {
	return new haxe.Template("foo");
}
Main.make_String = function() {
	return new String("foo");
}
Main.main = function() {
	var s = Main.make_String();
	var t = Main.make_haxe_Template();
}
\end{lstlisting}

\section{変性(バリアンス)}
\label{type-system-variance}

<<<<<<< HEAD:03-type-system.tex
変性とは他のものとの関連を表すもので、特に型パラメータに関するものが連想されます。そして、この文脈では驚くようなことがよく起こります。変性のエラーを起こすことはとても簡単です。
=======
While variance is also relevant in other places, it occurs particularly often with type parameters and comes as a surprise in this context. Additionally, it is very easy to trigger variance errors:
>>>>>>> english/master:HaxeManual/03-type-system.tex

\haxe{assets/Variance.hx}

見てわかるとおり、\type{Child}は\type{Base}に代入できるにもかかわらず、\type{Array<Child>}を\type{Array<Base>}に代入することはできません。この理由は少々予想外のものかもしれません。それはこの配列への書き込みが可能だからです。例えば、\expr{push()}メソッドです。この変性のエラーを無視してしまうことは簡単です。

\haxe{assets/Variance2.hx}

<<<<<<< HEAD:03-type-system.tex
ここでは\tref{cast}{expression-cast}を使って型チェッカーを破壊して、12行目の代入を可能にしてしまっています。\expr{bases}は元々の配列への参照を持っており、\type{Array<Base>}の型付けをされています。このため、\type{Base}に適合する別の型の\type{OtherChild}を配列に追加できます。しかし、元々の\expr{children}の参照は\type{Array<Child>}のままです。そのため良くないことに繰り返し処理の中で\type{OtherChild}のインスタンスに出くわします。

もし\type{Array}が\expr{push()}メソッドを持っておらず、他の編集方法も無かったならば、適合しない型を追加することができなくなるのでこの代入は安全になります。Haxeでは\tref{構造的部分型付け}{type-system-structural-subtyping}を使って型を適切に制限することでこれを実現できます。

\haxe{assets/Variance3.hx}

\expr{b}は\type{MyArray<Base>}として型付けされており、\type{MyArray}は\expr{pop()}メソッドしか持たないため、安全に代入することができます。\type{MyArray}には適合しない型を追加できるメソッドを持っておらず、このことは\emph{共変性}と呼ばれます。
=======
Here we subvert the type checker by using a \tref{cast}{expression-cast}, thus allowing the assignment after the commented line. With that we hold a reference \expr{bases} to the original array, typed as \type{Array<Base>}. This allows pushing another type compatible with \type{Base} (\type{OtherChild}) onto that array. However, our original reference \expr{children} is still of type \type{Array<Child>} and things go bad when we encounter the \type{OtherChild} instance in one of its elements while iterating.

If \type{Array} had no \expr{push()} method and no other means of modification, the assignment would be safe because no incompatible type could be added to it. In Haxe, we can achieve this by restricting the type accordingly using \tref{structural subtyping}{type-system-structural-subtyping}:

\haxe{assets/Variance3.hx}

We can safely assign with \expr{b} being typed as \type{MyArray<Base>} and \type{MyArray} only having a \expr{pop()} method. There is no method defined on \type{MyArray} which could be used to add incompatible types, it is thus said to be \emph{covariant}.

\define{Covariance}{define-covariance}{A \tref{compound type}{define-compound-type} is considered covariant if its component types can be assigned to less specific components, i.e. if they are only read, but never written.}

\define{Contravariance}{define-contravariance}{A \tref{compound type}{define-compound-type} is considered contravariant if its component types can be assigned to less generic components, i.e. if they are only written, but never read.}
>>>>>>> english/master:HaxeManual/03-type-system.tex

\define{共変性}{define-covariance}{\tref{複合型}{define-compound-type}がそれを構成する型よりも一般な型で構成される複合型に代入できる場合に、共変であるという。 つまり、読み込みのみが許されて書き込みができない場合です。}

\define{反変性}{define-contravariance}{\tref{複合型}{define-compound-type}がそれを構成する型よりも特殊な型で構成される複合型に代入できる場合に、反変であるという。 つまり、書き込みのみが許されて読み込みができない場合です。}

\section{単一化(ユニフィケーション)}
\label{type-system-unification}

\todo{Mention toString()/String conversion somewhere in this chapter.}

単一化は型システムの要であり、Haxeの堅牢さに大きく貢献しています。この節ではある型が他の型と適合するかどうかをチェックする過程を説明していきます。

\define{単一化}{define-unification}{型Aの型Bでの単一化というのは、AがBに代入可能かを調べる指向性を持つプロセスです。型が\tref{単相}{types-monomorph}の場合または単相を含む場合は、それを変化させることができます。}

単一化のエラーは簡単に起こすことができます。

\begin{lstlisting}
class Main {
	static public function main() {
    // Int should be String
		var s:String = 1;
	}
}
\end{lstlisting}

\type{Int}型の値を\type{String}型の変数に代入しようとしたので、コンパイラは\emph{IntをStringで単一化}しようと試みます。これはもちろん許可されておらず、コンパイラは\expr{"Int should be String"}というエラーを出力します。

このケースでは単一化は\emph{代入}によって引き起こされており、この文脈は「代入可能」という定義に対して直感的です。ただ、これは単一化が働くケースのうちの1つでしかありません。

\begin{description}
<<<<<<< HEAD:03-type-system.tex
	\item[代入:] \expr{a}が\expr{b}に代入された場合、\expr{a}の型は\expr{b}で単一化されます。
	\item[関数呼び出し:] \tref{関数}{types-function}の型の紹介ですでに触れています。一般的に言うと、コンパイラは渡された最初の引数の型を要求される最初の引数の型で単一化し、渡された二番目の引数の型を要求される二番目の引数の型で単一化するということを、すべての引数で行います。
	\item[関数のreturn:] 関数が\expr{return e}の式をもつ場合は常に\expr{e}の型は関数の戻り値の型で単一化されます。もし関数の戻り値の型が明示されていなければ、\expr{e}の型に型推論されて、それ以降の\expr{return}式は\expr{e}の型に推論されます。
	\item[配列の宣言:] コンパイラは、配列の宣言では与えられたすべての型に共通する最小の型を見つけようとします。詳細は\Fullref{type-system-unification-common-base-type}を参照してください。
	\item[オブジェクトの宣言:] オブジェクトを指定された型に対して宣言する場合、コンパイラは与えられたフィールドすべての型を要求されるフィールドの型で単一化します。
	\item[演算子の単一化:] 特定の型を要求する特定の演算子は、与えられた型をその型で単一化します。例えば、\expr{a \&\& b}という式は\expr{a}と\expr{b}両方を\type{Bool}で単一化します。式\expr{a == b}は\expr{a}を\expr{b}で単一化します。
=======
	\item[Assignment:] If \expr{a} is assigned to \expr{b}, the type of \expr{a} is unified with the type of \expr{b}.
	\item[Function call:] We have briefly seen this one while introducing the \tref{function}{types-function} type. In general, the compiler tries to unify the first given argument type with the first expected argument type, the second given argument type with the second expected argument type and so on until all argument types are handled.
	\item[Function return:] Whenever a function has a \expr{return e} expression, the type of \expr{e} is unified with the function return type. If the function has no explicit return type, it is inferred to the type of \expr{e} and subsequent \expr{return} expressions are inferred against it.
	\item[Array declaration:] The compiler tries to find a minimal type between all given types in an array declaration. Refer to \Fullref{type-system-unification-common-base-type} for details.
	\item[Object declaration:] If an object is declared ``against'' a given type, the compiler unifies each given field type with each expected field type.
	\item[Operator unification:] Certain operators expect certain types which the given types are unified against. For instance, the expression \expr{a \&\& b} unifies both \expr{a} and \expr{b} with \type{Bool} and the expression \expr{a == b} unifies \expr{a} with \expr{b}.
>>>>>>> english/master:HaxeManual/03-type-system.tex
\end{description}


\subsection{クラスとインターフェースの単一化}
\label{type-system-unification-between-classes-and-interfaces}

<<<<<<< HEAD:03-type-system.tex
クラスの間の単一化について定義を行う場合、単一化が指向性を持つことを心に留めておくべきです。より特殊なクラス(例えば、子クラス)を一般的なクラス(例えば、親クラス)に対して代入することはできますが、逆はできません。
=======
When defining unification behavior between classes, it is important to remember that unification is directional: We can assign a more specialized class (e.g. a child class) to a generic class (e.g. a parent class) but the reverse is not valid.
>>>>>>> english/master:HaxeManual/03-type-system.tex

以下のような、代入が許可されます。

\begin{itemize}
	\item 子クラスの親クラスへの代入
	\item クラスの実装済みのインターフェースへの代入
	\item インターフェースの親インターフェースへの代入
\end{itemize}

これらのルールは連結可能です。つまり、子クラスをその親クラスの親クラスへ代入可能であり、さらに親クラスが実装しているインターフェースへ代入可能であり、そのインターフェースの親インターフェースへ代入可能であるということです。

\todo{''parent class'' should probably be used here, but I have no idea what it means, so I will refrain from changing it myself.}

\subsection{構造的部分型付け}
\label{type-system-structural-subtyping}

\define{構造的部分型付け}{define-structural-subtyping}{構造的部分型付けは、同じ構造を持つ型の暗黙の関係を示します。}

<<<<<<< HEAD:03-type-system.tex
Haxeでは、構造的部分型付けはクラスインスタンスを構造体に代入するときのみ可能です。以下に、\tref{Haxe標準ライブラリ}{std}の\type{Lambda}の一部の例を挙げます。
=======
Structural sub-typing in Haxe is allowed when unifying

\begin{itemize}
	\item a \tref{class}{types-class-instance} with a \tref{structure}{types-anonymous-structure} and
	\item a structure with another structure.
\end{itemize}

The following example is part of the \type{Lambda} class of the \tref{Haxe Standard Library}{std}:
>>>>>>> english/master:HaxeManual/03-type-system.tex

\begin{lstlisting}
public static function empty<T>(it : Iterable<T>):Bool {
	return !it.iterator().hasNext();
}
\end{lstlisting}
<<<<<<< HEAD:03-type-system.tex

\expr{empty}メソッドは、\type{Iterable}が要素を持つかをチェックします。この目的では、引数の型について、それが列挙可能(Iterable)であること以外に何も知る必要がありません。Haxe標準ライブラリにはたくさんある\type{Iterable$<$T$>$}で単一化できる型すべてで、これを呼び出すことができるわけです。この種の型付けは非常に便利ですが、静的ターゲットでは大量に使うとパフォーマンスの低下を招くことがあります。詳しくは\Fullref{types-structure-performance}に書かれています。
=======
The \expr{empty}-method checks if an \type{Iterable} has an element. For this purpose, it is not necessary to know anything about the argument type other than the fact that it is considered an iterable. This allows calling the \expr{empty}-method with any type that unifies with \type{Iterable$<$T$>$} which applies to a lot of types in the Haxe Standard Library.

This kind of typing can be very convenient but extensive use may be detrimental to performance on static targets, which  is detailed in \Fullref{types-structure-performance}.
>>>>>>> english/master:HaxeManual/03-type-system.tex


\subsection{単相}
\label{type-system-monomorphs}

\tref{単相}{types-monomorph}である、あるいは単相を含む型についての単一化は\Fullref{type-system-type-inference}で詳しくあつかいます。

\subsection{関数の戻り値}
\label{type-system-unification-function-return}

<<<<<<< HEAD:03-type-system.tex
関数の戻り値の型の単一化では\tref{\type{Void}型}{types-void}も関連しており、\type{Void}型での単一化のはっきりとした定義が必要です。\type{Void}型は型の不在を表し、あらゆる型が代入できません。\type{Dynamic}でさえも代入できません。つまり、関数が明示的に\type{Dynamic}を返すと定義されている場合、\type{Void}を返してはいけません。
=======
Unification of function return types may involve the \tref{\type{Void}-type}{types-void} and requires a clear definition of what unifies with \type{Void}. With \type{Void} describing the absence of a type, it is not assignable to any other type, not even \type{Dynamic}. This means that if a function is explicitly declared as returning \type{Dynamic}, it cannot return \type{Void}.
>>>>>>> english/master:HaxeManual/03-type-system.tex

その逆も同じです。関数の戻り値が\type{Void}であると宣言しているなら、\type{Dynamic}やその他すべての型は返すことができません。しかし、関数の型を代入する時のこの方向の単一化は許可されています。

\begin{lstlisting}
var func:Void->Void = function() return "foo";
\end{lstlisting}
<<<<<<< HEAD:03-type-system.tex
=======

The right-hand function clearly is of type \type{Void->String}, yet we can assign it to the variable \expr{func} of type \type{Void->Void}. This is because the compiler can safely assume that the return type is irrelevant, given that it could not be assigned to any non-\type{Void} type.
>>>>>>> english/master:HaxeManual/03-type-system.tex

右辺の関数ははっきりと\type{Void->String}型ですが、これを\type{Void->Void}型の\expr{func}変数に代入することができます。これはコンパイラが戻り値は無関係だと安全に判断できるからで、その結果\type{Void}ではないあらゆる型を代入できるようになります。

\subsection{共通の基底型}
\label{type-system-unification-common-base-type}

複数の型の組み合わせが与えられたとき、そのすべての型が\emph{共通の基底型}で単一化されます。

\haxe{assets/UnifyMin.hx}

\type{Base}とは書かれていないにも関わらず、Haxeコンパイラは\type{Child1}と\type{Child2}の共通の型として\type{Base}を推論しています。Haxeコンパイラはこの方法の単一化を以下の場面で採用しています。

\begin{itemize}
	\item 配列の宣言
	\item \expr{if}/\expr{else}
	\item \expr{switch}のケース
\end{itemize}


\section{型推論}
\label{type-system-type-inference}

型推論はこのドキュメントで何度も出てきており、これ以降でも重要です。型推論の動作の簡単なサンプルをお見せします。

\haxe{assets/TypeInference.hx}
<<<<<<< HEAD:03-type-system.tex

この特殊な構文\expr{\$type}は、\Fullref{types-function}の型の説明をわかりやすくするためにも使っていました。それではここで公式な説明をしましょう。
=======
The special construct \expr{\$type} was previously mentioned in order to simplify the explanation of the \Fullref{types-function} type, so let us now introduce it officially:
>>>>>>> english/master:HaxeManual/03-type-system.tex

%TODO: $type
\define[Construct]{\expr{\$type}}{define-dollar-type}{\expr{\$type}は関数のように呼び出せるコンパイル時の仕組みで、一つの引数を持ちます。コンパイラは引数の式を評価し、そしてその式の型を出力します。}

上記の例では、最初の\expr{\$type}では\expr{Unknown<0>}が表示されます。これは\tref{単相}{types-monomorph}で、未知の型です。次の行の\expr{x = "foo"}で定数値の\type{String}を\expr{x}に代入しており、\type{String}の単相での\tref{単一化}{type-system-unification}が起こります。そして、\expr{x}がこのとき\type{String}に変わったことがわかります。

\Fullref{types-dynamic}以外の型が、単相での単一化を行うと単相はその型になります(その型に変形(\emph{morph})します)。このため、この型はもう別の型には変形できません。これが単相(monomorph)の\emph{mono}の部分です。

以下が単一化のルールです。型推論は複合型でも起こります。

\haxe{assets/TypeInference2.hx}

変数\expr{x}は初め空の\type{Array}で初期化されています。この時点で\expr{x}の型は配列であると言えますが、配列の要素の型については未知です。その結果\expr{x}の型は、\type{Array<Unknown<0>>}となります。この配列が\type{Array<String>}だとわかるには、\type{String}をこの配列にプッシュするだけで十分です。

\subsection{トップダウンの推論}
\label{type-system-top-down-inference}

多くの場合、ある型はその型で要求される型を推論します。しかし一部では、要求される型で型を推論します。これを\emph{トップダウンの推論}と呼びます。

\define{要求される型}{define-expected-type}{要求される型は、式の型が式が型付けされるより前にわかっている場合に現れます。例えば、式が関数の呼び出しの引数の場合です。この場合、\tref{トップダウンの推論}{type-system-top-down-inference}と呼ばれる方法で、式に型が伝搬します。}

良い例は型の混ざった配列です。\Fullref{types-dynamic}で書いた通り、\expr{[1, "foo"]}は要素の型を決定できないので、コンパイラはこれを拒絶します。これはトップダウンの推論を使えば解決します。

\haxe{assets/TopDownInference.hx}

ここでは、\expr{[1, "foo"]}に型付けするとき、要求される型が\type{Array<Dynamic>}であり、その要素は\type{Dynamic}であるとわかります。コンパイラが\tref{共通の基底型}{type-system-unification-common-base-type}を探す(そして失敗する)通常の単一化の挙動の代わりに、個々の要素が\type{Dynamic}で単一化され、型付けされます。

\tref{ジェネリック型パラメータのコンスラクト}{type-system-generic-type-parameter-construction}の紹介で、もう一つトップダウンの推論の面白い利用例を見ています。

\haxe{assets/GenericTypeParameter.hx}

\type{String}と\type{haxe.Template}の明示された型が、\expr{make}の戻り値の型の決定に使われています。これは、\expr{make()}の戻り値が変数へ代入されるのがわかっているので動作します。この方法を使うと、未知の型\type{T}をそれぞれ\type{String}と\type{haxe.Template}に紐づけることが可能です。

% this is not really top down inference
%Top-down inference is also utilized when dealing with \tref{enum constructors}{types-enum-constructor}:

%\haxe{assets/TopDownInference2.hx}

%The constructors \expr{TObject} and \expr{TFunction} of type \expr{ValueType} are recognized even though their containing module \type{Type} is not \tref{imported}{Import}. This is possible because the return type of \expr{Type.typeof("foo")} is known to be \expr{ValueType}.


\subsection{制限}
\label{type-system-inference-limitations}

ローカル変数をあつかう場合、型推論のおかげで多くの手動の型付けを省略できますが、型システムが助けを必要とする場面もあります。実際、\tref{変数フィールド}{class-field-variable}や\tref{プロパティ}{class-field-property}では、直接の初期化をしていない限りは型推論されません。

<<<<<<< HEAD:03-type-system.tex
また、再帰的な関数呼び出しでも型推論が制限される場面があります。型がまだ(完全に)わかっていない場合、型推論が間違って特殊すぎる型を推論する場合があります。

\section{モジュールとパス}
=======
A different kind of limitation involves the readability of code. If type inference is overused it might be difficult to understand parts of a program due to the lack of visible types. This is particularly true for method signatures. It is recommended to find a good balance between type inference and explicit type hints.


\section{Modules and Paths}
>>>>>>> english/master:HaxeManual/03-type-system.tex
\label{type-system-modules-and-paths}

\define{モジュール}{define-module}{
すべてのHaxeのコードはモジュールに属しており、パスを使って指定されます。要するに、.hxファイルそれぞれが一つのモジュールを表し、その中にいくつか型を置くことができます。型は\expr{private}にすることが可能で、その場合はその型の属するモジュールからしかアクセスできません。}

モジュールとそれに含まれる型との区別は意図的に不明瞭です。実際、\expr{haxe.ds.StringMap<Int>}の指定は、\expr{haxe.ds.StringMap.StringMap<Int>}の省略形とも考えられます。後者は4つ部位で構成されています。

\begin{enumerate}
	\item パッケージ \expr{haxe.ds}
	\item モジュール名 \expr{StringMap}
	\item 型名 \type{StringMap}
	\item 型パラメータ \type{Int}
\end{enumerate}

モジュールと型の名前が同じの場合、重複を取り除くことが可能で、これで\expr{haxe.ds.StringMap<Int>}という省略形が使えます。しかし、長い記述について知っていれば、\tref{モジュールの従属型}{type-system-module-sub-types}の指定の仕方の理解しやすくなります。

パスは、\tref{import}{type-system-import}を使ってパッケージの部分を省略することで、さらに短くすることができます。importの利用は不適切な識別子を作ってしまう場合があるので、\tref{解決順序}{type-system-resolution-order}についての理解が必要です。

\define{型のパス}{define-type-path}{(ドット区切りの)型のパスはパッケージ、モジュール名、型名から成ります。この一般的な形は\expr{pack1.pack2.packN.ModuleName.TypeName}です。} 


\subsection{モジュールの従属型}
\label{type-system-module-sub-types}

モジュール従属型とは、その型を定義しているモジュールの名前と異なる名前の型です。これにより、一つの.hxファイルに複数の型の定義が可能になり、これらの型はモジュール内では無条件でアクセス可能で、ほかのモジュールからも\expr{package.Module.Type}の形式でアクセスできます。

\begin{lstlisting}
var e:haxe.macro.Expr.ExprDef;
\end{lstlisting}

ここでは\expr{haxe.macro.Expr}の従属型\type{ExprDef}にアクセスしています。

<<<<<<< HEAD:03-type-system.tex
従属型の関係は、実行時には影響を与えません。publicの従属型はそのパッケージのメンバーになります。このため、同じパッケージの別々のモジュールで同じ従属型を定義した場合に衝突を起こします。
The sub-type relation is not reflected at runtime. That is, public sub-types become a member of their containing package, which could lead to conflicts if two modules within the same package try to define the same sub-type. Naturally the Haxe compiler detects these cases and reports them accordingly. In the example above, \type{ExprDef} is generated as \type{haxe.macro.ExprDef}.
=======
The sub-type relation is not reflected at run-time. That is, public sub-types become a member of their containing package, which could lead to conflicts if two modules within the same package tried to define the same sub-type. Naturally, the Haxe compiler detects these cases and reports them accordingly. In the example above \type{ExprDef} is generated as \type{haxe.macro.ExprDef}.
>>>>>>> english/master:HaxeManual/03-type-system.tex

Sub-types can also be made private:

\begin{lstlisting}
private class C { ... }
private enum E { ... }
private typedef T { ... }
private abstract A { ... }
\end{lstlisting}

\define{Private type}{define-private-type}{A type can be made private by using the \expr{private} modifier. As a result, the type can only be directly accessed from within the \tref{module}{define-module} it is defined in.

Private types, unlike public ones, do not become a member of their containing package.}

The accessibility of types can be controlled more fine-grained by using \tref{access control}{lf-access-control}.



\subsection{Import}
\label{type-system-import}

If a type path is used multiple times in a .hx file, it might make sense to use an \expr{import} to shorten it. This allows omitting the package when using the type:

\haxe{assets/Import.hx}

With \expr{haxe.ds.StringMap} being imported in the first line, the compiler is able to resolve the unqualified identifier \expr{StringMap} in the \expr{main} function to this package. The module \type{StringMap} is said to be \emph{imported} into the current file.

In this example, we are actually importing a \emph{module}, not just a specific type within that module. This means that all types defined within the imported module are available:

\haxe{assets/Import2.hx}

The type \type{Binop} is an \tref{enum}{types-enum-instance} declared in the module \type{haxe.macro.Expr}, and thus available after the import of said module. If we were to import only a specific type of that module, e.g. \expr{import haxe.macro.Expr.ExprDef}, the program would fail to compile with \expr{Class not found : Binop}.

There are several aspects worth knowing about importing:

\begin{itemize}
	\item The bottommost import takes priority (detailed in \Fullref{type-system-resolution-order}).
	\item The \tref{static extension}{lf-static-extension} keyword \expr{using} implies the effect of \expr{import}.
	\item If an enum is imported (directly or as part of a module import), all its \tref{enum constructors}{types-enum-constructor} are also imported (this is what allows the \expr{OpAdd} usage in the above example).
\end{itemize}

Furthermore, it is also possible to import \tref{static fields}{class-field} of a class and use them unqualified:

\haxe{assets/Import3.hx}

Special care has to be taken with field names or local variable names that conflict with a package name: Since they take priority over packages, a local variable named \expr{haxe} blocks off usage the entire \expr{haxe} package.

\paragraph{Wildcard import}

Haxe allows using \expr{.*} to allow import of all modules in a package, all types in a module or all static fields in a type. It is important to understand that this kind of import only crosses a single level as we can see in the following example:

\haxe{assets/ImportWildcard.hx}

Using the wildcard import on \expr{haxe.macro} allows accessing \type{Expr} which is a module in this package, but it does not allow accessing \type{ExprDef} which is a sub-type of the \type{Expr} module. This rule extends to static fields when a module is imported.

When using wildcard imports on a package the compiler does not eagerly process all modules in that package. This means that these modules are never actually seen by the compiler unless used explicitly and are then not part of the generated output.

\paragraph{Import with alias}

If a type or static field is used a lot in an importing module it might help to alias it to a shorter name. This can also be used to disambiguate conflicting names by giving them a unique identifier.

\haxe{assets/ImportAlias.hx}

Here we import \expr{String.fromCharCode} as \expr{f} which allows us to use \expr{f(65)} and \expr{f(66)}. While the same could be achieved with a local variable, this method is compile-time exclusive and guaranteed to have no run-time overhead.

\since{3.2.0}

Haxe also allows the more natural \expr{as} in place of \expr{in}.


\subsection{Resolution Order}
\label{type-system-resolution-order}

Resolution order comes into play as soon as unqualified identifiers are involved. These are \tref{expressions}{expression} in the form of \expr{foo()}, \expr{foo = 1} and \expr{foo.field}. The last one in particular includes module paths such as \expr{haxe.ds.StringMap}, where \expr{haxe} is an unqualified identifier.  

We describe the resolution order algorithm here, which depends on the following state:

\begin{itemize}
	\item the declared \tref{local variables}{expression-var} (including function arguments)
	\item the \tref{imported}{type-system-import} modules, types and statics
	\item the available \tref{static extensions}{lf-static-extension}
	\item the kind (static or member) of the current field
	\item the declared member fields on the current class and its parent classes
	\item the declared static fields on the current class
	\item the \tref{expected type}{define-expected-type}
	\item the expression being \expr{untyped} or not
\end{itemize}

\todo{proper label and caption + code/identifier styling for diagram}

\begin{flowchart}{type-system-resolution-order-diagram}{識別子`i'の解決順序}

\tikzset {
	level distance = 1.4cm,
	scale = 1
}
\tikzset{multiline/.style={align=center}}

\tikzstyle{noEdge} = [ auto = left, outer sep = 0.2cm ]

% Compact decision shape (cut off rectangle corners if you know how)
\tikzstyle{decisionc} = [
	decision,
	minimum height = 0.8cm,
	rectangle
]

\Tree
[.\node [decisionc] (dec1) {'i' == 'true'または'false'、'this'、'super'、'null'};
\edge [noEdge] node {いいえ};
[.\node [decisionc] (dec2) {ローカル変数'i'が存在する};
\edge [noEdge] node {いいえ};
[.\node [decisionc] (dec3) {現在のフィールドが静的フィールドである};
\edge [noEdge] node {いいえ};
[.\node [decisionc] (dec4) {現在のクラス、親クラスのいずれかにフィールド'i'が存在する};
\edge [noEdge] node {いいえ};
[.\node [decisionc] (dec5) {'this'の型の静的拡張があるか};
\edge [noEdge] node {いいえ};
[.\node [decisionc] (dec6) {現在のクラスに静的フィールド'i'があるか};
\edge [noEdge] node {いいえ};
[.\node [decisionc] (dec7) {インポートされたenumにコンストラクタ`i'があるか};
\edge [noEdge] node {いいえ};
[.\node [decisionc] (dec8) {静的フィールド`i'がインポートされているか};
\edge [noEdge] node {いいえ};
[.\node [decisionc] (dec9) {`i'が小文字から始まるか};
\edge [noEdge] node {いいえ};
[.\node [decisionc] (dec10) {型`i'がインポートされているか};
\edge [noEdge] node {いいえ};
[.\node [decisionc] (dec11) {現在のパッケージが型`i'をふくむモジュール`i'を持つか};
\edge [noEdge] node {いいえ};
[.\node [decisionc] (dec12) {トップレベルの型`i'が存在するか};
\edge [noEdge] node {いいえ};
[.\node [decisionc] (dec13) {untypedモードか};
\edge [noEdge] node {はい};
[.\node [decisionc] (dec14) {`i' == `__this__'};
\edge [noEdge] node {いいえ};
[.\node [decisionc] (dec15) {ローカル変数`i'を生成する};
]]]]]]]]]]]]]]]


\node [startstop, fill = green!70, xshift = 5cm] (resolve) at (dec15.east) {解決};

\tikzstyle{yesNode} = [above right, at start]

\coordinate (yesAnchor) at (resolve.north);

\draw [flowchartArrow] (dec1) -| (yesAnchor) node [yesNode] {はい};
\draw [flowchartArrow] (dec2) -| (yesAnchor) node [yesNode] {はい};
\draw [flowchartArrow] (dec4) -| (yesAnchor) node [yesNode] {はい};
\draw [flowchartArrow] (dec5) -| (yesAnchor) node [yesNode] {はい};
\draw [flowchartArrow] (dec6) -| (yesAnchor) node [yesNode] {はい};
\draw [flowchartArrow] (dec7) -| (yesAnchor) node [yesNode] {はい};
\draw [flowchartArrow] (dec8) -| (yesAnchor) node [yesNode] {はい};
\draw [flowchartArrow] (dec10) -| (yesAnchor) node [yesNode] {はい};
\draw [flowchartArrow] (dec11) -| (yesAnchor) node [yesNode] {はい};
\draw [flowchartArrow] (dec12) -| (yesAnchor) node [yesNode] {はい};
\draw [flowchartArrow] (dec14) -| (yesAnchor) node [yesNode] {はい};
\draw [flowchartArrow] (dec15) -- (resolve.west);

\draw [flowchartArrow] (dec3) to [out = 180, in = 180, distance = 4cm] (dec6);
\draw (dec3.west) node [above left] {はい};

\draw [flowchartArrow] (dec9) to [out = 180, in = 180, distance = 4cm] (dec13);
\draw (dec9.west) node [above left] {はい};

\node [startstop, fill = red!70, xshift = 3cm] (fail) at (dec13.east) {失敗};
\draw [flowchartArrow] (dec13.east) -- (fail.west) node [above right, at start] {いいえ};


\end{flowchart}

Given an identifier \expr{i}, the algorithm is as follows:

\begin{enumerate}
	\item If i is \expr{true}, \expr{false}, \expr{this}, \expr{super} or \expr{null}, resolve to the matching constant and halt.
	\item If a local variable named \expr{i} is accessible, resolve to it and halt.
	\item If the current field is static, go to \ref{resolution:static-lookup}.
	\item If the current class or any of its parent classes has a field named \expr{i}, resolve to it and halt.
	\item\label{resolution:static-extension} If a static extension with a first argument of the type of the current class is available, resolve to it and halt.
	\item\label{resolution:static-lookup} If the current class has a static field named \expr{i}, resolve to it and halt.
	\item\label{resolution:enum-ctor} If an enum constructor named \expr{i} is declared on an imported enum, resolve to it and halt.
	\item If a static named \expr{i} is explicitly imported, resolve to it and halt.
	\item If \expr{i} starts with a lower-case character, go to \ref{resolution:untyped}.
	\item\label{resolution:type} If a type named \expr{i} is available, resolve to it and halt.
	\item\label{resolution:untyped} If the expression is not in untyped mode, go to \ref{resolution:failure}
	\item If \expr{i} equals \expr{__this__}, resolve to the \expr{this} constant and halt.
	\item Generate a local variable named \expr{i}, resolve to it and halt.
	\item\label{resolution:failure} Fail
\end{enumerate}

For step \ref{resolution:type}, it is also necessary to define the resolution order of types:

\begin{enumerate}
	\item\label{resolution:import} If a type named \expr{i} is imported (directly or as part of a module), resolve to it and halt.
	\item If the current package contains a module named \expr{i} with a type named \expr{i}, resolve to it and halt.
	\item If a type named \expr{i} is available at top-level, resolve to it and halt.
	\item Fail
\end{enumerate}

For step \ref{resolution:import} of this algorithm as well as steps \ref{resolution:static-extension} and \ref{resolution:enum-ctor} of the previous one, the order of import resolution is important:

\begin{itemize}
	\item Imported modules and static extensions are checked from bottom to top with the first match being picked.
	\item Within a given module, types are checked from top to bottom.
	\item For imports, a match is made if the name equals.
	\item For \tref{static extensions}{lf-static-extension}, a match is made if the name equals and the first argument \tref{unifies}{type-system-unification}. Within a given type being used as static extension, the fields are checked from top to bottom.
\end{itemize}

\chapter{クラスフィールド}
\label{class-field}

\define{クラスフィールド}{define-class-field}{クラスフィールドはクラスに属する変数、プロパティまたはメソッドです。これは静的、または非静的になることができます。\emph{静的メソッド}と\emph{メンバ変数}といった名前を使うのと同じように、非静的フィールドについては\emph{メンバ}フィールドと呼びます。}

ここまで、一般的なHaxeのプログラムと型がどのように構成されているのかを見てきました。
このクラスフィールドに関する章では、構成に関する話題をまとめて、Haxeの動作に関する話題へのかけ橋とします。これはクラスフィールドが\tref{式}{expression}を持つ場所だからです。

クラスフィールドには3種類あります。

\begin{description}
	\item[変数:] \tref{変数}{class-field-variable}クラスフィールドにはある型の値が入っていて、参照、または代入することがができます。
	\item[プロパティ:] \tref{プロパティ}{class-field-property}クラスフィールドはアクセスされた時のカスタムの動作を定義します。クラスの外からは変数フィールドのように見えます。
	\item[メソッド:] \tref{メソッド}{class-field-method}はコードを実行するために呼び出すことのできる関数です。
\end{description}

厳密に言うと、変数は特定のアクセス方法を持つプロパティであると見なせます。実際に、Haxeコンパイラは変数とプロパティを型付けの段階では区別していません。しかし、構文レベルでは区別されます。

用語について補足すると、メソッドは(静的また非静的の)クラスに属する関数であり、式の中で現れる\tref{ローカル関数}{expression-function}のようなその他の関数はメソッドではありません。

\section{変数}
\label{class-field-variable}

変数フィールドについては、すでに前章でいくつかのサンプルコードで見てきました。変数フィールドは値を保持するもので、その性質はほとんどプロパティと共通しています(すべてでは無い)。

\haxe{assets/VariableField.hx}

ここから変数が以下のようなものだとわかります。

\begin{enumerate}
	\item 名前を持つ(ここでは\expr{member}),
	\item 型を持つ(ここでは\type{String}),
	\item 一定の初期値を持つ場合がある(ここでは\expr{"bar"}) and
	\item \tref{アクセス修飾子}{class-field-access-modifier}を持つ場合がある(ここでは\expr{static})
\end{enumerate}

上の例は最初に\expr{member}の初期値を出力した後、\expr{"foo"}を割り当ててから新しい値を出力しています。アクセス修飾子の効果は3種類のクラスフィールドで共通しており、その内容については後の節で説明します。

変数フィールドが初期値をもつ場合には、型の明示は不要になります。この場合、コンパイラが\tref{推論}{type-system-type-inference}を行います。

\begin{flowchart}{class-field-variable-init-values}{変数フィールドの値の初期化}
\Tree[.\node [decision] {\expr{inline}};
	\edge node[auto=right] {はい};
	[.\node [decision] {\expr{static}};
		\edge node[auto=right] {いいえ};
		\node [startstop, valueNone] {不正};
		\edge node[auto=left] {はい};
		\node [startstop, valueSome] {定数のみ};
	]
	\edge node[auto=left] {いいえ};
	[.\node [decision] {\expr{extern}};
		\edge node[auto=right] {いいえ};
		[.\node [decision] {\expr{static}};
			\edge node[auto=right] {いいえ};
			\node [startstop, valueSome] {'this'を使えない};
			\edge node[auto=left] {はい};
			\node [startstop, valueAll] {何でも};
		]
		\edge node[auto=left] {はい};
		\node [startstop, valueNone] {なし};
	]
]
\end{flowchart}


\section{プロパティ}
\label{class-field-property}

\tref{変数}{class-field-variable}に続き、プロパティがクラスにデータ持つ2番目の方法になります。変数とは異なり、プロパティはどのようなアクセスが許可されるかと、どのように生成されるかのより細かい制御が要求されます。よくある使い方は、例えば以下のようなものです。

\begin{itemize}
	\item どこからでも読み込み可能だが、書き込みは定義しているクラスからのみのフィールドを作る
	\item 読み込みアクセスがされたときに\emph{ゲッター}メソッドが実行されるフィールドを作る。
	\item 書き込みアクセスがされたときに\emph{セッター}メソッドが実行されるフィールドを作る。
\end{itemize}

プロパティをあつかう場合、2種類のアクセスについて理解することが重要です。

\define{読み込みアクセス}{define-read-access}{読み込みアクセスは右辺側で\tref{フィールドアクセス式}{expression-field-access}が使われると発生します。これには\expr{obj.field()}の形の関数呼び出しもふくまれるため、この\expr{field}も読み込みアクセスがされます。}

\define{書き込みアクセス}{define-write-access}{フィールドへの書き込みアクセスは、\tref{フィールドアクセス式}{expression-field-access}に\expr{obj.field = value}の形式で値の代入することで発生します。また、\expr{obj.field += value}の式\expr{+=}のような特殊な代入演算子を使うと、書き込みアクセスと\tref{読み込みアクセス}{define-read-access}の両方が発生します。} 

読み込みアクセスと書き込みアクセスを以下の構文を使って直接指定します。

\haxe{assets/Property.hx}

大部分は変数の構文と同じで、同じルールが適用されます。プロパティは以下の点で異なります。

\begin{itemize}
	\item フィールド名の後から小かっこが始まります(\expr{(})。
	\item 次に、特殊な\emph{アクセス識別子}が来ます(ここでは\expr{default})。
	\item カンマ(\expr{,})で区切ります。
	\item もう一つ特殊なアクセス識別子が続きます(ここでは\expr{null})。
	\item 小かっこを閉じます(\expr{)})。
\end{itemize}

1つ目のアクセス識別子はフィールドの読み込み、2つ目は書き込み時の挙動を決定します。アクセス識別子には以下の値が使用できます。

\begin{description}
	\item[\expr{default}:] フィールドの可視性が\expr{public}の場合、通常のフィールドと同じです。その他の場合、\expr{null}アクセスと同じです。
	\item[\expr{null}:] 定義したクラスのみからアクセスできます。
	\item[\expr{get}/\expr{set}:] アクセス時に\emph{アクセサメソッド}を呼び出します。コンパイラが使用可能なアクセサの存在を確認します。
	\item[\expr{dynamic}:] \expr{get}/\expr{set}アクセスに似ていますが、アクセサフィールドの存在を確認しません。
	\item[\expr{never}:] いかなるアクセスも許可しません。
\end{description}

\define{アクセサメソッド}{define-accessor-method}{型が\type{T}でフィールド名が\expr{field}のフィールドに対する\emph{アクセサメソッド}は、\type{Void->T}型のフィールド名\expr{get_field}の\emph{ゲッター}または\type{T->T}型のフィールド名\expr{set_field}の\emph{セッター}です。アクセサメソッドは略して\emph{アクセサ}とも呼びます。}

\trivia{アクセサ名}{Haxe2では、アクセス識別子に自由な識別子を使うことが可能で、その場合はそれがカスタムのアクセサメソッド名となっていました。しかし、これにより実装は変則的なものになっていました。例えば、\expr{Reflect.getProperty()}と\expr{Reflect.setProperty()}はどのような名前が名前が使われていたとしても対応する必要がありました。そのため、ターゲット出力時に参照のためのメタ情報を生成する必要がありました。\\
この識別子の名前を\expr{get_}、\expr{set_}から始まるもののみに制限することで、実装を大きく簡略化することに成功しました。これがHaxe2と3の間の破壊的な変更の1つ。}

\subsection{よくあるアクセス識別子の組み合わせ}
\label{class-field-property-common-combinations}

次の例はプロパティのよくあるアクセス識別子の組み合わせです。

\haxe{assets/Property2.hx}

\expr{main}メソッドの\target{JavaScript}へのコンパイル結果は、フィールドアクセスがどのようなものなのか理解する助けになるでしょう。

\begin{lstlisting}
var Main = function() {
	var v = this.get_x();
	this.set_x(2);
	var _g = this;
	_g.set_x(_g.get_x() + 1);
};
\end{lstlisting}

このとおり、読み込みアクセスは\expr{get_x()}の呼び出しとなり、書き込みアクセスは\expr{x}への\expr{2}の代入が\expr{set_x(2)}の呼び出しになりました。\expr{+=}の場合の出力は最初は少し不思議に見えるかもしれませんが、次の例で簡単にわかるはずです。

\haxe{assets/Property3.hx}

\expr{main}メソッドの\expr{x}のフィールドアクセスについて、ここで起きる事象は複雑です。まずこの場合は、\type{Main}のインスタンス化という副作用があります。そのため、コンパイラは\expr{new Main().x = new Main().x + 1}という出力を行わないように、複雑な式をローカル変数にキャッシュします。

\begin{lstlisting}
Main.main = function() {
	var _g = new Main();
	_g.set_x(_g.get_x() + 1);
}
\end{lstlisting}

\subsection{型システムへの影響}
\label{class-field-property-type-system-impact}

プロパティの存在は型システム対して、いくつかの重要な影響をもたらします。もっとも重要なのはプロパティはコンパイル時の機能であり、\emph{型がわかっている}必要があるということです。クラスインスタンスを\type{Dynamic}に代入すると、フィールドアクセスはアクセサメソッドを参照\emph{しません}。同じようにアクセス制限も働かなくなり、すべてのアクセスは\expr{public}と同じになります。

\expr{get}または\expr{set}のアクセス識別子を使うと、コンパイラはゲッターとセッターが本当に存在するかを確認します。以下はコンパイルできません。

\haxe{assets/Property4.hx}

\expr{get_x}メソッドを忘れていますが、親クラスでそれが定義されていた場合は今のクラスでそれを定義する必要はなくなります。

\haxe{assets/Property5.hx}

\expr{dynamic}アクセス識別子は\expr{get}や\expr{set}と同じように動作しますが、この存在チェックは行われません。

\subsection{ゲッターとセッターのルール}
\label{class-field-property-rules}

アクセサメソッドの可視性は、プロパティの可視性に影響を与えません。つまり、プロパティが\expr{public}であってもそのゲッターは\expr{private}でも構わないということです。

ゲッターとセッターは、その物理的フィールドにアクセスしてデータを使用する場合があります。アクセサメソッド自身からそのフィールドへのアクセスが行われた場合、コンパイラはこれをアクセサメソッド経由しないアクセスと見なします。これにより無限ループが回避されます。

\haxe{assets/GetterSetter.hx}

しかし、フィールドが少なくとも1つ、\expr{default}または\expr{null}のアクセス識別子を持つ時のみ、コンパイラはその物理的フィールドが存在していると考えます。

\define{物理的フィールド}{define-physical-field}{以下のいずれかの場合にフィールドが\emph{物理的}であると考えられます
	\begin{itemize}
		\item \tref{変数}{class-field-variable}
		\item 読み込みアクセスか書き込みアクセスのアクセス識別子が\expr{default}または\expr{null}である\tref{プロパティ}{class-field-property}
		\item \expr{:isVar}\tref{メタデータ}{lf-metadata}がつけられた\tref{プロパティ}{class-field-property}
	\end{itemize}
}

これらのケースに含まれない場合、アクセサメソッド内での自身のフィールドへのアクセスはコンパイルエラーを起こします。

\haxe{assets/GetterSetter2.hx}

物理的フィールドが必要であれば、\expr{:isVar}\tref{メタデータ}{lf-metadata}をフィールドつけることでそれを強制できます。

\haxe{assets/GetterSetter3.hx}

\trivia{プロパティのセッターの型}{新しいHaxeのユーザーにとって、セッターの型が\type{T->Void}ではなくて\type{T->T}でなくてはいけないというのはなじみがなく、驚かれるかもしれません。ではなぜ\emph{setter}が値を返す必要があるのでしょうか?\\
それはセッターを使ったフィールドへの代入を右辺の式として使いたいからです。\expr{x = y = 1}のような連結された式は、\expr{x = (y = 1)}として評価されます。\expr{x}に\expr{y = 1}の結果を代入するためには、\expr{y = 1}が値を持たなければなりません。\expr{y}のセッターの戻り値が\type{Void}であれば、それは不可能です。}

\section{メソッド}
\label{class-field-method}

\tref{変数}{class-field-variable}がデータを保持する一方で、メソッドは\tref{式}{expression}をもってプログラムの動作を定義します。このマニュアルのさまざまなサンプルコード中で、メソッドフィールドを見てきました。最初の\tref{Hello World}{introduction-hello-world}の例ですら、\expr{main}メソッドとして現れています。

\haxe{assets/HelloWorld.hx}

メソッドは\expr{function}キーワードから始まることで識別されます。そして以下の要素を持ちます。

\begin{enumerate}
	\item 名前を持つ(ここでは\expr{main})。
	\item 引数のリストを持つ(ここでは空の\expr{()})。
	\item 戻り値を持つ(ここでは\type{Void})。
	\item \tref{アクセス修飾子}{class-field-access-modifier}を持つ場合がある(ここでは\expr{static}と\expr{public})
	\item 式を持つ場合がある(ここでは\expr{\{trace("Hello World");\}})。
\end{enumerate}

引数と戻り値の型について学ぶために次の例を見てみましょう。

\haxe{assets/MethodField.hx}

引数はフィールド名の後に、小かっこ(\expr{(})を続け、引数の詳細のリストをカンマ(\expr{,})区切りで並べて、小かっこを閉じる(\expr{)})ことで記述します。引数の詳細についての情報は\Fullref{types-function}で説明されています。

この例からは\tref{型推論}{type-system-type-inference}が引数と戻り値についてどのように動作するのかもわかります。\expr{myFunc}は2つの引数を持ちますが、最初の引数の\expr{f}のみで\type{String}の型が明示されていて、2つ目の引数の\expr{i}には型注釈がありません。コンパイラがこのメソッドの呼び出しから推論を行うように残してあります。同じように、メソッドの戻り値の型も\expr{return true}から推論されて\type{Bool}になります。

\subsection{メソッドのオーバーライド(override)}
\label{class-field-overriding}

フィールドのオーバーライドは、クラスの階層構造を作る助けになります。オーバーライドはさまざまなデザインパターンで活用されますが、ここでは基本的な機能のみを説明します。クラスでオーバーライドを使うためには、\tref{親クラス}{types-class-inheritance}を持つ必要があります。次の例を見てみましょう。

\haxe{assets/Override.hx}

ここで重要なのは以下の要素です。

\begin{itemize}
	\item \type{Base}クラスは\expr{myMethod}メソッドとコンストラクタを持つ。
	\item \type{Child}は\type{Base}を継承しており、\expr{override}を宣言した\expr{myMethod}を持つ。
	\item \type{Main}クラスは\expr{main}メソッドで\expr{Child}をインスタンス化して、\type{Base}型を明示した\expr{child}変数に代入して、その\expr{myMethod()}を呼び出している。
\end{itemize}

\expr{child}変数の\type{Base}型を明示することで、コンパイル時には\type{Base}型であっても、実行時には\type{Child}型の\expr{myMethod}メソッドが実行されるという重要なことを強調しました。これはフィールドのアクセスが実行時に動的に解決されるからです。

\type{Child}クラスでは\expr{super.methodName()}を呼び出して、オーバーライドされたメソッドにアクセスすることができます。

\haxe{assets/OverrideCallParent.hx}

\expr{new}コンストラクタ内での\expr{super()}の使用については、\Fullref{types-class-inheritance}の節で説明してあります。

\subsection{変性とアクセス修飾子の影響}
\label{class-field-override-effects}

オーバーライドは\tref{変性}{type-system-variance}のルールと深い関わりがあります。というのは、引数の型が\emph{反変性} (より一般的な型)を許容し、戻り値の型は\emph{共変性}(より具体的な型)を許容するからです。

\haxe{assets/OverrideVariance.hx}

直観的には、この挙動は引数は関数へ「書き込み」であり戻り値は「読み込み」であるという事実から来ています。

この例から、\tref{可視性}{class-field-visibility}が変更できるということもわかります。オーバーライドされる側のフィールドが\expr{private}の場合は、\expr{public}のフィールドでオーバーライドすることができます。ただし、そのほかの場合は、可視性の変更はできません。

\tref{\expr{inline}}{class-field-inline}の宣言をされたフィールドもオーバーライドできません。これはインライン化がコンパイル時に関数の中身で書き換えを行う一方で、オーバーライドのフィールドは実行時に解決される、という衝突を避けるためです。	
	
\section{アクセス修飾子}
\label{class-field-access-modifier}
\state{NoContent}

\subsection{可視性}
\label{class-field-visibility}

フィールドはデフォルトでは\emph{private}です。つまり、そのクラス自身とその子クラスからしかアクセスできません。\expr{public}アクセス修飾子を使うことでどこからでもアクセスができるようにフィールドを公開できます。

\haxe{assets/Visibility.hx}

\type{Main}から\type{MyClass}の\expr{available}フィールドへのアクセスは、フィールドが\expr{public}なので許可されます。しかし\expr{unavailable}については、\type{MyClass}からのアクセスは許可されますが\type{Main}からは許可されません。これはフィールドが\expr{private}だからです(ここでは無くてもいい明示的宣言を行っています)。

この例では\emph{static}フィールドを使って可視性の実演をしていますが、メンバフィールドでもこのルールは同じです。次の例は\tref{継承}{types-class-inheritance}がある場合の可視性について実演しています。

\haxe{assets/Visibility2.hx}

\type{Child2}からの、\type{Child1}という異なる型の\expr{child1.baseField()}へのアクセスが許可されていることがわかります。これはこのフィールドが共通の親クラスの\type{Base}で定義されているからです。反対に\expr{child1Field}については、\type{Child2}からはアクセスできません。

可視性の修飾子の省略はデフォルトでは\expr{private}になることが多いですが、以下の場合は例外的に\expr{public}になります。

\begin{enumerate}
	\item クラスが\expr{extern}として宣言されている。
	\item \tref{インターフェース}{types-interfaces}で宣言しているフィールドである。
	\item \expr{public}フィールドを\tref{オーバーライド}{class-field-overriding}している。
\end{enumerate}

\trivia{protected}{HaxeにはJavaやC++やその他のオブジェクト指向言語で知られる\expr{protected}キーワードはありません。しかし、\expr{private}の挙動がこれらの言語の\expr{protected}の挙動に当たります。つまり、Haxeにはこれらの言語の\expr{private}に当たる挙動がありません。}

\subsection{inline(インライン化)}
\label{class-field-inline}

\expr{inline}キーワードはその関数の式を関数を呼び出した位置に直接挿し込みできるようにします。これは強力な最適化手段ですが、すべての関数にインライン化の挙動を持つ資格があるわけでありません。基本的な使い方は以下の通りです。

\haxe{assets/Inline.hx}

\target{JavaScript}出力を見るとインライン化の効果がわかります。

\begin{lstlisting}
(function () { "use strict";
var Main = function() { }
Main.main = function() {
	var a = 1;
	var b = 2;
	var c = (a + b) / 2;
}
Main.main();
})();
\end{lstlisting}

見てのとおり\expr{mid}フィールドの関数本体の\expr{(s1 + s2) / 2}が、\expr{mid(a, b)}の位置で\expr{s1}を\expr{a}に\expr{s2}を\expr{b}に置き換えられて出力されています。ターゲットによっては消えない場合もありますが、関数呼び出しが消滅しており、これが大きなパフォーマンスの改善になりえます。

インライン化するべきかの判断は簡単ではありません。書き込み処理の無い短い関数(\expr{=}の代入のみといった)は、たいていインライン化すると良いですし、より複雑な関数であってもインライン化する候補となりえます。一方で、インライン化がパフォーマンスに悪影響を与える場合もあります(複雑な式では、コンパイラが一時変数を作るなどのため)。

\expr{inline}キーワードは、インライン化されることを保証しません。コンパイラはさまざまな理由でインライン化をキャンセルします。例えば、コマンドラインの引数で\ic{--no-inline}が与えられた場合です。例外としてクラスが\tref{extern}{lf-externs}か、フィールドが\expr{:extern}\tref{メタデータ}{lf-metadata}を付けられていた場合、インライン化が強制されます。インライン化ができない場合、コンパイラはエラーを出力します。

これはインライン化を使う上で重要なので覚えておきましょう。

\haxe{assets/InlineRelying.hx}

\expr{error}の呼び出しがインライン化できれば、制御フローのチェッカーはインライン化された\tref{throw}{expression-throw}に満足してプログラムは正しくコンパイルされます。インライン化されなければ、コンパイラは関数呼び出しのみを見て、\expr{A return is missing here}(ここにリターンが足りません)というエラーを出力します。

\subsection{dynamic}
\label{class-field-dynamic}

メソッドは\expr{dynamic}キーワードをつけることで、束縛のしなおしをできるようにします。

\haxe{assets/DynamicFunction.hx}

最初の\expr{test()}の呼び出しではもともとの関数を実行して\expr{"original"}の文字列を返します。つぎの行で、\expr{test()}に新しい関数が代入されます。これが\expr{dynamic}が可能にする関数の再束縛です。その結果として、次の\expr{test()}の呼び出しでは\expr{"new"}の文字列が返っています。

\expr{dynamic}フィールドは\expr{inline}フィールドにできません。その理由は明らかです。インライン化はコンパイル時に行われますが、\expr{dynamic}な関数は実行時に解決されます。

%TODO: performance estimation %

\subsection{override}
\label{class-field-override}

\expr{override}アクセス修飾子は\tref{親クラス}{types-class-inheritance}ですでに存在するフィールドを定義するときに必要です。これはクラスの継承関係が大きい場合でも、書き手がオーバーライドに気づくようにするためです。同様に、実際には何もオーバーライドしないフィールドに\expr{override}しようとするとエラーになります(例えば、スペルミスの場合)。

オーバーライドの効果については、\Fullref{class-field-overriding}で詳しく説明しています。この修飾子は\tref{メソッド}{class-field-method}フィールドのみに使用可能です。

\chapter{式}
\label{expression}

Haxeの式は、プログラムが\emph{何をするか}を決定します。ほとんどの式は\tref{メソッド}{class-field-method}に書かれ、そのメソッドが何をすべきかをその式の合わせによって表現します。この章では、さまざまな種類の式を説明していきます。

ここに、いくつか定義を示しておきます。

\define{名前}{define-name}{
名前は次のいずれかにひもづきます。
\begin{itemize}
	\item 型
	\item ローカル変数
	\item ローカル関数
	\item フィールド
\end{itemize}}

\define{識別子}{define-identifier}{
Haxeの識別子は、アンダースコア\expr{_}、ドル\expr{\$}、小文字\expr{a-z}、大文字\expr{A-Z}のいずれかから始まり、任意の\expr{_}、\expr{A-Z}、\expr{a-z}、\expr{0-9}のつなぎ合わせが続きます。

さらに使用する状況によって以下の制限が加わります。これらは、型付けの時にチェックされます。
\begin{itemize}
	\item 型の名前は大文字\expr{A-Z}か、アンダースコア\expr{_}で始まる。
	\item \tref{名前}{define-name}では、先頭にドル記号は使えません。(ドル記号はほとんどの場合、\tref{マクロの実体化}{macro-reification}に使われます)
\end{itemize}}


\section{ブロック}
\label{expression-block}

Haxeのブロックは中かっこで\expr{\{}から始まり、\expr{\}}で終わります。ブロックはいくつかの式をふくみ、各式はセミコロンで終わります。一般の構文としては以下のとおりです。

\begin{lstlisting}
{
	式1;
	式2;
	...
	式N;
}
\end{lstlisting}

ブロック式の値とその型は、ブロック式がふくむ最後の式の値と型と同じになります。

ブロック内では、\tref{\expr{var}式}{expression-var}を使ったローカル変数の定義と\tref{\expr{function}式}{expression-function}を使ったローカル関数の定義が可能です。これらのローカル変数とローカル関数は、そのブロックとさらに入れ子のブロックの中では使用することができますが、ブロックの外では利用できません。また、定義よりも後でしか使えません。次の例では\expr{var}を使っていますが、同じルールが\expr{function}の場合でも使用されます。

\begin{lstlisting}
{
	a; // error, a is not declared yet
	var a = 1; // declare a
	a; // ok, a was declared
	{
		a; // ok, a is available in sub-blocks
	}
  // ok, a is still available after
	// sub-blocks	
	a;
}
a; // error, a is not available outside
\end{lstlisting}

実行時には、ブロックは上から下へと評価されていきます。フロー制御(例えば、\tref{例外}{expression-try-catch}や\tref{return式}{expression-return}など)によって、すべての式が評価される前に中断されることもあります。

\section{定数値}
\label{expression-constants}

Haxeの構文では以下の定数値をサポートしています。

\begin{description}
<<<<<<< HEAD
	\item[Int:] \expr{0}、\expr{1}、\expr{97121}、\expr{-12}、\expr{0xFF0000}といった、\tref{整数}{define-int}
	\item[Float:] \expr{0.0}、\expr{1.}、\expr{.3}、\expr{-93.2}といった\tref{浮動小数点数}{define-float}
	\item[String:] \expr{""}、\expr{"foo"}、\expr{''}、\expr{'bar'}といった\tref{文字列}{define-string}
	\item[true,false:] \tref{真偽値}{define-bool}
	\item[null:] null値
=======
	\item[Int:] An \tref{integer}{define-int}, such as \expr{0}, \expr{1}, \expr{97121}, \expr{-12}, \expr{0xFF0000}.
	\item[Float:] A \tref{floating point number}{define-float}, such as \expr{0.0}, \expr{1.}, \expr{.3}, \expr{-93.2}.
	\item[String:] A \tref{string of characters}{define-string}, such as \expr{""}, \expr{"foo"}, \expr{'{'}}, \expr{'bar'}.
	\item[true,false:] A \tref{boolean}{define-bool} value.
	\item[null:] The null value.
>>>>>>> english/master
\end{description}

また内部の構文木では、\tref{識別子}{define-identifier}は定数値としてあつかわれます。これは、\tref{マクロ}{macro}を使っているときに関係してくる話題です。

\section{2項演算子}
\label{expression-binops}

\section{単項演算子}
\label{expression-unops}

\section{配列の宣言}
\label{expression-array-declaration}

配列は\expr{,}で区切った値を、大かっこ\expr{[]}で囲んで初期化します。空の\expr{[]}は空の配列を表し、\expr{[1, 2, 3]}は\expr{1}、\expr{2}、\expr{3}の3つの要素を持つ配列になります。

配列の初期化をサポートしていないプラットフォームでは、生成されたコードはあまり簡潔ではないかもしれません。本質的には以下のようなコードに見えるでしょう。

\begin{lstlisting}
var a = new Array();
a.push(1);
a.push(2);
a.push(3);
\end{lstlisting}

つまり、関数を\tref{インライン化}{class-field-inline}するかを決める場合には、この構文で見えているよりも多くのコードがインライン化されることがあることを考慮すべきです。

より高度な初期化方法は、\Fullref{lf-array-comprehension}で説明します。

\section{オブジェクトの宣言}
\label{expression-object-declaration}

オブジェクトの宣言は、中かっこ\expr{\{}で始まり、\expr{キー:値}のペアがカンマ\expr{,}で区切られながら続いて、中かっこ\expr{\}}で終わります。

\begin{lstlisting}
{
	key1:value1,
	key2:value2,
	...
	keyN:valueN
}
\end{lstlisting}
さらに詳しいオブジェクトの宣言については\tref{匿名構造体}{types-anonymous-structure}の節で書かれています。

\section{フィールドへのアクセス}
\label{expression-field-access}

フィールドへのアクセスは、ドット\expr{.}の後にフィールドの名前を続けることで表現します。

\begin{lstlisting}
object.fieldName
\end{lstlisting}

この構文は\expr{pack.Type}の形でパッケージ内の型にアクセスするのにも使われます。

型付け機は、アクセスされたフィールドが本当に存在するかを確認し、フィールドの種類に依存した変更を適用します。もしフィールドへのアクセスが複数の意味にとれる場合は、\tref{解決順序}{type-system-resolution-order}の理解が役に立つでしょう。

\section{配列アクセス}
\label{expression-array-access}

配列アクセスは、大かっこ\expr{[}で始まり、インデックスを表す式が続き、大かっこ\expr{]}で閉じます。

\begin{lstlisting}
expr[indexExpr]
\end{lstlisting}

この記法については任意の式で許可されていますが、型付けの段階では以下の特定の組み合わせのみが許可されます。

\begin{itemize}
	\item \expr{expr}は\type{Array}か\type{Dynamic}であり、\expr{indexExpr}が\type{Int}である。
	\item \expr{expr}は\tref{抽象型}{types-abstract}であり、マッチする\tref{配列アクセス}{types-abstract-array-access}が定義されている。
\end{itemize}

\section{関数呼び出し}
\label{expression-function-call}

関数呼び出しは、任意の式を対象として、小かっこ\expr{(}を続け、引数の式のリストを\expr{,}で区切って並べて、小かっこ\expr{)}で閉じることで行います。

\begin{lstlisting}
subject(); // call with no arguments
subject(e1); // call with one argument
subject(e1, e2); // call with two arguments
// call with multiple arguments
subject(e1, ..., eN);
\end{lstlisting}

\section{var(変数宣言)}
\label{expression-var}

\expr{var}キーワードは、カンマ\expr{,}で区切って、複数の変数を宣言することができます。すべての変数は、正当な\tref{識別子}{define-identifier}を持ち、オプションとして\expr{=}を続けて値の代入を行うこともできます。また変数に明示的な型注釈をあたえることもできます。

\begin{lstlisting}
var a; // declare local a
var b:Int; // declare variable b of type Int
// declare variable c, initialized to value 1
var c = 1;
// declare variable d and variable e
// initialized to value 2
var d,e = 2;
\end{lstlisting}

ローカル変数のスコープについての挙動は\Fullref{expression-block}に書かれています。

\section{ローカル関数}
\label{expression-function}

Haxeはファーストクラス関数をサポートしており、式の中でローカル関数を宣言することができます。この構文は\tref{クラスフィールドメソッド}{class-field-method}にならいます。

\haxe{assets/LocalFunction.hx}

\expr{myLocalFunction}を、\expr{main}クラスフィールドの\tref{ブロック式}{expression-block}の中で宣言しました。このローカル関数は1つの引数\expr{i}を取り、それをスコープの外のvalueに足しています。

スコープについては、\tref{変数の場合}{expression-var}と同じで、多くの面で名前を持つローカル関数は、ローカル変数に対する匿名関数の代入と同じです。

\begin{lstlisting}
var myLocalFunction = function(a) { }
\end{lstlisting}

しかしながら、関数の場所による型パラメータに関する違いがあります。これは定義時に何にも代入されていない「左辺値」の関数と、それ以外の「右辺値」の関数についての違いで、以下の通りです。

\begin{itemize}
	\item 左辺値の関数は名前が必要で、\tref{型パラメータ}{type-system-type-parameters}を持ちます。
	\item 右辺値の関数については名前はあってもなくてもかまいませんが、型パラメータを使うことができません。
\end{itemize}

\section{new(インスタンス化)}
\label{expression-new}

\expr{new}キーワードは、\tref{クラス}{types-class-instance}と\tref{抽象型}{types-abstract}のインスタンス化を行います。\expr{new}の後にはインスタンス化される\tref{型のパス}{define-type-path}が続きます。場合によっては、\expr{<>}で囲んでカンマ\expr{,}で区切った、\tref{型パラメータ}{type-system-type-parameters}の記述がされます。その後に、小かっこ\expr{(}、カンマ\expr{,}区切りのコンストラクタの引数が続き、小かっこ\expr{)}で閉じます。

\haxe{assets/New.hx}

\expr{main}メソッドの中では、型パラメータ\type{Int}の明示付き、引数が\expr{12}と\expr{"foo"}で、\type{Main}クラス自身のインスタンス化を行っています。私たちが知っているように、この構文は、\tref{関数呼び出し}{expression-function-call}とよく似ており、「コンストラクタ呼び出し」と呼ぶことが多いです。

\section{for}
\label{expression-for}

Haxeは、C言語で知られる伝統的なforループはサポートしていません。\expr{for}キーワードの後には、小かっこ\expr{(}、変数の識別子、\expr{in}キーワード、くり返しの処理を行うコレクションの任意の式が続き、小かっこ\expr{)}で閉じられて、最後にくり返しの本体の任意の式で終わります。

\begin{lstlisting}
for (v in e1) e2;
\end{lstlisting}

型付け機は、\expr{e1}の型がくり返し可能であるかを確認します。くり返し可能というのは、\expr{iterator}メソッドが\type{Iterator<T>}を返すか、\type{Iterator<T>}自身である場合です。

変数vは、ループ本体の\expr{e2}の中で利用可能で、コレクション\expr{e1}の個々の要素の値が保持されます。

Haxeには、ある範囲のくり返しを表す特殊な範囲演算子があります。これは、\expr{min...max}といった2つの\type{Int}をとり、\expr{min}(自身をふくむ)から\expr{max}の一つ前までをくり返す\expr{IntIterator}を返す2項演算子です。\expr{max}が\expr{min}より小さくしないように気をつけてください。

\begin{lstlisting}
for (i in 0...10) trace(i); // 0 to 9
\end{lstlisting}

\expr{for}式の型は常に\type{Void}です。つまり、値は持たず、右辺の式としては使えません。

ループは、\tref{\expr{break}}{expression-break}と、\tref{\expr{continue}}{expression-continue}の式を使って、フロー制御が行えます。

\section{whileループ}
\label{expression-while}

通常の\expr{while}ループは、\expr{while}キーワードから始まり、小かっこ\expr{(}、条件式が続き、小かっこ\expr{)}を閉じて、ループ本体の式で終わります。

\begin{lstlisting}
while(condition) expression;
\end{lstlisting}

条件式は\type{Bool}型でなくてはいけません。

各くり返しで条件式は評価されます。\expr{false}と評価された場合ループは終了します。そうでない場合、ループ本体の式が評価されます。

\haxe{assets/While.hx}

この種類の\expr{while}ループは、ループ本体が一度も評価されないことがあります。条件式が最初から\expr{false}だった場合です。この点が\tref{do-whileループ}{expression-do-while}との違いです。

\section{do-whileループ}
\label{expression-do-while}

do-whileループは、\expr{do}キーワードから始まり、次にループ本体の式が来ます。その後に\expr{while}、小かっこ\expr{(}、条件式、小かっこ\expr{)}となります。

\begin{lstlisting}
do expression while(condition);
\end{lstlisting}

条件式は\type{Bool}型でなくてはいけません。

この構文を見てわかるとおり、\tref{while}{expression-while}ループの場合とは違ってループ本体の式は少なくとも一度は評価をされます。

\section{if}
\label{expression-if}

条件分岐式は、\expr{if}キーワードから始まり、小かっこ\expr{()}で囲んだ条件式、条件が真だった場合に評価される式となります。

\begin{lstlisting}
if (condition) expression;
\end{lstlisting}

条件式は\type{Bool}型でなくてはいけません。

オプションとして、\expr{else}キーワードを続けて、その後に、元の条件が偽だった場合に実行される式を記述することができます。

\begin{lstlisting}
if (condition) expression1 else expression2;
\end{lstlisting}

\expr{expression2}は以下のように、また別の\expr{if}式を持つかもしれません。

\begin{lstlisting}
if (condition1) expression1
else if(condition2) expression2
else expression3
\end{lstlisting}

\expr{if}式に値が要求される場合(たとえば、\expr{var x = if(condition) expression1 else expression2}という風に)、型付け機は\expr{expression1}と\expr{expression2}の型を\tref{単一化}{type-system-unification}します。\expr{else}式がなかった場合、型は\type{Void}であると推論されます。

\section{switch}
\label{expression-switch}

基本的なスイッチ式は、\expr{switch}キーワードと、その分岐対象の式から始まり、中かっこ\expr{\{\}}にはさまれてケース式が並びます。各ケース式は、\expr{case}キーワードからのパターン式か、\expr{default}キーワードで始まります。どちらの場合も、コロンが続き、オプショナルなケース本体の式が来ます。

\begin{lstlisting}
switch subject {
	case pattern1: case-body-expression-1;
	case pattern2: case-body-expression-2;
	default: default-expression;
}
\end{lstlisting}

ケース本体の式に、「フォールスルー」は起きません。このため、Haxeでは\tref{\expr{break}}{expression-break}キーワードは使用しません。

スイッチ式は値としてあつかうことができます。その場合、すべてのケース本体の式の型は\tref{単一化}{type-system-unification}できなくてはいけません。

パターン式については、\Fullref{lf-pattern-matching}で詳しく説明されています。

\section{try/catch}
\label{expression-try-catch}

Haxeでは、\expr{try/catch}構文を使うことで値を捕捉することができます。

\begin{lstlisting}
try try-expr
catch(varName1:Type1) catch-expr-1
catch(varName2:Type2) catch-expr-2
\end{lstlisting}

実行時に、\expr{try-expression}の評価が、\tref{\expr{throw}}{expression-throw}を引き起こすと、後に続く\expr{catch}ブロックのいずれかに捕捉されます。これらのブロックは以下から構成されます

\begin{itemize}
	\item \expr{throw}された値を割り当てる変数の名前。
	\item 捕捉する値の型を決める、明示的な型注釈
	\item 捕捉したときに実行される式
\end{itemize}

Haxeでは、あらゆる種類の値を\expr{throw}して、\expr{catch}することができます。その型は特定の例外やエラークラスに限定されません。\expr{catch}ブロックは上から下へとチェックされていき、投げられた値と型が適合する最初のブロックが実行されます。

この過程は、コンパイル時の\tref{単一化}{type-system-unification}に似ています。しかし、この判定は実行時に行われるものでいくつかの制限があります。

\begin{itemize}
	\item 型は実行時に存在するものでなければならない。\tref{クラスインスタンス}{types-class-instance}、\tref{列挙型インスタンス}{types-enum-instance}、\tref{コアタイプ抽象型}{types-abstract-core-type}、\tref{Dynamic}{types-dynamic}.
	\item 型パラメータは、\tref{Dynamic}{types-dynamic}でなければならない。
\end{itemize}

\section{return}
\label{expression-return}

\expr{return}式は、値をとるものと、とらないものの両方があります。

\begin{lstlisting}
return;
return expression;
\end{lstlisting}

\expr{return}式は、最も内側に定義されている関数のフロー制御からぬけ出します。最も内側というのは\tref{ローカル関数}{expression-function}の場合での特徴です。

\begin{lstlisting}
function f1() {
	function f2() {
		return;
	}
	f2();
	expression;
}
\end{lstlisting}

\expr{return}により、ローカル関数\expr{f2}からはぬけ出しますが、\expr{f1}からはぬけ出しません。つまり、\expr{expression}は評価されます。

\expr{return}が、値の式なしで使用された場合、型付け機はその関数の戻り値が\type{Void}型であることを確認します。\expr{return}が値の式を持つ場合、型付け機はその関数の戻り値の型(明示的に与えられているか、前のreturnによって推論されている場合)と、\expr{return}している値の型を\tref{単一化}{type-system-unification}します。

\section{break}
\label{expression-break}

\expr{break}キーワードは、そのキーワードをふくむ最も内側にあるループ(\expr{for}でも、\expr{while}でも)の制御フローからぬけ出して、くり返し処理を終了させます。

\begin{lstlisting}
while(true) {
	expression1;
	if (condition) break;
	expression2;
}
\end{lstlisting}

\expr{expression1}はすべてのくり返しで評価されますが、\expr{condition}が偽になると\expr{expression2}は、実行されません。

型付け機は\expr{break}がループの内部のみで使用されていることを確認します。\tref{\expr{switch}のケース}{expression-switch}に対する\expr{break}は、Haxeではサポートしていません。

\section{continue}
\label{expression-continue}

\expr{continue}キーワードは、そのキーワードをふくむ最も内側にあるループ(\expr{for}でも、\expr{while}でも)の現在のくり返しを終了します。そして、次のくり返しのためのループ条件チェックが行われます。

\begin{lstlisting}
while(true) {
	expression1;
	if(condition) continue;
	expression2;
}
\end{lstlisting}

\expr{expression1}は、各くり返しすべてで評価されますが、\expr{condition}が偽の時は、その回のくり返しについては評価がされません。\expr{break}は異なりループ処理自体は続きます。

型付け機は\expr{continue}がループの内部のみで使用されていることを確認します。

\section{throw}
\label{expression-throw}

Haxeでは、以下の構文で、値の\expr{throw}をすることができます。

\begin{lstlisting}
throw expr
\end{lstlisting}

\expr{throw}された値は、\tref{\expr{catch}ブロック}{expression-try-catch}で捕捉できます。捕捉されなかった場合の挙動はターゲット依存です。

\section{cast}
\label{expression-cast}

Haxeには、以下の2種類のキャストがあります。

\begin{lstlisting}
cast expr; // unsafe cast
cast (expr, Type); // safe cast
\end{lstlisting}

\subsection{非セーフキャスト}
\label{expression-cast-unsafe}

非セーフキャストは型システムを無力化するのに役立ちます。コンパイラは\expr{expr}を通常通りに型付けを行い、それを\tref{単相}{types-monomorph}としてつつみ込みます。これにより、その式をあらゆるものに割り当てすることが可能です。

非セーフキャストは、以下の例が示すように、\tref{Dynamic}{types-dynamic}への型変更ではありません。

\haxe{assets/UnsafeCast.hx}

変数\expr{i}は\type{Int}と型付けされて、非セーフキャスト\expr{cast i}を使って変数\expr{s}に代入しました。\expr{s}は、\type{Unknown}型、つまり単相となりました。その後は、通常の\tref{単一化}{type-system-unification}のルールに従って、あらゆる型へと結びつけることが可能です。例では、\type{String}型となりました。

これらのキャストは「非セーフ」と呼ばれます。これは、実行時の不正なキャストが定義されてないためです。 ほとんどの\tref{動的ターゲット}{define-dynamic-target}では動作する可能性が高いですが、\tref{静的ターゲット}{define-static-target}では未知のエラーの原因になりえます。

非セーフキャストは実行時のオーバーヘッドは、ほぼ、または全くありません。

\subsection{セーフキャスト}
\label{expression-cast-safe}

\tref{非セーフキャスト}{expression-cast-unsafe}とは異なり、実行時のキャスト失敗の挙動を持つのがセーフキャストです。

\haxe{assets/SafeCast.hx}

この例では、最初に\type{Child1}から\type{Base}へとキャストしています。これは、\type{Child1}が\type{Base}型の\tref{子クラス}{types-class-inheritance}なので、成功しています。次に\type{Child2}へキャストしていますが、\type{Child1}のインスタンスは\type{Child2}ではないので失敗しています。

Haxeコンパイラは、この場合\type{String}型の\tref{例外を投げます}{expression-throw}。この例外は、\tref{\expr{try/catch}ブロック}{expression-try-catch}を使って捕捉できます。

セーフキャストは実行時のオーバーヘッドがあります。重要なのは、コンパイラがすでにチェックを行っているので、\expr{Std.is}のようなチェックを自分で入れるのは、余分だということです。\type{String}型の例外を捕捉する、try-catchを行うのがセーフキャストで意図された用途です。

\section{型チェック}
\label{expression-type-check}
\since{3.1.0}

以下の構文でコンパイルタイムの型チェックをつけることが可能です。

\begin{lstlisting}
(expr : type)
\end{lstlisting}

小かっこは必須です。\tref{セーフキャスト}{expression-cast-safe}とは異なり、実行時に影響はありません。これは、コンパイル時の以下の2つの挙動を持ちます。

\begin{enumerate}
\item \tref{トップダウンの型推論}{type-system-top-down-inference}が\expr{expr}に対して\expr{type}の型で適用されます。
\item その結果、\expr{type}の型との\tref{単一化}{type-system-unification}がされます。
\end{enumerate}

この2つの操作には、\tref{解決順序}{type-system-resolution-order}が発生している場合や、\tref{抽象型キャスト}{types-abstract-implicit-casts}で、期待する型へと変化させる、便利な効果があります。

\chapter{言語機能}
\label{lf}

\emph{\tref{抽象型}{types-abstract}:}

抽象型は実行時には別の形として提供されるコンパイル時の構成要素です。これにより、すでに存在する型に別の意味をあたえることができます。

\emph{\tref{externクラス}{lf-externs}:}

externを使うことで、型安全のルールにしたがってターゲット固有の連携を記述することができます。

\emph{\tref{匿名構造体}{types-anonymous-structure}:}

匿名構造体を使うことでデータを簡単にまとめて、小さなデータクラスの必要性を減らすことができます。

\begin{lstlisting}
var point = { x: 0, y: 10 };
point.x += 10;
\end{lstlisting}

\emph{\tref{配列内包表記}{lf-array-comprehension}:}

ループと条件分岐を使って、素早く配列を生成して受け渡すことができます。

\begin{lstlisting}
var evenNumbers = [ for (i in 0...100) if (i\%2==0) i ];
\end{lstlisting}

\emph{\tref{クラス、インターフェース、継承}{types-class-instance}:}

Haxeは、クラスを使ったコードの構造化ができる、オブジェクト指向言語です。継承やインターフェースといったJavaでサポートされるようなオブジェクト指向言語の標準的な機能を備えています。

\emph{\tref{条件付きコンパイル}{lf-condition-compilation}:}

条件付きコンパイルを使うことで、コンパイルのパラメータごとに固有のコードをコンパイルすることができます。これはターゲットごとの違いを抽象化する手助けになるだけでなく、詳細のデバッグ機能を提供するなどその他の用途にも使用できます。

\begin{lstlisting}
\#if js
    js.Lib.alert("Hello");
\#elseif sys
    Sys.println("Hello");
\#end
\end{lstlisting}

\emph{\tref{(一般化)代数的データ型}{types-enum-instance}:}

Haxeではenumとして知られる、代数的データ型(ADT)を使って構造体を表現することができます。さらに、HaxeはGADTとして知られる一般化されたヴァリアントもサポートしています。

\begin{lstlisting}
enum Result {
    Success(data:Array<Int>);
    UserError(msg:String);
    SystemError(msg:String, position:PosInfos);
}
\end{lstlisting}

\emph{\tref{インライン呼び出し}{class-field-inline}:}

関数はインラインとして指定して、呼び出し場所にその関数のコードを挿入させることができます。これにより、手作業でのインライン化のようなコードの重複を発生させること無く、価値あるパフォーマンスの改善を得ることできます。

\emph{\tref{イテレータ(反復子)}{lf-iterators}:}

Haxeはイテレータを適切にあつかっているので、値のセット(例えば、配列)の反復処理がとても簡単です。自前のクラスであっても、イテレータ機能の実装をすることで素早く反復可能にすることができます。n.

\begin{lstlisting}
for (i in [1, 2, 3]) {
    trace(i);
}
\end{lstlisting}

\emph{\tref{ローカル関数とクロージャ}{expression-function}:}

Haxeでは関数はクラスフィールドに限定されず、式の中で定義することができます。その場合、強力なクロージャも利用可能です。

\begin{lstlisting}
var buffer = "";
function append(s:String) {
    buffer += s;
}
append("foo");
append("bar");
trace(buffer); // foobar
\end{lstlisting}

\emph{\tref{メタデータ}{lf-metadata}:}

フィールド、クラス、式に対してメタデータを追加できます。これにより、コンパイラ、マクロ、実行時のクラスに情報の受け渡しができます。

\begin{lstlisting}
class MyClass {
    @range(1, 8) var value:Int;
}
trace(haxe.rtti.Meta.getFields(MyClass).value.range); // [1,8]
\end{lstlisting}

\emph{\tref{静的拡張}{lf-static-extension}:}

既に存在するクラスやその他の型に対して、静的拡張を使うことで追加の機能を足すことができます。

\begin{lstlisting}
using StringTools;
"  Me & You    ".trim().htmlEscape();
\end{lstlisting}

\emph{\tref{文字列中の変数展開}{lf-string-interpolation}:}

シングルクオテーションを使って宣言した文字列では現在の文脈中の変数へのアクセスができます。

\begin{lstlisting}
trace('My name is $name and I work in ${job.industry}');
\end{lstlisting}

\emph{\tref{関数の部分適用}{lf-function-bindings}:} 

すべての関数は部分適用を使って、いくつかの引数だけに値を適用して残りの引数を後で指定できるように残すことができます。

\begin{lstlisting}
var map = new haxe.ds.IntMap();
var setToTwelve = map.set.bind(_, 12);
setToTwelve(1);
setToTwelve(2);
\end{lstlisting}

\emph{\tref{パターンマッチング}{lf-pattern-matching}:} 

複雑な構造体は、enumや構造体から情報を抽出したり、特定の演算子で値の組み合わせを指定したりしながら、パターンを当てはめてマッチングすることができます。

\begin{lstlisting}
var a = { foo: 12 };
switch (a) {
    case { foo: i }: trace(i);
    default:
}
\end{lstlisting}

\emph{\tref{プロパティ}{class-field-property}:}

変数のクラスフィールドは、カスタムの読み込み書き込みアクセスを指定するプロパティが使えます。これにより、より良いアクセス制御が実現できます。

\begin{lstlisting}
public var color(get,set);
function get_color() {
    return element.style.backgroundColor;
}
function set_color(c:String) {
    trace('Setting background of element to $c');
    return element.style.backgroundColor = c;
}
\end{lstlisting}

\emph{\tref{アクセス制御}{lf-access-control}:}

Haxeでは、メタデータの構文を使って、クラスやフィールドに対してアクセスを許可したりこじ開けたりといったアクセス制御をを行うことできます。

\emph{\tref{型パラメータ、共変性、反変性}{type-system-type-parameters}:}

型には型パラメータをつけて、型付きのコンテナなど複雑なデータ構造を表現できます。型パラメータは特定の型への制限が可能で、また、変性のルールに従います。

\begin{lstlisting}
class Main<A> {
    static function main() {
        new Main<String>("foo");
        new Main(12); // use type inference
    }

    function new(a:A) { }
}
\end{lstlisting}

\section{条件付きコンパイル}
\label{lf-condition-compilation}

Haxeでは、\expr{\#if}、\expr{\#elseif}、\expr{\#else}を使って\emph{コンパイラフラグ}を確認することで条件付きコンパイルが可能です。

\define{コンパイラフラグ}{define-compiler-flag}{コンパイラフラグはコンパイルの過程に影響をあたえる、設定可能な値です。このフラグは\expr{-D key=value}あるいは単に\expr{-D key}(この場合デフォルト値の\expr{"1"}になる)の形式でコマンドラインから指定できます。そのほかにも、コンパイラはコンパイルの過程で別のステップへ情報伝達するために、内部的にいくつかのフラグを設定します。}

以下は条件付きコンパイラの利用例のデモです。

\haxe{assets/ConditionalCompilation.hx}

これをフラグ無しでコンパイルした場合、\expr{main}メソッドの\expr{trace("ok");}が実行されて終了します。他の分岐はファイルを構文解析する際に切り捨てられます。他の分岐についても、正しいHaxeの構文である必要がありますが、型チェックはされません。

\expr{\#if}と\expr{\#elseif}の直後の条件には以下の式が使えます。

\begin{itemize}
	\item すべての識別子は同名のコンパイラフラグの値で置きかえられます。コマンドラインから\expr{-D some-flag}を指定すると\expr{some-flag}と\expr{some\_flag}のフラグが定義されることに気を付けてください。
	\item \type{String}、\type{Int}、\type{Float}の定数値は直接使用されます。
		\item \type{Bool}の演算\expr{\&\&} (and)、\expr{||} (or)、\expr{!} (not) は期待どおりに動作しますが、式全体を小かっこでかこむ必要があります。
	\item \expr{==}、\expr{!=}、\expr{>}、\expr{>=}、\expr{<}、\expr{<=}の演算子が値の比較に使えます。
	\item 小かっこ\expr{()}は通常通り、式をグループ化するのに使えます。
\end{itemize}

Haxeの構文解析器は、\expr{some-flag}を一つの句として認識しません、\expr{some - flag}の2項演算として読み取ります。このような場合は、アンダースコアを使う\expr{some_flag}の版を使用する必要があります。

\paragraph{ビルトインのコンパイラフラグ}

ビルトインのコンパイラフラグの完全なリストは、Haxeコンパイラを\expr{--help-defines}の引数をつけて呼び出すことで手に入れることができます。Haxeのコンパイラは、コンパイルごとに複数の\expr{-D}フラグを指定できます。

\tref{コンパイラフラグ一覧}{lf-condition-compilation-flags}も確認してみてください。

\subsection{グローバルコンパイラフラグ}
\label{lf-condition-compilation-flags}

Haxe 3.0以降では\expr{haxe --help-defines}を実行することで、サポートしている\tref{コンパイラフラグ}{lf-condition-compilation}の一覧を取得することができます。

\begin{center}
\begin{tabular}{| l | l |}
	\hline
	\multicolumn{2}{|c|}{グローバルコンパイラフラグ} \\ \hline
	フラグ &  説明 \\ \hline
	\expr{absolute-path} &  \expr{trace}の出力を絶対パスで行います。 \\
	\expr{advanced-telemetry}  &  SWFをMonocleのツールで測定できるようにします。 \\
	\expr{analyzer}  &  静的解析器を使った最適化を行います(実験的)。 \\
	\expr{as3} &  flash9のas3のソースコードを出力する場合に定義されます。 \\
	\expr{check-xml-proxy}  &  xmlプロキシの使用済みフィールドを確認します。 \\
	\expr{core-api}  & core APIの文脈で定義されています。 \\
	\expr{core-api-serialize}  &  C\#で、いくつかのcore APIクラスをSerializable属性でマークします。 \\
	\expr{cppia}  &  実験的にC++のインストラクションアセンブリを出力します \\
	\expr{dce=<mode:std|full|no>}  &  \tref{デッドコード削除}{cr-dce}のモードを設定します(デフォルトではstd)。 \\
	\expr{dce-debug}  &  Show \tref{dead code elimination}{cr-dce} log \\
	\expr{debug}  &  \expr{-debug}をつけてコンパイルした場合に有効化されます。 \\
	\expr{display}  &  補完中に有効化されます。 \\
	\expr{dll-export}  &  実験的なリンクをつけてC++生成します。 \\
	\expr{dll-import}  &  実験的なリンクをつけてC++生成します。 \\
	\expr{doc-gen}  &  正しくドキュメントを生成するため、削除と変更をしないように振る舞います。 \\
	\expr{dump}  &  dumpサブディレクトリに、完全な型付け済みの抽象構文木を出力します。Haxeに似た形式で出力するには\expr{dump=pretty}を使ってください。 \\
	\expr{dump-dependencies}  &  dumpサブディレクトリに、クラスの依存関係を出力をします。 \\
	\expr{dump-ignore-var-ids}  &  prettyではないdumpから、変数IDを削除します。(diffを取るのに役立ちます) \\
	\expr{erase-generics}  &  C\#でジェネリッククラスを取り消します。 \\
	\expr{fdb}  &  FDBの対話的なデバッグのために、flashのデバッグ情報をすべて有効化します。 \\
	\expr{file-extension}  &  C++ソースコードで拡張子を出力します。 \\
	\expr{flash-strict}  &  Flash出力でより厳密な型付けを行います。 \\
	\expr{flash-use-stage}  &  SWFライブラリを初期のstageに保ちます。 \\
	\expr{force-lib-check}  &  コンパイラが-net-libと-java-libの追加クラスを確認するように強制します(内部用)。 \\
	\expr{force-native-property}  &  3.1の互換性のために、すべてプロパティに\expr{:nativeProperty}のタグ付けをします。 \\
	\expr{format-warning}  &  2.xの互換性のために、フォーマットされた文字列に対して警告を出します。 \\
	\expr{gencommon-debug}  &  GenCommonの内部用 \\
	\expr{haxe-boot}  &  flashのbootクラスに生成された名前の代わりに'haxe'という名前を使います。 \\
	\expr{haxe-ver}  &  現在のHaxeのバージョンの値です。 \\
	\expr{hxcpp-api-level}  &  hxcppのバージョン間の互換性を保ちます。 \\
	\expr{include-prefix}  &  含有している出力ファイルにパスを付加します。 \\
	\expr{interp}  &  \expr{--interp}でコンパイルされて実行される場合に定義されます。 \\
	\expr{java-ver=[version:5-7]}  &  ターゲットとするJavaのバージョンを設定します。 \\
	\expr{js-classic}  &  JavaScript出力にfunctionラッパーと、strictモードを使いません。 \\
	\expr{js-es5}  &  ES5に準拠した実行環境のためのJavaScriptを出力します。 \\
	\expr{js-unflatten}  &  packageや型でネストしたオブジェクトを出力します。 \\
	\expr{keep-old-output}  &  出力ディレクトリの古いコードのファイルを残します(C\#/Java)。 \\
	\expr{loop-unroll-max-cost}  & ループ展開がキャンセルされる最大コスト(expressions * iterations、デフォルトでは250)。 \\
	\expr{macro} & \tref{マクロの文脈}{macro}でコンパイルされた場合に定義されます。 \\
	\expr{macro-times} & \expr{--times}と一緒に使用された場合にマクロごとの時間を表示します。 \\
	\expr{net-ver=<version:20-45>}  &  ターゲットとする.NETのバージョンを設定します。 \\
	\expr{net-target=<name>}  &  .NETのターゲット名を設定します。xbox、micro \_(Micro Framework)\_、compact \_(Compact Framework)\_が、正当な名前です。 \\
	\expr{neko-source} & Nekoのバイトコードではなくソースコードを出力します。 \\
	\expr{neko-v1} &  Nekoの1.xとの互換性を保ちます。 \\
	\expr{network-sandbox}  &  ローカルファイルアクセスの代わりに、ローカルネットワークサンドボックスを使います。 \\
	\expr{no-compilation}  &  C++の最終コンパイルを無効化します。 \\
	\expr{no-copt}  &  コンパイル時の最適化を無効化します \_(デバッグ用途)\_ \\
	\expr{no-debug}  &  C++出力からすべてのデバッグマクロを取り除きます。 \\
	\expr{no-deprecation-warnings} & \expr{@:deprecated}のフィールドが使われたことによる警告を無効化します。
	\expr{no-flash-override}  &  flashのみで、いくつかの基本クラスでのoverrideをHXサフィックスのついたメソッドで代替します。 \\
	\expr{no-opt}  &  最適化を無効化します。 \\
	\expr{no-pattern-matching}  &  パターンマッチングを無効化します。 \\
	\expr{no-inline}  &  \tref{インライン化}{class-field-inline}を無効化します。 \\
	\expr{no-root}  &  GenCSの内部用 \\
	\expr{no-macro-cache}  &  マクロの文脈でのキャッシュを無効化します。 \\
	\expr{no-simplify}  &  簡易化のフィルタを無効化します。 \\
	\expr{no-swf-compress}  &  SWF出力の圧縮を無効化します。 \\
	\expr{no-traces}  &  すべての\expr{trace}呼び出しを無効化します。 \\
	\expr{php-prefix}  &  \expr{--php-prefix}をつけてコンパイルした場合です。 \\
	\expr{real-position}  &  C\#ターゲットで、Haxeのソースマップを無効化します。 \\
	\expr{replace-files}  &  GenCommonの内部用です。 \\
	\expr{scriptable}  &  GenCPPの内部用です。 \\
	\expr{shallow-expose}  &  Haxeが生成したクロージャのスコープについて、windowオブジェクトの記述なしでのアクセスを許可します。 \\
	\expr{source-map-content}  &  JSのソースマップの一部として、.hxのソースコードを含ませます。 \\
	\expr{swc}  &  SWFの代わりにSWCを出力します。 \\
	\expr{swf-compress-level=<level:1-9>}  &  SWF出力の圧縮レベルを指定します。 \\
	\expr{swf-debug-password=<yourPassword>}  &  デバッグ用のパスワードを指定します。このパスワードはMD5アルゴリズムを使って暗号化されて、swfをデバッグするための認証解除を防ぎます。-D fdbを指定しない場合パスワードは使われません。 \\
	\expr{swf-direct-blit}  &  グラフィックの転送をするのにハードウェアアクセラレーションを使います。 \\
	\expr{swf-gpu}  &  グラフィックを描画するのにGPUを使います。 \\
	\expr{swf-metadata=<file.xml>}  &  swf内に\expr{<file.xml>}をメタデータとして埋め込みます。 \\
	\expr{swf-preloader-frame}  &  SWFの最初に空白フレームを挿入します。\expr{-D flash-use-stage}、\expr{-swf-lib}と合わせて使います。 \\
	\expr{swf-protected}  &  Haxeのprivateを、SWF内でpublicではなくprotectedを使うようにコンパイルします。 \\
	\expr{swf-script-timeout}  &  ActionScriptがタイムアウトのダイアログを表示するまでの最大時間を設定します(秒数で)。 \\
	\expr{swf-use-doabc}  &  DoAbcDefineのswfタグの代わりにDoAbcを使います。 \\
	\expr{sys}  &  システムのすべてのプラットフォームで定義されています。 \\
	\expr{unsafe}  &  C\#ターゲットでunsafeのコードを許容します \\
	\expr{use-nekoc}  &  内部のものの代わりにnekocのコンパイラを使います。 \\
	\expr{use-rtti-doc}  &  コンパイル中にドキュメントにアクセスできるようにします。 \\
	\expr{vcproj}  &  GenCPPの内部用。 \\
\end{tabular}
\end{center}

\section{extern}
\label{lf-externs}

externは、ターゲット固有の連携を型安全のルールに従って記述するために使います。宣言は普通のクラスに似た形で、以下の要素が必要です。

\begin{itemize}
	\item \expr{class}キーワードの前に\expr{extern}キーワードを置きます。
	\item \tref{メソッド}{class-field-method}は式を持ちません。
	\item すべての引数と戻り値の型を明示します。
\end{itemize}

\tref{Haxe標準ライブラリ}{std}の\type{Math}クラスがちょうどいい例です。その一部を抜粋します。

\begin{lstlisting}
extern class Math
{
	static var PI(default,null) : Float;
	static function floor(v:Float):Int;
}
\end{lstlisting}

\expr{extern}が、メソッドと変数の両方を定義できることがわかります(実際のところ、\expr{PI}は読み込み専用の\tref{プロパティ}{class-field-property})を定義しています。一度この情報がコンパイラで利用可能になると、型がわかり、フィールドへのアクセスが出来るようになります。

\haxe{assets/Extern.hx}

\expr{floor}メソッドの戻り値が\type{Int}して定義されているため、このように動作します。

Haxe標準ライブラリは、多くの\expr{extern}を\target{Flash}、\target{JavaScript}ターゲット用にもっています。これにより、ネイティブのAPIに型安全のルールに従ってアクセス可能にし、より高いレベルのAPI設計の助けになります。\tref{haxelib}{haxelib}でも、多くのネイティブのライブラリの\expr{extern}を入手できます。

\target{Flash}、\target{Java}、\target{C\#}ターゲットでは、\tref{コマンドライン}{compiler-usage}から直接ネイティブライブラリの取り込みを行うことができます。ターゲットごとの詳細は\Fullref{target-details}のそれぞれの節で説明されています。

\target{Python}や、\target{JavaScript}といったターゲットでは、\expr{extern}クラスをネイティブのモジュールから読み込むために追加の「インポート」が必要になる場合があります。Haxeはそのような依存関係を宣言する仕組みを提供しているので、それらを\Fullref{target-details}のそれぞれの節で説明します。

\paragraph{可変長引数と、型の選択肢}
\since{3.2.0}

haxe.externパッケージは、ネイティブの概念をHaxeに対応させるため、2つの型を提供しています。

\begin{description}
	\item[\type{Rest<T>}:] この型は関数の最後の引数として使って、可変長の引数を追加で渡すことを可能にします。型パラメータは引数を特定の型に制限するのに使います。
	\item[\type{EitherType<T1,T2>}:] この型はパラメータのどちらかの型を使うことができるようにする。つまり、型の選択肢を表現できます。3つ以上の型を選ばせたい場合はネストさせて使います。
\end{description}

以下にデモを用意しました。

\haxe{assets/RestAndEitherType.hx}


\section{静的拡張}
\label{lf-static-extension}

\define{静的拡張}{define-static-extension}{静的拡張は、すでに存在している型に対して元のソースコードを変更することなく、見せかけの拡張を行います。Haxeの静的拡張は、最初の引数が拡張する対象の型である静的メソッドを宣言して、それ\expr{using}を使って記述しているクラス内に持ちこむことで使用できます。}

静的拡張は、実際に型の変更を行うことなく型を強化する強力なツールです。以下の例で、その使い方を実演します。

\haxe{assets/StaticExtension.hx}

\type{Int}がもともと\expr{triple}メソッドを持っていないのは明らかですが、このプログラムは期待通り\expr{36}を出力します。\expr{12.triple()}の呼び出しが\expr{IntExtender.triple(12)}に変形されるためです。これには必要な条件が3つあります。

\begin{enumerate}
	\item 定数値の\expr{12}と、\expr{triple}の最初の引数の型がともに\type{Int}である
	\item \type{IntExtender}クラスが\expr{using Main.IntExtender}を使って現在の文脈に読み込まれている。
	\item \type{Int}自身は\expr{triple}フィールドを持っていない(持っていた場合、静的拡張よりも高い優先度になる)。
\end{enumerate}

静的拡張は糖衣構文であると認識されており、実際そのとおりですが、コードの可読性に大きな影響を与えることには注目する価値がありますあ。\expr{f1(f2(f3(f4(x))))}の形のネストされた呼び出しの代わりに、\expr{x.f4().f3().f2().f1()}のチェーンの形での呼び出しが可能になります。

優先順位のルールは\Fullref{type-system-resolution-order}で、すでに説明されているとおり、\expr{using}式が複数ある場合は下から上へと確認がされ、各モジュールでは各型のフィールドが上から下へと確認がされます。モジュールを静的拡張として\expr{using}すると、そのすべての型が現在の文脈にインポートされます(モジュール内の特定の型の場合とは対照的です。詳しくは、\Fullref{type-system-modules-and-paths}を見てください)。

\subsection{Haxe標準ライブラリについて}
\label{lf-static-extension-in-std}

Haxeの標準ライブラリのいくつかのクラスを、静的拡張の用途に合うようになっています。次の例からは\type{StringTools}の使い方がわかります。

\haxe{assets/StaticExtension2.hx}

\type{String}自身は\expr{replace}を持っていませんが、\expr{using StringTools}の静的拡張によって提供されます。いつものように、\target{JavaScript}への変換を見るとよくわかります。

\begin{lstlisting}
Main.main = function() {
	StringTools.replace("adc","d","b");
}
\end{lstlisting}

Haxe標準ライブラリでは以下のクラスが静的拡張として使うように設計されています。

\begin{description}
	\item[\type{StringTools}:] 置換やトリミングといった、文字列に対する拡張を提供します。
	\item[\type{Lambda}:] \type{Iterable}に対する関数型のメソッドを提供します。 
	\item[\type{haxe.EnumTools}:] \expr{enum}とそのインスタンスに対して型の情報を得る機能を提供します。
	\item[\type{haxe.macro.Tools}:] マクロをあつかう際のさまざまな拡張を提供します(詳しくは、\Fullref{macro-tools})。
\end{description}

\trivia{``using'' using}{\expr{using}キーワードが追加されて以降、\expr{using}を使う(using using)ときの問題や、その影響についての会話がよくされるようになりました。"using using"のせいでさまざまな場面でわかりにくい英語が生まれたため、このマニュアルの著者はこの機能をその正体から静的拡張と呼ぶことに決めました。}

\section{パターンマッチング}
\label{lf-pattern-matching}
\state{NoContent}

\subsection{導入}
\label{lf-pattern-matching-introduction}

パターンマッチングは、与えられた、おそらく深いパターンに一致する値に応じて分岐する処理のことです。Haxeでは、すべてのパターンマッチングは\tref{\expr{switch}式}{expression-switch}の個々の\expr{case}式が書き表すパターンに従って行われます。それでは以下のデータ構造を使って、さまざまなパターンの構文を見ていきましょう。

\haxe[firstline=1,lastline=4]{assets/PatternMatching.hx}

以下は、パターンマッチングの基本事項です。

\begin{itemize}
	\item パターンは上から下へとマッチングされます。
	\item 入力値にマッチする最上位のパターンが持っている式が実行されます。
	\item \expr{_}はすべてにマッチします。このため、\expr{case _:}は\expr{default:}と同じです。
\end{itemize}

\subsection{enumマッチング}
\label{lf-pattern-matching-enums}

enumはそのコンストラクタを自然な方法でマッチングすることができます。

\haxe[firstline=8,lastline=21]{assets/PatternMatching.hx}

パターンマッチングでは、ケースを上から下へと確認していき、入力値とマッチする最初のものを見つけ出します。以下の各ケースの実行の流れについての説明は、その過程を理解するのに役に立ちます。

\begin{description}
	\item[\expr{case Leaf(_)}:] \expr{myTree}は\expr{Node}なので、マッチングに失敗します。
	\item[\expr{case Node(_, Leaf(_))}:] \expr{myTree}の右側の子要素は\expr{Leaf}ではなく、\expr{Node}なので失敗します。
	\item[\expr{case Node(_, Node(Leaf("bar"), _))}:] マッチングに成功します。
	\item[\expr{case _}:] 前のケースでマッチングが成功したので確認が行われません。
\end{description}

\subsection{変数の取り出し}
\label{lf-pattern-matching-variable-capture}

パターンの一部のあらゆる値は、識別子を使ってマッチングさせて取り出すことができます。

\haxe[firstline=24,lastline=30]{assets/PatternMatching.hx}

これは、以下の流れにしたがって\expr{return}を行います。

\begin{itemize}
	\item \expr{myTree}が\expr{Leaf}の場合、その名前が返る。
	\item \expr{myTree}が\expr{Node}でその左の子要素が\expr{Leaf}の場合、その名前が返る(上の例の場合、これが適用されて\expr{"foo"}が返る)。
	\item そのほかの場合、\expr{"none"}が返る。
\end{itemize}

マッチされた値を取り出すのに\expr{=}を使うこともできます。

\haxe[firstline=32,lastline=36]{assets/PatternMatching.hx}

\expr{leafNode}には\expr{Leaf("foo")}が割り当てられているので、これにマッチします。そのほかのケースでは、\expr{myTree}自身が返ります。\expr{case x}は\expr{case _}と同じようにすべてにマッチしますが、\expr{x}のような識別子が使われるとマッチした値がその変数に対して割り当てられます。

\subsection{構造体マッチング}
\label{lf-pattern-matching-structure}

匿名構造体とインスタンスのフィールドに対してマッチさせることも可能です。

\haxe[firstline=38,lastline=50]{assets/PatternMatching.hx}

2番目のケースでは、\expr{rating}が\expr{"awesome"}にマッチすると、\expr{name}フィールドが識別子\expr{n}に割り当てられます。もちろん、この構造体を先の例のTreeに入れて、構造体と\expr{enum}を合わせたマッチングを行うこともできます。

クラスインスタンスについては、その親クラスのフィールドについてはマッチングできないという制限があります。

\subsection{配列マッチング}
\label{lf-pattern-matching-array}

配列は固定長のマッチングを行うことができます。

\haxe[firstline=52,lastline=60]{assets/PatternMatching.hx}

この例では、\expr{array[1]}が\expr{6}にマッチし、\expr{array[0]}は何でもよいので、\expr{1}が出力されます。

\subsection{orパターン}
\label{lf-pattern-matching-or}

\expr{|}演算子は、複数のパターンが許容されることを示す用途で、パターン内のあらゆる箇所に使うことができます。

\haxe[firstline=63,lastline=68]{assets/PatternMatching.hx}

orパターン内で変数の取得をしたい場合、その子要素両方で行わなくてはいけません。

\subsection{ガード}
\label{lf-pattern-matching-guards}

\expr{case ... if(condition):}の構文を使ってパターンをさらに限定することができます。

\haxe[firstline=71,lastline=79]{assets/PatternMatching.hx}

最初のケースは追加のガード条件\expr{if (b > a)}を持っています。このケースはこの条件が正だった場合のみ選択され、それ以外の場合は次のケースとのマッチングが続きます。

\subsection{複数の値のマッチング}
\label{lf-pattern-matching-tuples}

配列の構文は複数の値のマッチングにも使えます。

\haxe[firstline=82,lastline=87]{assets/PatternMatching.hx}

これは通常の配列のマッチングによく似ていますが、以下の点で違います。

\begin{itemize}
	\item 要素数は固定です。このためパターンの配列の長さが違ってはいけません。
	\item switchしている値を取得できません。例えば、\expr{case x}は使えません(\expr{case x}は使えます)。
\end{itemize}

\subsection{抽出子(エクストラクタ)}
\label{lf-pattern-matching-extractors}
\since{3.1.0}

抽出子(エクストラクタ)はマッチした値に変更を適用することができます。マッチした値に小さな変更を適用して、さらにマッチングを行う場合に便利です。

\haxe{assets/Extractor2.hx}

この場合、\expr{TString}列挙型コンストラクタの引数の値を、\expr{temp}に割り当てて、さらにネストした\expr{temp.toLowerCase()}に対する\expr{switch}を行っています。見てのとおり、\expr{TString}が\expr{"foo"}のケース違いのものを持っているので、このマッチングは成功します。これは抽出子を使うことで簡略化できます。

\haxe{assets/Extractor.hx}

抽出子は\expr{extractorExpression => match}の式によって認識されます。コンパイラはその前の例と同じようなコードを出力しますが、記述する構文はずいぶんと簡略化されました。抽出子は\expr{=>}で分断される以下の2つの部品からなります。

\begin{enumerate}
\item 左側はあらゆる式が可能で、アンダースコア\expr{_}が出現する箇所すべてが、現在マッチする値で置き換えられます。
\item 右側は左側を評価した結果をマッチングするためのパターンです。
\end{enumerate}

右側はパターンですから、さらに別の抽出子を使うことが可能です。以下の例では2つの抽出子をチェーンさせています。

\haxe{assets/Extractor4.hx}

これは\expr{3}がマッチして\expr{add(3, 1)}を呼び出し、その結果の\expr{4}がマッチして\expr{mul(4, 3)}呼び出された結果として、\expr{12}が出力されます。2つ目の\expr{=>}の右側の\expr{a}は、\tref{変数取り出し}{lf-pattern-matching-variable-capture}であることに注意してください。

現在は\tref{orパターン}{lf-pattern-matching-or}内で抽出子を使うことはできません。

\haxe{assets/Extractor5.hx}

しかし、orパターンを抽出子の右側に使うことはできます。そのため、上の例は小かっこ無しの場合ではコンパイル可能です。

\subsection{網羅性のチェック}
\label{lf-pattern-matching-exhaustiveness}

コンパイラは、起こりうるケースが忘れ去られてないかのチェックを行います。

\begin{lstlisting}
switch(true) {
    case false:
} // Unmatched patterns: true
\end{lstlisting}

マッチング対象の\type{Bool}型は、\expr{true}と\expr{false}の2つの値を取り得ますが、\expr{false}のみがチェックされています。

\todo{Figure out wtf our rules are now for when this is checked.}

\subsection{無意味なパターンのチェック}
\label{lf-pattern-matching-unused}

同じように、コンパイラはどのような入力値に対してもマッチしないパターンを禁止します。

\begin{lstlisting}
switch(Leaf("foo")) {
    case Leaf(_)
       | Leaf("foo"): // This pattern is unused
    case Node(l,r):
    case _: // This pattern is unused
}
\end{lstlisting}

\section{文字列補間}
\label{lf-string-interpolation}

Haxe3では、\emph{文字列補間}のおかげで、手動で文字列をつなげ合わせる必要がなくなりました。シングルクオート\expr{'}で囲まれた文字列の中で、ドル記号\expr{\$}に続けて識別子を記述すると、その識別子を評価してつなげ合わせてくれます。

\begin{lstlisting}
var x = 12;
// The value of x is 12
trace('The value of x is $x');
\end{lstlisting}

さらに、\expr{\$$\left\{expr\right\}$}を使うことで文字列内に式そのものを含めることが可能になります。この\expr{expr}は、Haxeの正当な式であれば、なんでもかまいません。

\begin{lstlisting}
var x = 12;
// The sum of 12 and 3 is 15
trace('The sum of $x and 3 is ${x + 3}');
\end{lstlisting}

文字列補間はコンパイル時の機能なので、実行時には影響を与えません。上の例は、手動のつなげ合わせと同じです。コンパイラは以下と同様のコードを生成します。

\begin{lstlisting}
trace("The sum of " + x +
  " and 3 is " + (x + 3));
\end{lstlisting}

もちろん、一切の補間なしでシングルクオートで囲んだ文字列を使用することができますが、\$の文字が補間の引き金として予約されてしまっていることに気を付けてください。文字列内で、ドル記号そのものを使いたい場合は\expr{\$\$}で使えます。

\trivia{Haxe3以前の文字列補間}{文字列補間自体はバージョン2.09からHaxeの機能として存在しています。そのころは、\expr{Std.format}のマクロが使われいました。これは新しい文字列補間の構文よりも、遅くてあまり快適でないものでした。}

\section{配列内包表記}
\label{lf-array-comprehension}

\todo{Comprehensions are only listing Arrays, not Maps}

Haxeの配列内包表記は、既存の構文を配列の初期化をより簡単にするためにも使えるようにするものです。配列内包表記は、\expr{for}または\expr{while}のキーワードによって識別されます。

\haxe{assets/ArrayComprehension.hx}

変数\expr{a}は、0から9までの数値を要素として持つ配列として初期化されます。コンパイラはループを作ってその繰り返しの一つ一つで要素を追加するコードを出力します。つまり以下のコードと等価です。

\begin{lstlisting}
var a = [];
for (i in 0...10) a.push(i);
\end{lstlisting}

変数\expr{b}も同じ値に初期化されますが、\expr{for}ではなく\expr{while}という異なる内包表記の形式を使っています。そして、これは以下のコードと等価です。

ループの式は、条件分岐やループのネストを含めて、いかなる式でもかまいません。ですから、以下の式は期待通りに動作します。

\haxe{assets/AdvArrayComprehension.hx}

\section{イテレータ(反復子)}
\label{lf-iterators}

Haxeでは、カスタムのイテレータや反復可能(iterable)なデータ型を簡単に定義できます。これらの概念は、\type{Iterator<T>}型と\type{Iterable<T>}型を使って以下のように表現されています。

\begin{lstlisting}
typedef Iterator<T> = {
	function hasNext() : Bool;
	function next() : T;
}

typedef Iterable<T> = {
	function iterator() : Iterator<T>;
}
\end{lstlisting}

これらの型のいずれかで\tref{構造的に単一化できる}{type-system-structural-subtyping}あらゆる\tref{class}{types-class-instance}は、\tref{forループ}{expression-for}で反復処理を行うことができます。つまり、型が合うように\expr{hasNext}と\expr{next}メソッドを定義すればそのクラスはイテレータであるし、\type{Iterator<T>}を返す\expr{iterator}メソッドを定義すれば反復可能な型です。

\haxe{assets/Iterator.hx}

この例での\type{MyStringIterator}は、\type{Bool}型を返す\expr{hasNext}と\type{String}型を返す\expr{next}メソッドを定義しているので、イテレータであると見なされます。また\expr{next}の戻り値の型から、これは\type{Iterator<String>}です。\expr{main}メソッドでこれをインスタンス化して反復処理を行っています。

\haxe{assets/Iterable.hx}


Here we do not setup a full iterator like in the previous example, but instead define that the \type{MyArrayWrap<T>} has a method \expr{iterator}, effectively forwarding the iterator method of the wrapped \type{Array<T>} type. 


\section{関数の束縛(bind)}
\label{lf-function-bindings}

Haxe3では、部分的に引数を適用して関数を束縛することが可能です。すべての関数型は\expr{bind}フィールドを持っており、これを呼び出すことで引数の数を減らした新しい関数を作りだすことができます。その実例を示します。

\haxe{assets/Bind.hx}

行4では、\expr{map.set}関数に2番目の引数に\expr{12}を適用し、\expr{f}という変数に割り当てました。アンダースコア\expr{_}はその引数を束縛しないことを表すのに使います。このことは\expr{map.set}と、\expr{f}の型の比較でもわかります。束縛された\type{String}型の引数が取り除かれたので、\expr{Int->String->Void}型が\expr{Int->Void}型に変わっています。

\expr{f(1)}を呼び出したことで実際には\expr{map.set(1, "12")}が実行され、\expr{f(2)}、\expr{f(3)}の呼び出しでも同じ関係性が成り立ちます。最後の行で、3つのインデックスすべてに紐づく値が\expr{"12"}になっていることが確認できます。

アンダースコア\expr{_}は、末尾の引数では省略することができます。つまり、\expr{map.set.bind(1)}で最初の引数を束縛した場合、インデックス\expr{1}について新しい値を設定する\expr{String->Void}関数が提供されます。

\trivia{コールバック}{Haxe3よりも前のバージョンでは、\expr{callback}キーワードに1つの関数の引数と任意の個数の束縛する引数をつけて呼び出しをしていました。この束縛する機能に対してコールバック関数という名前が使われるようになっていました。\\
\expr{callback}は、左から右への束縛のみでアンダースコア\expr{_}はサポートしていませんでした。アンダースコアを使うという選択肢は論争を生み、そのほかの案もいくつか現れましたがこれより優れているものはありませんでした。少なくともアンダースコア\expr{_}は、「ここに値を入れて」と言っているように見えるので、この意味を書き表すのに適しているという結論にいたりました。}

\section{メタデータ}
\label{lf-metadata}

以下の要素は、メタデータで属性をつけることができます。

\begin{itemize}
	\item \expr{class}、\expr{enum}の定義
	\item クラスフィールド
	\item 列挙型コンストラクタ
	\item 式
\end{itemize}

これらのメタデータの情報は、\type{haxe.rtti.Meta}APIを使って実行時に利用することが可能です。

\haxe{assets/Meta.hx}

メタデータは\expr{@}の文字で始まり、メタデータの名前が続き、その後にオプションでカンマで区切った定数値の引数が小かっこで囲まれている、ということで簡単に識別できます。

\begin{itemize}
	\item \type{MyClass}クラスは\expr{"Nicolas"}という文字列の引数1つを持つ\expr{author}メタデータと、引数を持たない\expr{debug}メタデータを持ちます。
	\item メンバ変数\expr{value}は、\expr{1}と\expr{8}の2つの整数の引数を持つ\expr{range}メタデータを持ちます。
	\item 静的メソッド\expr{method}は引数なしの\expr{broken}メタデータと、引数なしの\expr{:noCompletion}メタデータを持ちます。
\end{itemize}

\expr{main}メソッドでは、APIを通してこれらのメタデータへアクセスしています。この出力からは取得可能なデータの構造が分かります。

\begin{itemize}
	\item 各メタデータについてフィールドがあり、フィールドの名前はメタデータの名前です。
	\item フィールドの値はメタデータの引数に一致します。引数がない場合、フィールドの値は\expr{null}です。その他の場合、フィールドの値は引数1つが要素1つになった配列です。
	\item \expr{:}から始まるメタデータは省略されます。このメタデータは、\emph{コンパイラメタデータ}として知られます。
\end{itemize}

メタデータの引数の値は以下が使用できます。

\begin{itemize}
	\item \tref{定数値}{expression-constants}
	\item \tref{配列の宣言}{expression-array-declaration} (すべての要素がこのリストのいずれか)
	\item \tref{オブジェクトの宣言}{expression-object-declaration} (すべての要素がこのリストのいずれか)
\end{itemize}

\paragraph{ビルトインのコンパイラメタデータ}
コマンドラインから\expr{haxe --help-metas}を実行することで、定義済みメタデータの完全なリストを得ることができます。

詳しくは\tref{コンパイラメタデータのリスト}{cr-metadata}を見てください。

\section{アクセス制御}
\label{lf-access-control}

基本的な\tref{可視性}{class-field-visibility}のオプションで十分でない場合、アクセス制御が役に立ちます。アクセス制御は\emph{クラスレベル}と\emph{フィールドレベル}、そして以下の2方向の適用が可能です。

\begin{description}
	\item[アクセス許可:] \expr{:allow(target)}\tref{メタデータ}{lf-metadata}を使うことで、対象を与えられたクラスやフィールドからのアクセスを許容するようにします。
	\item[アクセス強制:] \expr{:access(target)}\tref{メタデータ}{lf-metadata}を使うことで、対象からの与えられたクラスやフィールドへのアクセスを強制的に可能にします。
\end{description}

このとき、\expr{target}には以下の\tref{ドットパス}{define-type-path}を使うことができます。

\begin{itemize}
	\item \emph{クラスフィールド}
	\item \emph{クラス}、\emph{抽象型}
	\item \emph{パッケージ}
\end{itemize}

\expr{target}はインポートを参照しません。つまり、完全なパスを正しく記述する必要があります。

クラスや抽象型の場合、アクセスの変更はその型のすべてのフィールドに反映されます。同じように、パッケージの場合、アクセスの変更はそのパッケージ内のすべての型のすべてのフィールドに反映されます。

\haxe{assets/ACL.hx}

\expr{MyClass.foo}は、\type{MyClass}に\expr{@:allow(Main)}を適用しているので、\expr{main}メソッドからアクセスできます。このコードは、\expr{@:allow(Main.main)}でも動作しますし、以下のように\type{MyClass}クラスの\expr{foo}フィールドにメタデータをつけても動作します。

\haxe{assets/ACL2.hx}

もし型にこのようなアクセスの変更ができない場合は、アクセス強制の方法が役立つかもしれません。

\haxe{assets/ACL3.hx}

\expr{@:access(MyClass.foo)}のメタデータは、\expr{main}メッソドからの\expr{foo}の可視性を変更します。

\trivia{メタデータという選択肢}{アクセス制御の言語機能には、新しい構文の導入ではなく、Haxeのメタデータの構文を使いました。これには以下のいくつかの理由があります。
\begin{itemize}
	\item 追加の構文は言語の構文解析を複雑にして、さらにはキーワードを増やしてしまします。
	\item 追加の構文は、言語のユーザーに追加の学習を要求します。メタデータであれば、それは既知のものです。
	\item メタデータは、この拡張を行うのに十分な表現力を持っています。
	\item メタデータはHaxeのマクロから、アクセスし、生成し、編集することが可能です。
\end{itemize}
もちろん、メタデータ構文の主な不利益はメタデータの名前、クラスやパッケージ名についてスペルミス(例えば、@:acesss)をした場合に何のエラーも出ないことです。しかし、この機能では実際に\expr{private}フィールドにアクセスしようとした場合にエラーがでるので、エラーが沈黙しているということにはなりえません。}

\since{3.1.0}

アクセスが\tref{インターフェース}{types-interfaces}に対して許可される場合、そのインターフェースを実装しているすべてのクラスに対してそれが引き継がれます。

\haxe{assets/ACL4.hx}

これは親クラスの場合も同様です。その場合、子クラスに対して引き継ぎがされます。

\trivia{壊れた機能}{子クラスや実装クラスへのアクセスの継承は、Haxe3.0への導入を予定されており、そしてドキュメントまでも作られていました。しかし、このマニュアルを作る過程でこのアクセス制御の実装がぬけ落ちていることを発見しました。}

\section{インラインコンストラクタ}
\label{lf-inline-constructor}
\since{3.1.0}

コンストラクタに、\tref{inline}{class-field-inline}の宣言をつけると、コンパイラは特定の場合において最適化を試みます。この最適化が動作するためにはいくつかの必要事項があります。

\begin{itemize}
	\item コンストラクタの呼び出しの結果はローカル変数への直接の代入でなければいけない。
	\item コンストラクタフィールドの式は、そのフィールドへの代入のみでなければならない。
\end{itemize}

以下に、コンストラクタのインライン化の実例を挙げます。

\haxe{assets/NewInline.hx}

\target{JavaScript}出力をみると、その効果がわかります。

\begin{lstlisting}
Main.main = function() {
	var pt_x = 1.2;
	var pt_y = 9.3;
};
\end{lstlisting}


\part{Compiler Reference}
\chapter{コンパイラの使い方}
\label{compiler-usage}

\paragraph{基本的な使い方}

Haxeコンパイラは基本的にはコマンドラインから以下の2つの質問に答える引数をつけて呼び出します。

\begin{itemize}
	\item 何をコンパイルするのか?
	\item 何を出力するのか?
\end{itemize}

最初の質問に答えるためには、\ic{-cp path}引数でクラスパスを指定し、\ic{-main dot_path}引数でコンパイル対象のメインクラスを指定すれば十分です。これでHaxeコンパイラはメインクラスのファイルを解決しコンパイルを始めます。

2つ目の質問に答えるためには、目的のターゲット特有の引数を指定します。Haxeターゲットはそれぞれ専用のコマンドラインオプションを持っています。例えば、\target{JavaScript}は\ic{-js file_name}、\target{PHP}は\ic{-php directory}です。ターゲットによってファイル名を指定するもの(\ic{-js}、\ic{-swf}、\ic{-neko}、\ic{-python}が該当)と、ディレクトリを指定するものがあります。

\paragraph{よく使う引数}

\emph{入力:}

\begin{description}
	\item[\ic{-cp path}] \ic{.hx}のソースファイルまたはパッケージ(サブディレクトリ)が置かれているディレクトリのパスを追加します。
	\item[\ic{-lib library_name}] \Fullref{haxelib}のライブラリを追加します。
	\item[\ic{-main dot_path}] メインクラスを設定します。
\end{description}

\emph{出力:}

\begin{description}
	\item[\ic{-js file_name}] 指定されたファイルに\tref{JavaScript}{target-javascript}のソースコードを出力します。
	\item[\ic{-as3 directory}] 指定されたディレクトリにActionScript3のソースコードを出力します。
	\item[\ic{-swf file_name}] 指定されたファイルに\tref{Flash}{target-flash}の.swfを出力します。
	\item[\ic{-neko file_name}] 指定されたファイルに\tref{Neko}{target-neko}のバイナリを出力します。
	\item[\ic{-php directory}] 指定されたディレクトリに\tref{PHP}{target-php}のソースコードを出力します。
	\item[\ic{-cpp directory}] 指定されたディレクトリに\tref{C++}{target-cpp}のソースコードを出力して、ネイティブのC++コンパイラでコンパイルします。
	\item[\ic{-cs directory}] 指定されたディレクトリに\tref{C\#}{target-cs}のソースコードを出力します。
	\item[\ic{-java directory}] 指定されたディレクトリに\tref{Java}{target-java}のソースコードを出力して、Javaコンパイラでコンパイルします。
	\item[\ic{-python file_name}] 指定されたファイルに\tref{Python}{target-python}のソースコードを出力します。
\end{description}

\chapter{コンパイラの機能}
\label{cr-features}
\state{NoContent}

\section{ビルトインのコンパイラメタデータ}
\label{cr-metadata}

Haxe 3.0以降では、\expr{haxe --help-metas}を実行することで定義済みのコンパイラメタデータのリストを得ることができます。

\begin{center}
\begin{tabular}{| l | l | l |}
	\hline
	\multicolumn{3}{|c|}{グローバルメタデータ} \\ \hline
	メタデータ &  説明  &  プラットフォーム \\ \hline
	@:abi & ABI呼び出し規約を使う  & cpp \\
	@:abstract &  基底クラスの実装を\tref{抽象型}{types-abstract}として使います。  &  cs  java \\
	@:access \_(Target path)\_  &   型またはフィールドへのプライベートなアクセスを強制する。詳しくは\tref{アクセス制御}{lf-access-control}  &  all \\
	@:allow \_(Target path)\_  &   型またはフィールドからのプライベートなアクセスを許可する。詳しくは\tref{アクセス制御}{lf-access-control}  &  all \\
	@:analyzer & 静的アナライザを設定する  &  all \\
	@:annotation  &  Annotation (\expr{@interface}) definitions on \expr{-java-lib} imports will be annotated with this metadata. Haxeでコンパイルした型には影響ありません   &  java \\
	@:arrayAccess  &  抽象型への\tref{配列アクセス}{types-abstract-array-access}を許可する  &  all \\
	@:autoBuild \_(Build macro call)\_  &   \expr{@:build}メタデータをその型のすべての子クラス、実装クラスに反映する。詳しくは\tref{autobuildマクロ}{macro-auto-build}  &  all \\
	@:bind  &  SWFライブラリで定義されているクラスを上書きします  &  flash \\
	@:bitmap \_(Bitmap file path)\_  &  与えられたビットマップデータをクラスに埋め込む(\expr{flash.display.BitmapData}の継承が必要)   &  flash \\
	@:bridgeProperties  &  クラスのすべてのHaxeプロパティに、ネイティブのプロパティのブリッジを用意する  &  cs \\
	@:build \_(Build macro call)\_  &   マクロからクラスまたは列挙型を構築する。詳しくは\tref{型構築}{macro-type-building}  &  all \\
	@:buildXml  &  Build.xmlに挿入するxmlデータを指定する。  &  cpp \\
	@:callable  &  抽象型で基底型の関数呼び出しのアクセスをできるようにする。  &  all \\
	@:classCode  &  クラスにプラットフォームネイティブのコードを挿入する  &  cs  java \\
	@:commutative  &  抽象型の2項演算子の各項を交換可能にする  &  all \\
	@:compilerGenerated  &  コンパイラから生成されたフィールドをマークする。ユーザーは使用すべきでは無い。  &  cs  java \\
	@:coreApi &  このクラスをコアAPIのクラスとして認識する。(APIチェックを強制する)  &  all \\
	@:coreType  &  抽象型を\tref{コアタイプ}{types-abstract-core-type}として認識する。そのため、型には実装があってはいけない。  &  all \\
	@:cppFileCode  &  C++の出力ファイルに挿入するコード  &  cpp \\
	@:cppInclude  &  C++の出力ファイルに含めるファイル  &  cpp \\
	@:cppNamespaceCode  &    &  cpp \\
	@:dce  &  \expr{-dce full}が指定されていない場合でも、\tref{デッドコード削除}{cr-dce}を強制する  &  all \\
	@:debug  &  \expr{-debug}が指定されていない場合でも、出力するSWFにデバッグ情報を強制する   &  flash \\
	@:decl   &     &  cpp \\
	@:defParam  &    &  all \\
	@:delegate  &  \expr{-net-lib}のデリゲートで自動的に追加される   &  cs \\
	@:depend  &     &  cpp \\
	@:deprecated   &  \expr{-java-lib}の\expr{@Deprecated}で修飾されているフィールドで自動的に追加される。Haxeでコンパイルされる型には何の影響も与えない。  &  java \\
	@:event  &  \expr{-net-lib}のイベントに自動的に追加される。Haxeでコンパイルされる型には何の影響も与えない。   &  cs \\
	@:enum  &  抽象型の定義につけることで有限の値のセットを定義する。詳しくは\tref{抽象型列挙体}{types-abstract-enum}  &  all \\
	@:expose \_(?Name=Class path)\_  &  クラスを\expr{window}オブジェクトか、node.jsの場合は\expr{exports}で使用可能にする。詳しくは\tref{HaxeのクラスをJavaScriptに露出させる}{target-javascript-expose} &  js \\
	@:extern  &  フィールドを\expr{extern}としてマークする。結果として出力されなくなる。  &  all \\
	@:fakeEnum \_(Type name)\_  &  列挙型を指定した型の集合としてあつかう。  &  all \\
	@:file(File path)  &  SWFターゲットに指定したバイナリファイルを含めてそのクラスと関連づける。(\expr{flash.utils.ByteArray}の継承が必要)  &  flash \\
	@:final  &  クラスが継承されるのを妨害する  &  all \\
	@:font \_(TTF path Range String)\_  &  指定したTrueTypeフォントをクラスに埋め込む(\expr{flash.text.Font}の継承が必要)  &  flash \\
	@:forward \_(List of field names)\_  & 基底型から\tref{フィールドアクセスの繰り上げ}{types-abstract-forward}を行う。  &  all \\
	@:from   &  抽象型のフィールドで、その型からのキャスト操作をその関数で定義する。詳しくは\tref{暗黙のキャスト}{types-abstract-implicit-casts}。  &  all \\
	@:functionCode  &     &  cpp \\
	@:functionTailCode  &    &  cpp \\
	@:generic &  クラスまたはクラスフィールドを\tref{ジェネリック}{type-system-generic}としてマークして、すべてのパラメータの組み合わせについて出力を行う。  &  all \\
	@:genericBuild  &  型のインスタンスを指定したマクロを使って生成する。   &  all \\
	@:getter \_(Class field name)\_  &  指定したフィールドにネイティブのゲッター関数を生成する。   &  flash \\
	@:hack   &  \expr{@:final}のマークがされているクラスの継承を許可する。  &  all \\
	@:headerClassCode  &  そのヘッダーで、生成されたクラスにコードを挿入する。  &  cpp \\
	@:headerCode   &  生成されたヘッダファイルにコードを挿入する。  &  cpp \\
	@:headerNamespaceCode  &    &  cpp \\
	@:hxGen  &  Haxeによって生成されたexternクラスに付く  &  cs  java \\
	@:ifFeature \_(Feature name)\_  &  指定された機能がコンパイルに含まれていた場合に、フィールドを\tref{デッドコード削除}{cr-dce}から保護する。  &  all \\
	@:include &     &  cpp \\
	@:initPackage  &    &  all \\
	@:internal  &  クラスやフィールドに\expr{internal}アクセスの修飾をする。  &  cs  java \\
	@:isVar  &  物理的フィールドが不要なプロパティに対して、物理的フィールドを強制する。  &  all \\
	@:javaCanonical \_(Output type package,Output type name)\_ &  Javaターゲットで型の正規パスを指定するのに使われる。  &  java \\
	@:jsRequire  &  その\expr{extern}に必要なJavaScriptモジュールを出力する。  &  js \\
	@:keep   &  \tref{デッドコード削除}{cr-dce}から、フィールドや型を保護する。  &  all \\
	@:keepInit  &  クラスからすべてのフィールドが削除された場合でも、\tref{デッドコード削除}{cr-dce}から保護する。  &  all \\
	@:keepSub &  すべての実装クラス、子孫クラスに\expr{@:keep}メタデータを継承する。  &  all \\
	@:macro  &  \_(deprecated)\_  &  all \\
	@:mergeBlock  &  修飾したブロックを現在のスコープにマージする。 &  all \\
	@:meta   &  内部的にクラスフィールドがメタデータフィールドであるとマークするのに使われる。  &  all \\
	@:multiType \_(Relevant type parameters)\_  &  抽象型でその\expr{@:to}関数の中からthisの型を選択する。  &  all \\
	@:native \_(Output type path)\_  &  クラスと列挙型のパスを出力の過程で書き換える。  &  all \\
	@:nativeChildren  &  型のそのすべての子を\expr{extern}であるかのように扱う。(プラットフォームネイティブ)  &  cs java \\
	@:nativeGen  &  型を\expr{extern}であるかのように扱う。(プラットフォームネイティブ)  &  cs  java \\
	@:nativeProperty  &  動的な使用がされる場合であっても、ネイティブのプロパティを使う。  &  cpp \\
	@:noCompletion  &  そのフィールドをコンパイラの\tref{補完}{cr-completion}の候補に含めない。  &  all \\
	@:noDebug &  \expr{-debug}が設定されている場合でもSWFにデバッグ情報を含めない。   &  flash \\
	@:noDoc  &  型がドキュメント出力に含まれないようにする。  &  all \\
	@:noImportGlobal  &  \expr{import Class.*}で静的フィールドが含まれるのを妨害します。  &  all \\
	@:noPrivateAccess  &  指定した式からの、プライベートアクセスを禁止します。  &  all \\
	@:noStack &     &  cpp \\
	@:noUsing &  フィールドが\expr{using}で使用されるのを防ぎます。  &  all \\
	@:nonVirtual &  関数をvirtualでないと宣言する。  &  cpp \\
	@:notNull &  抽象型が\tref{\expr{null}値}{types-nullability}を許容しないことを宣言する。  &  all \\
	@:ns  &  SWFのジェネレータが名前空間を扱うために内部的に使う。   &  flash \\
	@:op \_(The operation)\_  &  抽象型のフィールドを\tref{演算子オーバーロード}{types-abstract-operator-overloading}として定義する。  &  all \\
	@:optional  &  構造体のフィールドをオプションのものとしてマークする。詳しくは\tref{オプション引数}{types-nullability-optional-arguments}  &  all \\
	@:overload \_(Function specification)\_  &  フィールドが異なる引数の型で呼び出しされるのを許可する。対象の関数は式を持てない。  &  all \\
	@:privateAccess  &  指定された式からのあらゆるプライベートアクセスを許可する。  &  all \\
	@:property  &  プロパティフィールドをネイティブのC\#プロパティにコンパイルされるようにマークする。   &  cs \\
	@:protected  &  クラスフィールドを\expr{protected}としてマークする。  &  all \\
	@:public  &  クラスフィールドを\expr{public}としてマークする。 &  all \\
	@:publicFields  &  クラスのすべてのクラスフィールドを\expr{public}にする。  &  all \\
	@:pythonImport  &  \expr{extern}クラス対してpythonのインポート文を生成する。  &  python \\
	@:readOnly  &  ネイティブの\expr{readonly}キーワードを付けたフィールドを出力する。   &  cs \\
	@:remove  &  インターフェースをすべての実装クラスから出力前に削除する。  &  all \\
	@:require \_(Compiler flag to check)\_  &  特定の\tref{コンパイラフラグ}{lf-condition-compilation}が指定されている場合のみアクセスを許可する。  &  all \\
	@:rtti   &  実行時型情報を追加する。詳しくは\tref{RTTI}{cr-rtti}。  &  all \\
	@:runtime  &    &  all \\
	@:runtimeValue  &  抽象型を実行時の値としてマークする。  &  all \\
	@:selfCall  &  メソッド呼び出しをオブジェクトそのものの呼び出しに変換する。  &  js \\
	@:setter \_(Class field name)\_  &  指定したフィールドのネイティブのセッター関数を生成する。   &  flash \\
	@:sound \_(File path)\_  &  \_.wav\_または\_.mp3\_ファイルをSWFを埋め込んでクラスに紐づける。(\expr{flash.media.Sound}の継承が必要)  &  flash \\
	@:sourceFile  &  外部クラスのソースコードのファイル名。  &  cpp \\
	@:strict  &  C\#のネイティブ属性、Javaのメタデータを宣言するのに使う。型チェックがされる。  &  cs java \\
	@:struct  &  クラスを構造体としてマークする。   &  cs \\
	@:structAccess  &  \expr{extern}クラスをポインタ('->')ではなく構造体アクセス('.')を使うようにマークする。  &  cpp \\
	@:suppressWarnings  &  出力されるJavaクラスにSuppressWarningsの修飾を行う。  &  java \\
	@:throws \_(Type as String)\_  &  \expr{throws}宣言を出力される関数に追加する。   &  java \\
	@:to  &  抽象型のフィールドで、その型へのキャスト操作をその関数で定義する。詳しくは\tref{暗黙のキャスト}{types-abstract-implicit-casts}。  & all \\
	@:transient  &  クラスフィールドに\expr{transient}フラグを追加する。  &  java \\
	@:unbound  &  コンパイラの内部で無制限のグローバル変数を表すのに使う。  &  all \\
	@:unifyMinDynamic  &  型の集合を\type{Dynamic}に単一化するのを許可する。  &  all \\
	@:unreflective  &    &  cpp \\
	@:unsafe  &  クラスまたはメソッドでC\#の\expr{unsafe}フラグを宣言する。   &  cs \\
	@:usage  &    &  all \\
	@:value  &  フィールドや関数の引数のデフォルト値を記録するのに使う。  &  all \\
	@:void  &  C++ネイティブの'void'を戻り値に使う。  &  cpp \\
	@:volatile  &    &  cs  java \\
\end{tabular}
\end{center}

\section{デッドコード削除}
\label{cr-dce}

デッドコード削除(Dead Code Elimination、\emph{DCE})は、未使用のコードを出力から取り除くコンパイラ機能です。型付けの後に、デッドコード削除の始点(多くの場合はmainメソッド)から再帰的にたどっていき、どのフィールドと型が使用されているかを決定します。これにより使用済みのフィールドはマークされ、マークされていないフィールドはクラスから取り除かれます。

デッドコード削除には3つのモードがあり、コマンドラインからの呼び出し時に指定します。

\begin{description}
	\item[-dce std:] Haxeの標準ライブラリのクラスのみがデッドコード削除の影響を受けます。これがすべてのターゲットでのデフォルト値です。
	\item[-dce no:] デッドコード削除されません。
	\item[-dce full:] すべてのクラスがデッドコード削除の影響を受けます。
\end{description}
デッドコード削除のアルゴリズムは型付けされたコードではうまく働きますが、\tref{Dynamic}{types-dynamic}や\tref{リフレクション}{std-reflection}を使っていると失敗する場合があります。そのような場合は、以下のメタデータを使ったクラスやフィールドの明示的な修飾が有効かもしれません。

\begin{description}
	\item[\expr{@:keep}:] クラスに使用するとすべてのフィールドがデッドコード削除の対象から除外されます。フィールドに使用するとそのフィールドがデッドコード削除の対象になりません。
	\item[\expr{@:keepSub}:] クラスに使用すると、その子孫クラスすべてを\expr{@:keep}で修飾したのと同様の動作をします。
	\item[\expr{@:keepInit}:] 通常、クラスはすべてのフィールドがデッドコード削除によって削除されると(あるいは最初から空だと)出力から削除されます。このメタデータを使うと、空のクラスが保護されます。
\end{description}

ソースコードを編集するのではなくコマンドラインからクラスを\expr{@:keep}としてマークしたい場合、コンパイラマクロの\expr{--macro keep('type dot path')}を使うことでそれが可能です。このマクロについて詳しくは\href{http://api.haxe.org/haxe/macro/Compiler.html#keep}{haxe.macro.Compiler.keep API}をご覧ください。パッケージをマークするとそのモジュールやサブタイプがデッドコード削除から保護されて、コンパイルに含まれます。

コンパイラは現在のモードに応じて、自動的に\expr{dce}フラグの値を\expr{"std"}、\expr{"no"}、\expr{"full"}のいずれかに設定します。このフラグは\tref{条件付きコンパイル}{lf-condition-compilation}で使用できます。

\trivia{デッドコード削除の書き直し}{
デッドコード削除は元々Haxe 2.07で実装されましたが、その実装では関数が明示的に型付けされると使用されているという判定がされていました。このせいでいくつか機能で問題がありました。型安全性を確かめるためにすべてのクラスフィールドを型付けする必要があるインタフェースでとくに深刻でした。このせいでデッドコード削除は完全に破たんし、Haxe 2.10での書き直しにつながりました。}

\trivia{デッドコード削除とtry.haxe.org}{\url{http://try.haxe.org}のサイトが公開されたとき、\type{JavaScript}ターゲットのデッドコード削除は大きく改善されました。\target{JavaScript}の出力コードに対する反応はさまざまでしたが、これにより削除されるコードの選択がより細かく行われるようになりました。}


\section{コンパイラサービス}
\label{cr-completion}
\state{NoContent}

\subsection{概要}
\label{cr-completion-overview}

Haxeの豊富な\tref{型システム}{type-system}は、IDEやエディタが正確な補完情報を提供することを難しくしています。\tref{型推論}{type-system-type-inference}や\tref{マクロ}{macro}のせいで、必要な挙動を再現するのに相当な量の作業が必要です。これがHaxeのコンパイラが、サードパーティーのソフトウェアが使うためのビルトインの補完モードを備えている理由です。

すべての補完は\ic{--display file@position[@mode]}のコンパイラ引数を使うことで開始されます。この引数は以下を要求します。

\begin{description}
	\item[file:] 補完のためチェックを行うファイルです。これは.hxファイルの絶対パスまたは相対パスです。クラスパスやライブラリではありません。
	\item[position:] 補完のためのチェックを行う、与えられたファイルのバイト位置です(文字数での位置ではありません)。
	\item[mode:] 使用する補完モードです(後述)。
\end{description}

補完モードの詳細については以下の通りです。

\begin{description}
	\item[\tref{フィールドアクセス}{cr-completion-field-access}:] その型のアクセス可能なフィールドのリストを提供します。
	\item[\tref{関数の引数}{cr-completion-call-argument}:] 呼び出そうとしている関数の型を取得します。
	\item[\tref{型のパス}{cr-completion-type-path}:] 子パッケージとサブタイプ、静的フィールドをリストアップします。
	\item[\tref{使用状況}{cr-completion-usage}:] コンパイルされるファイルのすべて中から指定された型またはフィールド、変数の出現位置をリストアップします。(\ic{"usage"}モード)
	\item[\tref{定義位置}{cr-completion-position}:] 指定された型またはフィールド、変数の定義位置を取得します。(\ic{"position"}モード)
	\item[\tref{トップレベル}{cr-completion-top-level}:] 指定した位置で使用可能なすべての識別子をリストアップします。(\ic{"toplevel"}モード)
\end{description}

Haxeのコンパイラは非常に速いので多くの場合では通常のコンパイラの呼び出しを使っても問題がありませんが、より大きなHaxeプロジェクトのために\tref{サーバーモード}{cr-completion-server}を用意してあります。これにより、ファイルに実際に変更があった場合や依存関係の更新がされたときだけ再コンパイルがされます。

\paragraph{インターフェースについての注意事項}
\label{cr-completion-interface-notes}

\begin{itemize}
	\item ファイルの目的の位置にパイプライン\ic{|}の文字を置くことで、\ic{position}引数を0に設定することができます。これは、バイト数のカウント無しでIDEが何をすべきなのかをテストしたり実演したりするのに、便利です。この章のサンプルはこの機能を使っていきます。この機能は\ic{|}が不正な構文になる位置に置かれている場合のみ動作します。例えば、ドットの直後(\ic{.|})や小かっこの直後(\ic{(|})です。
	\item 出力はHTMLエスケープされます。つまり\ic{\&}、\ic{<}、\ic{>}はそれぞれ\ic{\&amp;}、\ic{\&lt;}、\ic{\&gt;}に変換されます。
	\item ドキュメントの出力は、ソースコード内のものと同じように改行とタブ文字を含みます。
	\item 補完モードの実行ではコンパイラはなるべくエラーを出力せずにエラーからの復帰を試みます。補完中に致命的なエラーが発生した場合、Haxeのコンパイラは補完結果の代わりにエラーメッセージを出力します。XMLでは無いあらゆる出力は致命的なエラーメッセージであると判断できます。
\end{itemize}

\subsection{フィールドアクセス補完}
\label{cr-completion-field-access}

フィールドの補完はドット(\ic{.})文字の後から開始されて、その型で利用可能なフィールドをリストアップします。コンパイラは補完の位置までのすべての構文解析と型付けを行い、関連する情報を標準エラー出力に出力します。

\begin{lstlisting}
class Main {
  public static function main() {
    trace("Hello".|
  }
}
\end{lstlisting}

このファイルをMain.hxとして保存すると、補完を\ic{haxe --display Main.hx@0}のコマンドで呼び出せます。その出力は以下のようなものでしょう(いくつかの情報を可読性のために削ったりフォーマットをかけたりしています)。

\begin{lstlisting}
<list>
<i n="length">
  <t>Int</t>
  <d>
    The number of characters in `this` String.
  </d>
</i>
<i n="charAt">
  <t>index : Int -&gt; String</t>
  <d>
    Returns the character at position `index` of `this` String.
    If `index` is negative or exceeds `this.length`, the empty String
    "" is returned.
  </d>
</i>
<i n="charCodeAt">
  <t>index : Int -&gt; Null&lt;Int&gt;</t>
  <d>
    Returns the character code at position `index` of `this` String.
    If `index` is negative or exceeds `this.length`, null is returned.
    To obtain the character code of a single character, "x".code can
    be used instead to inline the character code at compile time.
    Note that this only works on String literals of length 1.
  </d>
</i>
</list>
\end{lstlisting}

このXMLの構造は以下の通りです。

\begin{itemize}
	\item ドキュメント直下の\ic{list}ノードはいくつかの\ic{i}ノードを持ち、そのそれぞれが1つフィールドを表現しています。
	\item \ic{n}属性はフィールドの名前です。
	\item \ic{t}ノードはフィールドの型です。
	\item \ic{d}ノードはフィールドのドキュメントです。
\end{itemize}

\since{3.2.0}

\ic{-D display-details}をつけてコンパイルすると、各フィールドに\ic{var}と\ic{method}のいずれかの\ic{k}属性が付きます。これにより、関数型の変数フィールドとメソッドフィールドを区別できます。

\subsection{呼び出し引数の補完}
\label{cr-completion-call-argument}

呼び出し引数の補完は小かっこ(\ic{(})の後から開始されて、呼び出しをしようとしている関数の型を返します。これはコンストラクタの呼び出しを含むすべての関数呼び出しで使用できます。

\begin{lstlisting}
class Main {
  public static function main() {
    trace("Hello".split(|
  }
}
\end{lstlisting}

このファイルをMain.hxとして保存すると、補完を\ic{haxe --display Main.hx@0}のコマンドで呼び出せます。その出力は以下のようなになります。

\begin{lstlisting}
<type>
delimiter : String -&gt; Array&lt;String&gt;
</type>
\end{lstlisting}

IDEはここから、呼び出す関数に\ic{delimiter}という\type{String}型の引数が1つあって\type{Array<String>}を返すということを、読み取れます。

\trivia{出力構造の問題}{私たちは現在のフォーマットはほんの少しのわずらわしい自前の構文解析が必要になることを認めます。特に関数については、将来的にはより構造化された出力を提供するようになるかもしれません。}

\subsection{型のパスの補完}
\label{cr-completion-type-path}

型のパスの補完は\tref{import宣言}{type-system-import}、\tref{using宣言}{lf-static-extension}あるいはあらゆる位置での型の記述で発生します。そしてこれは以降の3種類に分けることができます。

\paragraph{パッケージの補完}

以下はhaxeパッケージに属する子パッケージとモジュールのすべてをリストアップします。

\begin{lstlisting}
import haxe.|
\end{lstlisting}

\begin{lstlisting}
<list>
<i n="CallStack"><t></t><d></d></i>
<i n="Constraints"><t></t><d></d></i>
<i n="DynamicAccess"><t></t><d></d></i>
<i n="EnumFlags"><t></t><d></d></i>
<i n="EnumTools"><t></t><d></d></i>
<i n="Http"><t></t><d></d></i>
<i n="Int32"><t></t><d></d></i>
<i n="Int64"><t></t><d></d></i>
<i n="Json"><t></t><d></d></i>
<i n="Log"><t></t><d></d></i>
<i n="PosInfos"><t></t><d></d></i>
<i n="Resource"><t></t><d></d></i>
<i n="Serializer"><t></t><d></d></i>
<i n="Template"><t></t><d></d></i>
<i n="Timer"><t></t><d></d></i>
<i n="Ucs2"><t></t><d></d></i>
<i n="Unserializer"><t></t><d></d></i>
<i n="Utf8"><t></t><d></d></i>
<i n="crypto"><t></t><d></d></i>
<i n="ds"><t></t><d></d></i>
<i n="extern"><t></t><d></d></i>
<i n="format"><t></t><d></d></i>
<i n="io"><t></t><d></d></i>
<i n="macro"><t></t><d></d></i>
<i n="remoting"><t></t><d></d></i>
<i n="rtti"><t></t><d></d></i>
<i n="unit"><t></t><d></d></i>
<i n="web"><t></t><d></d></i>
<i n="xml"><t></t><d></d></i>
<i n="zip"><t></t><d></d></i>
</list>
\end{lstlisting}


\paragraph{モジュールのインポートの補完}

以下は、\type{haxe.Unserializer}モジュールの\tref{サブタイプ}{type-system-module-sub-types}と、\type{haxe.Unserializer}の\expr{public static}なフィールド(これらもインポート可能なので)のすべてをリストアップします。

\begin{lstlisting}
import haxe.Unserializer.|
\end{lstlisting}

\begin{lstlisting}
<list>
<i n="DEFAULT_RESOLVER">
  <t>haxe.TypeResolver</t>
  <d>
    This value can be set to use custom type resolvers.

    A type resolver finds a Class or Enum instance from a given String.
    By default, the haxe Type Api is used.

    A type resolver must provide two methods:

    1. resolveClass(name:String):Class&lt;Dynamic&gt; is called to
      determine a Class from a class name
    2. resolveEnum(name:String):Enum&lt;Dynamic&gt; is called to
      determine an Enum from an enum name

    This value is applied when a new Unserializer instance is created.
    Changing it afterwards has no effect on previously created
    instances.
  </d>
</i>
<i n="run">
  <t>v : String -&gt; Dynamic</t>
  <d>
    Unserializes `v` and returns the according value.

    This is a convenience function for creating a new instance of
    Unserializer with `v` as buffer and calling its unserialize()
    method once.
  </d>
</i>
<i n="TypeResolver"><t></t><d></d></i>
<i n="Unserializer"><t></t><d></d></i>
</list>
\end{lstlisting}


\begin{lstlisting}
using haxe.Unserializer.|
\end{lstlisting}


\paragraph{その他のモジュールの補完}

以下は、\type{haxe.Unserializer}のすべての\tref{サブタイプ}{type-system-module-sub-types}をリストアップします。

\begin{lstlisting}
using haxe.Unserializer.|
\end{lstlisting}

\begin{lstlisting}
class Main {
  static public function main() {
    var x:haxe.Unserializer.|
  }
}
\end{lstlisting}

\begin{lstlisting}
<list>
<i n="TypeResolver"><t></t><d></d></i>
<i n="Unserializer"><t></t><d></d></i>
</list>
\end{lstlisting}


\subsection{使用状況の補完}
\label{cr-completion-usage}
\since{3.2.0}

使用状況の補完は\ic{"usage"}モードの引数を使うことで使用できます(詳しくは\Fullref{cr-completion-overview})。ローカル変数を使って実演しますが、フィールドと型についても同じように動作することも覚えておきましょう。

\begin{lstlisting}
class Main {
  public static function main() {
    var a = 1;
    var b = a + 1;
    trace(a);
    a.|
  }
}
\end{lstlisting}

このファイルをMain.hxとして保存すると、補完を\ic{haxe --display Main.hx@0@usage}のコマンドで呼び出せます。この出力は以下のようになります。

\begin{lstlisting}
<list>
<pos>main.hx:4: characters 9-10</pos>
<pos>main.hx:5: characters 7-8</pos>
<pos>main.hx:6: characters 1-2</pos>
</list>
\end{lstlisting}


\subsection{定義位置の補完}
\label{cr-completion-position}
\since{3.2.0}

定義位置の補完は\ic{"position"}モードの引数を使うことで使用できます(詳しくは\Fullref{cr-completion-overview})。フィールドを使って実演しますが、ローカル変数と型でも同じように動作することも覚えておきましょう。

\begin{lstlisting}
class Main {
  static public function main() {
    "foo".split.|
}
\end{lstlisting}

このファイルをMain.hxとして保存すると、補完を\ic{haxe --display Main.hx@0@position}のコマンドで呼び出せます。この出力は以下のようになります。

\begin{lstlisting}
<list>
<pos>std/string.hx:124: characters 1-54</pos>
</list>
\end{lstlisting}

\trivia{ターゲットの特定の省略による影響}{このサンプルでは\expr{std}のString.hxが取得されましたが、ここに実際の実装はありません。これはどのターゲットとも特定しなかったためであり、補完モードではそれでも構いません。例えば\ic{-neko neko.n}のコマンドラインが含められた場合、結果として取得される位置は代わりに\ic{std/neko/_std/string.hx:84: lines 84-98.}となるでしょう。}

\subsection{トップレベルの補完}
\label{cr-completion-top-level}
\since{3.2.0}

トップレベルの補完は、与えられた補完位置での使用可能な識別子をHaxeコンパイラが知るかぎりのすべて表示します。この補完機能だけは実演するのに、実際の位置を引数であたえる必要があります。

\begin{lstlisting}
class Main {
  static public function main() {
    var a = 1;
  }
}

enum MyEnum {
  MyConstructor1;
  MyConstructor2(s:String);
}
\end{lstlisting}

このファイルをMain.hxとして保存すると、補完を\ic{haxe --display Main.hx@63@toplevel}のコマンドで呼び出せます。その出力は以下のようになります(簡潔さのためにいくつかの要素を削っています)。

\begin{lstlisting}
<il>
<i k="local" t="Int">a</i>
<i k="static" t="Void -&gt; Unknown&lt;0&gt;">main</i>
<i k="enum" t="MyEnum">MyConstructor1</i>
<i k="enum" t="s : String -&gt; MyEnum">MyConstructor2</i>
<i k="package">sys</i>
<i k="package">haxe</i>
<i k="type" p="Int">Int</i>
<i k="type" p="Float">Float</i>
<i k="type" p="MyEnum">MyEnum</i>
<i k="type" p="Main">Main</i>
</il>
\end{lstlisting}

XMLの構造は各要素の\ic{k}属性によります。すべての場合で\ic{i}のノードはその値として名前を持ちます。

\begin{description}
	\item[\ic{local}, \ic{member}, \ic{static}, \ic{enum}, \ic{global}:] \ic{t}属性にその変数やフィールドの型を持ちます。
	\item[\ic{global}, \ic{type}:] \ic{p}属性にその型やフィールドが属するモジュールのパスを持ちます。
\end{description}

\subsection{補完サーバー}
\label{cr-completion-server}

コンパイルと補完を最速で行いたいのであれば、\ic{--wait}のコマンドラインパラメータでHaxeの補完サーバーを立ち上げることができます。また、\ic{-v}でサーバーがログを出力するようになります。以下が例です。

\begin{lstlisting}
haxe -v --wait 6000
\end{lstlisting}

こうすると、Haxeサーバーに接続して、コマンドラインパラメータを送って、最後にNULL文字を送ることで、レスポンスの読み取りができます(補完が成功の場合も失敗の場合も)。

\ic{--connect}のコマンドラインのパラメータを使うことで、Haxeはコンパイルコマンドを直接実行するのではなくサーバーに送るようになります。

\begin{lstlisting}
haxe --connect 6000 myproject.hxml
\end{lstlisting}

はじめに\ic{--cwd}のパラメータを使うことで、Haxeサーバーの現在の作業ディレクトリを変更することができることに気をつけてください。多くの場合クラスパスとそのほかのファイルはプロジェクトからの相対パスで指定されます。

\paragraph{動作の詳細}

コンパイルサーバーは以下の内容をキャッシュします。

\begin{description}
	\item[構文解析したファイル] ファイルは編集があせれたときが解析エラーになったときのみ再度構文解析が行われます。
	\item[Haxelibの呼び出し] 前回のHaxelibの呼び出し結果は再度利用されます(補完時のみです。コンパイルには関係ありません)
	\item[型付けされたモジュール] モジュールのコンパイル結果はコンパイル成功した場合にキャッシュされて、その依存関係が更新されるまで補完とコンパイルに再利用されます。
\end{description}

コマンドラインで\ic{--times}を追加することで、コンパイラが使用した正確な時間を取得して、コンパイルサーバーがどのような影響を与えたかを知ることができます。

\paragraph{プロトコル}
次のHaxe/Nekoの例からわかるとおり、サーバーのポートに単純につないで、1行づつコマンドを送って、NULL文字で終了します。その後に結果を読み取ります。

マクロやその他のコマンドはエラー以外のログを出力することができます。コマンドラインからの実行の場合は、標準出力と標準エラー出力にプリントされるものの違いがありますが、ソケットモードの場合は違います。この2つを区別するために、ログメッセージ(エラーではない)は\ic{x01}の文字から始まり、そのメッセージのすべての改行文字は\ic{x01}で置き換えられます。

警告やそのほかのメッセージはエラーと考えられますが、致命的なものではありません。致命的なエラーが起こると、\ic{x02}から始まる1行のメッセージが送られます。

以下にサーバーへ接続して、このプロトコルに従った処理を行うコードがあります。

\haxe{assets/CompletionServer.hx}

\paragraph{マクロの影響}

コンパイルサーバーは\tref{マクロの実行}{macro}に副作用を与えます。

\section{リソース}
\label{cr-resources}
\flag{fold}{true}

Haxeは単純なリソース埋め込みのシステムを提供しています。これによりファイルをコンパイル後のアプリケーションに直接埋め込むことができます。

この方法は画像や音楽のような巨大なファイルの埋め込みには適していないかもしれませんが、設定やXMLのようなより小さなデータを埋め込むのにはとても便利です。

\subsection{リソースの埋め込み}
\label{cr-resources-embed}

以下のように、\emph{-resource}のコンパイラ引数をつかって外部ファイルの埋め込みができます。

\todo{what to use for listing of non-haxe code like hxml?}
\begin{lstlisting}
-resource hello_message.txt@welcome
\end{lstlisting}

\emph{@}マークの後の文字列は\emph{リソースの識別子}です。コードからリソースを取得するのに使います。省略された場合(\emph{@}マークごと)、ファイル名がリソース識別子として使われます。

\subsection{テキストリソースを取得する}
\label{cr-resources-getString}

埋め込んだリソースを取得するには、\type{haxe.Resource}の\emph{getString}の静的メソッドにリソース識別子を渡して事項します。

\haxe{assets/ResourceGetString.hx}

上記のコードは先ほどの\emph{welcome}を識別子として使って\emph{hello_message.txt}ファイルの内容を表示します。

\subsection{バイナリリソースを取得する}
\label{cr-resources-getBytes}

巨大バイナリファイルをアプリケーションに埋め込むのは推奨されないものの、バイナリデータの埋め込みは便利です。埋め込んだリソースは\type{haxe.Resource}の\emph{getBytes}の静的メソッドを使うことでバイナリとして取得できます。

\haxe{assets/ResourceGetBytes.hx}

\emph{getBytes}メソッドの戻り値の型は、データの各バイトにアクセスできる\type{haxe.io.Bytes}です。

\subsection{実装の詳細}
\label{cr-resources-impl}

ターゲットのプラットフォームにリソースの埋め込み機能があればそれを使います。その他の場合、独自の実装を持ちます。

\begin{itemize}
\item \emph{Flash} リソースは\type{ByteArray}として定義されて埋め込まれる。
\item \emph{C\#} コンパイルされたアセンブリに含まれる。
\item \emph{Java} JARファイル内にパッケージされる。
\item \emph{C++} グローバルなバイト列の定数として記録される。
\item \emph{JavaScript} Haxeシリアル化フォーマットに従ってシリアル化されて\type{haxe.Resource}の静的フィールドに記録される。
\item \emph{Neko} 文字列として\type{haxe.Resource}クラスの静的フィールドに記録される。
\end{itemize}


\section{実行時型情報(RTTI)}
\label{cr-rtti}

Haxeコンパイラは\expr{:rtti}メタデータで修飾されたクラス、あるいはその子孫クラスに対して実行時型情報(RTTI)を生成します。

\since{3.2.0}

\type{haxe.rtti.Rtti}型がRTTIについての処理を簡単にするために導入されました。現在では、情報の取得はとても簡単です。

\haxe{assets/RttiUsage.hx}

\subsection{RTTIの構造}
\label{cr-rtti-structure}

\paragraph{一般的な型情報}

\begin{description}
	\item[path:] \tref{型のパス}{define-type-path}。
	\item[module:] その型を含んでいる\tref{モジュール}{define-module}のパス。
	\item[file:] その型を含む.hxのファイルのフルパス。例えば\tref{マクロ}{macro}から定義された場合など、そのようなファイルがない場合\expr{null}になる場合がある。
	\item[params:] その型が持つ\tref{型パラメータ}{type-system-type-parameters}を表す文字列の配列。Haxe 3.2.0では、\tref{制約}{type-system-type-parameter-constraints}についての情報を持ちません。
	\item[doc:] その型のドキュメント。この情報は\expr{-D use_rtti_doc}の\tref{コンパイラフラグ}{define-compiler-flag}を付けた場合のみ使用できます。フラグがないか、ドキュメントがない場合は\expr{null}です。
	\item[isPrivate:] その型が\tref{private}{define-private-type}かどうか。
	\item[platforms:] その型が利用可能なターゲットのリスト。
	\item[meta:] その型につけられているメタデータ。
\end{description}
	
\paragraph{クラスの型情報}
\label{cr-rtti-class-type-information}

\begin{description}
	\item[isExtern:] クラスが\tref{extern}{lf-externs}かどうか。
	\item[isInterface:] \tref{インターフェース}{types-interfaces}かどうか。
	\item[superClass:] その親クラスの型のパスと型パラメータのリスト。
	\item[interfaces:] そのクラスのインターフェースの型のパスと型パラメータのリストのリスト。
	\item[fields:] \Fullref{cr-rtti-class-field-information}に記載されている、\tref{クラスフィールド}{class-field}のリスト
	\item[statics:] \Fullref{cr-rtti-class-field-information}に記載されている、静的クラスフィールドのリスト。
	\item[tdynamic:] そのクラスに\tref{動的に実装}{types-dynamic-implemented}された型、あるいは型が存在しない場合は\expr{null}
\end{description}

\paragraph{列挙型の型情報}

\begin{description}
	\item[isExtern:] その列挙型が\tref{extern}{lf-externs}かどうか。
	\item[constructors:] その列挙型コンストラクタのリスト。
\end{description}

\paragraph{抽象型の型情報}

\begin{description}
	\item[to:] 定義されている\tref{暗黙のtoキャスト}{types-abstract-implicit-casts}の配列。
	\item[from:] 定義されている\tref{暗黙のfromキャスト}{types-abstract-implicit-casts}の配列。
	\item[impl:] 実装しているクラスの\tref{クラスの型情報}{cr-rtti-class-type-information}。
	\item[athis:] その抽象型の\tref{基底型}{define-underlying-type}。
\end{description}
	
	
\paragraph{クラスフィールド情報}
\label{cr-rtti-class-field-information}

\begin{description}
	\item[name:] フィールドの名前。
	\item[type:] フィールドの型。
	\item[isPublic:] フィールドが\tref{public}{class-field-visibility}かどうか。
	\item[isOverride:] フィールドが別のフィールドの\tref{オーバーライド}{class-field-override}かどうか。
	\item[doc:] フィールドのドキュメント。この情報は\expr{-D use_rtti_doc}の\tref{コンパイラフラグ}{define-compiler-flag}を付けた場合のみ使用できます。フラグがないか、ドキュメントがない場合は\expr{null}です。
	\item[get:] フィールドの\tref{読み込みアクセスの挙動}{define-read-access}。
	\item[set:] フィールドの\tref{書き込みアクセスの挙動}{define-write-access}。
	\item[params:] そのフィールドが持つ\tref{型パラメータ}{type-system-type-parameters}を表す文字列の配列。Haxe 3.2.0では、\tref{制約}{type-system-type-parameter-constraints}についての情報を持ちません。
	\item[platforms:] フィールドが使用可能なターゲットのリスト。
	\item[meta:] フィールドに付けられているメタデータ。
	\item[line:] フィールドが定義されている行番号。この情報はフィールドが式を持つ場合のみ使用できます。その他の場合は\expr{null}です
	\item[overloads:] このフィールドに対する利用可能なオーバーロードのリスト。存在しなければ\expr{null}です。
\end{description}

\paragraph{列挙型コンストラクタ情報}
\label{cr-rtti-enum-constructor-information}

\begin{description}
	\item[name:] コンストラクタ名。
	\item[args:] 引数のリスト。存在しなければ\expr{null}です。
	\item[doc:] コンストラクタのドキュメント。この情報は\expr{-D use_rtti_doc}の\tref{コンパイラフラグ}{define-compiler-flag}を付けた場合のみ使用できます。フラグがないか、ドキュメントがない場合は\expr{null}です。
	\item[platforms:] コンストラクタが使用可能なターゲットのリスト。
	\item[meta:] コンストラクタに付けられているメタデータ。
\end{description}

\chapter{マクロ}
\label{macro}

マクロは疑いようもなくHaxeの最も高度な機能です。マクロは少数の精鋭にとってのみこれをマスターする価値があるので黒魔術と呼ばれることがありますが、実際には魔法のようなものは何もありません(もちろん闇も)。

\define{抽象構文木(AST:Abstract Syntax Tree)}{define-ast}{抽象構文木はHaxeのコードを構文解析して型付けされた構造へと変換した結果です。この構造はHaxe標準ライブラリの\expr{haxe/macro/Expr.hx}ファイルで定義されている型をつかってマクロから利用可能です。}

\begin{flowchart}{macro-compilation-role}{コンパイルにおけるマクロの役割}

\tikzstyle{macro} = [ fill = orange!40 ]
\tikzstyle{macroEdge} = [ dashed, color = orange!70!black ]
\tikzstyle{edge} = [ midway, auto = left, outer sep = 0.2cm ]

\node (src) [process] {ソースコード};
\node (lexpar) [process, right = of src] {字句解析/構文解析器};
\node (ast1) [process, right = of lexpar] {抽象構文木(AST)};
\node (ast1t) [above = of ast1, xshift = 0.5cm, text width = 4cm] {
	\begin{itemize}
		\itemsep-0.2em
		\item 式
		\item 型の表現(ComplexType)
		\item haxe.macro.Expr
	\end{itemize}
};
\node (macro) [process, right = of ast1, macro] {マクロプロセッサ};
\node (ast2) [process, below = 4cm of macro, macro] {抽象構文木(AST)};
\node (typer) [process, left = of ast2] {型付け器};
\node (ast3) [process, left = of typer] {型付けされたAST};
\node (ast3t) [above = of ast3, xshift = 0.5cm, text width =4cm] {
	\begin{itemize}
		\itemsep-0.2em
		\item 型付けされた式
		\item 型
		\item haxe.macro.Type
	\end{itemize}
};
\node (gen) [process, left = of ast3] {ジェネレータ};
\node (out) [process, above = of gen] {出力};

\draw [flowchartArrow] (src) -- (lexpar);
\draw [flowchartArrow] (lexpar) -- (ast1) node[edge] {解析};
\draw [dashed] (ast1t.-144) -- (ast1t.144 |- ast1.north);
\draw [flowchartArrow, macroEdge] (ast1) -- (macro);
\draw [flowchartArrow, macroEdge] (macro) -- (ast2) node[edge] {変換};
\draw [flowchartArrow, macroEdge] (ast2) -- (typer);
\draw [flowchartArrow] (typer |- ast1.south) -- (typer);
\draw [flowchartArrow] (typer) -- (ast3) node[edge] {型付け};
\draw [dashed] (ast3t.-144) -- (ast3t.-144 |- ast3.north);
\draw [flowchartArrow] (ast3) -- (gen);
\draw [flowchartArrow] (gen) -- (out) node[edge] {出力};

\end{flowchart}

基本的なマクロの1つは\emph{構文変形}です。これは0個以上の式を受け取り、1つの式を返します。マクロが呼び出されると、その結果としてマクロを呼び出した位置にコードが挿入されます。この点はプリプロセッサに似ていますが、Haxeのマクロはテキストの置換ツールではありません。

マクロには種類があり、それぞれ異なるコンパイルの段階で動作します。

\begin{description}
	\item[初期化マクロ:] コマンドラインから\ic{--macro}コンパイラパラメータを使うことで使用します。コンパイラ引数が処理されて、\emph{型付けコンテクスト}が作成されたあとに実行されます。ただし、これはどの型付けが実行されるよりも前です(詳しくは\Fullref{macro-initialization})。
	\item[ビルドマクロ:] クラス、列挙型、抽象型を\expr{@:build}または\expr{@:autoBuild}\tref{メタデータ}{lf-metadata}を使って定義します。その型の生成(クラスの継承関係などの他の型との依存関係の解決も含めて)後に型ごとに実行されます。ただし、これはフィールドが型付けされるよりは前です(詳しくは\Fullref{macro-type-building})。
	\item[式マクロ:] 型付けされると同時に実行される普通の関数です。
\end{description}

\section{マクロコンテクスト}
\label{macro-context}

\define{マクロコンテクスト}{define-macro-context}{マクロコンテクストとはマクロが実行される環境です。マクロの種別によって、クラスのビルドや、関数の型付けなどを行います。コンテクストについての情報は\ic{haxe.macro.Context} APIを通して入手できます。}

Haxeマクロはマクロの種類に応じて異なる、コンテクストの情報にアクセスできます。\type{Context}のAPIでは情報の問い合わせをするだけでなく、新しい型を定義したり、特定のコールバックを登録したりといった編集も行えます。以下で説明している通り、すべてのマクロの種類ですべての情報が利用可能なわけではないことに注意してください。

\begin{itemize}
	\item 初期化マクロでは\expr{Context.getLocal*()}メソッドは\expr{null}を返します。初期化マクロのコンテクストでは、ローカルの型やメソッドはありません。
	\item ビルドマクロのみで\expr{Context.getBuildFields()}からしかるべき値が返ってきます。その他のマクロでは、ビルドされるフィールドはありません。
	\item ビルドマクロでは、ローカルの型があります(ただし、不完全)が、ローカルのメソッドはありません。\expr{Context.getLocalMethod()}は、\expr{null}を返します。
\end{itemize}

\type{Context}のAPIは、\expr{haxe.macro.Compiler}のAPIと合わせて完全なものになります。\expr{haxe.macro.Compiler}について、詳しくは\Fullref{macro-initialization}を参照してください。こちらのAPIはすべての種類のマクロで使用可能で、初期化マクロ以外からでもあらゆる編集が可能であることに注意しなくてはいけません。定義されていない\tref{ビルド順序}{macro-limitations-build-order}の自然な制限に逆らいます。つまり、例えば\expr{Compiler.define()}でのフラグの定義は、そのフラグに対する\tref{条件付きコンパイル}{lf-condition-compilation}のチェックの後でも先でも影響を与えます。

\section{引数}
\label{macro-arguments}

ほとんどの場合、マクロへの引数は式を\type{Expr}列挙型インスタンスとして表現したものです。これらは構文解析されていますが、型付けはされていません。つまり、Haxeの構文ルールに従うものであればどのようなものもあり得ます。マクロではその構造を解析したり、\expr{haxe.macro.Context.typeof()}を使ってその構造を調べたりできます。

マクロへの引数は評価できるとは限らないため、意図する副作用が起きる保証がないことに気を付けてください。また、引数の式はマクロで複製されて戻り値の式で複数使うことができるというのも重要です。

\haxe{assets/MacroArguments.hx}

\expr{add}マクロは、\expr{x++}を引数としてよびだされており、\tref{式の実体化}{macro-reification-expression}を使って\expr{x++ + x++}を返しており、このため2度インクリメントがされています。

\subsection{ExprOf}
\label{macro-ExprOf}

\type{Expr}はあらゆる入力と一致するため、Haxeでは\type{haxe.macro.ExprOf<T>}型を提供しています。ほとんどの面では\type{Expr}と同じですが、受け入れる式の型を強制することができます。これは\tref{静的拡張}{lf-static-extension}と合わせてマクロを使うときに便利です。

\haxe{assets/ExprOf.hx}

上2種類の\expr{identity}の呼び出しは両方とも問題ありません。たとえ引数が\expr{ExprOf<String>}で宣言されていてもです。\type{Int}型の\expr{1}が許容されることに驚くかもしれませんが、\tref{マクロの引数}{macro-arguments}での説明からの論理的な必然性があります。つまり引数の式は型付けされないので、コンパイラは\tref{単一化}{type-system-unification}の一致チェックができないというわけです。

次の2つの行の呼び出しは、静的拡張を使っている点で状況が異なります(\expr{using Main}に注目してください)。静的拡張では左側(\expr{"foo"}や\expr{1})の型によって、\expr{identity}のフィールドアクセスが意味を持ちます。これにより引数の型に対しての型チェックが可能になり、\expr{1.identity()}が\expr{Main.identity()}のフィールドが合わないという結果になっています。

\subsection{定数の式}
\label{macro-constant-arguments}

マクロは\tref{定数}{expression-constants}の引数を要求するように宣言することができます。

\haxe{assets/MacroArgumentsConst.hx}

これによりわざわざ式を経由するなく、コンパイラはその定数を直接使うことができます。

\subsection{残りの引数}
\label{macro-rest-argument}

マクロの引数の最後が\type{Array<Expr>}型だった場合、任意の個数の追加の引数を渡して、それを配列として利用できます。

\haxe{assets/MacroArgumentsRest.hx}

\section{実体化(レイフィケーション)}
\label{macro-reification}

Haxeコンパイラではマクロの活用を簡単にするために式、型、クラスの\emph{実体化(レイフィケーション)}が可能です。実体化の構文は\expr{macro expr}であり、\expr{expr}は正当なHaxe式であれば何でもかまいません。

\subsection{式の実体化}
\label{macro-reification-expression}

式の実体化を使うと、手軽に\type{haxe.macro.Expr}のインスタンスを作成できます。Haxeのコンパイラは通常のHaxeの構文を式のオブジェクトへと変換します。これにはエスケープの仕組みがあり、それらはすべて\expr{\$}の文字からはじまります。

\begin{description}
	\item[\expr{\$\{\}}または\expr{\$e\{\}}:] \type{Expr -> Expr} これは式の構築に使います。\expr{\{ \}}の中の式が評価されてその値がその位置に配置されます。
	\item[\expr{\$a\{\}}:] \type{Expr -> Array<Expr>} \type{Array<Expr>}が要求される場所(例えば、呼び出し引数や、ブロックの要素)で使用すると、\expr{\$a\{\}}の値を配列にします。そのほかの場合は、配列の宣言を生成します。
	\item[\expr{\$b\{\}}:] \type{Array<Expr> -> Expr} 与えられた配列からブロック式を生成します。
	\item[\expr{\$i\{\}}:] \type{String -> Expr} 与えられた文字列の識別子を生成します。
	\item[\expr{\$p\{\}}:] \type{Array<String> -> Expr} 文字列の配列から、フィールドアクセス式を生成します。
	\item[\expr{\$v\{\}}:] \type{Dynamic -> Expr} その引数の型にあわせて式を作ります。これは\tref{基本型}{types-basic-types}と\tref{列挙型インスタンス}{types-enum-instance}でのみ動作することが保証されています。
\end{description}

加えて\expr{@:pos(p)}\tref{メタデータ}{lf-metadata}を使って、実体化の場所の代わりに\expr{p}に式の位置を対応させられます。

この種類の実体化は式が期待されている場所でのみ動作します。また、\expr{object.\$\{fieldName\}}は動作しませんが、\expr{object.\$fieldName}は動作します。これはすべての文字列を期待する場所で同じです。

\begin{itemize}
	\item フィールドアクセス \expr{object.\$name}
	\item 変数名 \expr{var \$name = 1;}
\end{itemize}

\since{3.1.0}
\begin{itemize}
	\item フィールド名 \expr{\{ \$name: 1\} }
	\item 関数名 \expr{function \$name() \{ \}}
	\item キャッチの変数名 \expr{try e() catch(\$name:Dynamic) \{\}}
\end{itemize}

\subsection{型の実体化}
\label{macro-reification-type}

型の実体化を使うと、手軽に\type{haxe.macro.Expr.ComplexType}のインスタンスを生成できます。構文は\expr{macro : Type}で、\expr{Type}は正当な型のパスの式であれば何でもかまいません。この構文は通常の明示的な型注釈のコードに似ています(例えば、\expr{var x:Type}の変数宣言)。

\type{ComplexType}のコンストラクタごとに、以下の別々の構文があります。

\begin{description}
	\item[\expr{TPath}:] \expr{macro : pack.Type}
	\item[\expr{TFunction}:] \expr{macro : Arg1 -> Arg2 -> Return}
	\item[\expr{TAnonymous}:] \expr{macro : \{ field: Type \}}
	\item[\expr{TParent}:] \expr{macro : (Type)}
	\item[\expr{TExtend}:] \expr{macro : \{> Type, field: Type \}}
	\item[\expr{TOptional}:] \expr{macro : ?Type}
\end{description}

\subsection{クラスの実体化}
\label{macro-reification-class}

\type{haxe.macro.Expr.TypeDefinition}のインスタンスを取得するためにも、実体化は使えます。これには以下のような、\expr{macro class}の構文を使います。

\haxe{assets/ClassReification.hx}

生成された\type{TypeDefinition}のインスタンスは、多くの場合は\expr{haxe.macro.Context.defineType}に渡すことで、呼び出し対象のコンテクストに(マクロコンテクスト自体にではありません)新しい型を追加して使います。

この種類の実体化は\type{TypeDefinition}の\expr{fields}の配列から\expr{haxe.macro.Expr.Field}のインスタンスの取得するのにも便利です。

\section{Tools}
\label{macro-tools}

Haxe標準ライブラリには、マクロの活用を簡単にするツールクラス一式も用意されています。これらのクラスは\tref{静的拡張}{lf-static-extension}で使うのが最適で、\expr{using haxe.macro.Tools}で各コンテクストそれぞれやすべてに持ち込むことができます。

\begin{description}
	\item[\type{ComplexTypeTools}:] \type{ComplexType}のインスタンスを人間が読める形に出力したり、\type{ComplexType}対応する\type{Type}を見つけたりできます。
	\item[\type{ExprTools}:] \type{Expr}のインスタンスを人間が読める形で出力したり、式の繰り返しやマッピングの処理を行ったりできます。
	\item[\type{MacroStringTools}:] マクロコンテクストで有用な文字列と文字列式を扱う処理を提供します。
	\item[\type{TypeTools}:] \type{Type}インスタンスを人間が読める形で出力したり、\tref{単一化}{type-system-unification}や対応する\type{ComplexType}の取得といった型を扱うのに便利な機能を提供します。
\end{description}

さらに\type{haxe.macro.Printer}クラスが、様々な型を人間の読めるフォーマットで出力する\expr{public}メソッドを提供しています。これはマクロのデバッグをするのに便利です。

\trivia{ティンカーベルライブラリとなぜTools.hxが動作するのか}{
モジュールを\expr{using}することでそのすべての型が静的拡張のコンテクストに取り込まれることはこれまでに学んできました。つまるところ、その型というのは他の型指定する\tref{typedef}{type-system-typedef}でも良いわけです。コンパイラはよその型をモジュールの一部と認識して、それが静的拡張にも引き継がれるわけです。\\
このテクニックはJuraj Kirchheimの\emph{tinkerbell}\footnote{https://github.com/back2dos/tinkerbell}で同じ目的で初めて使われました。tinkerbellではHaxeコンパイラと標準ライブラリが提供するよりもずっと先に便利なマクロツールを提供していました。このライブラリは今でも追加のマクロのツールを得るためのライブラリとして第一候補でありつづけており、さらにその他の便利機能も提供しています。}


\section{型ビルド}
\label{macro-type-building}

型ビルドのマクロはいくつかの点で式マクロとは使い方が違います。

\begin{itemize}
	\item 式は返しません。その代わりクラスフィールドの配列を返します。戻り値の型は明示的に\type{Array<haxe.macro.Expr.Field>}を指定しないといけません。
	\item \tref{コンテクスト}{macro-context}にローカルメソッドとローカル変数が含まれません。
	\item コンテクストにビルドフィールドが含まれ、\expr{haxe.macro.Context.getBuildFields()}で使用可能です。
	\item 直接呼び出すのではなく、\tref{class}{types-class-instance}または\tref{enum}{types-enum-instance}の宣言に対する\expr{@:build}または\expr{@:autoBuild}\tref{メタデータ}{lf-metadata}の引数として指定します。
\end{itemize}

以下の例で型ビルドを実演しています。モジュールが\expr{macro}関数を含むとそのモジュールがマクロコンテクストで型付けされてしまうため、2つのファイルに分割していることに気を付けてください。ビルドされる型はビルドマクロが走る前は不完全な状態でしか読み込みがされないので、このことがよく問題になります。型ビルドのマクロは常にそれ用のモジュールに分けて定義することをオススメします。

\haxe{assets/TypeBuildingMacro.hx}
\haxe{assets/TypeBuilding.hx}

\type{TypeBuildingMacro}の\expr{build}メソッドは次の3つのステップを経て動作します。

\begin{enumerate}
	\item \expr{Context.getBuildFields()}を使ってビルドフィールドを取得する。
	\item \expr{funcName}マクロ引数をフィールド名として使って、新しい\type{haxe.macro.expr.Field}を宣言する。このフィールドは\type{String}\tref{変数}{class-field-variable}でデフォルト値は\expr{"my default"}(\expr{kind}フィールドより)で\expr{public static}です(\expr{access}フィールドより)。
	\item 新しいフィールドをビルドフィールドに追加してそれを返す。
\end{enumerate}

このマクロは\type{Main}クラスに対する\expr{@:build}メタデータの引数です。この型が必要になるとコンパイラは以下を行います。

\begin{enumerate}
	\item クラスフィールドも含めて、このモジュールを構文解析する。
	\item \tref{インターフェース}{types-interfaces}や\tref{継承}{types-class-inheritance}などの他の型との関係も含めて、型の設定をする。
	\item \expr{@:build}メタデータに従って、型ビルドのマクロを実行する。
	\item 型ビルドの返したフィールドに従って、クラスの型付けを通常通り続行する。
\end{enumerate}

こうして型ビルドマクロによって思いのままにクラスのフィールドを追加したり、編集したりができます。上の例では、マクロは\expr{"myFunc"}の引数で呼び出されて、\expr{Main.myFunc}を正当なフィールドアクセスにしています。

型ビルドのマクロで何も編集したくない場合、マクロで\expr{null}を返してもかまいません。これでコンパイラに何の変更もしないことが伝わります。\expr{Context.getBuildFields()}を返すよりも好ましいです。



\subsection{列挙型ビルド}
\label{macro-enum-building}

\tref{列挙型}{types-enum-instance}のビルドは、クラスのビルドと類似しており簡単な対応関係があります。

\begin{itemize}
	\item 引数を持たない列挙型は変数フィールド\expr{FVar}です。
	\item 引数を持つ列挙型はメソッドフィールド\expr{FFun}です。
\end{itemize}

\todo{Check if we can build GADTs this way.}

\haxe{assets/EnumBuildingMacro.hx}
\haxe{assets/EnumBuilding.hx}

列挙型\type{E}は\expr{:build}メタデータの修飾されており、呼び出されたマクロが2つのコンストラクタ\expr{A}と\expr{B}を追加しています。\expr{A}は\expr{FVar(null, null)}、つまり引数の無いコンストラクタとして追加されています。\expr{B}は\tref{実体化}{macro-reification-expression}を使って、\type{Int}引数1つを持つ\type{haxe.macro.Expr.Function}を取得しています。

\expr{main}メソッドは\tref{マッチング}{lf-pattern-matching}によって生成された列挙型の構造を証明しています。これで、生成された型が以下と同じだということがわかります。

\begin{lstlisting}
enum E {
	A;
	B(value:Int);
}
\end{lstlisting}


\subsection{@:autoBuild}
\label{macro-auto-build}

クラスが\expr{:autoBuild}メタデータを持つ場合、それを継承するすべてのクラスに\expr{:build}メタデータを生成します。インターフェースが\expr{:autoBuild}メタデータを持つ場合、それを継承するすべてのインターフェースとすべての実装クラスに\expr{:build}メタデータを生成します。\expr{:autoBuild}はそのクラスやインターフェース自身には\expr{:build}を適用しないことに気をつけてください。

\haxe{assets/AutoBuildingMacro.hx}
\haxe{assets/AutoBuilding.hx}

コンパイル中に以下の出力がされます。

\begin{lstlisting}
AutoBuildingMacro.hx:6:
  fromInterface: TInst(I2,[])
AutoBuildingMacro.hx:6:
  fromInterface: TInst(Main,[])
AutoBuildingMacro.hx:11:
  fromBaseClass: TInst(Main,[])
\end{lstlisting}

これらのマクロ実行順序は不定であることを留意しておきましょう、詳しくは\Fullref{macro-limitations-build-order}で説明されています。


\subsection{@:genericBuild}
\label{macro-generic-build}
\since{3.1.0}

通常の\tref{ビルドマクロ}{macro-type-building}は型ごとに実行するもので、これでも十分に強力です。いくつかの用途では、型の\emph{使用}ごと、つまりコードに出現するごとにビルドマクロが走ることが役立つものもあります。何より、これにより具体的な型パラメータにもアクセスできるようになります。

\expr{@:genericBuild}は\expr{@:build}と全くおなじように型に引数付きのマクロ呼び出しを追加することで使用します。

\haxe{assets/GenericBuildMacro1.hx}
\haxe{assets/GenericBuild1.hx}

この例を実行するとコンパイラは\ic{TAbstract(Int,[])}と\ic{TInst(String,[])}を出力することから、\type{MyType}の具体的な型が認識されたことがわかります。このマクロの処理では、この情報をカスタムの型の生成もできます(\expr{haxe.macro.Context.defineType}を使うことで)し、すでに存在する型の参照もできます。簡潔さのためにここでは\expr{null}を返して、コンパイラに型を\tref{推論}{type-system-type-inference}させています。

Haxe 3.1では\expr{@:genericBuild}マクロの戻り値は\type{haxe.macro.Type}でなくてはいけませんでした。Haxe 3.2では、
\type{haxe.macro.ComplexType}を返すことが許可(そして推奨)されています。多くの場合は、型はただパスで参照するだけで動作するのでこのほうが簡単です。

\paragraph{定数型パラメータ}

Haxeでは型パラメータ名が\expr{Const}の場合、\tref{定数値の式}{expression-constants}を型パラメータとして渡すことができます。\expr{@:genericBuild}マクロのコンテクストでマクロに直接情報を渡すのに役立ちます。

\haxe{assets/GenericBuildMacro2.hx}
\haxe{assets/GenericBuild2.hx}

このマクロの処理ではファイルを読み込んで、カスタムの型を生成することができます。

\section{制限}
\label{macro-limitations}
\state{NoContent}

\subsection{マクロ内のマクロ}
\label{macro-limitations-macro-in-macro}

\subsection{静的拡張}
\label{macro-limitations-static-extension}

マクロと\tref{静的拡張}{lf-static-extension}の概念には多少の衝突があります。静的拡張は使用される関数を決定するために既知の型を要求しますが、構文が型付けされる前に実行されます。ですからこの2つの機能を合わせて使うと問題が生じるというのは驚くことではありません。Haxe 3.0では型付けされた式を元の構文の式に戻す変換を試みます。これは必ず成功するわけではなく、重要な情報が失われることもあります。これらについては十分に気を付けたうえで使用することを推奨します。

\since{3.1.0}

静的拡張とマクロの合わせた使用について3.1.0のリリースで修正がされました。Haxeコンパイラはマクロの引数から元の式の復元を試行しなくなった代わりに、特殊な\expr{@:this this}の式を渡すようになりました。式の構造については何の情報も提供しない代わりに正しく型付けができます。

\haxe{assets/MacroStaticExtension.hx}

\subsection{ビルド順序}
\label{macro-limitations-build-order}

型のビルド順序は未定義であり、それは\tref{ビルドマクロ}{macro-type-building}の実行順序についても同じです。いくつかのルールは決まってはいますが、ビルドマクロは実行順序に依存しないようにすることを強く推奨します。もし型ビルドを複数回実行する必要があるなら、マクロのコードで直接解決してください。ビルドマクロが同じ型に対して複数回実行されるのを避けるためには、状態を静的変数にいれておくか、型に\tref{メタデータ}{lf-metadata}を追加するのが有効です。

\haxe{assets/MacroBuildOrder.hx}

\type{I1}と、\type{I2}の両方のインターフェースが\expr{:autoBuild}を持っており、\type{C}クラスに対して2度ビルドマクロが実行されます。ここではクラスに\expr{:processed}メタデータを足して、2度目の実行でそれを確認することで重複した処理を回避しています。


\subsection{型パラメータ}
\label{macro-limitations-type-parameters}


\section{初期化マクロ}
\label{macro-initialization}

初期化マクロはコマンドラインから\expr{--macro callExpr(args)}コマンドを使って呼び出します。これにより、コンテクストを生成した後の、\expr{-main}の引数が型付けされるより前に呼び出されるコールバックを登録します。これにより様々な方法でコンパイラの設定ができます。

\expr{--macro}の引数が単なる識別子の呼び出しだった場合、その識別子はHaxe標準ライブラリの\type{haxe.macro.Compiler}内から検索されます。このクラスには便利な初期化マクロがいくつもあります。詳しくは\href{http://api.haxe.org//haxe/macro/Compiler.html}{API}を記載されています。

例えば、\expr{include}マクロではパッケージをまるまる、必要であれば再帰的にコンパイルに含めることができます。その場合のコマンドライン引数は\expr{--macro include('some.pack', true)}といった形になります。

もちろん、カスタムの初期化マクロを定義して実際のコンパイルの前に様々な作業をさせることもできます。そういったマクロは\expr{--macro some.Class.theMacro(args)}の形で呼び出します。例えば、すべてのマクロに共通の\tref{コンテクスト}{macro-context}が使われるので、初期化マクロで他のマクロの設定のための静的フィールドに値を設定することができます。

\part{Standard Library}
\input{10-std.tex}

\part{Miscellaneous}
\input{11-haxelib.tex}
\input{12-target-details.tex}

\end{document}
