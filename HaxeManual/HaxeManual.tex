\documentclass{../haxe}

% todo-related
\usepackage[left=4.7cm, right=2cm, top=2cm, bottom=4.2cm]{geometry}
\usepackage[draft]{todonotes}
\reversemarginpar

% title (TODO: move this to class file once it looks good)

\renewcommand{\maketitle}{
   \begin{titlepage}
     \setcounter{page}{-1}
			\begin{center}
				~\\[3cm]
				\includegraphics[scale=1.25]{../assets/logo.pdf}~\\[1cm]
				{\huge \bfseries Haxe 3マニュアル}\\[7cm]
				Haxe Foundation\\
				April 12, 2015\\
				(訳 : \today)
			\end{center}
   \end{titlepage}
}


\input{../tikz}

% Conventions:

% run-time, compile-time
% Haxe, Haxelib (unless we are talking about the command itself)
% Haxe Standard Library, Haxe Compiler
% object-oriented

% code example width for ebooks: 47

\begin{document}
\title{Haxe 3マニュアル}
\author{Haxe Foundation}
\date{\today}
\maketitle


\clearpage
\todototoc
\listoftodos
\clearpage

\clearpage
\tableofcontents
\clearpage

\chapter{導入}
\label{introduction}
\state{NoContent}

\section{Haxeって何?}
\label{introduction-what-is-haxe}

\todo{Could we have a big Haxe logo in the First Manual Page (Introduction) under the menu (a bit like a book cover ?) It looks a bit empty now and is a landing page for "Manual"}

Haxeはオープンソースの高級プログラミング言語とコンパイラから成り、ECMAScript\footnote{http://www.ecma-international.org/publications/standards/Ecma-327.htm}を元にした構文のコードさまざまなターゲットの言語へとコンパイルすることを可能にします。適度な抽象化を行うため、1つのコードベースから複数のターゲットへコンパイルすることも可能です。

Haxeは強く型付けされている一方で、必要に応じて型付けを弱めることも可能です。型情報を活用すれば、ターゲットの言語では実行時にしか発見できないようなエラーをコンパイル時に検出することができます。さらに型情報は、ターゲットへの変換時に最適化や堅牢なコードを生成するためにも使用されます。

現在、Haxeには9つのターゲット言語があり、さまざまな用途に利用できます。

\begin{center}
\begin{tabular}{| l | l | l |}
	\hline
	名前 & 出力形式 & 主な用途 \\ \hline
	JavaScript & ソースコード & ブラウザ, デスクトップ, モバイル, サーバー \\
	Neko & バイトコード & デスクトップ, サーバー \\
	PHP & ソースコード & サーバー \\
	Python & ソースコード & デスクトップ, サーバー \\
	C++ & ソースコード & デスクトップ, モバイル, サーバー \\
	ActionScript 3 & ソースコード & ブラウザ, デスクトップ, モバイル \\
	Flash & バイトコード & ブラウザ, デスクトップ, モバイル \\ 
	Java & ソースコード & デスクトップ, サーバー \\
	C\# & ソースコード & デスクトップ, モバイル, サーバー \\ \hline
\end{tabular}
\end{center}

この\Fullref{introduction}の残りでは、Haxeのプログラムがどのようなものなのか、Haxeはが2005年に生まれてからどのように進化してきたのか、を概要でお送りします。

\Fullref{types}では、Haxeの7種類の異なる型についてとそれらがどう関わりあっているのかについて紹介します。型に関する話は、\Fullref{type-system}へと続き、\emph{単一化(Unification)}、\emph{型パラメータ}、\emph{型推論}についての解説がされます。

\Fullref{class-field}では、Haxeのクラスの構造に関する全てをあつかいます。加えて、\emph{プロパティ}、\emph{インラインフィールド}、\emph{ジェネリック関数}についてもあつかいます。

\Fullref{expression}では、\emph{式}を使用して実際にいくつかの動作をさせる方法をお見せします。

\Fullref{lf}では、\emph{パターンマッチング}、\emph{文字列補間}、\emph{デッドコード削除}のようなHaxeの詳細の機能について記述しています。ここで、Haxeの言語リファレンスは終わりです。

そして、Haxeのコンパイラリファレンスへと続きます。まずは\Fullref{compiler-usage}で基本的な内容を、そして、\Fullref{cr-features}で高度な機能をあつかいます。最後に\Fullref{macro}で、ありふれたタスクを\emph{Haxeマクロ}がどのように単純かするのかを見ながら、刺激的なマクロの世界に挑んでいきます。

次の\Fullref{std}のでは、Haxeの標準ライブラリに含まれる主要な型や概念を一つ一つ見ていきます。そして、\Fullref{haxelib}でHaxeのパッケージマネージャであるHaxelibについて学びます。

Haxeは様々なターゲット間の差を吸収してくれますが、場合によってはターゲットを直接的にあつかうことが重要になります。これが、\Fullref{target-details}の話題です。

\section{このドキュメントについて}
\label{introduction-about-this-document}

このドキュメントは、Haxe3の公式マニュアル(の日本語訳)です。そのため、初心者向けののチュートリアルではなく、プログラミングは教えません。しかし、項目は大まかに前から順番に読めるように並べてあり、前に出てきた項目と、次に出てくる項目との関連づけがされています。先の項目で後の項目でててくる情報に触れた方が説明しやすい場所では、先にその情報に触れています。そのような場面ではリンクがされています。リンク先は、ほとんどの場合で先に読むべき内容ではありません。

このドキュメントでは、理論的な要素を実物としてつなげるために、たくさんのHaxeのソースコードを使います。これらのコードのほとんどはmain関数を含む完全なコードでありそのままコンパイルが可能ですが、いくつかはそうではなくコードの重要な部分の抜き出しです。

ソースコードは以下のように示されます:

\begin{lstlisting}
Haxe code here
\end{lstlisting}

時々、Haxeがどのようなコードを出力をするかを見せるため、ターゲットの\target{JavaScript}などのコードも示します。

さらに、このドキュメントではいくつかの単語の定義を行います。定義は主に、新しい型やHaxe特有の単語を紹介するときに行われます。私たちが紹介するすべての新しい内容に対して定義をするわけではありません(例えば、クラスの定義など)。

定義は以下のように示されます。

\define{定義の名前}{define-definition}{定義の説明}

また、いくつかの場所には\emph{トリビア}欄を用意してます。トリビア欄では、Haxeの開発過程でどうしてそのような決定がなされたのか、なぜその機能が過去のHaxeのバージョンから変更されたのかなど非公開の情報をお届けします。この情報は一般的には重要ではない、些細な内容なので読み飛ばしても構いません。

\trivia{トリビアについて}{これはトリビアです}

\subsection{著者と貢献者}
\label{introduction-authors-and-contributions}

このドキュメントの大半の内容は、Haxe Foundation所属のSimon Krajewskiによって書かれました。そして、このドキュメントの貢献者である以下の方々に感謝の意を表します。

\begin{itemize}
	\item Dan Korostelev: 追加の内容と編集
	\item Caleb Harper: 追加の内容と編集
	\item Josefiene Pertosa: 編集
	\item Miha Lunar: 編集
	\item Nicolas Cannasse: Haxe創始者
\end{itemize}

\subsection{ライセンス}
\label{introduction-license}

\href{http://haxe.org/foundation}{Haxe Foundation}制作のHaxeマニュアルは、\href{http://creativecommons.org/licenses/by/4.0/}{クリエイティブコモンズ 表示-4.0-国際 ライセンス}で提供されています。元データは、\href{https://github.com/HaxeFoundation/HaxeManual}{https://github.com/HaxeFoundation/HaxeManual}で管理されています。

\paragraph{日本語訳のライセンス(訳注)}

日本語訳も、\href{http://creativecommons.org/licenses/by/4.0/}{クリエイティブコモンズ 表示-4.0-国際 ライセンス}で提供しています。元データは、\href{https://github.com/HaxeLocalizeJP/HaxeManual}{https://github.com/HaxeLocalizeJP/HaxeManual}で管理されています。

\section{Hello World}
\label{introduction-hello-world}

次のプログラムはコンパイルして実行をすると``Hello World''と表示します:

\haxe{assets/HelloWorld.hx}
<<<<<<< HEAD

上記のコードは、\ic{Main.hx}という名前で保存して、\ic{haxe -main Main --interp}というコマンドでHaxeコンパイラを呼び出すと実際に動作させることが可能です。これで\ic{Main.hx:3: Hello world}という出力がされるはずです。このことから以下のいくつかのことを学ぶことができます。

\todo{This generates the following output: too many 'this'. You may like a passive sentence: the following output will be generated...though this is to be avoided, generally}

\begin{itemize}
	\item Haxeのコードは\ic{.hx}という拡張子で保存する。
	\item Haxeのコンパイラはコマンドラインツールであり、\ic{-main Main}や\ic{--interp}のようなパラメータをつけて呼び出すことができる。
	\item Haxeのプログラムにはクラスがあり(\type{Main}、大文字から始まる)、クラスには関数がある(\expr{main}、小文字からはじまる)。 
	\item Haxeのmainクラスをふくむファイルは、そのクラス名と同じ名前を使う(この場合だと、\type{Main.hx})。
\end{itemize}

\section{歴史}
\label{introduction-haxe-history}
\state{Reviewed}

Haxeのプロジェクトは、2005年10月22日にフランスの開発者の\emph{Nicolas Cannasse}によって、オープンソースのActionScript2コンパイラである\emph{MTASC}(Motion-Twin Action Script Compiler)と、Motion-Twinの社内言語であり、実験的に型推論をオブジェクト指向に取り入れた\emph{MTypes}の後継として始まりました。Nicolasのプログラミング言語の設計に対する長年の情熱と、\emph{Motion-Twin}でゲーム開発者として働くことで異なる技術が混ざり合う機会を得たことが、まったく新しい言語の作成に結び付いたのです。

そのころのつづりは\emph{haXe}で、2006年の2月にAVMのバイトコードとNicolas自身が作成した\emph{Neko}バーチャルマシン\footnote{http://nekovm.org}への出力をサポートするベータ版がリリースされました。

この日からHaxeプロジェクトのリーダーであり続けるNicolas Cannasseは明確なビジョンをもってHaxeの設計を続け、そして2006年5月のHaxe1.0のリリースに導きました。この最初のメジャーリリースから\target{Javascript}のコード生成をサポートの始まり、型推論や構造的部分型などの現在のHaxeの機能のいくつかはすでにこのころからありました。

Haxe1では、2年間いくつかのマイナーリリースを行い、2006年8月に\target{Flash AVM2}ターゲットと\emph{haxelib}ツール、2007年3月に\target{ActionScript3}ターゲットを追加しました。この時期は安定性の改善に強く焦点が当てられ、その結果、数回のマイナーリリースが行われました。

Haxe2.0は2008年7月にリリースされました。\emph{Franco Ponticelli}の好意により、このリリースには\target{PHP}ターゲットが含まれました。同様に、\emph{Hugh Sanderson}の貢献により、2009年7月のHaxe2.04リリースで\target{C++}ターゲットが追加されました。

Haxe1と同じように、以降の数か月で安定性のためのリリースを行いました。そして2011年1月、\emph{macros}をサポートするHaxe2.07がリリースされました。このころに、\emph{Bruno Garcia}が\target{JavaScript}ターゲットのメンテナとしてチームに加わり、 2.08と2.09のリリースで劇的な改善が行われました。

2.09のリリース後、\emph{Simon Krajewski}がチームに加わり、Haxe3の出発に向けて働き始めました。さらに\emph{Cau\^{e} Waneck}の\target{Java}と\target{C\#}のターゲットがHaxeのビルドに取り込まれました。そしてHaxe2は次で最後のリリースとなることが決まり、2012年1月にHaxe2.10がリリースされました。

2012年の終盤、Haxe3にスイッチを切り替えて、Haxeコンパイラチームは、新しく設立された\emph{Haxe Foundation}\footnote{http://haxe-foundation.org}の援助を受けながら、次のメジャーバージョンに向かっていきました。そして、Haxe3は2013年の5月にリリースされました。


\part{言語リファレンス}
\chapter{型}
\label{types}

Haxeコンパイラは豊かな型システムを持っており、これがコンパイル時に型エラーを検出することを手助けします。型エラーとは、文字列による割り算や、整数のフィールドへのアクセス、不十分な(あるいは多すぎる)引数での関数呼び出し、といった型が不正である演算のことです。

いくつかの言語では、この安全性を得るためには各構文での明示的な型の宣言が強いられるので、コストがかかります。

\begin{lstlisting}
var myButton:MySpecialButton = new MySpecialButton(); // AS3
MySpecialButton* myButton = new MySpecialButton(); // C++ 
\end{lstlisting}

一方、Haxeではコンパイラが型を\emph{推論}できるため、この明示的な型注釈は必要ではありません。

\begin{lstlisting}
var myButton = new MySpecialButton(); // Haxe
\end{lstlisting}

型推論の詳細については\Fullref{type-system-type-inference}で説明します。今のところは、上のコードの変数\expr{myButton}は\type{MySpecialButton}の\emph{クラスインスタンス}とわかると言っておけば十分でしょう。

Haxeの型システムは、以下の7つの型を認識します。

\begin{description}
 \item[\emph{クラスインスタンス}:] クラスかインターフェースのオブジェクト
 \item[\emph{列挙型インスタンス}:] Haxeの列挙型(enum)の値
 \item[\emph{構造体}:] 匿名の構造体。つまり、連想配列。
 \item[\emph{関数}:] 引数と戻り値1つの型の複合型。
 \item[\emph{ダイナミック}:] あらゆる型に一致する、なんでも型。
 \item[\emph{抽象(abstract)}:] 実行時には別の型となる、コンパイル時の型。
 \item[\emph{単相}:] 後で別の型が付けられる未知(Unknown)の型。
\end{description}

ここからの節で、それぞれの型のグループとこれらがどうかかわっているのかについて解説していきます。

\define{複合型}{define-compound-type}{
複合型というのは、型の一部として型を持つ型です。\tref{型パラメータ}{type-system-type-parameters}を持つ型や、\tref{関数}{types-function}型がこれに当たります。
}

\section{基本型}
\label{types-basic-types}

\emph{基本型}は\type{Bool}と\type{Float}と\type{Int}です。値の構文は以下のように簡単に区別できます。

\begin{itemize}
	\item \expr{true}と\expr{false}は\type{Bool}。
	\item \expr{1}、\expr{0}、\expr{-1}、\expr{0xFF0000}は\type{Int}。
	\item \expr{1.0}、\expr{0.0}、\expr{-1.0}、\expr{1e10}は\type{Float}。
\end{itemize}

Haxeでは基本型は\tref{クラス}{types-class-instance}ではありません。これらは\tref{抽象型}{types-abstract}として実装されており、以降の項で解説するとおり、コンパイラ内部の演算処理に結び付けられています。

\subsection{数値型}
\label{types-numeric-types}

\define[Type]{Float}{define-float}{IEEEの64bit倍精度浮動小数点数を表します。}

\define[Type]{Int}{define-int}{整数を表します。}

\type{Int}は\type{Float}が期待されるすべての場所で使用することが可能です (IntはFloatへの代入が可能で、Floatとして表現可能です)。しかし、逆はできません。 \type{Float}から\type{Int}への代入は精度を失ってしまう場合があり、信頼できません。

\subsection{オーバーフロー}
\label{types-overflow}

パフォーマンスのためにHaxeコンパイラはオーバーフローに対する挙動を矯正しません。オーバーフローに対する挙動は、ターゲットのプラットフォームが責任を持ちます。各プラットフォームごとのオーバーフローの挙動を以下にまとめています。

\begin{description}
	\item[C++, Java, C\#, Neko, Flash:] 一般的な挙動をもつ32Bit符号付き整数。
	\item[PHP, JS, Flash 8:] ネイティブの\emph{Int}型を持たない。Floatの上限(2\textsuperscript{52})を超えた場合に精度を失う。
\end{description}

代替手段として、プラットフォームごとの追加の計算を行う代わりに、正しいオーバーフローの挙動を持つ\emph{haxe.Int32}と\emph{haxe.Int64}クラスが用意されています。

\subsection{数値の演算子}
\label{types-numeric-operators}

\todo{make sure the types are right for inc, dec, negate, and bitwise negate}
\todo{While introducing the different operations, we should include that information as well, including how they differ with the "C" standard, see http://haxe.org/manual/operators}

以下は、Haxeの数値演算子です。優先度が降順になるようにグループ化して並べています。

\begin{center}
\begin{tabular}{| l | l | l | l | l |}
	\hline
	\multicolumn{5}{|c|}{算術演算} \\ \hline
	演算子 & 演算 & 引数1 & 引数2 & 戻り値 \\ \hline
	\expr{++} & 1増加 & \type{Int} & なし & \type{Int}\\
	& & \type{Float} & なし & \type{Float}\\
	\expr{--} & 1減少 & \type{Int} & なし & \type{Int}\\
	& & \type{Float} & なし & \type{Float}\\
	\expr{+} & 加算 & \type{Float} & \type{Float} & \type{Float} \\
	& & \type{Float} & \type{Int} & \type{Float} \\
	& & \type{Int} & \type{Float} & \type{Float} \\
	& & \type{Int} & \type{Int} & \type{Int} \\
	\expr{-} & 減算 & \type{Float} & \type{Float} & \type{Float} \\
	& & \type{Float} & \type{Int} & \type{Float} \\
	& & \type{Int} & \type{Float} & \type{Float} \\
	& & \type{Int} & \type{Int} & \type{Int} \\
	\expr{*} & 乗算 & \type{Float} & \type{Float} & \type{Float} \\
	& & \type{Float} & \type{Int} & \type{Float} \\
	& & \type{Int} & \type{Float} & \type{Float} \\
	& & \type{Int} & \type{Int} & \type{Int} \\	
	\expr{/} & 除算 & \type{Float} & \type{Float} & \type{Float} \\
	& & \type{Float} & \type{Int} & \type{Float} \\
	& & \type{Int} & \type{Float} & \type{Float} \\
	& & \type{Int} & \type{Int} & \type{Float} \\
	\expr{\%} & 剰余 & \type{Float} & \type{Float} & \type{Float} \\
	& & \type{Float} & \type{Int} & \type{Float} \\
	& & \type{Int} & \type{Float} & \type{Float} \\
	& & \type{Int} & \type{Int} & \type{Int} \\	 \hline
	\multicolumn{5}{|c|}{比較演算} \\ \hline
	演算子 & 演算 & 引数1 & 引数2 & 戻り値 \\ \hline
	\expr{==} & 等価 & \type{Float/Int} & \type{Float/Int} & \type{Bool} \\
	\expr{!=} & 不等価 & \type{Float/Int} & \type{Float/Int} & \type{Bool} \\
	\expr{<} & より小さい & \type{Float/Int} & \type{Float/Int} & \type{Bool} \\
	\expr{<=} & より小さいか等しい & \type{Float/Int} & \type{Float/Int} & \type{Bool} \\
	\expr{>} & より大きい & \type{Float/Int} & \type{Float/Int} & \type{Bool} \\
	\expr{>=} & より大きいか等しい & \type{Float/Int} & \type{Float/Int} & \type{Bool} \\ \hline
	\multicolumn{5}{|c|}{ビット演算} \\ \hline
	演算子 & 演算 & 引数1 & 引数2 & 戻り値 \\ \hline
	\expr{\textasciitilde} & ビット単位の否定(NOT) & \type{Int} & なし & \type{Int} \\	
	\expr{\&} & ビット単位の論理積(AND) & \type{Int} & \type{Int} & \type{Int} \\	
	\expr{|} & ビット単位の論理和(OR) & \type{Int} & \type{Int} & \type{Int} \\	
	\expr{\^} & ビット単位の排他的論理和(XOR) & \type{Int} & \type{Int} & \type{Int} \\	
	\expr{<<} & 左シフト & \type{Int} & \type{Int} & \type{Int} \\
	\expr{>>} & 右シフト & \type{Int} & \type{Int} & \type{Int} \\
	\expr{>>>} & 符号なしの右シフト & \type{Int} & \type{Int} & \type{Int} \\ \hline
	\multicolumn{5}{|c|}{Comparison} \\ \hline
\end{tabular}
\end{center}

\paragraph{等価性}

\emph{enum:}
\begin{description}
	\item[パラメータなしのEnum] 常に同じ値になるので、\expr{MyEnum.A == MyEnum.A}で比較できる。
	\item[パラメータありEnum] \expr{a.equals(b)}で比較できる。 (これは\expr{Type.enumEquals()}の短縮形である)。
\end{description}

\emph{Dynamic:}
1つ以上の\expr{Dynamic}な値に対する比較は、未定義であり、プラットフォーム依存です。

\subsection{Bool(真偽値)}
\label{types-bool}

\define[Type]{Bool}{define-bool}{真(\emph{true})または、偽(\emph{false})のどちらかになる値を表します。}

\type{Bool}型の値は、\tref{\expr{if}}{expression-if}や\tref{\expr{while}}{expression-while}のような\emph{条件文}によく表れます。以下の演算子は、\type{Bool}値を受け取って\type{Bool}値を返します。

\begin{itemize}
	\item \expr{\&\&} (and)
	\item \expr{||} (or)
	\item \expr{!} (not)
\end{itemize}

Haxeは、Bool値の2項演算は、実行時に左から右へ必要な分だけ評価することを保証します。例えば、\expr{A \&\& B}という式は、まず\expr{A}を評価して\expr{A}が\expr{true}だった場合のみ\expr{B}が評価されます。同じように、\expr{A || B}では\expr{A}が\expr{true}だった場合は、\expr{B}の値は意味を持たないので評価されません。

これは、以下のような場合に重要です。

\begin{lstlisting}
if (object != null && object.field == 1) { }
\end{lstlisting}

\expr{object}が\expr{null}の場合に\expr{object.field}にアクセスするとランタイムエラーになりますが、事前に\expr{object != null}のチェックをすることでエラーから守ることができます。

\subsection{Void}
\label{types-void}

\define[Type]{Void}{define-void}{Voidは型が存在しないことを表します。特定の場面(主に関数)で値を持たないことを表現するのに使います。}

Voidは型システムにおける特殊な場合です。Voidは実際には型ではありません。Voidは特に関数の引数と戻り値で型が存在しないことを表現するのに使います。私たちはすでに最初の``Hello World''の例でVoidを使用しています。
\todo{please review, doubled content}

\haxe{assets/HelloWorld.hx}

関数型について詳しくは\Fullref{types-function}で解説しますが、ここで軽く予習をしておきましょう。上の例の\expr{main}関数は\type{Void->Void}型です。これは``引数は無く、戻り値も無い''という意味です。

Haxeでは、フィールドや変数に対してVoidを指定することはできません。以下のように書こうとするとエラーが発生します。
\todo{review please, sounds weird}

\begin{lstlisting}
// 引数と変数の型にVoidは使えません。
var x:Void;
\end{lstlisting}



\section{null許容}
\label{types-nullability}

\define{null許容}{define-nullable}{Haxeでは、ある型が値として\expr{null}をとる場合にnull許容であるとみなす。}

プログラミング言語は、null許容についてなにか1つ明確な定義を持つのが一般的です。ですが、Haxeではターゲットとなる言語の元々の挙動に従うことで妥協しています。ターゲット言語のうちのいくつかは全てがデフォルト値として\expr{null}をとり、その他は特定の型では\expr{null}を許容しません。つまり、以下の2種類の言語を区別しなくてはいけません。

\define{静的ターゲット}{define-static-target}{静的ターゲットでは、その言語自体が基本型が\expr{null}を許容しないような型システムを持っています。この性質は\target{Flash}、\target{C++}、\target{Java}、\target{C\#}ターゲットに当てはまります。}
\define{動的ターゲット}{define-dynamic-target}{動的ターゲットは型に関して寛容で、基本型が\expr{null}を許容します。これは\target{JavaScript}と\target{PHP}、\target{Neko}、\target{Flash 6-8}ターゲットが当てはまります。}
\todo{for starters...please review}

\define{デフォルト値}{define-default-value}{
  基本型は、静的ターゲットではデフォルト値は以下になります。
  \begin{description}
		\item[\type{Int}:] \expr{0}。
		\item[\type{Float}:] \target{Flash}では\expr{NaN}。その他の静的ターゲットでは\expr{0.0}。
		\item[\type{Bool}:] \expr{false}。
	\end{description}
}

その結果、Haxeコンパイラは静的ターゲットでは基本型に対する\expr{null}を代入することはできません。\expr{null}を代入するためには、以下のように基本型を\type{Null$<$T$>$}で囲う必要があります。

\begin{lstlisting}
// 静的プラットフォームではエラー
var a:Int = null;
var b:Null<Int> = null; // こちらは問題ない
\end{lstlisting}

同じように、基本型は\type{Null$<$T$>$}で囲わなければ\expr{null}と比較することはできません。

\begin{lstlisting}
var a : Int = 0;
// 静的プラットフォームではエラー
if( a == null ) { ... }
var b : Null<Int> = 0;
if( b != null ) { ... } // 問題ない
\end{lstlisting}

この制限は\tref{unification}{type-system-unification}が動作するすべての状況でかかります。

\define[Type]{\expr{Null<T>}}{define-null-t}{静的ターゲットでは、\type{Null<Int>}、\type{Null<Float>}、\type{Null<Bool>}の型で\expr{null}を許容することが可能になります。動的ターゲットでは\expr{Null<T>}に効果はありません。また、\expr{Null<T>}はその型が\expr{null}を持つことを表すドキュメントとしても使うことができます。}

nullの値は隠匿されます。つまり、\type{Null$<$T$>$}や\type{Dynamic}のnullの値を基本型に代入した場合には、デフォルト値が使用されます。

\begin{lstlisting}
var n : Null<Int> = null;
var a : Int = n;
trace(a); // 静的プラットフォームでは0
\end{lstlisting}



\subsection{オプション引数とnull許容}
\label{types-nullability-optional-arguments}

null許容について考える場合、オプション引数についても考慮しなくてはいけません。

特に、null許容ではない\emph{ネイティブ}のオプション引数と、それとは異なる、null許容であるHaxe特有のオプション引数があることです。この違いは以下のように、オプション引数にクエスチョンマークを付けることで作ります。

\begin{lstlisting}
// xはネイティブのInt(null許容ではない)
function foo(x : Int = 0) {}
// y is Null<Int> (null許容)
function bar( ?y : Int) {}
// z is also Null<Int>
function opt( ?z : Int = -1) {}
\end{lstlisting}
\todo{Is there a difference between \type{?y : Int} and \type{y : Null$<$Int$>$} or can you even do the latter? Some more explanation and examples with native optional and Haxe optional arguments and how they relate to nullability would be nice.}

\trivia{アーギュメント(Argument)とパラメータ(Parameter)}{他のプログラミング言語では、よく\emph{アーギュメント}と\emph{パラメータ}は同様の意味として使われます。Haxeでは、関数に関連する場合に\emph{アーギュメント}を、\Fullref{type-system-type-parameters}と関連する場合に\emph{パラメータ}を使います。}

\section{クラスインスタンス}
\label{types-class-instance}


多くのオブジェクト指向言語と同じように、Haxeでも大抵のプログラムではクラスが最も重要なデータ構造です。Haxeのすべてのクラスは、明示された名前と、クラスの配置されたパスと、0個以上のクラスフィールドを持ちます。ここではクラスの一般的な構造とその関わり合いについて焦点を当てていきます。クラスフィールドの詳細については後で\Fullref{class-field}の章で解説をします。
\todo{please review future tense}

以下のサンプルコードが、この節で学ぶ基本になります。

\haxe{assets/Point.hx}

意味的にはこれは不連続の2次元空間上の点を表現するものですが、このことはあまり重要ではありません。代わりにその構造に注目しましょう。

\begin{itemize}
	\item \expr{class}のキーワードは、クラスを宣言していることを示すものです。
	\item \type{Point}はクラス名です。\tref{型の識別子のルール}{define-identifier}に従っているものが使用できます。
	\item クラスフィールドは\expr{$\left\{\right\}$}で囲われます。
	\item \type{Int}型の\expr{x}と\expr{y}の2つの\emph{変数}フィールドと、
	\item クラスの\emph{コンストラクタ}となる特殊な\emph{関数}フィールド\expr{new}と、
	\item 通常の関数\expr{toString}でクラスフィールドが構成されています。
\end{itemize}

また、Haxeにはすべてのクラスと一致する特殊な型があります。

\define[Type]{\expr{Class$<$T$>$}}{define-class-t}{
この型はすべてのクラスの型と一致します。つまり、すべてのクラス(インスタンスではなくクラス)をこれに代入することができます。コンパイル時に、\type{Class<T>}は全てのクラスの型の共通の親の型となります。しかし、この関係性は生成されたコードに影響を与えません。

この型は、任意のクラスを要求するようなAPIで役立ちます。例えば、\tref{HaxeリフレクションAPI}{std-reflection}のいくつかのメソッドがこれに当てはまります。
}

\subsection{クラスのコンストラクタ}
\label{types-class-constructor}

クラスのインスタンスは、クラスのコンストラクタを呼び出すことで生成されます。この過程は一般的に\emph{インスタンス化}と呼ばれます。クラスインスタンスは、別名として\emph{オブジェクト}と呼ぶこともあります。ですが、クラス/クラスインスタンスと、列挙型/列挙型インスタンスという似た概念を区別するために、クラスインスタンスと呼ぶことが好まれます。

\begin{lstlisting}
var p = new Point(-1, 65);
\end{lstlisting}

この例で、変数\expr{p}に代入されたのが\type{Point}クラスのインスタンスです。\type{Point}のコンストラクタは\expr{-1}と\expr{65}の2つの引数を受け取り、これらをそれぞれインスタンスの\expr{x}と\expr{y}の変数に代入しています(\Fullref{types-class-instance}で、定義を確認してください)。\expr{new}の正確な意味については、後の\ref{expression-new}の節で再習します。現時点では、\expr{new}はクラスのコンストラクタを呼び、適切なオブジェクトを返すものと考えておきましょう。


\subsection{継承}
\label{types-class-inheritance}

クラスは他のクラスから継承ができます。Haxeでは、継承は\expr{extends}キーワードを使って行います。

\haxe{assets/Point3.hx}

この関係は、よく"BはAである(is-a)"の関係とよく言われます。つまり、すべての\type{Point3}クラスのインスタンスは、同時に\type{Point}のインスタンスである、ということです。\type{Point}は\type{Point3}の\emph{親クラス}であると言い、\type{Point3}は\type{Point}の\emph{子クラス}であると言います。1つのクラスはたくさんの子クラスを持つことができますが、親クラスは1つしか持つことができません。ただし、``クラスXの親クラス''というのは、直接の親クラスだけでなく、親クラスの親クラスや、そのまた親、また親のクラスなどを指すこともよくあります。

上記のクラスは\type{Point}コンストラクタによく似ていますが、2つの新しい構文が登場しています。

\begin{itemize}
	\item \expr{extends Point} は\type{Point}からの継承を意味します。
	\item \expr{super(x, y)} は親クラスのコンストラクタを呼び出します。この場合は\expr{Point.new}です。
\end{itemize}

上の例ではコンストラクタを定義していますが、子クラスで自分自身のコンストラクタを定義する必要はありません。ただし、コンストラクタを定義する場合\expr{super()}の呼び出しが必須になります。他のよくあるオブジェクト指向言語とは異なり、\expr{super()}はコンストラクタの最初である必要はなく、どこで呼び出しても構いません。

また、クラスはその親クラスの\tref{メソッド}{class-field-method}を\expr{override}キーワードを明示して記述することで上書きすることができます。その効果と制限について詳しくは\Fullref{class-field-overriding}であつかいます。


\subsection{インターフェース}
\label{types-interfaces}

インターフェースはクラスのパブリックフィールドを記述するもので、クラスの署名ともいうべきものです。インターフェースは実装を持たず、構造に関する情報のみを与えます。

\begin{lstlisting}
interface Printable {
	public function toString():String;
}
\end{lstlisting}
この構文は以下の点をのぞいて、クラスによく似ています。

\begin{itemize}
	\item \expr{interface}キーワードを\expr{class}キーワードの代わりに使う。
	\item 関数が\tref{式}{expression}を持たない。
	\item すべてのフィールドが型を明示する必要がある。
\end{itemize}

インタフェースは、\tref{構造的部分型}{type-system-structural-subtyping}とは異なり、クラス間の\emph{静的な関係性}について記述します。以下のように明示的に宣言した場合にのみ、クラスはインターフェースと一致します。

\begin{lstlisting}
class Point implements Printable { }
\end{lstlisting}

\expr{implements}キーワードの記述により、"\type{Point}は\type{Printable}である(is-a)"の関係性が生まれます。つまり、すべての\type{Point}のインスタンスは、\type{Printable}のインスタンスでもあります。クラスは親のクラスを1つしか持てませんが、以下のように複数の\expr{implements}キーワードを使用することで複数のインターフェースを実装(implements)することが可能です。

\begin{lstlisting}
class Point implements Printable implements Serializable
\end{lstlisting}

コンパイラは実装が条件を満たしているかの確認を行います。つまり、クラスが実際にインターフェースで要求されるフィールドを実装しているかを確めます。フィールドの実装は、そのクラス自体と、その親となるいずれかのクラスの実装が考慮されます。

インターフェースのフィールドは、変数とプロパティのどちらであるかに対する制限は与えません:

\haxe{assets/InterfaceWithVariables.hx}

インターフェースは\expr{extends}キーワードで複数のインタフェースを継承することができます。

\trivia{Implementsの構文}{Haxeの3.0よりも前のバージョンでは、\expr{implements}キーワードはカンマで区切られていました。Javaのデファクトスタンダードに合わせるため、私たちはカンマを取り除くことに決定しました。これが、Haxe 2と3の間の破壊的な変更の1つです。}

\section{列挙型インスタンス}
\label{types-enum-instance}

Haxeは強力な列挙型(enum)をもっています。この列挙型は実際には\emph{代数的データ型} (ADT)\footnote{\url{https://ja.wikipedia.org/wiki/\%E4\%BB\%A3\%E6\%95\%B0\%E7\%9A\%84\%E3\%83\%87\%E3\%83\%BC\%E3\%82\%BF\%E5\%9E\%8B}}に当たります。列挙型は\tref{式}{expression}を持つことはできませんが、データ構造を表現するのに非常に役に立ちます。

\haxe{assets/Color.hx}

このコードでは、enumは、赤、緑、青のいずれかか、またはRGB値で表現した色、を書き表しています。この文法の構造は以下の通りです。

\begin{itemize}
	\item \expr{enum}キーワードが、列挙型について定義することを宣言しています。
	\item \type{Color}が列挙型の名前です。\tref{型の識別子のルール}{define-identifier}に従うすべてのものが使用できます。
	\item 中カッコ \expr{$\left\{\right\}$} で囲んだ中に\emph{列挙型のコンストラクタ}を記述します。
	\item \expr{Red}と\expr{Green}と\expr{Blue}には引数がありません。
	\item \expr{Rgb}は、\expr{r}、\expr{g}、\expr{b}の3つの\type{Int}型の引数を持ちます。
\end{itemize}

Haxの型システムには、すべての列挙型を統合する型があります。

\define[Type]{\expr{Enum$<$T$>$}}{define-enum-t}{すべての列挙型と一致する型です。コンパイル時に、\type{Enum<T>}は全ての列挙型の共通の親の型となります。しかし、この関係性は生成されたコードに影響を与えません。} 
\todo{Same as in 2.2, what is \type{Enum$<$T$>$} syntax?}

\subsection{列挙型のコンストラクタ}
\label{types-enum-constructor}

クラスと同じように、列挙型もそのコンストラクタを使うことでインスタンス化を行います。しかし、クラスとは異なり列挙型は複数のコンストラクタを持ち、以下のようにコンストラクタの名前を使って呼び出します。

\begin{lstlisting}
var a = Red;
var b = Green;
var c = Rgb(255, 255, 0);
\end{lstlisting}
このコードでは変数\expr{a}、\expr{b}、\expr{c}の型は\type{Color}です。変数\expr{c}は\expr{Rgb}コンストラクタと引数を使って初期化されています。
\todo{list arguments}

すべての列挙型のインスタンスは\type{EnumValue}という特別な型に対して代入が可能です。

\define[Type]{EnumValue}{define-enumvalue}{EnumValueはすべての列挙型のインスタンスと一致する特別な型です。この型はHaxeの標準ライブラリでは、すべての列挙型に対して可能な操作を提供するのに使われます。またユーザーのコードでは、特定の列挙型ではなく任意の列挙型のインスタンスを要求するAPIで利用できます。}

以下の例からわかるように、列挙型とそのインスタンスを区別することは大切です。

\haxe{assets/EnumUnification.hx}

もし、上でコメント化されている行のコメント化が解除された場合、このコードはコンパイルできなくなります。これは、列挙型のインスタンスである\expr{Red}は、列挙型である\type{Enum<Color>}型の変数には代入できないためです。

この関係性は、クラスとそのインスタンスの関係性に似ています。

\trivia{\type{Enum$<$T$>$の型パラメータを具体化する}}{このマニュアルのレビューアの一人は上のサンプルコードの\type{Color}と\type{Enum<Color>}の違いについて困惑しました。実際、型パラメータの具体化は意味のないもので、デモンストレーションのためのものでしかありませんでした。私たちはよく型を書くのを省いて、型についてあつかうのを\tref{型推論}{type-system-type-inference}にまかせてしまいます。

しかし、型推論では\type{Enum<Color>}ではないものが推論されます。コンパイラは、列挙型のコンストラクタをフィールドとしてみなした、仮の型を推論します。現在のHaxe 3.2.0では、この仮の型について表現することは不可能であり、また表現する必要もありません。}


\subsection{列挙型を使う}
\label{types-enum-using}

列挙型は、有限の種類の値のセットが許されることを表現するだけでも有用です。それぞれのコンストラクタについて多様性が示されるので、コンパイラはありうる全ての値が考慮されていることをチェックすることが可能です。これは、例えば以下のような場合です。

\haxe{assets/Color2.hx}

\expr{getColor()}の戻り値を\expr{color}に代入し、その値で\tref{\expr{switch}式}{expression-switch}の分岐を行います。

初めの\expr{Red}、\expr{Green}、\expr{Blue}の3ケースについては些細な内容で、ただColorの引数無しのコンストラクタとの一致するか調べています。最後の\expr{Rgb(r, g, b)}のケースでは、コンストラクタの引数の値をどうやって利用するのかがわかります。引数の値はケースの式の中で出てきたローカル変数として、\tref{\expr{var}の式}{expression-var}を使った場合と同じように、利用可能です。

\expr{switch}の使い方について、より高度な情報は後の\tref{パターンマッチング}{lf-pattern-matching}の節でお話します。


\section{匿名の構造体}
\label{types-anonymous-structure}

匿名の構造体は、型を明示せずに利用できるデータの集まりです。以下の例では、\expr{x}と\expr{name}の2つのフィールドを持つ構造体を生成して、それぞれを\expr{12}と\expr{"foo"}の値で初期化しています。

\haxe{assets/Structure.hx}

構文のルールは以下の通りです :

\begin{enumerate}
	\item 構造体は中カッコ \expr{$\left\{\right\}$} で囲う。
	\item \emph{カンマで区切られた} キーと値のペアのリストを持つ。
	\item \tref{識別子}{define-identifier}の条件を満たすカギと、値が\emph{コロン}で区切られる。
	\item\label{valueanytype} 値には、Haxeのあらゆる式が当てはまる。
\end{enumerate}
ルール\ref{valueanytype}は複雑にネストした構造体を含みます。例えば、以下のような。
\todo{please reformat}

\begin{lstlisting}
var user = {
  name : "Nicolas",
  age : 32,
  pos : [
    { x : 0, y : 0 },
    { x : 1, y : -1 }
  ],
};
\end{lstlisting}
構造体のフィールドは、クラスと同じように、\emph{ドット}(\expr{.})を使ってアクセスします。

\begin{lstlisting}
// 名前を取得する。ここでは"Nicolas"。
user.name;
// ageを33に設定。
user.age = 33;
\end{lstlisting}
特筆すべきは、匿名の構造体の使用は型システムを崩壊させないことです。コンパイラは実際に利用可能なフィールドにしかアクセスを許しません。つまり、以下のようなコードはコンパイルできません。

\begin{lstlisting}
class Test {
  static public function main() {
    var point = { x: 0.0, y: 12.0 };
    // { y : Float, x : Float } has no field z (zフィールドが足りない)
    point.z;
  }
}
\end{lstlisting}
このエラーメッセージはコンパイラが\expr{point}の型を知っていることを表します。この\expr{point}の型は、\expr{x}と\expr{y}の\type{Float}型のフィールドを持つ構造体であり、\expr{z}というフィールドは持たないのでアクセスに失敗しました。
この\expr{point}の型は\tref{型推論}{type-system-type-inference}により識別され、そのおかげでローカル変数では型を明示しなくて済みます。ただし、\expr{point}が、クラスやインスタンスのフィールドだった場合、以下のように型の明示が必要になります。

\begin{lstlisting}
class Path {
    var start : { x : Int, y : Int };
    var target : { x : Int, y : Int };
    var current : { x : Int, y : Int };
}
\end{lstlisting}

このような冗長な型の宣言をさけるため、特にもっと複雑な構造体の場合、以下のように\tref{typedef}{type-system-typedef}を使うことをお勧めします。

\begin{lstlisting}
typedef Point = { x : Int, y : Int }

class Path {
    var start : Point;
    var target : Point;
    var current : Point;
}
\end{lstlisting}


\subsection{JSONで構造体を書く}
\label{types-structure-json}

以下のように、\emph{文字列の定数値}をキーに使う\emph{JavaScript Object Notation(JSON)}の構文を構造体に使うこともできます。

\begin{lstlisting}
var point = { "x" : 1, "y" : -5 };
\end{lstlisting}

キーには\emph{文字列の定数値}すべてが使えますが、フィールドが\tref{Haxeの識別子}{define-identifier}として有効である場合のみ型の一部として認識されます。そして、Haxeの構文では識別子として無効なフィールドにはアクセスできないため、\tref{リフレクション}{std-reflection}の\expr{Reflect.field}と\expr{Reflect.setField}を使ってアクセスしなくてはいけません。

\subsection{構造体の型のクラス記法}
\label{types-structure-class-notation}

構造体の型を書く場合に、Haxeでは\Fullref{class-field}を書くときと同じ構文が使用できます。以下の\tref{typedef}{type-system-typedef}では、\type{Int}型の\expr{x}の\expr{y}変数フィールドを持つ\type{Point}型を定義しています。

\begin{lstlisting}
typedef Point = {
    var x : Int;
    var y : Int;
}
\end{lstlisting}

\subsection{オプションのフィールド}
\label{types-structure-optional-fields}

\todo{I don't really know how these work yet.}

\subsection{パフォーマンスへの影響}
\label{types-structure-performance}

構造体をつかって、さらに\tref{構造的部分型付け}{type-system-structural-subtyping}を使った場合、\tref{動的ターゲット}{define-dynamic-target}ではパフォーマンスに影響はありません。しかし、\tref{静的ターゲット}{define-static-target}では、動的な検査が発生するので通常は静的なフィールドアクセスよりも遅くなります。

\section{関数}
\label{types-function}

\todo{It seems a bit convoluted explanations. Should we maybe start by "decoding" the meaning of  Void -> Void, then Int -> Bool -> Float, then maybe have samples using \$type}

関数の型は、\tref{単相}{types-monomorph}と共に、Haxeのユーザーからよく隠れている型の1つです。コンパイル時に式の型を出力させる\expr{\$type}という特殊な識別子を使えば、この型を以下のように浮かび上がらせることが可能です。

\haxe{assets/FunctionType.hx}

初めの\expr{\$type}の出力は、test関数の定義と強い類似性があります。では、その相違点を見てみます。

\begin{itemize}
	\item \emph{関数の引数}は、カンマではなく\expr{->}で区切られる。
	\item \emph{引数の戻り値}の型は、もう一つ\expr{->}を付けた後に書かれる。
\end{itemize}

どちらの表記でも、\expr{test}関数が1つ目の引数として\type{Int}を受け取り、2つ目の引数として\type{String型}を受け取り、\type{Bool型}の値を返すことはよくわかります。2つ目の\expr{\$type}式の\expr{test(1, "foo")}のようにこの関数を呼び出すと、Haxeの型検査は\expr{1}が\type{Int}に代入可能か、\expr{"foo"}が\type{String}に代入可能かをチェックします。そして、その呼び出し後の型は、\expr{test}の戻り値の型の\type{Bool}となります。

もし、ある関数の型が、別の関数の型を引数か戻り値に含む場合、丸かっこをグループ化に使うことができます。例えば、\type{Int -> (Int -> Void) -> Void}は初めの引数の型が\type{Int}、2番目の引数が\type{Int -> Void}で、戻り値が\type{Void}の関数を表します。

\subsection{オプション引数}
\label{types-function-optional-arguments}

オプション引数は、引数の識別子の直前にクエスチョンマーク(\expr{?})を付けることで表現できます。

\haxe[label=assets/OptionalArguments.hx]{assets/OptionalArguments.hx}

\expr{test}関数は、2つのオプション引数を持ちます。\type{Int}型の\expr{i}と\type{String}型の\expr{s}です。これは3行目の関数型の出力に直接反映されています。

この例では、関数を4回呼び出しその結果を出力しています。

\begin{enumerate}
	\item 初めの呼び出しは引数無し。
	\item 2番目の呼び出しは\expr{1}のみの引数。
	\item 3番目の呼び出しは\expr{1}と\expr{"foo"}の2つの引数。
	\item 4番目の呼び出しは\expr{"foo"}のみの引数。
\end{enumerate}

この出力を見ると、オプション引数が呼び出し時に省略されると\expr{null}になることがわかります。つまり、これらの引数は\expr{null}が入る型でなくてはいけないことになり、ここで\tref{null許容}{types-nullability}に関する疑問が浮かび上がります。Haxeのコンパイラは\tref{静的ターゲット}{define-static-target}に出力する場合に、オプションの基本型の引数の型を\type{Null<T>}であると推論することで、オプション引数の型がnull許容であることを保証してます。

初めの3つの呼び出しは直観的なものですが、4つ目の呼び出しには驚くかもしれません。後の引数に代入可能な値が渡されたため、オプション引数はスキップされています。

\subsection{デフォルト値}
\label{types-function-default-values}

Haxeでは、引数のデフォルト値として定数値を割り当てることが可能です。

\haxe{assets/DefaultValues.hx}
この例は、\Fullref{types-function-optional-arguments}のものとよく似ています。違いは、関数の引数の\expr{i}と\expr{s}それぞれに\expr{12}と\expr{"bar"}を代入していることだけです。これにより、引数が省略された場合に\expr{null}ではなく、このデフォルト値が使われるようになります。

%TODO: Default values do not imply nullability, even if the value is \expr{null}. 

Haxeでのデフォルト値は、型の一部では無いので、出力時に呼び出し元で置き換えられるわけではありません(ただし、特有の動作を行う\tref{インライン}{class-field-inline}の関数を除く)。いくつかのターゲットでは、無視された引数に対してやはり\expr{null}を渡して、以下の関数と同じようなコードを生成します。

\begin{lstlisting}
	static function test(i = 12, s = "bar") {
		if (i == null) i = 12;
		if (s == null) s = "bar";
		return "i: " +i + ", s: " +s;
	}
\end{lstlisting}
つまり、パフォーマンスが要求されるコードでは、デフォルト値を使わない書き方をすることが重要だと考えてください。




\section{ダイナミック}
\label{types-dynamic}

Haxeは静的な型システムを持っていますが、この型システムは\type{Dynamic}型を使うことで事実上オフにすることが可能です。\emph{Dynamicな値}は、あらゆるものに割り当て可能です。逆に、\type{Dynamic}に対してはあらゆる値を割り当て可能です。これにはいくつかの弱点があります。

\begin{itemize}
	\item 代入、関数呼び出しなど、特定の型を要求される場面でコンパイラが型チェックをしなくなります。
	\item 特定の最適化が、特に静的ターゲットにコンパイルする場合に、効かなくなります。
	\item よくある間違い(フィールド名のタイポなど)がコンパイル時に検出できなくなって、実行時のエラーが起きやすくなります。
	\item \Fullref{cr-dce}は、\type{Dynamic}を通じて使用しているフィールドを検出できません。
\end{itemize}

\type{Dynamic}が実行時に問題を起こすような例を考えるのはとても簡単です。以下の2行を静的ターゲットへコンパイルすることを考えてください。

\begin{lstlisting}
var d:Dynamic = 1;
d.foo;
\end{lstlisting}

これをコンパイルしたプログラムを、Flash Playerで実行した場合、\texttt{Number にプロパティ foo が見つからず、デフォルト値もありません。}というエラーが発生します。\type{Dynamic}を使わなければ、このエラーはコンパイル時に検出できます。

\trivia{Haxe 3より前のDynamicの推論}{Haxe 3のコンパイラは型を\type{Dynamic}として推論することはないので、\type{Dynamic}を使いたい場合はそのことを明示しなければ行きません。以前のHaxeのバージョンでは、混ざった型のArrayを\type{Array<Dynamic>}として推論してました(例えば、\expr{[1, true, "foo"]})。私たちはこの挙動はたくさんの型の問題を生み出すことに気づき、この仕様をHaxe 3で取り除きました。}

実際のところ\type{Dynamic}は使ってしまいますが、多くの場面では他のもっと良い選択肢があるので\type{Dynamic}の使用は最低限にすべきです。例えば、Haxeの\Fullref{std-reflection}APIは、コンパイル時には構造のわからないカスタムのデータ構造をあつかう際に最も良い選択肢になりえます。

\type{Dynamic}は、\tref{単相(monomorph)}{types-monomorph}を\tref{単一化}{type-system-unification}する場合に、特殊な挙動をします。以下のような場合に、とんでもない結果を生んでしまうので、単相が\type{Dynamic}に拘束されることはありません。

\haxe{assets/DynamicInferenceIssue.hx}

\expr{Json.parse}の戻り値は\type{Dynamic}ですが、ローカル変数のjsonの型は\type{Dynamic}に拘束されません。単相のままです。そして、\expr{json.length}のフィールドにアクセスした時に\tref{匿名の構造体}{types-anonymous-structure}として推論されて、それにより\expr{json[0]}の配列アクセスでエラーになっています。これは、\expr{json}に対して、\expr{var json:Dynamic}というように明示的に\type{Dynamic}の型付けをすることで避けることができます。

\trivia{標準ライブラリでのDynamic}{DynamicはHaxe 3より前の標準ライブラリではかなり頻繁に表れていましたが、Haxe 3までの継続的な型システムの改善によってDynamicの出現頻度を減らすことができました。}

\subsection{型パラメータ付きのダイナミック}
\label{types-dynamic-with-type-parameter}

\type{Dynamic}は、\tref{型パラメータ}{type-system-type-parameters}を付けても付けなくても良いという点でも特殊な型です。型パラメータを付けた場合、\Fullref{types-dynamic}のすべてのフィールドがパラメータの型であることが強制されます。

\begin{lstlisting}
var att : Dynamic<String> = xml.attributes;
// 正当。値が文字列。
att.name = "Nicolas";
// 同上
att.age = "26";
// エラー。値が文字列ではない。
att.income = 0;
\end{lstlisting}


\subsection{ダイナミックを実装(implements)する}
\label{types-dynamic-implemented}

クラスは\type{Dynamic}と\type{Dynamic$<$T$>$}を\tref{実装}{types-interfaces}することができます。
これにより任意のフィールドへのアクセスが可能になります。\type{Dynamic}の場合、フィールドはあらゆる型になる可能性があり、\type{Dynamic$<$T$>$}の場合、フィールドはパラメータの型と矛盾しない型のみに強制されます。

\haxe{assets/ImplementsDynamic.hx}

\type{Dynamic}を実装しても、 他のインターフェースが要求する実装を満たすことにはなりません。明示的な実装が必要です。

型パラメータなしの\type{Dynamic}を実装したクラスでは、特殊なメソッド\expr{resolve}を利用することができます。\tref{読み込みアクセス}{define-read-access}がありフィールドが存在しなかった場合、\expr{resolve}メソッドが以下のように呼び出されます。

\haxe{assets/DynamicResolve.hx}

\section{抽象型(abstract)}
\label{types-abstract}

抽象(abstract)型は、実行時には別の型になる型です。抽象型は挙動を編集したり強化したりするために、具体型(=抽象型でない型)を``おおう''型を定義するコンパイル時の機能です。

\haxe[firstline=1,lastline=5]{assets/MyAbstract.hx}

上記のコードからは以下を学ぶことができます。

\begin{itemize}
	\item \expr{abstract}キーワードは、抽象型を定義することを宣言している。
	\item \type{AbstractInt}は抽象型の名前であり、型の識別子のルールを満たすものなら何でも使える。
	\item 丸かっこ\expr{()}の中は、その\emph{基底型}の\type{Int}である。
	\item 中カッコ\expr{$\left\{\right\}$}の中はフィールドで、
	\item \type{Int}型の\expr{i}のみを引数とするコンストラクタの\expr{new}関数がある。
\end{itemize}

\define{基底型}{define-underlying-type}{
抽象型の基底型は、実行時にその抽象型を表すために使われる型です。基底型はたいていの場合は具体型ですが、別の抽象型である場合もあります。
}

構文はクラスを連想させるもので、意味合いもよく似ています。実際、抽象型のボディ部分(中カッコの開始以降)は、クラスフィールドとして構文解析することが可能です。抽象型は\tref{メソッド}{class-field-method}と、\tref{実体}{define-physical-field}の無い\tref{プロパティ}{class-field-property}フィールドを持つことが可能です。

さらに、抽象型は以下のように、クラスと同じようにインスタンス化して使用することができます

\haxe[firstline=7,lastline=12]{assets/MyAbstract.hx}

はじめに書いたとおり、抽象型はコンパイル時の機能ですから、見るべきは上記のコードの実際の出力です。この出力例としては、簡潔なコードが出力される\target{JavaScript}が良いでしょう。上記のコードを\texttt{haxe -main MyAbstract -js myabstract.js}でコンパイルすると以下のような\target{JavaScript}が出力されます。

\begin{lstlisting}
var a = 12;
console.log(a);
\end{lstlisting}

抽象型の\type{AbstractInt}は出力から完全に消えてしまい、その基底型の\type{Int}の値のみが残っています。これは、\type{AbstractInt}のコンストラクタがインライン化されて、そのインラインの式が値を\expr{this}に代入します(インライン化については後の\Fullref{class-field-inline}で学びます)。これは、クラスのように考えていた場合、驚くべきことかもしれません。しかし、これこそが抽象型を使って表現したいことそのものです。
抽象型のすべての\emph{インラインのメンバメソッド}では\expr{this}への代入が可能で、これにより``内部の値''が編集できます。

``もしメンバ関数でinlineが宣言されていなかった場合、何が起こるのか?''というのは良い疑問です。そのようなコードははっきりと成立します。その場合、Haxeは実装クラスと呼ばれるprivateのクラスを生成します。この実装クラスは抽象型のメンバ関数を、最初の引数としてその基底型の\expr{this}を加えた静的な(static)関数で持ちます。さらに実装の詳細の話をすると、この実装クラスは\tref{選択的関数}{types-abstract-selective-functions}でも使われます。

\trivia{基本型と抽象型}{抽象型が生まれる前には、基本型はexternクラスと列挙型で実装されていました。\type{Int}型を\type{Float}型の``子クラス''としてあつかうなどのいくつかの面では便利でしたが、一方で問題も引き起こしました。例えば、\type{Float}がexternクラスなので、実際のオブジェクトしか受け入れないはずの空の構造体の型\expr{\{\}}として単一化できました。}



\subsection{暗黙のキャスト}
\label{types-abstract-implicit-casts}

クラスとは異なり抽象型は暗黙のキャストを許します。抽象型には2種類の暗黙のキャストがあります。

\begin{description}
	\item[直接:] 他の型から抽象型への直接のキャストを許します。これは\expr{to}と\expr{from}のルールを抽象型に設定することでできます。 これは、その抽象型の基底型に単一化可能な型のみで利用可能です。
	\item[クラスフィールド:] 特殊なキャスト関数を呼び出すことによるキャストを許します。この関数は\expr{@:to}と\expr{@:from}のメタデータを使って定義されます。この種類のキャストは全ての型で利用可能です。
\end{description}

下のコードは、直接キャストの例です。

\haxe{assets/ImplicitCastDirect.hx}

\expr{from Int}かつ\expr{to Int}の\type{MyAbstract}を定義しました。これは\type{Int}を代入することが可能で、かつ\type{Int}に代入することが可能だという意味です。このことは、9、10行目に表れています。まず、\type{Int}の12を\type{MyAbstract}型の変数\expr{a}に代入しています(これは\expr{from Int}の宣言により可能になります)。そして次に、\type{Int}型の変数\expr{b}に、抽象型のインスタンスを代入しています(これは\expr{to Int}の宣言により可能になります)。

クラスフィールドのキャストも同じ意味を持ちますが、定義の仕方はまったく異なります。

\haxe{assets/ImplicitCastField.hx}

静的な関数に\expr{@:from}を付けることで、その引数の型からその抽象型への暗黙のキャストを行う関数として判断されます。この関数はその抽象型の値を返す必要があります。\expr{static}を宣言する必要もあります。

同じように関数に\expr{@:to}を付けることで、その抽象型からその戻り値の型への暗黙のキャストを行う関数として判断されます。この関数は普通はメンバ関数ですが、\expr{static}でも構いません。そして、これは\tref{選択的関数}{types-abstract-selective-functions}として働きます。

上の例では、\expr{fromString}メソッドが\expr{"3"}の値を\type{MyAbstract}型の変数\expr{a}への代入を可能にし、
\expr{toArray}メソッドがその抽象型インスタンスを\type{Array<Int>}型の変数\expr{b}への代入を可能にします。

この種類のキャストを使った場合、必要な場所でキャスト関数の呼び出しが発生します。このことは\target{JavaScript}出力を見ると明らかです。

\begin{lstlisting}
var a = _ImplicitCastField.MyAbstract_Impl_.fromString("3");
var b = _ImplicitCastField.MyAbstract_Impl_.toArray(a);
\end{lstlisting}

これは2つのキャスト関数で\tref{インライン化}{class-field-inline}を行うことでさらなる最適化を行うことができます。これにより出力は以下のように変わります。

\begin{lstlisting}
var a = Std.parseInt("3");
var b = [a];
\end{lstlisting}


型\expr{A}から時の型\expr{B}への代入の時にどちらかまたは両方が抽象型である場合に使われるキャストの\emph{選択アルゴリズム}は簡単です。

\begin{enumerate}
	\item \expr{A}が抽象型でない場合は3へ。
	\item \expr{A}が、\expr{B}\emph{への}変換を持っている場合、これを適用して6へ。
	\item \expr{B}が抽象型でない場合は5へ。
	\item \expr{B}が、\expr{A}\emph{からの}変換を持っている場合、これを適用して6へ。
	\item 単一化失敗で、終了。
	\item 単一化成功で、終了。
\end{enumerate}

\begin{flowchart}{types-abstract-implicit-casts-selection-algorithm}{選択アルゴリズムのフローチャート}

\tikzset {
	level distance = 1.8cm
}

\tikzstyle{edgeBelow} = [ auto = left, outer sep = 0.2cm ]

\Tree
[.\node [decisionc] (dec1) {\expr{A}は抽象型};
\edge [edgeBelow] node {はい};
[.\node [decisionc] (dec2) {\expr{A}に\expr{B}への\expr{to}変換がある};
\edge [edgeBelow] node {いいえ};
[.\node [decisionc] (dec3) {\expr{B}は抽象型};
\edge [edgeBelow] node {はい};
[.\node [decisionc] (dec4) {\expr{B}に\expr{A}からの\expr{from}変換がある};
\edge [edgeBelow] node {いいえ};
[.\node [startstop, fill = red!70] (fail) {単一化失敗};
]]]]]


\node [startstop, fill = green!70, xshift = 3cm] (success) at (fail.east) {単一化成功};

\tikzstyle{altNode} = [above right, at start]
\tikzstyle{skipNode} = [above left, at start]
\tikzstyle{skipArrow} = [out = 195, in = 165, looseness = 1.6]
\coordinate (altAnchor) at (success.north);

\draw [flowchartArrow] (dec1.west) [skipArrow] to node [skipNode] {いいえ} (dec3.north west);
\draw [flowchartArrow] (dec3.south west) [skipArrow] to node [skipNode] {いいえ} (fail.west);
\draw [flowchartArrow] (dec2) -| (altAnchor) node [altNode] {はい};
\draw [flowchartArrow] (dec4) -| (altAnchor) node [altNode] {はい};

\end{flowchart}

意図的に暗黙のキャストは連鎖的ではありません。これは以下の例でわかります。

\haxe{assets/ImplicitTransitiveCast.hx}

\type{A}から\type{B}、\type{B}から\type{C}への個々のキャストは可能ですが、\type{A}から\type{C}への連鎖的なキャストはできません。これは、キャスト方法が複数生まれてしまうことは避けて、選択アルゴリズムの簡潔さを保つためです。


\subsection{演算子オーバーロード}
\label{types-abstract-operator-overloading}

抽象型ではクラスフィールドに\expr{@:op}メタデータを付けることで、単項演算子と2項演算子のオーバーロードが可能です。

\haxe{assets/AbstractOperatorOverload.hx}

\expr{@:op(A * B)}を宣言することで、\expr{repeat}関数は、左辺が\type{MyAbstract}で右辺が\type{Int}の場合の\expr{*}演算子による乗算の関数として利用されます。これは18行目で利用されています。この部分は\target{JavaScript}にコンパイルすると以下のようになります。

\begin{lstlisting}
console.log(_AbstractOperatorOverload.
  MyAbstract_Impl_.repeat(a,3));
\end{lstlisting}


\tref{クラスフィールドによる暗黙の型変換}{types-abstract-implicit-casts}と同様に、オーバーロードメソッドも要求された場所で呼び出しが発生します。上記の例の\expr{repeat}関数は可換ではありません。\expr{MyAbstract * Int}は動作しますが、\expr{Int * MyAbstract}では動作しません。\expr{Int * MyAbstract}でも動作させたい場合は\expr{@:commutative}のメタデータが使えます。逆に、\expr{MyAbstract * Int}ではなく\expr{Int * MyAbstract}でのみ動作させてたい場合、1つ目の引数で\type{Int}型、2つ目の引数で\type{MyAbstract}型を受け取る静的な関数をオーバーロードメソッドにすることができます。

単項演算子の場合もこれによく似ています。

\haxe{assets/AbstractUnopOverload.hx}

2項演算子と単項演算子の両方とも、戻り値の型は何でも構いません。

\paragraph{基底型の演算を公開する}

基底型が抽象型でそこで許容されている演算子でかつ戻り値を元の抽象型に代入可能なものについては、\expr{@:op}関数のボディを省略することが可能です。

\haxe{assets/AbstractExposeTypeOperations.hx}

\todo{please review for correctness}

\subsection{配列アクセス}
\label{types-abstract-array-access}

配列アクセスは、配列の特定の位置の値にアクセスするのに伝統的に使われている特殊な構文です。これは大抵の場合、\type{Int}のみを引数としますが、抽象型の場合はカスタムの配列アクセスを定義することが可能です。\tref{Haxeの標準ライブラリ}{std}では、これを\type{Map}型に使っており、これには以下の2つのメソッドがあります。
\todo{You have marked ``Map'' for some reason}

\begin{lstlisting}
@:arrayAccess
public inline function get(key:K) {
  return this.get(key);
}
@:arrayAccess
public inline function arrayWrite(k:K, v:V):V {
	this.set(k, v);
	return v;
}
\end{lstlisting}

配列アクセスのメソッドは以下の2種類があります。

\begin{itemize}
	\item \expr{@:arrayAccess}メソッドが1つの引数を受け取る場合、それは読み取り用です。
	\item \expr{@:arrayAccess}メソッドが2つの引数を受け取る場合、それは書き込み用です。
\end{itemize}

上記のコードの\expr{get}メソッドと\expr{arrayWrite}メソッドは、以下のように使われます。

\haxe{assets/AbstractArrayAccess.hx}

ここでは以下のように出力に配列アクセスのフィールドの呼び出しが入ることになりますが、驚かないでください。

\begin{lstlisting}
map.set("foo",1);
console.log(map.get("foo")); // 1
\end{lstlisting}

\paragraph{配列アクセスの解決順序}
\label{types-abstract-array-access-order}

Haxe 3.2以前では、バグのため\expr{:arrayAccess}のフィールドがチェックされる順序は定義されていませんでした。これは、Haxe 3.2では修正されて一貫して上から下へと確認が行われるようになりました。

\haxe{assets/AbstractArrayAccessOrder.hx}

\expr{a[0]}の配列アクセスは、\expr{getInt1}のフィールドとして解決されて、小文字の\expr{f}が返ります。バージョン3.2以前のHaxeでは、結果が異なる場合があります。

\tref{暗黙のキャスト}{types-abstract-implicit-casts}が必要な場合であっても、先に定義されている方が優先されます。

\subsection{選択的関数}
\label{types-abstract-selective-functions}

コンパイラは抽象型のメンバ関数を静的な(static)関数へと変化させるので、手で静的な関数を記述してそれを抽象型のインスタンスで使うことができます。この意味は、関数の最初の引数の型で、その関数が使えるようになる\tref{静的拡張}{lf-static-extension}に似ています。

\haxe{assets/SelectiveFunction.hx}

抽象型の\type{MyAbstract}の\expr{getString}のメソッドは、最初の引数として\type{MyAbstract$<$String$>$}を受け取ります。これにより、14行目の変数\expr{a}の関数呼び出しが可能になります(\expr{a}の型が\type{MyAbstract$<$String$>$}なので)。しかし、\type{MyAbstract$<$Int$>$}の変数\expr{b}では使えません。

\trivia{偶然の機能}{
実際のところ選択的関数は意図して作られたというよりも、発見された機能です。この機能について初めて言及されてから実際に動作させるまでに必要だったのは軽微な修正のみでした。この発見が、Mapのような複数の型の抽象型にもつながっています。}


\subsection{抽象型列挙体}
\label{types-abstract-enum}
\since{3.1.0}

抽象型の宣言に\expr{@:enum}のメタデータを追加することで、その値を有限の値のセットを定義して使うことができます。

\haxe{assets/AbstractEnum.hx}

以下の\target{JavaScript}への出力を見ても明らかなように、Haxeコンパイラは抽象型\type{HttpStatus}の全てのフィールドへのアクセスをその値に変換します。

\begin{lstlisting}
Main.main = function() {
	var status = 404;
	var msg = Main.printStatus(status);
};
Main.printStatus = function(status) {
	switch(status) {
	case 404:
		return "Not found";
	case 405:
		return "Method not allowed";
	}
};
\end{lstlisting}

これは\tref{インライン変数}{class-field-inline}によく似ていますが、いくつかの利点があります。

\begin{itemize}
	\item コンパイラがそのセットのすべての値が正しく型付けされていることを保証できます。
	\item パターンマッチで、抽象型列挙体への\tref{マッチング}{lf-pattern-matching}を行う場合に\tref{網羅性}{lf-pattern-matching-exhaustiveness}がチェックされます。
	\item 少ない構文でフィールドを定義できます。
\end{itemize}


\subsection{抽象型フィールドの繰り上げ}
\label{types-abstract-forward}
\since{3.1.0}

基底型をラップした場合、その機能性のを``保ちたい''場合があります。繰り上がりの関数を手で書くのは面倒なので、Haxeでは\expr{@:forward}メタデータを利用できるようにしています。

\haxe{assets/AbstractExpose.hx}

この例では、抽象型の\type{MyArray}が\type{Array}をラップしています。この\expr{@:forward}メタデータは、基底型から繰り上がらせるフィールド2つを引数として与えられています。上記の例の\expr{main}関数は、\type{MyArray}をインスタンス化して、その\expr{push}と\expr{pop}のメソッドにアクセスしています。コメント化されている行は、\expr{length}フィールドは利用できないことを実演するものです。

ではどのようなコードが出力されるのか、いつものように\target{JavaScript}への出力を見てみましょう。

\begin{lstlisting}
Main.main = function() {
	var myArray = [];
	myArray.push(12);
	myArray.pop();
};
\end{lstlisting}

全てのフィールドを繰り上げる場合は、引数なしの\expr{@:forward}を利用できます。もちろんこの場合でも、Haxeコンパイラは基底型にそのフィールドが存在していることを保証します。

\trivia{マクロとして実装}{\expr{@:enum}と\expr{@:forward}の両機能は、もともとは\tref{ビルドマクロ}{macro-type-building}を利用して実装していました。この実装はマクロなしのコードから使う場合はうまく動作していましたが、マクロからこれらの機能を使った場合に問題を起こしました。このため、これらの機能はコンパイラへと移されました。}


\subsection{コアタイプの抽象型}
\label{types-abstract-core-type}

Haxeの標準ライブラリは、基本型のセットをコアタイプの抽象型として定義しています。これらは\expr{@:coreType}メタデータを付けることで識別されて、基底型の定義を欠きます。これらの抽象型もまた異なる型の表現として考えることができます。
そして、その型はHaxeのターゲットのネイティブの型です。

カスタムのコアタイプの抽象型の導入は、Haxeのターゲットにその意味を理解させる必要があり、ほとんどのユーザーのコードで必要ないでしょう。ですが、マクロを使いたい人や、新しいHaxeのターゲットを作りたい人にとっては興味深い利用例があります

コアタイプの抽象型は、不透過の抽象型(他の型をラップする抽象型のこと)とは異なる以下の性質をもちます。

\begin{itemize}
	\item 基底型を持たない。
	\item \expr{@:notNull}メタデータの注釈を付けない限り、null許容としてあつかわれる。
	\item 式の無い\tref{配列アクセス}{types-abstract-array-access}関数を定義できる。
	\item Haxeの制限から離れた、式を持たない\tref{演算子オーバーロードのフィールド}{types-abstract-operator-overloading}が可能。
\end{itemize}



\section{単相(モノモーフ)}
\label{types-monomorph}

単相は、\tref{単一化}{type-system-unification}の過程で、他の異なる型へと形を変える型です。これについて詳しくは\tref{型推論}{type-system-type-inference}の節で話します。

\chapter{型システム}
\label{type-system}

私たちは\Fullref{types}の章でさまざまな種類の型について学んできました。ここからはそれらがお互いにどう関連しあっているかを見ていく時間です。まず、複雑な型に対して名前(別名)を与える仕組みである\tref{Typedef}{type-system-typedef}の紹介から簡単に始めます。typedefは特に、\tref{型パラメータ}{type-system-type-parameters}を持つ型で役に立ちます。

任意の2つの型について、その上位にある型のグループが矛盾しないかをチェックすることで多くの型安全性が得られます。これがコンパイラが試みる\emph{単一化}であり、\Fullref{type-system-unification}の節で詳しく説明します。

すべての型は\emph{モジュール}に所属し、\emph{パス}を通して呼び出されます。\Fullref{type-system-modules-and-paths}では、これらに関連した仕組みについて詳しい説明を行います。

\section{Typedef}
\label{type-system-typedef}

typedefは\tref{匿名構造体}{types-anonymous-structure}の節で、すでに登場しています。そこでは複雑な構造体の型について名前を与えて簡潔にあつかう方法を見ています。この利用法はtypedefが一体なにに良いのかを的確に表しています。構造体の型に対して名前を与えるのは、typedefの主たる用途かもしれません。実際のところ、この用途が一般的すぎて、多くのHaxeユーザーがtypdefを構造体のためのものだと思ってしまっています。

typedefは他のあらゆる型に対して名前を与えることが可能です。

\begin{lstlisting}
typedef IA = Array<Int>;
\end{lstlisting}

これにより\expr{Array$<$Int$>$}が使われる場所で、代わりに\expr{IA}を使うことが可能になります。この場合、はほんの数回のタイプ数しか減らせませんが、より複雑な複合型の場合は違います。これこそが、typedefと構造体が強く結びついて見える理由です。

\begin{lstlisting}
typedef User = {
    var age : Int;
    var name : String;
}
\end{lstlisting}

typedefはテキスト上の置き換えではなく、実は本物の型です。Haxe標準ライブラリの\type{Iterable}のように\tref{型パラメータ}{type-system-type-parameters}を持つことができます。

\begin{lstlisting}
typedef Iterable<T> = {
	function iterator() : Iterator<T>;
}
\end{lstlisting}

\subsection{拡張}
\label{type-system-extensions}

% TODO: move to structures? %

拡張は、構造体が与えられた型のフィールドすべてと、加えていくつかのフィールドを持っていることを表すために使われます。

\haxe{assets/Extension.hx}
大なりの演算子を使うことで、追加のクラスフィールドを持つ\type{Iterable$<$T$>$}の拡張が作成されました。このケースでは、読み込み専用の\tref{プロパティ}{class-field-property} である\type{Int}型の\expr{length}が要求されます。 

\type{IterableWithLength$<$T$>$}に適合するためには、\type{Iterable$<$T$>$}にも適合してさらに読み込み専用の\type{Int}型のプロパティ\expr{length}を持ってなきゃいけません。例では、Arrayが割り当てられており、これはこれらの条件をすべて満たしています。

\since{3.1.0}

複数の構造体を拡張することもできます。

\haxe{assets/Extension2.hx}

\section{型パラメータ}
\label{type-system-type-parameters}

\tref{クラスフィールド}{class-field}や\tref{列挙型コンストラクタ}{types-enum-constructor}のように、Haxeではいくつかの型についてパラメータ化を行うことができます。型パラメータは山カッコ\expr{$<>$}内にカンマ区切りで記述することで、定義することができます。シンプルな例は、Haxe標準ライブラリの\type{Array}です。

\begin{lstlisting}
class Array<T> {
	function push(x : T) : Int;
}
\end{lstlisting}
\type{Array}のインスタンスが作られると、型パラメータ\type{T}は単相となります。つまり、1度に1つの型であれば、あらゆる型を適用することができます。この適用は以下の方法で行います

\begin{description}
	\item[明示的に、]\expr{new Array$<$String$>$()}のように型を記述してコンストラクタを呼び出して適用する。
	\item[暗黙に]、\tref{型推論}{type-system-type-inference}で適用する。例えば、\expr{arrayInstance.push("foo")}を呼び出す。
\end{description}

型パラメータが付くクラスの定義の内部では、その型パラメータは不定の型となります。\tref{制約}{type-system-type-parameter-constraints}が追加されない限り、コンパイラはその型パラメータはあらゆる型になりうるものと決めつけることになります。その結果、型パラメータの\tref{cast}{expression-cast}を使わなければ、その型のフィールドにアクセスできなくなります。また、\tref{一般化}{type-system-generic}をして適切な制約をつけない限り、その型パラメータの型の新しいインスタンスを生成することもできません。

以下は、型パラメータが使用できる場所についての表です。

\begin{center}
\begin{tabular}{| l | l | l |}
	\hline
	パラメータが付く場所 & 型を適用する場所 & 備考 \\ \hline
	Class & インスタンス作成時 & メンバフィールドにアクセスする際に型を適用することもできる \\
	Enum & インスタンス作成時 & \\
	Enumコンストラクタ & インスタンス作成時 & \\
	関数 & 呼び出し時 & メソッドと名前付きのローカル関数で利用可能	\\
	構造体 & インスタンス作成時 & \\ \hline
\end{tabular}
\end{center}

関数の型パラメータは呼び出し時に適用される、この型パラメータは(制約をつけない限り)あらゆる型を許容します。しかし、一回の呼び出しにつき適用は1つの型のみ可能です。このことは関数が複数の引数を持つ場合に役立ちます。

\haxe{assets/FunctionTypeParameter.hx}

\expr{equals}関数の\expr{expected}と\expr{actual}の引数両方が、\type{T}型になっています。これはすべての\expr{equals}の呼び出しで、2つの引数の型が同じでなければならないことを表しています。コンパイラは最初(両方の引数が\type{Int}型)と2つめ(両方の引数が\type{String}型)の呼び出しは認めていますが、3つ目の呼び出しはコンパイルエラーにします。

\trivia{式の構文内での型パラメータ}{なぜ、\expr{method<String>(x)}のようにメソッドに型パラメータをつけた呼び出しができないのか?という質問をよくいただきます。このときのエラーメッセージはあまり参考になりませんが、これには単純な理由があります。それは、このコードでは、\expr{<}と\expr{>}の両方が2項演算子として構文解析されて、\expr{(method < String) > (x)}と見なされるからです。}

\subsection{制約}
\label{type-system-type-parameter-constraints}

型パラメータは複数の型で制約を与えることができます。

\haxe{assets/Constraints.hx}

\expr{test}メソッドの型パラメータ\type{T}は、\type{Iterable$<$String$>$}と\type{Measurable}の型に制約されます。\type{Measurable}の方は、便宜上\tref{typedef}{type-system-typedef}を使って、\type{Int}型の読み込み専用\tref{プロパティ}{class-field-property}\expr{length}を要求しています。つまり、以下の条件を満たせば、これらの制約と矛盾しません。

\begin{itemize}
	\item \type{Iterable$<$String$>$}である
	\item かつ、\type{Int}型の\expr{length}を持つ
\end{itemize}

7行目では空の配列で、8行目では\type{Array$<$String$>$}で\expr{test}関数を呼び出すことができることを確認しました。しかし、10行目の\type{String}の引数では制約チェックで失敗しています。これは、\type{String}は\type{Iterable$<$T$>$}と矛盾するからです。

\section{ジェネリック}
\label{type-system-generic}

大抵の場合、Haxeコンパイラは型パラメータが付けられていた場合でも、1つのクラスや関数を生成します。これにより自然な抽象化が行われて、ターゲット言語のコードジェネレータは出力先の型パラメータはあらゆる型になりえると思い込むことになります。つまり、生成されたコードで型チェックが働き、動作が邪魔されることがあります。

クラスや関数は、\expr{:generic} \tref{メタデータ}{lf-metadata}で\emph{ジェネリック}属性をつけることで一般化することができます。これにより、コンパイラは型パラメータの組み合わせごとのクラスまたは関数を修飾された名前で書き出します。このような設計は\tref{静的ターゲット}{define-static-target}で出力サイズの巨大化と引き換えに、パフォーマンスが重要なコード部位での速度を得られます。

\haxe{assets/GenericClass.hx}

あまり使わない明示的な\type{MyArray<String>}の型宣言があり、よく使う\tref{型推論}{type-system-type-inference}であつかわせていますが、これが重要です。コンパイラは、コンストラクタの呼び出し時にジェネリッククラスの正確な型な型を知っている必要があります。この\target{JavaScript}出力は以下のような結果になります。

\begin{lstlisting}
(function () { "use strict";
var Main = function() { }
Main.main = function() {
	var a = new MyArray_String();
	var b = new MyArray_Int();
}
var MyArray_Int = function() {
};
var MyArray_String = function() {
};
Main.main();
})();
\end{lstlisting}

\type{MyArray<String>}と\type{MyArray<Int>}は、それぞれ\type{MyArray_String}と\type{MyArray_Int}になっています。これはジェネリック関数でも同じです。

\haxe{assets/GenericFunction.hx}

\target{JavaScript}出力を見れば明白です。

\begin{lstlisting}
(function () { "use strict";
var Main = function() { }
Main.method_Int = function(t) {
}
Main.method_String = function(t) {
}
Main.main = function() {
	Main.method_String("foo");
	Main.method_Int(1);
}
Main.main();
})();
\end{lstlisting}


\subsection{ジェネリック型パラメータのコンストラクト}
\label{type-system-generic-type-parameter-construction}

\define{ジェネリック型パラメータ}{define-generic-type-parameter}{型パラメータを持っているクラスまたはメソッドがジェネリックであるとき、その型パラメータもジェネリックであるという。}

普通の型パラメータでは、\expr{new T()}のようにその型をコンストラクトすることはできません。これは、Haxeが1つの関数を生成するために、そのコンストラクトが意味をなさないからです。しかし、型パラメータがジェネリックの場合は違います。これは、コンパイラはすべての型パラメータの組み合わせに対して別々の関数を生成しています。このため\expr{new T()}の\type{T}を実際の型に置き換えることができます。

\haxe{assets/GenericTypeParameter.hx}

ここでは、\type{T}の実際の型の決定には、\tref{トップダウンの推論}{type-system-top-down-inference}が行われることに注意してください。この方法での型パラメータのコンストラクトを行うには2つの必須事項があります。

\begin{enumerate}
	\item ジェネリックであること
	\item 明示的に、\tref{コンストラクタ}{types-class-constructor}を持つように\tref{制約}{type-system-type-parameter-constraints}されていること
\end{enumerate}

先ほどの例は、1つ目は\expr{make}が\expr{@:generic}メタデータを持っており、2つ目\type{T}が\type{Constructible}に制約されています。\type{String}と\type{haxe.Template}の両方とも1つ\type{String}の引数のコンストラクタを持つのでこの制約に当てはまります。確かにJavascript出力は予測通りのものになっています。

\begin{lstlisting}
var Main = function() { }
Main.__name__ = true;
Main.make_haxe_Template = function() {
	return new haxe.Template("foo");
}
Main.make_String = function() {
	return new String("foo");
}
Main.main = function() {
	var s = Main.make_String();
	var t = Main.make_haxe_Template();
}
\end{lstlisting}

\section{Variance}
\label{type-system-variance}

While variance is relevant in other places, it occurs particularly often with type parameters and often comes as a surprise in this context. It is also very easy to trigger variance errors:

\haxe{assets/Variance.hx}

Apparently, an \type{Array<Child>} cannot be assigned to an \type{Array<Base>}, even though \type{Child} can be assigned to \type{Base}. The reason for this might be somewhat unexpected: It is not allowed because arrays can be written to, e.g. via their \expr{push()} method. It is easy to generate problems by ignoring variance errors:

\haxe{assets/Variance2.hx}

What happens here is that we subvert the type checker by using a \tref{cast}{expression-cast}, thus allowing the assignment in line 12. With that we hold a reference \expr{bases} to the original array, typed as \type{Array<Base>}. This allows pushing another type compatible with \type{Base}, \type{OtherChild}, onto that array. However, our original reference \expr{children} is still of type \type{Array<Child>}, and things go bad when we encounter the \type{OtherChild} instance in one of its elements while iterating.

If \type{Array} had no \expr{push()} method and no other means of modification, the assignment would be safe because no incompatible type could be added to it. We can achieve this in Haxe by restricting the type accordingly using \tref{structural subtyping}{type-system-structural-subtyping}:

\haxe{assets/Variance3.hx}

With \expr{b} being typed as \type{MyArray<Base>} and \type{MyArray} only having a \expr{pop()} method, we can safely assign. There is no method defined on \type{MyArray} which could be used to add incompatible types, it is thus said to be \emph{covariant}.

\define{Covariance}{define-covariance}{A \tref{compound type}{define-compound-type} is considered covariant if its component types can be assigned to less specific components, i.e. if they are only read, but never written.}

\define{Contravariance}{define-contravariance}{A \tref{compound type}{define-compound-type} is considered contravariant if its component types can be assigned to less generic components, i.e. if they are only written, but never read.}




\section{Unification}
\label{type-system-unification}

\todo{Mention toString()/String conversion somewhere in this chapter.}

Unification is the heart of the type system and contributes immensely to the robustness of Haxe programs. It describes the process of checking if a type is compatible to another type.

\define{Unification}{define-unification}{Unification between two types A and B is a directional process which answers the question if A \emph{can be assigned to} B. It may \emph{mutate} either type if it is or has a \tref{monomorph}{types-monomorph}.}

Unification errors are very easy to trigger:

\begin{lstlisting}
class Main {
	static public function main() {
    // Int should be String
		var s:String = 1;
	}
}
\end{lstlisting}
We try to assign a value of type \type{Int} to a variable of type \type{String}, which causes the compiler to try and \emph{unify Int with String}. This is, of course, not allowed and makes the compiler emit the error \expr{Int should be String}.

In this particular case, the unification is triggered by an \emph{assignment}, a context in which the ``is assignable to'' definition is intuitive. It is one of several cases where unification is performed:

\begin{description}
	\item[Assignment:] If \expr{a} is assigned to \expr{b}, the type of \expr{a} is unified with the type of \expr{b}.
	\item[Function call:] We have briefly seen this one while introducing the \tref{function}{types-function} type. In general, the compiler tries to unify the first given argument type with the first expected argument type, the second given argument type with the second expected argument type and so on until all argument types are handled.
	\item[Function return:] Whenever a function has a \expr{return e} expression, the type of \expr{e} is unified with the function return type. If the function has no explicit return type, it is infered to the type of \expr{e} and subsequent \expr{return} expressions are infered against it.
	\item[Array declaration:] The compiler tries to find a minimal type between all given types in an array declaration. Refer to \Fullref{type-system-unification-common-base-type} for details.
	\item[Object declaration:] If an object is declared ``against'' a given type, the compiler unifies each given field type with each expected field type.
	\item[Operator unification:] Certain operators expect certain types which given types are unified against. For instance, the expression \expr{a \&\& b} unifies both \expr{a} and \expr{b} with \type{Bool} and the expression \expr{a == b} unifies \expr{a} with \expr{b}.
\end{description}


\subsection{Between Class/Interface}
\label{type-system-unification-between-classes-and-interfaces}

When defining unification behavior between classes, it is important to remember that unification is directional: We can assign a more specialized class (e.g. a child class) to a generic class (e.g. a parent class), but the reverse is not valid.

The following assignments are allowed:

\begin{itemize}
	\item child class to parent class
	\item class to implementing interface
	\item interface to base interface
\end{itemize}
These rules are transitive, meaning that a child class can also be assigned to the base class of its base class, an interface its base class implements, the base interface of an implementing interface and so on.
\todo{''parent class'' should probably be used here, but I have no idea what it means, so I will refrain from changing it myself.}

\subsection{Structural Subtyping}
\label{type-system-structural-subtyping}

\define{Structural Subtyping}{define-structural-subtyping}{Structural subtyping defines an implicit relation between types that have the same structure.}

In Haxe, structural subtyping is only possible when assigning a class instance to a structure. The following example is part of the \type{Lambda} class of the \tref{Haxe Standard Library}{std}:

\begin{lstlisting}
public static function
empty<T>(it : Iterable<T>):Bool {
	return !it.iterator().hasNext();
}
\end{lstlisting}
The \expr{empty}-method checks if an \type{Iterable} has an element. For this purpose, it is not necessary to know anything about the argument type other than the fact that it is considered an iterable. This allows calling the \expr{empty}-method with any type that unifies with \type{Iterable$<$T$>$}, which applies to a lot of types in the Haxe Standard Library.

This kind of typing can be very convenient, but extensive use may be detrimental to performance on static targets, which is detailed in \Fullref{types-structure-performance}.


\subsection{Monomorphs}
\label{type-system-monomorphs}

Unification of types having or being a \tref{monomorph}{types-monomorph} is detailed in \Fullref{type-system-type-inference}.


\subsection{Function Return}
\label{type-system-unification-function-return}

Unification of function return types may involve the \tref{\type{Void}-type}{types-void} and require a clear definition of what unifies with \type{Void}. With \type{Void} describing the absence of a type, it is not assignable to any other type, not even \type{Dynamic}. This means that if a function is explicitly declared as returning \type{Dynamic}, it must not return \type{Void}.

The opposite applies as well: If a function declares a return type of \type{Void}, it cannot return \type{Dynamic} or any other type. However, this direction of unification is allowed when assigning function types:

\begin{lstlisting}
var func:Void->Void = function() return "foo";
\end{lstlisting}
The right-hand function clearly is of type \type{Void->String}, yet we can assign it to variable \expr{func} of type \type{Void->Void}. This is because the compiler can safely assume that the return type is irrelevant, given that it could not be assigned to any non-\type{Void} type.


\subsection{Common Base Type}
\label{type-system-unification-common-base-type}

Given a set of multiple types, a \emph{common base type} is a type which all types of the set unify against:

\haxe{assets/UnifyMin.hx}
Although \type{Base} is not mentioned, the Haxe Compiler manages to infer it as the common type of \type{Child1} and \type{Child2}. The Haxe Compiler employs this kind of unification in the following situations:

\begin{itemize}
	\item array declarations
	\item \expr{if}/\expr{else}
	\item cases of a \expr{switch}
\end{itemize}




\section{Type Inference}
\label{type-system-type-inference}

The effects of type inference have been seen throughout this document and will continue to be important. A simple example shows type inference at work:

\haxe{assets/TypeInference.hx}
The special construct \expr{\$type} was previously mentioned in order to simplify the explanation of the \Fullref{types-function} type, so let us introduce it officially now:

%TODO: $type
\define[Construct]{\expr{\$type}}{define-dollar-type}{\expr{\$type} is a compile-time mechanism being called like a function, with a single argument. The compiler evaluates the argument expression and then outputs the type of that expression.}

In the example above, the first \expr{\$type} prints \expr{Unknown<0>}. This is a \tref{monomorph}{types-monomorph}, a type that is not yet known. The next line \expr{x = "foo"} assigns a \type{String} literal to \expr{x}, which causes the \tref{unification}{type-system-unification} of the monomorph with \type{String}. We then see that the type of \expr{x} indeed has changed to \type{String}.

Whenever a type other than \Fullref{types-dynamic} is unified with a monomorph, that monomorph \emph{becomes} that type: it \emph{morphs} into that type. Therefore it cannot morph into a different type afterwards, a property expressed in the \emph{mono} part of its name.

Following the rules of unification, type inference can occur in compound types:

\haxe{assets/TypeInference2.hx}
Variable \expr{x} is first initialized to an empty \type{Array}. At this point we can tell that the type of \expr{x} is an array, but we do not yet know the type of the array elements. Consequentially, the type of \expr{x} is \type{Array<Unknown<0>>}. It is only after pushing a \type{String} onto the array that we know the type to be \type{Array<String>}.


\subsection{Top-down Inference}
\label{type-system-top-down-inference}

Most of the time, types are inferred on their own and may then be unified with an expected type. In a few places, however, an expected type may be used to influence inference. We then speak of \emph{top-down inference}.

\define{Expected Type}{define-expected-type}{Expected types occur when the type of an expression is known before that expression has been typed, e.g. because the expression is argument to a function call. They can influence typing of that expression through what is called \tref{top-down inference}{type-system-top-down-inference}.}

A good example are arrays of mixed types. As mentioned in \Fullref{types-dynamic}, the compiler refuses \expr{[1, "foo"]} because it cannot determine an element type. Employing top-down inference, this can be overcome:

\haxe{assets/TopDownInference.hx}

Here, the compiler knows while typing \expr{[1, "foo"]} that the expected type is \type{Array<Dynamic>}, so the element type is \type{Dynamic}. Instead of the usual unification behavior where the compiler would attempt (and fail) to determine a \tref{common base type}{type-system-unification-common-base-type}, the individual elements are typed against and unified with \type{Dynamic}.

We have seen another interesting use of top-down inference when \tref{construction of generic type parameters}{type-system-generic-type-parameter-construction} was introduced:

\haxe{assets/GenericTypeParameter.hx}

The explicit types \type{String} and \type{haxe.Template} are used here to determine the return type of \expr{make}. This works because the method is invoked as \expr{make()}, so we know the return type will be assigned to the variables. Utilizing this information, it is possible to bind the unknown type \type{T} to \type{String} and \type{haxe.Template} respectively.

% this is not really top down inference
%Top-down inference is also utilized when dealing with \tref{enum constructors}{types-enum-constructor}:

%\haxe{assets/TopDownInference2.hx}

%The constructors \expr{TObject} and \expr{TFunction} of type \expr{ValueType} are recognized even though their containing module \type{Type} is not \tref{imported}{Import}. This is possible because the return type of \expr{Type.typeof("foo")} is known to be \expr{ValueType}.


\subsection{Limitations}
\label{type-system-inference-limitations}

Type inference saves a lot of manual type hints when working with local variables, but sometimes the type system still needs some help. In fact, it does not even try to infer the type of a \tref{variable}{class-field-variable} or \tref{property}{class-field-property} field unless it has a direct initialization.

There are also some cases involving recursion where type inference has limitations. If a function calls itself recursively while its type is not (completely) known yet, type inference may infer a wrong, too specialized type.




\section{Modules and Paths}
\label{type-system-modules-and-paths}

\define{Module}{define-module}{All Haxe code is organized in modules, which are addressed using paths. In essence, each .hx file represents a module which may contain several types. A type may be \expr{private}, in which case only its containing module can access it.}

The distinction of a module and its containing type of the same name is blurry by design. In fact, addressing \expr{haxe.ds.StringMap<Int>} can be considered shorthand for \expr{haxe.ds.StringMap.StringMap<Int>}. The latter version consists of four parts:

\begin{enumerate}
	\item the package \expr{haxe.ds}
	\item the module name \expr{StringMap}
	\item the type name \type{StringMap}
	\item the type parameter \type{Int}
\end{enumerate}
If the module and type name are equal, the duplicate can be removed, leading to the \expr{haxe.ds.StringMap<Int>} short version. However, knowing about the extended version helps with understanding how \tref{module sub-types}{type-system-module-sub-types} are addressed.

Paths can be shortened further by using an \tref{import}{type-system-import}, which typically allows omitting the package part of a path. This may lead to usage of unqualified identifiers, for which understanding the \tref{resolution order}{type-system-resolution-order} is required.

\define{Type path}{define-type-path}{The (dot-)path to a type consists of the package, the module name and the type name. Its general form is \expr{pack1.pack2.packN.ModuleName.TypeName}.} 


\subsection{Module Sub-Types}
\label{type-system-module-sub-types}

A module sub-type is a type declared in a module with a different name than that module. This allows a single .hx file to contain multiple types, which can be accessed unqualified from within the module, and by using \expr{package.Module.Type} from other modules:

\begin{lstlisting}
var e:haxe.macro.Expr.ExprDef;
\end{lstlisting}

Here the sub-type \type{ExprDef} within module \expr{haxe.macro.Expr} is accessed. 

The sub-type relation is not reflected at runtime. That is, public sub-types become a member of their containing package, which could lead to conflicts if two modules within the same package try to define the same sub-type. Naturally the Haxe compiler detects these cases and reports them accordingly. In the example above, \type{ExprDef} is generated as \type{haxe.macro.ExprDef}.

Sub-types can also be made private:

\begin{lstlisting}
private class C { ... }
private enum E { ... }
private typedef T { ... }
private abstract A { ... }
\end{lstlisting}

\define{Private type}{define-private-type}{A type can be made private by using the \expr{private} modifier. As a result, the type can only be directly accessed from within the \tref{module}{define-module} it is defined in.

Private types, unlike public ones, do not become a member of their containing package.}

The accessibility of types can be controlled more fine-grained by using \tref{access control}{lf-access-control}.



\subsection{Import}
\label{type-system-import}

If a type path is used multiple times in a .hx file, it might make sense to use an \expr{import} to shorten it. This allows omitting the package when using the type:

\haxe{assets/Import.hx}

With \expr{haxe.ds.StringMap} being imported in the first line, the compiler is able to resolve the unqualified identifier \expr{StringMap} in the \expr{main} function to this package. The module \type{StringMap} is said to be \emph{imported} into the current file.

In this example, we are actually importing a \emph{module}, not just a specific type within that module. This means that all types defined within the imported module are available:

\haxe{assets/Import2.hx}

The type \type{Binop} is an \tref{enum}{types-enum-instance} declared in the module \type{haxe.macro.Expr}, and thus available after the import of said module. If we were to import only a specific type of that module, e.g. \expr{import haxe.macro.Expr.ExprDef}, the program would fail to compile with \expr{Class not found : Binop}.

There several aspects worth knowing about importing:

\begin{itemize}
	\item The bottommost import takes priority (detailed in \Fullref{type-system-resolution-order}).
	\item The \tref{static extension}{lf-static-extension} keyword \expr{using} implies the effect of \expr{import}.
	\item If an enum is imported (directly or as part of a module import), all its \tref{enum constructors}{types-enum-constructor} are also imported (this is what allows the \expr{OpAdd} usage in above example).
\end{itemize}

Furthermore, it is also possible to import \tref{static fields}{class-field} of a class and use them unqualified:

\haxe{assets/Import3.hx}

\todo{Describe import a.*}

Special care has to be taken with field names or local variable names that conflict with a package name: Since they take priority over packages, a local variable named \expr{haxe} blocks off usage the entire \expr{haxe} package.

\subsection{Resolution Order}
\label{type-system-resolution-order}

Resolution order comes into play as soon as unqualified identifiers are involved. These are \tref{expressions}{expression} in the form of \expr{foo()}, \expr{foo = 1} and \expr{foo.field}. The last one in particular includes module paths such as \expr{haxe.ds.StringMap}, where \expr{haxe} is an unqualified identifier.  

We describe the resolution order algorithm here, which depends on the following state:

\begin{itemize}
	\item the delared \tref{local variables}{expression-var} (including function arguments)
	\item the \tref{imported}{type-system-import} modules, types and statics
	\item the available \tref{static extensions}{lf-static-extension}
	\item the kind (static or member) of the current field
	\item the declared member fields on the current class and its parent classes
	\item the declared static fields on the current class
	\item the \tref{expected type}{define-expected-type}
	\item the expression being \expr{untyped} or not
\end{itemize}

\todo{proper label and caption + code/identifier styling for diagram}

\begin{flowchart}{type-system-resolution-order-diagram}{識別子 `i' の解決順序}

\tikzset {
	level distance = 1.4cm,
	scale = 1
}
\tikzset{multiline/.style={align=center}}

\tikzstyle{noEdge} = [ auto = left, outer sep = 0.2cm ]

% Compact decision shape (cut off rectangle corners if you know how)
\tikzstyle{decisionc} = [
	decision,
	minimum height = 0.8cm,
	rectangle
]

\Tree
[.\node [decisionc] (dec1) {'i' == 'true'または'false'、'this'、'super'、'null'};
\edge [noEdge] node {No};
[.\node [decisionc] (dec2) {ローカル変数'i'が存在する};
\edge [noEdge] node {No};
[.\node [decisionc] (dec3) {現在のフィールドが静的フィールドである};
\edge [noEdge] node {No};
[.\node [decisionc] (dec4) {現在のクラス、親クラスのいずれかにフィールド'i'が存在する};
\edge [noEdge] node {No};
[.\node [decisionc] (dec5) {'this'の型の静的拡張があるか};
\edge [noEdge] node {No};
[.\node [decisionc] (dec6) {現在のクラスに静的フィールド'i'があるか};
\edge [noEdge] node {No};
[.\node [decisionc] (dec7) {インポートされたenumにコンストラクタ`i'があるか};
\edge [noEdge] node {No};
[.\node [decisionc] (dec8) {静的フィールド`i'がインポートされているか};
\edge [noEdge] node {No};
[.\node [decisionc] (dec9) {`i'が小文字から始まるか};
\edge [noEdge] node {No};
[.\node [decisionc] (dec10) {型`i'がインポートされているか};
\edge [noEdge] node {No};
[.\node [decisionc] (dec11) {現在のパッケージが型`i'をふくむモジュール`i'を持つか};
\edge [noEdge] node {No};
[.\node [decisionc] (dec12) {トップレベルの型`i'が存在するか};
\edge [noEdge] node {No};
[.\node [decisionc] (dec13) {untypedモードか};
\edge [noEdge] node {Yes};
[.\node [decisionc] (dec14) {`i' == `__this__'};
\edge [noEdge] node {No};
[.\node [decisionc] (dec15) {ローカル変数`i'を生成する};
]]]]]]]]]]]]]]]


\node [startstop, fill = green!70, xshift = 5cm] (resolve) at (dec15.east) {解決};

\tikzstyle{yesNode} = [above right, at start]

\coordinate (yesAnchor) at (resolve.north);

\draw [flowchartArrow] (dec1) -| (yesAnchor) node [yesNode] {はい};
\draw [flowchartArrow] (dec2) -| (yesAnchor) node [yesNode] {はい};
\draw [flowchartArrow] (dec4) -| (yesAnchor) node [yesNode] {はい};
\draw [flowchartArrow] (dec5) -| (yesAnchor) node [yesNode] {はい};
\draw [flowchartArrow] (dec6) -| (yesAnchor) node [yesNode] {はい};
\draw [flowchartArrow] (dec7) -| (yesAnchor) node [yesNode] {はい};
\draw [flowchartArrow] (dec8) -| (yesAnchor) node [yesNode] {はい};
\draw [flowchartArrow] (dec10) -| (yesAnchor) node [yesNode] {はい};
\draw [flowchartArrow] (dec11) -| (yesAnchor) node [yesNode] {はい};
\draw [flowchartArrow] (dec12) -| (yesAnchor) node [yesNode] {はい};
\draw [flowchartArrow] (dec14) -| (yesAnchor) node [yesNode] {はい};
\draw [flowchartArrow] (dec15) -- (resolve.west);

\draw [flowchartArrow] (dec3) to [out = 180, in = 180, distance = 3cm] (dec6);
\draw (dec3.west) node [above left] {はい};

\draw [flowchartArrow] (dec9) to [out = 180, in = 180, distance = 3cm] (dec13);
\draw (dec9.west) node [above left] {はい};

\node [startstop, fill = red!70, xshift = 2cm] (fail) at (dec13.east) {失敗};
\draw [flowchartArrow] (dec13.east) -- (fail.west) node [above right, at start] {いいえ};


\end{flowchart}

Given an identifier \expr{i}, the algorithm is as follows:

\begin{enumerate}
	\item If i is \expr{true}, \expr{false}, \expr{this}, \expr{super} or \expr{null}, resolve to the matching constant and halt.
	\item If a local variable named \expr{i} is accessible, resolve to it and halt.
	\item If the current field is static, go to \ref{resolution:static-lookup}.
	\item If the current class or any of its parent classes has a field named \expr{i}, resolve to it and halt.
	\item\label{resolution:static-extension} If a static extension with a first argument of the type of the current class is available, resolve to it and halt.
	\item\label{resolution:static-lookup} If the current class has a static field named \expr{i}, resolve to it and halt.
	\item\label{resolution:enum-ctor} If an enum constructor named \expr{i} is declared on an imported enum, resolve to it and halt.
	\item If a static named \expr{i} is explicitly imported, resolve to it and halt.
	\item If \expr{i} starts with a lower-case character, go to \ref{resolution:untyped}.
	\item\label{resolution:type} If a type named \expr{i} is available, resolve to it and halt.
	\item\label{resolution:untyped} If the expression is not in untyped mode, go to \ref{resolution:failure}
	\item If \expr{i} equals \expr{__this__}, resolve to the \expr{this} constant and halt.
	\item Generate a local variable named \expr{i}, resolve to it and halt.
	\item\label{resolution:failure} Fail
\end{enumerate}

For step \ref{resolution:type}, it is also necessary to define the resolution order of types:

\begin{enumerate}
	\item\label{resolution:import} If a type named \expr{i} is imported (directly or as part of a module), resolve to it and halt.
	\item If the current package contains a module named \expr{i} with a type named \expr{i}, resolve to it and halt.
	\item If a type named \expr{i} is available at top-level, resolve to it and halt.
	\item Fail
\end{enumerate}

For step \ref{resolution:import} of this algorithm as well as steps \ref{resolution:static-extension} and \ref{resolution:enum-ctor} of the previous one, the order of import resolution is important:

\begin{itemize}
	\item Imported modules and static extensions are checked from bottom to top with the first match being picked.
	\item Within a given module, types are checked from top to bottom.
	\item For imports, a match is made if the name equals.
	\item For \tref{static extensions}{lf-static-extension}, a match is made if the name equals and the first argument \tref{unifies}{type-system-unification}. Within a given type being used as static extension, the fields are checked from top to bottom.
\end{itemize}

\chapter{Class Fields}
\label{class-field}

\define{クラスフィールド}{define-class-field}{クラスフィールドは、クラスに属する変数、プロパティまたはメソッドです。これは静的、または非静的になることができます。非静的フィールドは\emph{メンバ}フィールドと呼ばれるので、例えば、\emph{静的メソッド}と\emph{メンバ変数}と言います。}

ここまで、私達は型と一般的なHaxeのプログラムがどのように構成されているのかを見てきました。
このクラスフィールドに関する章では、この話題についてまとめ、Haxeの動作の一部につなげます。これは、クラスフィールドが\tref{式}{expression}を持つ場所だからです。

% この部分は現段階では未遂行なのでコメントにしています。
% ...are at home や "the structural part"の日本語訳をもう少し練る予定です
%So far we have seen how types and Haxe programs in general are structured. This section about class fields concludes the structural part and at the same time bridges to the behavioral part of Haxe. This is because class fields are the place where \tref{expressions}{expression} are at home.

There are three kinds of class fields:

\begin{description}
	\item[Variable:] A \tref{variable}{class-field-variable} class field holds a value of a certain type, which can be read or written.
	\item[Property:] A \tref{property}{class-field-property} class field defines a custom access behavior for something that, outside the class, looks like a variable field.
	\item[Method:] A \tref{method}{class-field-method} is a function which can be called to execute code.
\end{description}
Strictly speaking, a variable could be considered to be a property with certain access modifiers. Indeed, the Haxe Compiler does not distinguish variables and properties during its typing phase, but they remain separated at syntax level.

Regarding terminology, a method is a (static or non-static) function belonging to a class. Other functions, such as a \tref{local functions}{expression-function} in expressions, are not considered methods.


\section{Variable}
\label{class-field-variable}

We have already seen variable fields in several code examples of previous sections. Variable fields hold values, a characteristic which they share with most (but not all) properties:

\haxe{assets/VariableField.hx}
We can learn from this that a variable

\begin{enumerate}
	\item has a name (here: \expr{member}),
	\item has a type (here: \type{String}),
	\item may have a constant initialization (here: \expr{"bar"}) and
	\item may have \tref{access modifiers}{class-field-access-modifier} (here: \expr{static})
\end{enumerate}

The example first prints the initialization value of \expr{member}, then sets it to \expr{"foo"} before printing its new value. The effect of access modifiers is shared by all three class field kinds and explained in a separate section.

It should be noted that the explicit type is not required if there is an initialization value. The compiler will \tref{infer}{type-system-type-inference} it in this case.

\begin{flowchart}{class-field-variable-init-values}{変数フィールドの値の初期化}
\Tree[.\node [decision] {\expr{inline}};
	\edge node[auto=right] {はい};
	[.\node [decision] {\expr{static}};
		\edge node[auto=right] {いいえ};
		\node [startstop, valueNone] {不正};
		\edge node[auto=left] {はい};
		\node [startstop, valueSome] {定数のみ};
	]
	\edge node[auto=left] {いいえ};
	[.\node [decision] {\expr{extern}};
		\edge node[auto=right] {いいえ};
		[.\node [decision] {\expr{static}};
			\edge node[auto=right] {いいえ};
			\node [startstop, valueSome] {'this'を使えない};
			\edge node[auto=left] {はい};
			\node [startstop, valueAll] {何でも};
		]
		\edge node[auto=left] {はい};
		\node [startstop, valueNone] {なし};
	]
]
\end{flowchart}


\section{Property}
\label{class-field-property}

Next to \tref{variables}{class-field-variable}, properties are the second option for dealing with data on a class. Unlike variables however, they offer more control of which kind of field access should be allowed and how it should be generated. Common use cases include:

\begin{itemize}
	\item Have a field which can be read from anywhere, but only be written from within the defining class.
	\item Have a field which invokes a \emph{getter}-method upon read-access.
	\item Have a field which invokes a \emph{setter}-method upon write-access.
\end{itemize}

When dealing with properties, it is important to understand the two kinds of access:

\define{Read Access}{define-read-access}{A read access to a field occurs when a right-hand side \tref{field access expression}{expression-field-access} is used. This includes calls in the form of \expr{obj.field()}, where \expr{field} is accessed to be read.}

\define{Write Access}{define-write-access}{A write access to a field occurs when a \tref{field access expression}{expression-field-access} is assigned a value in the form of \expr{obj.field = value}. It may also occur in combination with \tref{read access}{define-read-access} for special assignment operators such as \expr{+=} in expressions like \expr{obj.field += value}.} 

Read access and write access are directly reflected in the syntax, as the following example shows:

\haxe{assets/Property.hx}

For the most part, the syntax is similar to variable syntax, and the same rules indeed apply. Properties are identified by

\begin{itemize}
	\item the opening parenthesis \expr{(} after the field name,
	\item followed by a special \emph{access identifier} (here: \expr{default}),
	\item with a comma \expr{,} separating
	\item another special access identifier (here: \expr{null})
	\item before a closing parenthesis \expr{)}.
\end{itemize}

The access identifiers define the behavior when the field is read (first identifier) and written (second identifier). The accepted values are:

\begin{description}
	\item[\expr{default}:] Allows normal field access if the field has public visibility, otherwise equal to \expr{null} access.
	\item[\expr{null}:] Allows access only from within the defining class.
	\item[\expr{get}/\expr{set}:] Access is generated as a call to an \emph{accessor method}. The compiler ensures that the accessor is available.
	\item[\expr{dynamic}:] Like \expr{get}/\expr{set} access, but does not verify the existence of the accessor field.
	\item[\expr{never}:] Allows no access at all.
\end{description}

\define{Accessor method}{define-accessor-method}{An \emph{accessor method} (or short \emph{accessor}) for a field named \expr{field} of type \type{T} is a \emph{getter} named \expr{get_field} of type \type{Void->T} or a \emph{setter} named \expr{set_field} of type \type{T->T}.}

\trivia{Accessor names}{In Haxe 2, arbitrary identifiers were allowed as access identifiers and would lead to custom accessor method names to be admitted. This made parts of the implementation quite tricky to deal with. In particular, \expr{Reflect.getProperty()} and \expr{Reflect.setProperty()} had to assume that any name could have been used, requiring the target generators to generate meta-information and perform lookups.\\
We disallowed these identifiers and went for the \expr{get_} and \expr{set_} naming convention which greatly simplified implementation. This was one of the breaking changes between Haxe 2 and 3.}

\subsection{Common accessor identifier combinations}
\label{class-field-property-common-combinations}

The next example shows common access identifier combinations for properties:

\haxe{assets/Property2.hx}

The \target{Javascript} output helps understand what the field access in the \expr{main}-method is compiled to:

\begin{lstlisting}
var Main = function() {
	var v = this.get_x();
	this.set_x(2);
	var _g = this;
	_g.set_x(_g.get_x() + 1);
};
\end{lstlisting}

As specified, the read access generates a call to \expr{get_x()}, while the write access generates a call to \expr{set_x(2)} where \expr{2} is the value being assigned to \expr{x}. The way the \expr{+=} is being generated might look a little odd at first, but can easily be justified by the following example:

\haxe{assets/Property3.hx}

What happens here is that the expression part of the field access to \expr{x} in the \expr{main} method is \emph{complex}: It has potential side-effects, such as the construction of \type{Main} in this case. Thus, the compiler cannot generate the \expr{+=} operation as \expr{new Main().x = new Main().x + 1} and has to cache the complex expression in a local variable:

\begin{lstlisting}
Main.main = function() {
	var _g = new Main();
	_g.set_x(_g.get_x() + 1);
}
\end{lstlisting}



\subsection{Impact on the type system}
\label{class-field-property-type-system-impact}

The presence of properties has several consequences on the type system. Most importantly, it is necessary to understand that properties are a compile-time feature and thus \emph{require the types to be known}. If we were to assign a class with properties to \type{Dynamic}, field access would \emph{not} respect accessor methods. Likewise, access restrictions no longer apply and all access is virtually public.

When using \expr{get} or \expr{set} access identifier, the compiler ensures that the getter and setter actually exists. The following problem does not compile:

\haxe{assets/Property4.hx}

The method \expr{get_x} is missing, but it need not be declared on the class defining the property itself as long as a parent class defines it:

\haxe{assets/Property5.hx}

The \expr{dynamic} access modifier works exactly like \expr{get} or \expr{set}, but does not check for the existence



\subsection{Rules for getter and setter}
\label{class-field-property-rules}

Visibility of accessor methods has no effect on the accessibility of its property. That is, if a property is \expr{public} and defined to have a getter, that getter may me defined as \expr{private} regardless.

Both getter and setter may access their physical field for data storage. The compiler ensures that this kind of field access does not go through the accessor method if made from within the accessor method itself, thus avoiding infinite recursion:

\haxe{assets/GetterSetter.hx}

However, the compiler assumes that a physical field exists only if at least one of the access identifiers is \expr{default} or \expr{null}.

\define{Physical field}{define-physical-field}{A field is considered to be \emph{physical} if it is either
	\begin{itemize}
		\item a \tref{variable}{class-field-variable}
		\item a \tref{property}{class-field-property} with the read-access or write-access identifier being \expr{default} or \expr{null}
		\item a \tref{property}{class-field-property} with \expr{:isVar} \tref{metadata}{lf-metadata}
	\end{itemize}
}

If this is not the case, access to the field from within an accessor method causes a compilation error:

\haxe{assets/GetterSetter2.hx}

If a physical field is indeed intended, it can be forced by attributing the field in question with the \expr{:isVar} \tref{metadata}{lf-metadata}:

\haxe{assets/GetterSetter3.hx}


\trivia{Property setter type}{It is not uncommon for new Haxe users to be surprised by the type of a setter being required to be \type{T->T} instead of the seemingly more natural \type{T->Void}. After all, why would a \emph{setter} have to return something?\\
The rationale is that we still want to be able to use field assignments using setters as right-side expressions. Given a chain like \expr{x = y = 1}, it is evaluated as \expr{x = (y = 1)}. In order to assign the result of \expr{y = 1} to \expr{x}, the former must have a value. If \expr{y} had a setter returning \type{Void}, this would not be possible.}


\section{Method}
\label{class-field-method}

While \tref{variables}{class-field-variable} hold data, methods are defining behavior of a program by hosting \tref{expressions}{expression}. We have seen method fields in every code example of this document with even the initial \tref{Hello World}{introduction-hello-world} example containing a \expr{main} method:

\haxe{assets/HelloWorld.hx}

Methods are identified by the \expr{function} keyword. We can also learn that they

\begin{enumerate}
	\item have a name (here: \expr{main}),
	\item have an argument list (here: empty \expr{()}),
	\item have a return type (here: \type{Void}),
	\item may have \tref{access modifiers}{class-field-access-modifier} (here: \expr{static} and \expr{public}) and
	\item may have an expression (here: \expr{\{trace("Hello World");\}}).
\end{enumerate}

We can also look at the next example to learn more about arguments and return types:

\haxe{assets/MethodField.hx}

Arguments are given by an opening parenthesis \expr{(} after the field name, a comma \expr{,} separated list of argument specifications and a closing parenthesis \expr{)}. Additional information on the argument specification is described in \Fullref{types-function}.

The example demonstrates how \tref{type inference}{type-system-type-inference} can be used for both argument and return types. The method \expr{myFunc} has two arguments but only explicitly gives the type of the first one, \expr{f}, as \type{String}. The second one, \expr{i}, is not type-hinted and it is left to the compiler to infer its type from calls made to it. Likewise, the return type of the method is inferred from the \expr{return true} expression as \type{Bool}.

\subsection{Overriding Methods}
\label{class-field-overriding}

Overriding fields is instrumental for creating class hierarchies. Many design patterns utilize it, but here we will explore only the basic functionality. In order to use overrides in a class, it is required that this class has a \tref{parent class}{types-class-inheritance}. Let us consider the following example:

\haxe{assets/Override.hx}

The important components here are

\begin{itemize}
	\item the class \type{Base} which has a method \expr{myMethod} and a constructor,
	\item the class \type{Child} which \expr{extends Base} and also has a method \expr{myMethod} being declared with \expr{override}, and
	\item the \type{Main} class whose \expr{main} method creates an instance of \expr{Child}, assigns it to a variable \expr{child} of explicit type \type{Base} and calls \expr{myMethod()} on it.
\end{itemize}

The variable \expr{child} is explicitly typed as \type{Base} to highlight an important difference: At compile-time the type is known to be \type{Base}, but the runtime still finds the correct method \expr{myMethod} on class \type{Child}. It is then obvious that the field access is resolved dynamically at runtime.

\subsection{Effects of variance and access modifiers}
\label{class-field-override-effects}

Overriding adheres to the rules of \tref{variance}{type-system-variance}. That is, their argument types allow \emph{contravariance} (less specific types) while their return type allows \emph{covariance} (more specific types):

\haxe{assets/OverrideVariance.hx}

Intuitively, this follows from the fact that arguments are ``written to'' the function and the return value is ``read from'' it.

The example also demonstrates how \tref{visibility}{class-field-visibility} may be changed: An overriding field may be \expr{public} if the overridden field is \expr{private}, but not the other way around.

It is not possible to override fields which are declared as \tref{\expr{inline}}{class-field-inline}. This is due to the conflicting concepts: While inlining is done at compile-time by replacing a call with the function body, overriding fields necessarily have to be resolved at runtime.
	
	
	
\section{Access Modifier}
\label{class-field-access-modifier}
\state{NoContent}

\subsection{Visibility}
\label{class-field-visibility}

Fields are by default \emph{private}, meaning that only the class and its sub-classes may access them. They can be made \emph{public} by using the \expr{public} access modifier, allowing access from anywhere.

\haxe{assets/Visibility.hx}

Access to field \expr{available} of class \type{MyClass} is allowed from within \type{Main} because it is denoted as being \expr{public}. However, while access to field \expr{unavailable} is allowed from within class \type{MyClass}, it is not allowed from within class \type{Main} because it is \expr{private} (explicitly, although this identifier is redundant here).

The example demonstrates visibility through \emph{static} fields, but the rules for member fields are equivalent. The following example demonstrates visibility behavior for when \tref{inheritance}{types-class-inheritance} is involved.

\haxe{assets/Visibility2.hx}

We can see that access to \expr{child1.baseField()} is allowed from within \type{Child2} even though \expr{child1} is of a different type, \type{Child1}. This is because the field is defined on their common ancestor class \type{Base}, contrary to field \expr{child1Field} which can not be accessed from within \type{Child2}.

Omitting the visibility modifier usually defaults the visibility to \expr{private}, but there are exceptions where it becomes \expr{public} instead:

\begin{enumerate}
	\item If the class is declared as \expr{extern}.
	\item If the field is declared on an \tref{interface}{types-interfaces}.
	\item If the field \tref{overrides}{class-field-overriding} a public field.
\end{enumerate}

\trivia{Protected}{Haxe has no notion of a \expr{protected} keyword known from Java, C++ and other object-oriented languages. However, its \expr{private} behavior is equal to those language's protected behavior, so Haxe actually lacks their real private behavior.}

\subsection{Inline}
\label{class-field-inline}

The \expr{inline} keyword allows function bodies to be directly inserted in place of calls to them. This can be a powerful optimization tool, but should be used judiciously as not all functions are good candidates for inline behavior. The following example demonstrates the basic usage:

\haxe{assets/Inline.hx}

The generated \target{Javascript} output reveals the effect of inline:

\begin{lstlisting}
(function () { "use strict";
var Main = function() { }
Main.main = function() {
	var a = 1;
	var b = 2;
	var c = (a + b) / 2;
}
Main.main();
})();
\end{lstlisting}

As evident, the function body \expr{(s1 + s2) / 2} of field \expr{mid} was generated in place of the call to \expr{mid(a, b)}, with \expr{s1} being replaced by \expr{a} and \expr{s2} being replaced by \expr{b}. This avoids a function call which, depending on the target and frequency of occurrences, may yield noticeable performance improvements.

It is not always easy to judge if a function qualifies for being inline. Short functions that have no writing expressions (such as a \expr{=} assignment) are usually a good choice, but even more complex functions can be candidates. However, in some cases inlining can actually be detrimental to performance, e.g. because the compiler has to create temporary variables for complex expressions.

Inline is not guaranteed to be done. The compiler might cancel inlining for various reasons or a user could supply the \ic{--no-inline} command line argument to disable inlining. The only exception is if the class is \tref{extern}{lf-externs} or if the class field has the \expr{:extern} \tref{metadata}{lf-metadata}, in which case inline is forced. If it cannot be done, the compiler emits an error.

It is important to remember this when relying on inline:

\haxe{assets/InlineRelying.hx}

If the call to \expr{error} is inlined the program compiles correctly because the control flow checker is satisfied due to the inlined \tref{throw}{expression-throw} expression. If inline is not done, the compiler only sees a function call to \expr{error} and emits the error \expr{A return is missing here}.


\subsection{Dynamic}
\label{class-field-dynamic}

Methods can be denoted with the \expr{dynamic} keyword to make them (re-)bindable:

\haxe{assets/DynamicFunction.hx}

The first call to \expr{test()} invokes the original function which returns the \type{String} \expr{"original"}. In the next line, \expr{test} is \emph{assigned} a new function. This is precisely what \expr{dynamic} allows: Function fields can be assigned a new function. As a result, the next invocation of \expr{test()} returns the \type{String} \expr{"new"}.

Dynamic fields cannot be \expr{inline} for obvious reasons: While inlining is done at compile-time, dynamic functions necessarily have to be resolved at runtime.

%TODO: performance estimation %

\subsection{Override}
\label{class-field-override}

The access modifier \expr{override} is required when a field is declared which also exists on a \tref{parent class}{types-class-inheritance}. Its purpose is to ensure that the author of a class is aware of the override as this may not always be obvious in large class hierarchies. Likewise, having \expr{override} on a field which does not actually override anything (e.g. due to a misspelled field name) triggers an error.

The effects of overriding fields are detailed in \Fullref{class-field-overriding}. This modifier is only allowed on \tref{method}{class-field-method} fields.

\chapter{式}
\label{expression}

Haxeの式は、プログラムが\emph{何をするか}を決定します。ほとんどの式は\tref{メソッド}{class-field-method}に書かれ、そのメソッドが何をすべきかをその式の合わせによって表現します。この章では、さまざまな種類の式を説明していきます。

ここに、いくつか定義を示しておきます。

\define{名前}{define-name}{
名前は次のいずれかにひもづきます。
\begin{itemize}
	\item 型
	\item ローカル変数
	\item ローカル関数
	\item フィールド
\end{itemize}}

\define{識別子}{define-identifier}{
Haxeの識別子は、アンダースコア\expr{_}、ドル\expr{\$}、小文字\expr{a-z}、大文字\expr{A-Z}のいずれかから始まり、任意の\expr{_}、\expr{A-Z}、\expr{a-z}、\expr{0-9}のつなぎ合わせが続きます。

さらに使用する状況によって以下の制限が加わります。これらは、型付けの時にチェックされます。
\begin{itemize}
	\item 型の名前は大文字\expr{A-Z}か、アンダースコア\expr{_}で始まる。
	\item \tref{名前}{define-name}では、先頭にドル記号は使えません。(ドル記号はほとんどの場合、\tref{マクロの実体化}{macro-reification}に使われます)
\end{itemize}}


\section{ブロック}
\label{expression-block}

Haxeのブロックは中かっこで\expr{\{}から始まり、\expr{\}}で終わります。ブロックはいくつかの式をふくみ、各式はセミコロンで終わります。一般の構文としては以下のとおりです。

\begin{lstlisting}
{
	式1;
	式2;
	...
	式N;
}
\end{lstlisting}

ブロック式の値とその型は、ブロック式がふくむ最後の式の値と型と同じになります。

ブロック内では、\tref{\expr{var}式}{expression-var}を使ったローカル変数の定義と\tref{\expr{function}式}{expression-function}を使ったローカル関数の定義が可能です。これらのローカル変数とローカル関数は、そのブロックとさらに入れ子のブロックの中では使用することができますが、ブロックの外では利用できません。また、定義よりも後でしか使えません。次の例では\expr{var}を使っていますが、同じルールが\expr{function}の場合でも使用されます。

\begin{lstlisting}
{
	a; // error, a is not declared yet
	var a = 1; // declare a
	a; // ok, a was declared
	{
		a; // ok, a is available in sub-blocks
	}
  // ok, a is still available after
	// sub-blocks	
	a;
}
a; // error, a is not available outside
\end{lstlisting}

実行時には、ブロックは上から下へと評価されていきます。フロー制御(例えば、\tref{例外}{expression-try-catch}や\tref{return式}{expression-return}など)によって、すべての式が評価される前に中断されることもあります。

\section{定数値}
\label{expression-constants}

Haxeの構文では以下の定数値をサポートしています。

\begin{description}
	\item[Int:] \expr{0}、\expr{1}、\expr{97121}、\expr{-12}、\expr{0xFF0000}といった、\tref{整数}{define-int}
	\item[Float:] \expr{0.0}、\expr{1.}、\expr{.3}、\expr{-93.2}といった\tref{浮動小数点数}{define-float}
	\item[String:] \expr{""}、\expr{"foo"}、\expr{''}、\expr{'bar'}といった\tref{文字列}{define-string}
	\item[true,false:] \tref{真偽値}{define-bool}
	\item[null:] null値
\end{description}

また内部の構文木では、\tref{識別子}{define-identifier}は定数値としてあつかわれます。これは、\tref{マクロ}{macro}を使っているときに関係してくる話題です。

\section{2項演算子}
\label{expression-binops}

\section{単項演算子}
\label{expression-unops}

\section{配列の宣言}
\label{expression-array-declaration}

配列は\expr{,}で区切った値を、大かっこ\expr{[]}で囲んで初期化します。空の\expr{[]}は空の配列を表し、\expr{[1, 2, 3]}は\expr{1}、\expr{2}、\expr{3}の3つの要素を持つ配列になります。

配列の初期化をサポートしていないプラットフォームでは、生成されたコードはあまり簡潔ではないかもしれません。本質的には以下のようなコードに見えるでしょう。

\begin{lstlisting}
var a = new Array();
a.push(1);
a.push(2);
a.push(3);
\end{lstlisting}

つまり、関数を\tref{インライン化}{class-field-inline}するかを決める場合には、この構文で見えているよりも多くのコードがインライン化されることがあることを考慮すべきです。

より高度な初期化方法は、\Fullref{lf-array-comprehension}で説明します。

\section{オブジェクトの宣言}
\label{expression-object-declaration}

オブジェクトの宣言は、中かっこ\expr{\{}で始まり、\expr{キー:値}のペアがカンマ\expr{,}で区切られながら続いて、中かっこ\expr{\}}で終わります。

\begin{lstlisting}
{
	key1:value1,
	key2:value2,
	...
	keyN:valueN
}
\end{lstlisting}
さらに詳しいオブジェクトの宣言については\tref{匿名構造体}{types-anonymous-structure}の節で書かれています。

\section{フィールドへのアクセス}
\label{expression-field-access}

フィールドへのアクセスは、ドット\expr{.}の後にフィールドの名前を続けることで表現します。

\begin{lstlisting}
object.fieldName
\end{lstlisting}

この構文は\expr{pack.Type}の形でパッケージ内の型にアクセスするのにも使われます。

型付け機は、アクセスされたフィールドが本当に存在するかを確認し、フィールドの種類に依存した変更を適用します。もしフィールドへのアクセスが複数の意味にとれる場合は、\tref{解決順序}{type-system-resolution-order}の理解が役に立つでしょう。

\section{配列アクセス}
\label{expression-array-access}

配列アクセスは、大かっこ\expr{[}で始まり、インデックスを表す式が続き、大かっこ\expr{]}で閉じます。

\begin{lstlisting}
expr[indexExpr]
\end{lstlisting}

この記法については任意の式で許可されていますが、型付けの段階では以下の特定の組み合わせのみが許可されます。

\begin{itemize}
	\item \expr{expr}は\type{Array}か\type{Dynamic}であり、\expr{indexExpr}が\type{Int}である。
	\item \expr{expr}は\tref{抽象型}{types-abstract}であり、マッチする\tref{配列アクセス}{types-abstract-array-access}が定義されている。
\end{itemize}

\section{関数呼び出し}
\label{expression-function-call}

関数呼び出しは、任意の式を対象として、小かっこ\expr{(}を続け、引数の式のリストを\expr{,}で区切って並べて、小かっこ\expr{)}で閉じることで行います。

\begin{lstlisting}
subject(); // call with no arguments
subject(e1); // call with one argument
subject(e1, e2); // call with two arguments
// call with multiple arguments
subject(e1, ..., eN);
\end{lstlisting}

\section{var(変数宣言)}
\label{expression-var}

\expr{var}キーワードは、カンマ\expr{,}で区切って、複数の変数を宣言することができます。すべての変数は、正当な\tref{識別子}{define-identifier}を持ち、オプションとして\expr{=}を続けて値の代入を行うこともできます。また変数に明示的な型注釈をあたえることもできます。

\begin{lstlisting}
var a; // declare local a
var b:Int; // declare variable b of type Int
// declare variable c, initialized to value 1
var c = 1;
// declare variable d and variable e
// initialized to value 2
var d,e = 2;
\end{lstlisting}

ローカル変数のスコープについての挙動は\Fullref{expression-block}に書かれています。

\section{ローカル関数}
\label{expression-function}

Haxeはファーストクラス関数をサポートしており、式の中でローカル関数を宣言することができます。この構文は\tref{クラスフィールドメソッド}{class-field-method}にならいます。

\haxe{assets/LocalFunction.hx}

\expr{myLocalFunction}を、\expr{main}クラスフィールドの\tref{ブロック式}{expression-block}の中で宣言しました。このローカル関数は1つの引数\expr{i}を取り、それをスコープの外のvalueに足しています。

スコープについては、\tref{変数の場合}{expression-var}と同じで、多くの面で名前を持つローカル関数は、ローカル変数に対する匿名関数の代入と同じです。

\begin{lstlisting}
var myLocalFunction = function(a) { }
\end{lstlisting}

しかしながら、関数の場所による型パラメータに関する違いがあります。これは定義時に何にも代入されていない「左辺値」の関数と、それ以外の「右辺値」の関数についての違いで、以下の通りです。

\begin{itemize}
	\item 左辺値の関数は名前が必要で、\tref{型パラメータ}{type-system-type-parameters}を持ちます。
	\item 右辺値の関数については名前はあってもなくてもかまいませんが、型パラメータを使うことができません。
\end{itemize}

\section{new(インスタンス化)}
\label{expression-new}

\expr{new}キーワードは、\tref{クラス}{types-class-instance}と\tref{抽象型}{types-abstract}のインスタンス化を行います。\expr{new}の後にはインスタンス化される\tref{型のパス}{define-type-path}が続きます。場合によっては、\expr{<>}で囲んでカンマ\expr{,}で区切った、\tref{型パラメータ}{type-system-type-parameters}の記述がされます。その後に、小かっこ\expr{(}、カンマ\expr{,}区切りのコンストラクタの引数が続き、小かっこ\expr{)}で閉じます。

\haxe{assets/New.hx}

\expr{main}メソッドの中では、型パラメータ\type{Int}の明示付き、引数が\expr{12}と\expr{"foo"}で、\type{Main}クラス自身のインスタンス化を行っています。私たちが知っているように、この構文は、\tref{関数呼び出し}{expression-function-call}とよく似ており、「コンストラクタ呼び出し」と呼ぶことが多いです。

\section{for}
\label{expression-for}

Haxeは、C言語で知られる伝統的なforループはサポートしていません。\expr{for}キーワードの後には、小かっこ\expr{(}、変数の識別子、\expr{in}キーワード、くり返しの処理を行うコレクションの任意の式が続き、小かっこ\expr{)}で閉じられて、最後にくり返しの本体の任意の式で終わります。

\begin{lstlisting}
for (v in e1) e2;
\end{lstlisting}

型付け機は、\expr{e1}の型がくり返し可能であるかを確認します。くり返し可能というのは、\expr{iterator}メソッドが\type{Iterator<T>}を返すか、\type{Iterator<T>}自身である場合です。

変数vは、ループ本体の\expr{e2}の中で利用可能で、コレクション\expr{e1}の個々の要素の値が保持されます。

Haxeには、ある範囲のくり返しを表す特殊な範囲演算子があります。これは、\expr{min...max}といった2つの\type{Int}をとり、\expr{min}(自身をふくむ)から\expr{max}の一つ前までをくり返す\expr{IntIterator}を返す2項演算子です。\expr{max}が\expr{min}より小さくしないように気をつけてください。

\begin{lstlisting}
for (i in 0...10) trace(i); // 0 to 9
\end{lstlisting}

\expr{for}式の型は常に\type{Void}です。つまり、値は持たず、右辺の式としては使えません。

ループは、\tref{\expr{break}}{expression-break}と、\tref{\expr{continue}}{expression-continue}の式を使って、フロー制御が行えます。

\section{whileループ}
\label{expression-while}

通常の\expr{while}ループは、\expr{while}キーワードから始まり、小かっこ\expr{(}、条件式が続き、小かっこ\expr{)}を閉じて、ループ本体の式で終わります。

\begin{lstlisting}
while(condition) expression;
\end{lstlisting}

条件式は\type{Bool}型でなくてはいけません。

各くり返しで条件式は評価されます。\expr{false}と評価された場合ループは終了します。そうでない場合、ループ本体の式が評価されます。

\haxe{assets/While.hx}

この種類の\expr{while}ループは、ループ本体が一度も評価されないことがあります。条件式が最初から\expr{false}だった場合です。この点が\tref{do-whileループ}{expression-do-while}との違いです。

\section{do-whileループ}
\label{expression-do-while}

do-whileループは、\expr{do}キーワードから始まり、次にループ本体の式が来ます。その後に\expr{while}、小かっこ\expr{(}、条件式、小かっこ\expr{)}となります。

\begin{lstlisting}
do expression while(condition);
\end{lstlisting}

条件式は\type{Bool}型でなくてはいけません。

この構文を見てわかるとおり、\tref{while}{expression-while}ループの場合とは違ってループ本体の式は少なくとも一度は評価をされます。

\section{if}
\label{expression-if}

条件分岐式は、\expr{if}キーワードから始まり、小かっこ\expr{()}で囲んだ条件式、条件が真だった場合に評価される式となります。

\begin{lstlisting}
if (condition) expression;
\end{lstlisting}

条件式は\type{Bool}型でなくてはいけません。

オプションとして、\expr{else}キーワードを続けて、その後に、元の条件が偽だった場合に実行される式を記述することができます。

\begin{lstlisting}
if (condition) expression1 else expression2;
\end{lstlisting}

\expr{expression2}は以下のように、また別の\expr{if}式を持つかもしれません。

\begin{lstlisting}
if (condition1) expression1
else if(condition2) expression2
else expression3
\end{lstlisting}

\expr{if}式に値が要求される場合(たとえば、\expr{var x = if(condition) expression1 else expression2}という風に)、型付け機は\expr{expression1}と\expr{expression2}の型を\tref{単一化}{type-system-unification}します。\expr{else}式がなかった場合、型は\type{Void}であると推論されます。

\section{switch}
\label{expression-switch}

基本的なスイッチ式は、\expr{switch}キーワードと、その分岐対象の式から始まり、中かっこ\expr{\{\}}にはさまれてケース式が並びます。各ケース式は、\expr{case}キーワードからのパターン式か、\expr{default}キーワードで始まります。どちらの場合も、コロンが続き、オプショナルなケース本体の式が来ます。

\begin{lstlisting}
switch subject {
	case pattern1: case-body-expression-1;
	case pattern2: case-body-expression-2;
	default: default-expression;
}
\end{lstlisting}

ケース本体の式に、「フォールスルー」は起きません。このため、Haxeでは\tref{\expr{break}}{expression-break}キーワードは使用しません。

スイッチ式は値としてあつかうことができます。その場合、すべてのケース本体の式の型は\tref{単一化}{type-system-unification}できなくてはいけません。

パターン式については、\Fullref{lf-pattern-matching}で詳しく説明されています。

\section{try/catch}
\label{expression-try-catch}

Haxeでは、\expr{try/catch}構文を使うことで値を捕捉することができます。

\begin{lstlisting}
try try-expr
catch(varName1:Type1) catch-expr-1
catch(varName2:Type2) catch-expr-2
\end{lstlisting}

実行時に、\expr{try-expression}の評価が、\tref{\expr{throw}}{expression-throw}を引き起こすと、後に続く\expr{catch}ブロックのいずれかに捕捉されます。これらのブロックは以下から構成されます

\begin{itemize}
	\item \expr{throw}された値を割り当てる変数の名前。
	\item 捕捉する値の型を決める、明示的な型注釈
	\item 捕捉したときに実行される式
\end{itemize}

Haxeでは、あらゆる種類の値を\expr{throw}して、\expr{catch}することができます。その型は特定の例外やエラークラスに限定されません。\expr{catch}ブロックは上から下へとチェックされていき、投げられた値と型が適合する最初のブロックが実行されます。

この過程は、コンパイル時の\tref{単一化}{type-system-unification}に似ています。しかし、この判定は実行時に行われるものでいくつかの制限があります。

\begin{itemize}
	\item 型は実行時に存在するものでなければならない。\tref{クラスインスタンス}{types-class-instance}、\tref{列挙型インスタンス}{types-enum-instance}、\tref{コアタイプ抽象型}{types-abstract-core-type}、\tref{Dynamic}{types-dynamic}.
	\item 型パラメータは、\tref{Dynamic}{types-dynamic}でなければならない。
\end{itemize}

\section{return}
\label{expression-return}

\expr{return}式は、値をとるものと、とらないものの両方があります。

\begin{lstlisting}
return;
return expression;
\end{lstlisting}

\expr{return}式は、最も内側に定義されている関数のフロー制御からぬけ出します。最も内側というのは\tref{ローカル関数}{expression-function}の場合での特徴です。

\begin{lstlisting}
function f1() {
	function f2() {
		return;
	}
	f2();
	expression;
}
\end{lstlisting}

\expr{return}により、ローカル関数\expr{f2}からはぬけ出しますが、\expr{f1}からはぬけ出しません。つまり、\expr{expression}は評価されます。

\expr{return}が、値の式なしで使用された場合、型付け機はその関数の戻り値が\type{Void}型であることを確認します。\expr{return}が値の式を持つ場合、型付け機はその関数の戻り値の型(明示的に与えられているか、前のreturnによって推論されている場合)と、\expr{return}している値の型を\tref{単一化}{type-system-unification}します。

\section{break}
\label{expression-break}

\expr{break}キーワードは、そのキーワードをふくむ最も内側にあるループ(\expr{for}でも、\expr{while}でも)の制御フローからぬけ出して、くり返し処理を終了させます。

\begin{lstlisting}
while(true) {
	expression1;
	if (condition) break;
	expression2;
}
\end{lstlisting}

\expr{expression1}はすべてのくり返しで評価されますが、\expr{condition}が偽になると\expr{expression2}は、実行されません。

型付け機は\expr{break}がループの内部のみで使用されていることを確認します。\tref{\expr{switch}のケース}{expression-switch}に対する\expr{break}は、Haxeではサポートしていません。

\section{continue}
\label{expression-continue}

\expr{continue}キーワードは、そのキーワードをふくむ最も内側にあるループ(\expr{for}でも、\expr{while}でも)の現在のくり返しを終了します。そして、次のくり返しのためのループ条件チェックが行われます。

\begin{lstlisting}
while(true) {
	expression1;
	if(condition) continue;
	expression2;
}
\end{lstlisting}

\expr{expression1}は、各くり返しすべてで評価されますが、\expr{condition}が偽の時は、その回のくり返しについては評価がされません。\expr{break}は異なりループ処理自体は続きます。

型付け機は\expr{continue}がループの内部のみで使用されていることを確認します。

\section{throw}
\label{expression-throw}

Haxeでは、以下の構文で、値の\expr{throw}をすることができます。

\begin{lstlisting}
throw expr
\end{lstlisting}

\expr{throw}された値は、\tref{\expr{catch}ブロック}{expression-try-catch}で捕捉できます。捕捉されなかった場合の挙動はターゲット依存です。

\section{cast}
\label{expression-cast}

Haxeには、以下の2種類のキャストがあります。

\begin{lstlisting}
cast expr; // unsafe cast
cast (expr, Type); // safe cast
\end{lstlisting}

\subsection{非セーフキャスト}
\label{expression-cast-unsafe}

非セーフキャストは型システムを無力化するのに役立ちます。コンパイラは\expr{expr}を通常通りに型付けを行い、それを\tref{単相}{types-monomorph}としてつつみ込みます。これにより、その式をあらゆるものに割り当てすることが可能です。

非セーフキャストは、以下の例が示すように、\tref{Dynamic}{types-dynamic}への型変更ではありません。

\haxe{assets/UnsafeCast.hx}

変数\expr{i}は\type{Int}と型付けされて、非セーフキャスト\expr{cast i}を使って変数\expr{s}に代入しました。\expr{s}は、\type{Unknown}型、つまり単相となりました。その後は、通常の\tref{単一化}{type-system-unification}のルールに従って、あらゆる型へと結びつけることが可能です。例では、\type{String}型となりました。

これらのキャストは「非セーフ」と呼ばれます。これは、実行時の不正なキャストが定義されてないためです。 ほとんどの\tref{動的ターゲット}{define-dynamic-target}では動作する可能性が高いですが、\tref{静的ターゲット}{define-static-target}では未知のエラーの原因になりえます。

非セーフキャストは実行時のオーバーヘッドは、ほぼ、または全くありません。

\subsection{セーフキャスト}
\label{expression-cast-safe}

\tref{非セーフキャスト}{expression-cast-unsafe}とは異なり、実行時のキャスト失敗の挙動を持つのがセーフキャストです。

\haxe{assets/SafeCast.hx}

この例では、最初に\type{Child1}から\type{Base}へとキャストしています。これは、\type{Child1}が\type{Base}型の\tref{子クラス}{types-class-inheritance}なので、成功しています。次に\type{Child2}へキャストしていますが、\type{Child1}のインスタンスは\type{Child2}ではないので失敗しています。

Haxeコンパイラは、この場合\type{String}型の\tref{例外を投げます}{expression-throw}。この例外は、\tref{\expr{try/catch}ブロック}{expression-try-catch}を使って捕捉できます。

セーフキャストは実行時のオーバーヘッドがあります。重要なのは、コンパイラがすでにチェックを行っているので、\expr{Std.is}のようなチェックを自分で入れるのは、余分だということです。\type{String}型の例外を捕捉する、try-catchを行うのがセーフキャストで意図された用途です。

\section{型チェック}
\label{expression-type-check}
\since{3.1.0}

以下の構文でコンパイルタイムの型チェックをつけることが可能です。

\begin{lstlisting}
(expr : type)
\end{lstlisting}

小かっこは必須です。\tref{セーフキャスト}{expression-cast-safe}とは異なり、実行時に影響はありません。これは、コンパイル時の以下の2つの挙動を持ちます。

\begin{enumerate}
\item \tref{トップダウンの型推論}{type-system-top-down-inference}が\expr{expr}に対して\expr{type}の型で適用されます。
\item その結果、\expr{type}の型との\tref{単一化}{type-system-unification}がされます。
\end{enumerate}

この2つの操作には、\tref{解決順序}{type-system-resolution-order}が発生している場合や、\tref{抽象型キャスト}{types-abstract-implicit-casts}で、期待する型へと変化させる、便利な効果があります。

\chapter{言語機能}
\label{lf}

\paragraph{\tref{抽象型}{types-abstract}:}

抽象型は実行時には別の形として提供されるコンパイル時の構成要素です。これにより、すでに存在する型に別の意味をあたえることができます。

\paragraph{\tref{externクラス}{lf-externs}:}

externを使うことで、型安全のルールにしたがってターゲット固有の連携を記述することができます。

\paragraph{\tref{匿名構造体}{types-anonymous-structure}:}

匿名構造体を使うことでデータを簡単にまとめて、小さなデータクラスの必要性を減らすことができます。

\begin{lstlisting}
var point = { x: 0, y: 10 };
point.x += 10;
\end{lstlisting}

\paragraph{\tref{配列内包表記}{lf-array-comprehension}:}

ループと条件分岐を使って、素早く配列を生成して受け渡すことができます。

\begin{lstlisting}
var evenNumbers = [ for (i in 0...100) if (i\%2==0) i ];
\end{lstlisting}

\paragraph{\tref{クラス、インターフェース、継承}{types-class-instance}:}

Haxeはクラスを使ったコードの構造化ができる、オブジェクト指向言語です。継承やインターフェースといったJavaでサポートされるようなオブジェクト指向言語の標準的な機能を備えています。

\paragraph{\tref{条件付きコンパイル}{lf-condition-compilation}:}

条件付きコンパイルを使うことで、コンパイルのパラメータごとに固有のコードをコンパイルすることができます。これはターゲットごとの違いを抽象化する手助けになるだけでなく、詳細のデバッグ機能を提供するなどその他の用途にも使用できます。

\begin{lstlisting}
#if js
    js.Lib.alert("Hello");
#elseif sys
    Sys.println("Hello");
#end
\end{lstlisting}

\paragraph{\tref{(一般化)代数的データ型}{types-enum-instance}:}

Haxeではenumとして知られる、代数的データ型(ADT)を使ってデータ構造を表現することができます。さらに、Haxeは一般化されたヴァリアント(GADT)もサポートしています。

\begin{lstlisting}
enum Result {
    Success(data:Array<Int>);
    UserError(msg:String);
    SystemError(msg:String, position:PosInfos);
}
\end{lstlisting}

\paragraph{\tref{インライン呼び出し}{class-field-inline}:}

関数をインラインとして指定して、呼び出し場所にその関数のコードを挿入させることができます。これにより、手作業でのインライン化のようなコードの重複を発生させること無く、価値あるパフォーマンスの改善を得ることできます。

\paragraph{\tref{イテレータ(反復子)}{lf-iterators}:}

Haxeはイテレータを適切にあつかっているので、値のセット(例えば、配列)の反復処理がとても簡単です。自前のクラスであっても、イテレータ機能の実装をすることで素早く反復可能にすることができます。

\begin{lstlisting}
for (i in [1, 2, 3]) {
    trace(i);
}
\end{lstlisting}

\paragraph{\tref{ローカル関数とクロージャ}{expression-function}:}

Haxeでは関数はクラスフィールドに限定されず、式の中で定義することができます。その場合、強力なクロージャも使用可能です。

\begin{lstlisting}
var buffer = "";
function append(s:String) {
    buffer += s;
}
append("foo");
append("bar");
trace(buffer); // foobar
\end{lstlisting}

\paragraph{\tref{メタデータ}{lf-metadata}:}

フィールド、クラス、式に対してメタデータを追加できます。これにより、コンパイラ、マクロ、実行時のクラスに情報の受け渡しができます。

\begin{lstlisting}
class MyClass {
    @range(1, 8) var value:Int;
}
trace(haxe.rtti.Meta.getFields(MyClass).value.range); // [1,8]
\end{lstlisting}

\paragraph{\tref{静的拡張}{lf-static-extension}:}

既に存在するクラスやその他の型に対して、静的拡張を使うことで追加の機能を足すことができます。

\begin{lstlisting}
using StringTools;
"  Me & You    ".trim().htmlEscape();
\end{lstlisting}

\paragraph{\tref{文字列中の変数展開}{lf-string-interpolation}:}

シングルクオテーションを使って宣言した文字列では現在の文脈中の変数へのアクセスができます。

\begin{lstlisting}
trace('My name is $name and I work in ${job.industry}');
\end{lstlisting}

\paragraph{\tref{関数の部分適用}{lf-function-bindings}:} 

すべての関数は部分適用を使って、いくつかの引数だけに値を適用して残りの引数を後で指定できるように残すことができます。

\begin{lstlisting}
var map = new haxe.ds.IntMap();
var setToTwelve = map.set.bind(_, 12);
setToTwelve(1);
setToTwelve(2);
\end{lstlisting}

\paragraph{\tref{パターンマッチング}{lf-pattern-matching}:} 

複雑な構造体はenumや構造体から情報を抽出したり、特定の演算子で値の組み合わせを指定したりしながら、パターンを当てはめてマッチングすることができます。

\begin{lstlisting}
var a = { foo: 12 };
switch (a) {
    case { foo: i }: trace(i);
    default:
}
\end{lstlisting}

\paragraph{\tref{プロパティ}{class-field-property}:}

変数のクラスフィールドにはカスタムの読み込み書き込みアクセスを指定するプロパティが使えます。これにより、より良いアクセス制御が実現できます。

\begin{lstlisting}
public var color(get,set);
function get_color() {
    return element.style.backgroundColor;
}
function set_color(c:String) {
    trace('Setting background of element to $c');
    return element.style.backgroundColor = c;
}
\end{lstlisting}

\paragraph{\tref{アクセス制御}{lf-access-control}:}

Haxeでは、メタデータの構文を使って、クラスやフィールドに対してアクセスを許可したり強制したりといったアクセス制御を行うことできます。

\paragraph{\tref{型パラメータ、共変性、反変性}{type-system-type-parameters}:}

型には型パラメータをつけて、型付きのコンテナなど複雑なデータ構造を表現できます。型パラメータは特定の型への制限が可能で、また、変性のルールに従います。

\begin{lstlisting}
class Main<A> {
    static function main() {
        new Main<String>("foo");
        new Main(12); // 型推論を使う。
    }

    function new(a:A) { }
}
\end{lstlisting}

\section{条件付きコンパイル}
\label{lf-condition-compilation}

Haxeでは、\expr{\#if}、\expr{\#elseif}、\expr{\#else}を使って\emph{コンパイラフラグ}を確認することで条件付きコンパイルが可能です。

\define{コンパイラフラグ}{define-compiler-flag}{コンパイラフラグはコンパイルの過程に影響をあたえる、設定可能な値です。このフラグは\expr{-D key=value}あるいは単に\expr{-D key}(この場合デフォルト値の\expr{"1"}になる)の形式でコマンドラインから指定できます。そのほかにも、コンパイラはコンパイルの過程で別のステップへ情報伝達するために、内部的にいくつかのフラグを設定します。}

以下は条件付きコンパイルの利用例のデモです。

\haxe{assets/ConditionalCompilation.hx}

これをフラグ無しでコンパイルした場合、\expr{main}メソッドの\expr{trace("ok");}が実行されて終了します。他の分岐はファイルを構文解析する際に切り捨てられます。他の分岐についても、正しいHaxeの構文である必要がありますが、型チェックはされません。

\expr{\#if}と\expr{\#elseif}の直後の条件には以下の式が使えます。

\begin{itemize}
	\item すべての識別子は同名のコンパイラフラグの値で置きかえられます。コマンドラインから\expr{-D some-flag}を指定すると\expr{some-flag}と\expr{some\_flag}のフラグが定義されることに気を付けてください。
	\item \type{String}、\type{Int}、\type{Float}の定数値は直接使用されます。
		\item \type{Bool}の演算\expr{\&\&} (and)、\expr{||} (or)、\expr{!} (not) は期待どおりに動作しますが、式全体を小かっこでかこむ必要があります。
	\item \expr{==}、\expr{!=}、\expr{>}、\expr{>=}、\expr{<}、\expr{<=}の演算子が値の比較に使えます。
	\item 小かっこ\expr{()}は通常通り、式をグループ化するのに使えます。
\end{itemize}

Haxeの構文解析器は\expr{some-flag}を一つの句として認識しません、\expr{some - flag}の2項演算として読み取ります。このような場合はアンダースコアを使う\expr{some_flag}の版を使用する必要があります。

\paragraph{ビルトインのコンパイラフラグ}

ビルトインのコンパイラフラグの完全なリストはHaxeコンパイラを\expr{--help-defines}の引数をつけて呼び出すことで手に入れることができます。Haxeのコンパイラはコンパイルごとに複数の\expr{-D}フラグを指定できます。

\tref{コンパイラフラグ一覧}{lf-condition-compilation-flags}も確認してみてください。

\subsection{グローバルコンパイラフラグ}
\label{lf-condition-compilation-flags}

Haxe 3.0以降では\expr{haxe --help-defines}を実行することで、サポートしている\tref{コンパイラフラグ}{lf-condition-compilation}の一覧を取得することができます。

\begin{center}
\begin{tabular}{| l | l |}
	\hline
	\multicolumn{2}{|c|}{グローバルコンパイラフラグ} \\ \hline
	フラグ &  説明 \\ \hline
	\expr{absolute-path} &  \expr{trace}の出力を絶対パスで行います。 \\
	\expr{advanced-telemetry}  &  SWFをMonocleのツールで測定できるようにします。 \\
	\expr{analyzer}  &  静的解析器を使った最適化を行います(実験的)。 \\
	\expr{as3} &  flash9のas3のソースコードを出力する場合に定義されます。 \\
	\expr{check-xml-proxy}  &  xmlプロキシの使用済みフィールドを確認します。 \\
	\expr{core-api}  & core APIの文脈で定義されています。 \\
	\expr{core-api-serialize}  &  C\#で、いくつかのcore APIクラスをSerializable属性でマークします。 \\
	\expr{cppia}  &  実験的にC++のインストラクションアセンブリを出力します \\
	\expr{dce=<mode:std|full|no>}  &  \tref{デッドコード削除}{cr-dce}のモードを設定します(デフォルトではstd)。 \\
	\expr{dce-debug}  &  Show \tref{dead code elimination}{cr-dce} log \\
	\expr{debug}  &  \expr{-debug}をつけてコンパイルした場合に有効化されます。 \\
	\expr{display}  &  補完中に有効化されます。 \\
	\expr{dll-export}  &  実験的なリンクをつけてC++生成します。 \\
	\expr{dll-import}  &  実験的なリンクをつけてC++生成します。 \\
	\expr{doc-gen}  &  正しくドキュメントを生成するため、削除と変更をしないように振る舞います。 \\
	\expr{dump}  &  dumpサブディレクトリに、完全な型付け済みの抽象構文木を出力します。Haxeに似た形式で出力するには\expr{dump=pretty}を使ってください。 \\
	\expr{dump-dependencies}  &  dumpサブディレクトリに、クラスの依存関係を出力をします。 \\
	\expr{dump-ignore-var-ids}  &  prettyではないdumpから、変数IDを削除します。(diffを取るのに役立ちます) \\
	\expr{erase-generics}  &  C\#でジェネリッククラスを取り消します。 \\
	\expr{fdb}  &  FDBの対話的なデバッグのために、flashのデバッグ情報をすべて有効化します。 \\
	\expr{file-extension}  &  C++ソースコードで拡張子を出力します。 \\
	\expr{flash-strict}  &  Flash出力でより厳密な型付けを行います。 \\
	\expr{flash-use-stage}  &  SWFライブラリを初期のstageに保ちます。 \\
	\expr{force-lib-check}  &  コンパイラが-net-libと-java-libの追加クラスを確認するように強制します(内部用)。 \\
	\expr{force-native-property}  &  3.1の互換性のために、すべてプロパティに\expr{:nativeProperty}のタグ付けをします。 \\
	\expr{format-warning}  &  2.xの互換性のために、フォーマットされた文字列に対して警告を出します。 \\
	\expr{gencommon-debug}  &  GenCommonの内部用 \\
	\expr{haxe-boot}  &  flashのbootクラスに生成された名前の代わりに'haxe'という名前を使います。 \\
	\expr{haxe-ver}  &  現在のHaxeのバージョンの値です。 \\
	\expr{hxcpp-api-level}  &  hxcppのバージョン間の互換性を保ちます。 \\
	\expr{include-prefix}  &  含有している出力ファイルにパスを付加します。 \\
	\expr{interp}  &  \expr{--interp}でコンパイルされて実行される場合に定義されます。 \\
	\expr{java-ver=[version:5-7]}  &  ターゲットとするJavaのバージョンを設定します。 \\
	\expr{js-classic}  &  JavaScript出力にfunctionラッパーと、strictモードを使いません。 \\
	\expr{js-es5}  &  ES5に準拠した実行環境のためのJavaScriptを出力します。 \\
	\expr{js-unflatten}  &  packageや型でネストしたオブジェクトを出力します。 \\
	\expr{keep-old-output}  &  出力ディレクトリの古いコードのファイルを残します(C\#/Java)。 \\
	\expr{loop-unroll-max-cost}  & ループ展開がキャンセルされる最大コスト(expressions * iterations、デフォルトでは250)。 \\
	\expr{macro} & \tref{マクロの文脈}{macro}でコンパイルされた場合に定義されます。 \\
	\expr{macro-times} & \expr{--times}と一緒に使用された場合にマクロごとの時間を表示します。 \\
	\expr{net-ver=<version:20-45>}  &  ターゲットとする.NETのバージョンを設定します。 \\
	\expr{net-target=<name>}  &  .NETのターゲット名を設定します。xbox、micro \_(Micro Framework)\_、compact \_(Compact Framework)\_が、正当な名前です。 \\
	\expr{neko-source} & Nekoのバイトコードではなくソースコードを出力します。 \\
	\expr{neko-v1} &  Nekoの1.xとの互換性を保ちます。 \\
	\expr{network-sandbox}  &  ローカルファイルアクセスの代わりに、ローカルネットワークサンドボックスを使います。 \\
	\expr{no-compilation}  &  C++の最終コンパイルを無効化します。 \\
	\expr{no-copt}  &  コンパイル時の最適化を無効化します \_(デバッグ用途)\_ \\
	\expr{no-debug}  &  C++出力からすべてのデバッグマクロを取り除きます。 \\
	\expr{no-deprecation-warnings} & \expr{@:deprecated}のフィールドが使われたことによる警告を無効化します。 \\
	\expr{no-flash-override}  &  flashのみで、いくつかの基本クラスでのoverrideをHXサフィックスのついたメソッドで代替します。 \\
	\expr{no-opt}  &  最適化を無効化します。 \\
	\expr{no-pattern-matching}  &  パターンマッチングを無効化します。 \\
	\expr{no-inline}  &  \tref{インライン化}{class-field-inline}を無効化します。 \\
	\expr{no-root}  &  GenCSの内部用 \\
	\expr{no-macro-cache}  &  マクロの文脈でのキャッシュを無効化します。 \\
	\expr{no-simplify}  &  簡易化のフィルタを無効化します。 \\
	\expr{no-swf-compress}  &  SWF出力の圧縮を無効化します。 \\
	\expr{no-traces}  &  すべての\expr{trace}呼び出しを無効化します。 \\
	\expr{php-prefix}  &  \expr{--php-prefix}をつけてコンパイルした場合です。 \\
	\expr{real-position}  &  C\#ターゲットで、Haxeのソースマップを無効化します。 \\
	\expr{replace-files}  &  GenCommonの内部用です。 \\
	\expr{scriptable}  &  GenCPPの内部用です。 \\
	\expr{shallow-expose}  &  Haxeが生成したクロージャのスコープについて、windowオブジェクトの記述なしでのアクセスを許可します。 \\
	\expr{source-map-content}  &  JSのソースマップの一部として、.hxのソースコードを含ませます。 \\
	\expr{swc}  &  SWFの代わりにSWCを出力します。 \\
	\expr{swf-compress-level=<level:1-9>}  &  SWF出力の圧縮レベルを指定します。 \\
	\expr{swf-debug-password=<yourPassword>}  &  デバッグ用のパスワードを指定します。このパスワードはMD5アルゴリズムを使って暗号化されて、swfをデバッグするための認証解除を防ぎます。-D fdbを指定しない場合パスワードは使われません。 \\
	\expr{swf-direct-blit}  &  グラフィックの転送をするのにハードウェアアクセラレーションを使います。 \\
	\expr{swf-gpu}  &  グラフィックを描画するのにGPUを使います。 \\
	\expr{swf-metadata=<file.xml>}  &  swf内に\expr{<file.xml>}をメタデータとして埋め込みます。 \\
	\expr{swf-preloader-frame}  &  SWFの最初に空白フレームを挿入します。\expr{-D flash-use-stage}、\expr{-swf-lib}と合わせて使います。 \\
	\expr{swf-protected}  &  Haxeのprivateを、SWF内でpublicではなくprotectedを使うようにコンパイルします。 \\
	\expr{swf-script-timeout}  &  ActionScriptがタイムアウトのダイアログを表示するまでの最大時間を設定します(秒数で)。 \\
	\expr{swf-use-doabc}  &  DoAbcDefineのswfタグの代わりにDoAbcを使います。 \\
	\expr{sys}  &  システムのすべてのプラットフォームで定義されています。 \\
	\expr{unsafe}  &  C\#ターゲットでunsafeのコードを許容します \\
	\expr{use-nekoc}  &  内部のものの代わりにnekocのコンパイラを使います。 \\
	\expr{use-rtti-doc}  &  コンパイル中にドキュメントにアクセスできるようにします。 \\
	\expr{vcproj}  &  GenCPPの内部用。 \\
\end{tabular}
\end{center}

\section{extern}
\label{lf-externs}

externはターゲット固有の連携を型安全のルールに従って記述するために使います。宣言は普通のクラスに似た形で、以下の要素が必要です。

\begin{itemize}
	\item \expr{class}キーワードの前に\expr{extern}キーワードを置きます。
	\item \tref{メソッド}{class-field-method}は式を持ちません。
	\item すべての引数と戻り値の型を明示します。
\end{itemize}

\tref{Haxe標準ライブラリ}{std}の\type{Math}クラスがちょうどいい例です。その一部を抜粋します。

\begin{lstlisting}
extern class Math
{
	static var PI(default,null) : Float;
	static function floor(v:Float):Int;
}
\end{lstlisting}

\expr{extern}が、メソッドと変数の両方を定義できることがわかります(実際のところ、\expr{PI}は読み込み専用の\tref{プロパティ}{class-field-property})を定義しています。一度この情報がコンパイラで使用可能になると、型がわかり、フィールドへのアクセスが出来るようになります。

\haxe{assets/Extern.hx}

\expr{floor}メソッドの戻り値が\type{Int}して定義されているため、このように動作します。

Haxe標準ライブラリは多くの\expr{extern}を\target{Flash}、\target{JavaScript}ターゲット用にもっています。これにより、ネイティブのAPIに型安全のルールに従ってアクセス可能にし、より高いレベルのAPI設計の助けになります。\tref{haxelib}{haxelib}でも、多くのネイティブのライブラリの\expr{extern}を入手できます。

\target{Flash}、\target{Java}、\target{C\#}ターゲットでは、\tref{コマンドライン}{compiler-usage}から直接ネイティブライブラリの取り込みを行うことができます。ターゲットごとの詳細は\Fullref{target-details}のそれぞれの節で説明されています。

\target{Python}や、\target{JavaScript}といったターゲットでは、\expr{extern}クラスをネイティブのモジュールから読み込むために追加の「インポート」が必要になる場合があります。Haxeはそのような依存関係を宣言する仕組みを提供しているので、それらを\Fullref{target-details}のそれぞれの節で説明します。

\paragraph{可変長引数と、型の選択肢}
\since{3.2.0}

haxe.externパッケージはネイティブの概念をHaxeに対応させるため、2つの型を提供しています。

\begin{description}
	\item[\type{Rest<T>}:] この型は関数の最後の引数として使って、可変長の引数を追加で渡すことを可能にします。型パラメータは引数を特定の型に制限するのに使います。
	\item[\type{EitherType<T1,T2>}:] この型はパラメータのどちらかの型を使うことができるようにする。つまり、型の選択肢を表現できます。3つ以上の型を選ばせたい場合はネストさせて使います。
\end{description}

以下にデモを用意しました。

\haxe{assets/RestAndEitherType.hx}


\section{静的拡張}
\label{lf-static-extension}

\define{静的拡張}{define-static-extension}{静的拡張はすでに存在している型に対して、元のソースコードを変更することなく見せかけの拡張を行います。Haxeの静的拡張は最初の引数が拡張する対象の型である静的メソッドを宣言して、それ\expr{using}を使って記述しているクラス内に持ちこむことで使用できます。}

静的拡張は実際に型の変更を行うことなく型を強化する強力なツールです。以下の例で、その使い方を実演します。

\haxe{assets/StaticExtension.hx}

\type{Int}は元々\expr{triple}メソッドを持っていませんが、このプログラムは期待通り\expr{36}を出力します。\expr{12.triple()}の呼び出しが\expr{IntExtender.triple(12)}に変形されるためです。これには必要な条件が3つあります。

\begin{enumerate}
	\item 定数値の\expr{12}と\expr{triple}の最初の引数の型が、共に\type{Int}である
	\item \type{IntExtender}クラスが\expr{using Main.IntExtender}を使って現在の文脈に読み込まれている。
	\item \type{Int}自身は\expr{triple}フィールドを持っていない(持っていた場合、静的拡張よりも高い優先度になる)。
\end{enumerate}

静的拡張はシンタックスシュガーですが、コードの可読性に大きな影響を与えることには注目する価値があります。\expr{f1(f2(f3(f4(x))))}の形のネストされた呼び出しの代わりに、\expr{x.f4().f3().f2().f1()}のチェーンの形での呼び出しが可能になります。

優先順位のルールは\Fullref{type-system-resolution-order}ですでに説明されているとおり、\expr{using}式が複数ある場合は下から上へと確認がされ、各モジュールでは各型のフィールドが上から下へと確認がされます。モジュールを静的拡張として\expr{using}すると、そのすべての型が現在の文脈にインポートされます(モジュール内の特定の型の場合とは対照的です。詳しくは\Fullref{type-system-modules-and-paths}を見てください)。

\subsection{Haxe標準ライブラリについて}
\label{lf-static-extension-in-std}

Haxeの標準ライブラリのいくつかのクラスは静的拡張の用途に合うように設計されています。次の例からは\type{StringTools}の使い方がわかります。

\haxe{assets/StaticExtension2.hx}

\type{String}自身は\expr{replace}を持っていませんが、\expr{using StringTools}の静的拡張によって提供されます。いつものように、\target{JavaScript}への変換を見るとよくわかります。

\begin{lstlisting}
Main.main = function() {
	StringTools.replace("adc","d","b");
}
\end{lstlisting}

Haxe標準ライブラリでは以下のクラスが静的拡張として使うように設計されています。

\begin{description}
	\item[\type{StringTools}:] 置換やトリミングといった、文字列に対する拡張を提供します。
	\item[\type{Lambda}:] \type{Iterable}に対する関数型のメソッドを提供します。 
	\item[\type{haxe.EnumTools}:] 列挙型とそのインスタンスについての情報を得る機能を提供します。
	\item[\type{haxe.macro.Tools}:] マクロをあつかう際のさまざまな拡張を提供します(詳しくは\Fullref{macro-tools})。
\end{description}

\trivia{``using'' using}{\expr{using}キーワードが追加されて以降、\expr{using}を使う(using using)ときの問題や、その影響についての会話がよくされるようになりました。"using using"のせいでさまざまな場面でわかりにくい英語が生まれたため、このマニュアルの著者はこの機能をその実際の性質から静的拡張と呼ぶことに決めました。}

\section{パターンマッチング}
\label{lf-pattern-matching}
\state{NoContent}

\subsection{導入}
\label{lf-pattern-matching-introduction}

パターンマッチングは、与えられたパターンと値がマッチするかで分岐をする処理のことです。Haxeでは、すべてのパターンマッチングは\tref{\expr{switch}式}{expression-switch}の個々の\expr{case}式が表すパターンに従って行われます。それでは以下のデータ構造を使って、さまざまなパターンの構文を見ていきましょう。

\haxe[firstline=1,lastline=4]{assets/PatternMatching.hx}

以下はパターンマッチングの基本事項です。

\begin{itemize}
	\item パターンは上から下へとマッチングされます。
	\item 入力値にマッチする最上位のパターンの持っている式が実行されます。
	\item \expr{_}はすべてにマッチします。このため、\expr{case _:}は\expr{default:}と同じです。
\end{itemize}

\subsection{enumマッチング}
\label{lf-pattern-matching-enums}

enumのコンストラクタは直観的な方法でマッチングできます。

\haxe[firstline=8,lastline=21]{assets/PatternMatching.hx}

パターンマッチングでは、ケースを上から下へと確認していき、入力値とマッチする最初のものを見つけ出します。以下の各ケースを実行する流れの説明で、その過程を理解してください。

\begin{description}
	\item[\expr{case Leaf(_)}:] \expr{myTree}は\expr{Node}なので、マッチングに失敗します。
	\item[\expr{case Node(_, Leaf(_))}:] \expr{myTree}の右側の子要素は\expr{Leaf}ではなく、\expr{Node}なので失敗します。
	\item[\expr{case Node(_, Node(Leaf("bar"), _))}:] マッチングに成功します。
	\item[\expr{case _}:] 前のケースでマッチングが成功したので確認が行われません。
\end{description}

\subsection{変数の取り出し}
\label{lf-pattern-matching-variable-capture}

パターンの一部のあらゆる値は識別子を使ってマッチングさせて、取り出すことができます。

\haxe[firstline=24,lastline=30]{assets/PatternMatching.hx}

これは以下の流れにしたがって\expr{return}を行います。

\begin{itemize}
	\item \expr{myTree}が\expr{Leaf}の場合、その名前が返る。
	\item \expr{myTree}が\expr{Node}でその左の子要素が\expr{Leaf}の場合、その名前が返る(上の例の場合、これが適用されて\expr{"foo"}が返る)。
	\item そのほかの場合、\expr{"none"}が返る。
\end{itemize}

マッチされた値を取り出すのに\expr{=}を使うこともできます。

\haxe[firstline=32,lastline=36]{assets/PatternMatching.hx}

\expr{leafNode}には\expr{Leaf("foo")}が割り当てられているので、これにマッチします。そのほかのケースでは、\expr{myTree}自身が返ります。\expr{case x}は\expr{case _}と同じようにすべてにマッチしますが、\expr{x}のような識別子が使われるとマッチした値がその変数に対して割り当てられます。

\subsection{構造体マッチング}
\label{lf-pattern-matching-structure}

匿名構造体とインスタンスのフィールドに対してマッチさせることも可能です。

\haxe[firstline=38,lastline=50]{assets/PatternMatching.hx}

2番目のケースでは、\expr{rating}が\expr{"awesome"}にマッチすると、\expr{name}フィールドが識別子\expr{n}に割り当てられます。もちろん、この構造体を先の例のTreeに入れて、構造体と\expr{enum}を合わせたマッチングを行うこともできます。

クラスインスタンスについては、その親クラスのフィールドについてはマッチングできないという制限があります。

\subsection{配列マッチング}
\label{lf-pattern-matching-array}

配列は固定長のマッチングを行うことができます。

\haxe[firstline=52,lastline=60]{assets/PatternMatching.hx}

この例では、\expr{array[1]}が\expr{6}にマッチし、\expr{array[0]}は何でもよいので、\expr{1}が出力されます。

\subsection{orパターン}
\label{lf-pattern-matching-or}

\expr{|}演算子は複数のパターンが許容されることを示す用途で、パターン内のあらゆる箇所に使うことができます。

\haxe[firstline=63,lastline=68]{assets/PatternMatching.hx}

orパターン内で変数の取得をしたい場合、その子要素両方で行わなくてはいけません。

\subsection{ガード}
\label{lf-pattern-matching-guards}

\expr{case ... if(condition):}の構文を使ってパターンをさらに限定することができます。

\haxe[firstline=71,lastline=79]{assets/PatternMatching.hx}

最初のケースは追加のガード条件\expr{if (b > a)}を持っています。このケースはこの条件が正だった場合のみ選択され、それ以外の場合は次のケースとのマッチングが続きます。

\subsection{複数の値のマッチング}
\label{lf-pattern-matching-tuples}

配列の構文は複数の値のマッチングにも使えます。

\haxe[firstline=82,lastline=87]{assets/PatternMatching.hx}

これは通常の配列のマッチングによく似ていますが、以下の点で違います。

\begin{itemize}
	\item 要素数は固定です。このためパターンの配列の長さが違ってはいけません。
	\item switchしている値を取得できません。例えば、\expr{case x}は使えません(\expr{case _}は使えます)。
\end{itemize}

\subsection{抽出子(エクストラクタ)}
\label{lf-pattern-matching-extractors}
\since{3.1.0}

抽出子(エクストラクタ)はマッチした値に変更を適用することができます。マッチした値に小さな変更を適用して、さらにマッチングを行う場合に便利です。

\haxe{assets/Extractor2.hx}

この場合、\expr{TString}列挙型コンストラクタの引数の値を、\expr{temp}に割り当てて、さらにネストした\expr{temp.toLowerCase()}に対する\expr{switch}を行っています。見てのとおり、\expr{TString}が\expr{"foo"}の一部大文字のものを持っているので、このマッチングは成功します。これは抽出子を使うことで簡略化できます。

\haxe{assets/Extractor.hx}

抽出子は\expr{extractorExpression => match}の式によって認識されます。コンパイラはその前の例と同じようなコードを出力しますが、記述する構文はずいぶんと簡略化されました。抽出子は\expr{=>}で分断される以下の2つの部品からなります。

\begin{enumerate}
\item 左側はあらゆる式が可能で、アンダースコア(\expr{_})が出現する箇所すべてが、現在マッチする値で置き換えられます。
\item 右側は左側を評価した結果をマッチングするためのパターンです。
\end{enumerate}

右側はパターンですから、さらに別の抽出子を使うことが可能です。以下の例では2つの抽出子をチェーンさせています。

\haxe{assets/Extractor4.hx}

これは\expr{3}がマッチして\expr{add(3, 1)}を呼び出し、その結果の\expr{4}がマッチして\expr{mul(4, 3)}呼び出された結果として、\expr{12}が出力されます。2つ目の\expr{=>}の右側の\expr{a}は\tref{変数取り出し}{lf-pattern-matching-variable-capture}であることに注意してください。

現在は\tref{orパターン}{lf-pattern-matching-or}内で抽出子を使うことはできません。

\haxe{assets/Extractor5.hx}

しかし、orパターンを抽出子の右側に使うことはできます。そのため、上の例は小かっこ無しの場合ではコンパイル可能です。

\subsection{網羅性のチェック}
\label{lf-pattern-matching-exhaustiveness}

コンパイラは起こりうるケースが忘れ去られてないかのチェックを行います。

\begin{lstlisting}
switch(true) {
    case false:
} // Unmatched patterns: true (trueにマッチするパターンが無い)
\end{lstlisting}

マッチング対象の\type{Bool}型は\expr{true}と\expr{false}の2つの値を取り得ますが、\expr{false}のみがチェックされています。

\todo{Figure out wtf our rules are now for when this is checked.}

\subsection{無意味なパターンのチェック}
\label{lf-pattern-matching-unused}

同じように、コンパイラはどのような入力値に対してもマッチしないパターンを禁止します。

\begin{lstlisting}
switch(Leaf("foo")) {
    case Leaf(_)
       | Leaf("foo"): // This pattern is unused (このパターンは使用されない)
    case Node(l,r):
    case _: // This pattern is unused (このパターンは使用されない)
}
\end{lstlisting}

\section{文字列補間}
\label{lf-string-interpolation}

Haxe3では、\emph{文字列補間}のおかげで、手動で文字列をつなげ合わせる必要がなくなりました。シングルクオート(\expr{'})で囲まれた文字列の中で、ドル記号(\expr{\$})に続けて識別子を記述すると、その識別子を評価してつなげ合わせてくれます。

\begin{lstlisting}
var x = 12;
// xの値は12
trace('xの値は$x');
\end{lstlisting}

さらに、\expr{\$$\left\{expr\right\}$}を使うことで文字列内に式そのものを含めることが可能になります。この\expr{expr}はHaxeの正当な式であれば、なんでもかまいません。

\begin{lstlisting}
var x = 12;
// 12足す3は15
trace('$x足す3は${x + 3}');
\end{lstlisting}

文字列補間はコンパイル時の機能なので、実行時には影響を与えません。上の例は手動のつなげ合わせと同じです。コンパイラは以下と同様のコードを生成します。

\begin{lstlisting}
trace(x + "足す3は" + (x + 3));
\end{lstlisting}

もちろん、一切の補間なしでシングルクオートで囲んだ文字列を使用することができますが、\$の文字が補間のトリガーとして予約されてしまっていることに気を付けてください。文字列内でドル記号そのものを使いたい場合は\expr{\$\$}を使います。

\trivia{Haxe3以前の文字列補間}{文字列補間自体はバージョン2.09からHaxeの機能として存在しています。そのころは\expr{Std.format}のマクロが使われいました。これは新しい文字列補間の構文よりも遅くてあまり快適でないものでした。}

\section{配列内包表記}
\label{lf-array-comprehension}

\todo{Comprehensions are only listing Arrays, not Maps}

Haxeの配列内包表記は既存の構文を配列の初期化をより簡単にするためにも使えるようにするものです。配列内包表記は\expr{for}または\expr{while}のキーワードによって識別されます。

\haxe{assets/ArrayComprehension.hx}

変数\expr{a}は0から9までの数値を要素として持つ配列として初期化されます。コンパイラはループを作ってその繰り返しの一つ一つで要素を追加するコードを出力します。つまり以下のコードと等価です。

\begin{lstlisting}
var a = [];
for (i in 0...10) a.push(i);
\end{lstlisting}

変数\expr{b}も同じ値に初期化されますが、\expr{for}ではなく\expr{while}という異なる内包表記の形式を使っています。そして、これは以下のコードと等価です。

ループの式は条件分岐やループのネストを含めて、いかなる式でもかまいません。ですから、以下の式は期待通りに動作します。

\haxe{assets/AdvArrayComprehension.hx}

\section{イテレータ(反復子)}
\label{lf-iterators}

Haxeでは、カスタムのイテレータや反復可能(iterable)なデータ型を簡単に定義できます。これらの概念は\type{Iterator<T>}型と\type{Iterable<T>}型を使って以下のように表現されています。

\begin{lstlisting}
typedef Iterator<T> = {
	function hasNext() : Bool;
	function next() : T;
}

typedef Iterable<T> = {
	function iterator() : Iterator<T>;
}
\end{lstlisting}

これらの型のいずれかで\tref{構造的に単一化できる}{type-system-structural-subtyping}あらゆる\tref{class}{types-class-instance}は、\tref{forループ}{expression-for}で反復処理を行うことができます。つまり、型が合うように\expr{hasNext}と\expr{next}メソッドを定義すればそのクラスはイテレータであるし、\type{Iterator<T>}を返す\expr{iterator}メソッドを定義すれば反復可能な型です。

\haxe{assets/Iterator.hx}

この例での\type{MyStringIterator}は\type{Bool}型を返す\expr{hasNext}と\type{String}型を返す\expr{next}メソッドを定義しているので、イテレータであると見なされます。また\expr{next}の戻り値の型から、これは\type{Iterator<String>}です。\expr{main}メソッドでこれをインスタンス化して反復処理を行っています。

\haxe{assets/Iterable.hx}

こちらは1つ前の例とは異なり自前の\expr{Iterator}を準備していませんが、代わりに\type{MyArrayWrap<T>}は\type{Array<T>}の\expr{iterator}関数を効果的に利用しています。


\section{関数の束縛(bind)}
\label{lf-function-bindings}

Haxe3では、部分的に引数を適用して関数を束縛することが可能です。すべての関数型は\expr{bind}フィールドを持っており、これを呼び出すことで引数の数を減らした新しい関数を作りだすことができます。その実例を示します。

\haxe{assets/Bind.hx}

行4では、\expr{map.set}関数に2番目の引数に\expr{12}を適用し、\expr{f}という変数に割り当てました。アンダースコア(\expr{_})はその引数を束縛しないことを表すのに使います。このことは\expr{map.set}と、\expr{f}の型の比較でもわかります。束縛された\type{String}型の引数が取り除かれたので、\expr{Int->String->Void}型が\expr{Int->Void}型に変わっています。

\expr{f(1)}を呼び出したことで実際には\expr{map.set(1, "12")}が実行されます。\expr{f(2)}、\expr{f(3)}の呼び出しでも同じ関係性が成り立ちます。最後の行で、3つのインデックスすべてに紐づく値が\expr{"12"}になっていることが確認できます。

アンダースコア(\expr{_})は末尾の引数では省略することができます。つまり、\expr{map.set.bind(1)}で最初の引数を束縛した場合、インデックス\expr{1}について新しい値を設定する\expr{String->Void}関数が提供されます。

\trivia{コールバック}{Haxe3よりも前のバージョンでは、\expr{callback}キーワードに1つの関数の引数と任意の個数の束縛する引数をつけて呼び出しをしていました。この束縛する機能に対してコールバック関数という名前が使われるようになっていました。\\
\expr{callback}は左から右への束縛のみでアンダースコア(\expr{_})はサポートしていませんでした。アンダースコアを使うという選択肢は論争を生み、そのほかの案もいくつか現れましたがこれより優れているものはありませんでした。少なくともアンダースコア(\expr{_})は「ここに値を入れて」と言っているように見えるので、この意味を書き表すのに適しているという結論にいたりました。}

\section{メタデータ}
\label{lf-metadata}

以下の要素はメタデータで属性をつけることができます。

\begin{itemize}
	\item \expr{class}、\expr{enum}の定義
	\item クラスフィールド
	\item 列挙型コンストラクタ
	\item 式
\end{itemize}

これらのメタデータの情報は\type{haxe.rtti.Meta}のAPIを使って実行時に利用することが可能です。

\haxe{assets/Meta.hx}

メタデータは\expr{@}の文字で始まり、メタデータの名前が続き、その後にオプションでカンマで区切った定数値の引数が小かっこで囲まれている、ということで簡単に識別できます。

\begin{itemize}
	\item \type{MyClass}クラスは\expr{"Nicolas"}という文字列の引数1つを持つ\expr{author}メタデータと、引数を持たない\expr{debug}メタデータを持ちます。
	\item メンバ変数\expr{value}は\expr{1}と\expr{8}の2つの整数の引数を持つ\expr{range}メタデータを持ちます。
	\item 静的メソッド\expr{method}は引数なしの\expr{broken}メタデータと、引数なしの\expr{:noCompletion}メタデータを持ちます。
\end{itemize}

\expr{main}メソッドでは、APIを通してこれらのメタデータへアクセスしています。この出力からは取得可能なデータの構造が分かります。

\begin{itemize}
	\item 各メタデータについてフィールドがあり、フィールドの名前はメタデータの名前です。
	\item フィールドの値はメタデータの引数に一致します。引数がない場合、フィールドの値は\expr{null}です。その他の場合、フィールドの値は引数1つが要素1つになった配列です。
	\item \expr{:}から始まるメタデータは省略されます。このメタデータは\emph{コンパイラメタデータ}として知られます。
\end{itemize}

メタデータの引数の値は以下が使用できます。

\begin{itemize}
	\item \tref{定数値}{expression-constants}
	\item \tref{配列の宣言}{expression-array-declaration} (すべての要素がこのリストのいずれか)
	\item \tref{オブジェクトの宣言}{expression-object-declaration} (すべての要素がこのリストのいずれか)
\end{itemize}

\paragraph{ビルトインのコンパイラメタデータ}
コマンドラインから\expr{haxe --help-metas}を実行することで、定義済みメタデータの完全なリストを得ることができます。

詳しくは\tref{コンパイラメタデータのリスト}{cr-metadata}を見てください。

\section{アクセス制御}
\label{lf-access-control}

基本的な\tref{可視性}{class-field-visibility}のオプションで十分でない場合、アクセス制御が役に立ちます。アクセス制御は\emph{クラスレベル}と\emph{フィールドレベル}、そして以下の2方向の適用が可能です。

\begin{description}
	\item[アクセス許可:] \expr{:allow(target)}\tref{メタデータ}{lf-metadata}を使うことで、対象を与えられたクラスやフィールドからのアクセスを許容するようにします。
	\item[アクセス強制:] \expr{:access(target)}\tref{メタデータ}{lf-metadata}を使うことで、対象からの与えられたクラスやフィールドへのアクセスを強制的に可能にします。
\end{description}

このとき、\expr{target}には以下の\tref{ドットパス}{define-type-path}を使うことができます。

\begin{itemize}
	\item \emph{クラスフィールド}
	\item \emph{クラス}、\emph{抽象型}
	\item \emph{パッケージ}
\end{itemize}

\expr{target}はインポートを参照しません。つまり、完全なパスを正しく記述する必要があります。

クラスや抽象型の場合、アクセスの変更はその型のすべてのフィールドに反映されます。同じように、パッケージの場合、アクセスの変更はそのパッケージ内のすべての型のすべてのフィールドに反映されます。

\haxe{assets/ACL.hx}

\expr{MyClass.foo}は\type{MyClass}に\expr{@:allow(Main)}を適用しているので、\expr{main}メソッドからアクセスできます。このコードは\expr{@:allow(Main.main)}でも動作しますし、以下のように\type{MyClass}クラスの\expr{foo}フィールドにメタデータをつけても動作します。

\haxe{assets/ACL2.hx}

もし型にこのようなアクセスの変更ができない場合は、アクセス強制の方法が役立つかもしれません。

\haxe{assets/ACL3.hx}

\expr{@:access(MyClass.foo)}のメタデータは\expr{main}メッソドからの\expr{foo}の可視性を変更します。

\trivia{メタデータという選択肢}{アクセス制御の言語機能には、新しい構文の導入ではなく、Haxeのメタデータの構文を使いました。これには以下のいくつかの理由があります。
\begin{itemize}
	\item 追加の構文は言語の構文解析を複雑にして、さらにはキーワードを増やしてしまします。
	\item 追加の構文は言語のユーザーに追加の学習を要求します。メタデータであれば、それは既知のものです。
	\item メタデータはこの拡張を行うのに十分な表現力を持っています。
	\item メタデータはHaxeのマクロから、アクセスし、生成し、編集することが可能です。
\end{itemize}
もちろん、メタデータ構文の主な不利益はメタデータの名前、クラスやパッケージ名についてスペルミス(例えば、@:acesss)をした場合に何のエラーも出ないことです。しかし、この機能では実際に\expr{private}フィールドにアクセスしようとした場合にエラーがでるので、エラーが沈黙しているということにはなりえません。}

\since{3.1.0}

アクセスが\tref{インターフェース}{types-interfaces}に対して許可される場合、そのインターフェースを実装しているすべてのクラスに対してそれが引き継がれます。

\haxe{assets/ACL4.hx}

これは親クラスの場合も同様です。その場合、子クラスに対して引き継ぎがされます。

\trivia{壊れた機能}{子クラスや実装クラスへのアクセスの継承はHaxe3.0への導入を予定されており、そしてドキュメントまでも作られていました。しかし、このマニュアルを作る過程でこのアクセス制御の実装がぬけ落ちていることを発見しました。}

\section{インラインコンストラクタ}
\label{lf-inline-constructor}
\since{3.1.0}

コンストラクタに、\tref{inline}{class-field-inline}の宣言をつけると、コンパイラは特定の場合において最適化を試みます。この最適化が動作するためにはいくつかの必要事項があります。

\begin{itemize}
	\item コンストラクタの呼び出しの結果はローカル変数への直接の代入でなければいけない。
	\item コンストラクタフィールドの式はそのフィールドへの代入のみでなければならない。
\end{itemize}

以下に、コンストラクタのインライン化の実例を挙げます。

\haxe{assets/NewInline.hx}

\target{JavaScript}出力をみると、その効果がわかります。

\begin{lstlisting}
Main.main = function() {
	var pt_x = 1.2;
	var pt_y = 9.3;
};
\end{lstlisting}


\part{コンパイラリファレンス}
\chapter{コンパイラの使い方}
\label{compiler-usage}

\paragraph{基本的な使い方}

Haxeコンパイラは基本的には、以下の2つの質問に答える引数をつけてコマンドラインから呼び出します。

\begin{itemize}
	\item 何をコンパイルするのか?
	\item 何を出力するのか?
\end{itemize}

最初の質問に答えるためには、\ic{-cp path}引数でクラスパスを指定して、\ic{-main dot_path}引数でコンパイル対象のメインクラスを指定すれば十分です。これでHaxeコンパイラはメインクラスのファイルを解決しコンパイルを始めます。

2つ目の質問に答えるためには、目的のターゲットごとの引数を指定します。Haxeのターゲットはそれぞれ専用のコマンドラインオプションを持っています。例えば、\target{JavaScript}は\ic{-js file_name}、\target{PHP}は\ic{-php directory}です。ターゲットによってファイル名を指定するもの(\ic{-js}、\ic{-swf}、\ic{-neko}、\ic{-python}が該当)と、ディレクトリを指定するものがあります。

\paragraph{よく使う引数}

■\emph{入力:}

\begin{description}
	\item[\ic{-cp path}] \ic{.hx}のソースファイルまたはパッケージ(サブディレクトリ)が置かれているディレクトリのパスを追加します。
	\item[\ic{-lib library_name}] \Fullref{haxelib}のライブラリを追加します。
	\item[\ic{-main dot_path}] メインクラスを設定します。
\end{description}

■\emph{出力:}

\begin{description}
	\item[\ic{-js file_name}] 指定されたファイルに\tref{JavaScript}{target-javascript}のソースコードを出力します。
	\item[\ic{-as3 directory}] 指定されたディレクトリにActionScript3のソースコードを出力します。
	\item[\ic{-swf file_name}] 指定されたファイルに\tref{Flash}{target-flash}の.swfを出力します。
	\item[\ic{-neko file_name}] 指定されたファイルに\tref{Neko}{target-neko}のバイナリを出力します。
	\item[\ic{-php directory}] 指定されたディレクトリに\tref{PHP}{target-php}のソースコードを出力します。
	\item[\ic{-cpp directory}] 指定されたディレクトリに\tref{C++}{target-cpp}のソースコードを出力して、ネイティブのC++コンパイラでコンパイルします。
	\item[\ic{-cs directory}] 指定されたディレクトリに\tref{C\#}{target-cs}のソースコードを出力します。
	\item[\ic{-java directory}] 指定されたディレクトリに\tref{Java}{target-java}のソースコードを出力して、Javaコンパイラでコンパイルします。
	\item[\ic{-python file_name}] 指定されたファイルに\tref{Python}{target-python}のソースコードを出力します。
\end{description}

\chapter{Compiler Features}
\label{cr-features}
\state{NoContent}

\section{Built-in Compiler Metadata}
\label{cr-metadata}

Starting from Haxe 3.0, you can get the list of defined compiler metadata by running \expr{haxe --help-metas}

\begin{center}
\begin{tabular}{| l | l | l |}
	\hline
	\multicolumn{3}{|c|}{Global metadata} \\ \hline
	Metadata &  Description  &  Platform \\ \hline
	@:abi & Function ABI/calling convention  & cpp \\
	@:abstract &  Sets the underlying class implementation as \tref{abstract type}{types-abstract}  &  cs  java \\
	@:access \_(Target path)\_  &   Forces private access to package  type or field,  see \tref{Access Control}{lf-access-control}  &  all \\
	@:allow \_(Target path)\_  &   Allows private access from package  type or field,  see \tref{Access Control}{lf-access-control}  &  all \\
	@:analyzer & Used to configure the static analyzer  &  all \\
	@:annotation  &  Annotation (\expr{@interface}) definitions on \expr{-java-lib} imports will be annotated with this metadata. Has no effect on types compiled by Haxe   &  java \\
	@:arrayAccess  &  Allows \tref{Array access}{types-abstract-array-access} on an abstract  &  all \\
	@:autoBuild \_(Build macro call)\_  &   Extends \expr{@:build} metadata to all extending and implementing classes. See \tref{Macro autobuild}{macro-auto-build}  &  all \\
	@:bind  &  Override Swf class declaration  &  flash \\
	@:bitmap \_(Bitmap file path)\_  &  \_Embeds given bitmap data into the class (must extend \expr{flash.display.BitmapData})   &  flash \\
	@:bridgeProperties  &  Creates native property bridges for all Haxe properties in this class  &  cs \\
	@:build \_(Build macro call)\_  &   Builds a class or enum from a macro. See \tref{Type Building}{macro-type-building}  &  all \\
	@:buildXml  &  Specify xml data to be injected into Build.xml  &  cpp \\
	@:callable  &  Abstract forwards call to its underlying type  &  all \\
	@:classCode  &  Used to inject platform-native code into a class  &  cs  java \\
	@:commutative  &  Declares an abstract operator as commutative  &  all \\
	@:compilerGenerated  &  Marks a field as generated by the compiler. Shouldn't be used by the end user  &  cs  java \\
	@:coreApi &  Identifies this class as a core api class (forces Api check)  &  all \\
	@:coreType  &  Identifies an abstract as \tref{core type}{types-abstract-core-type} so that it requires no implementation  &  all \\
	@:cppFileCode  &  Code to be injected into generated cpp file  &  cpp \\
	@:cppInclude  &  File to be included in generated cpp file  &  cpp \\
	@:cppNamespaceCode  &    &  cpp \\
	@:dce  &  Forces \tref{Dead Code Elimination}{cr-dce} even when not \expr{-dce full} is specified  &  all \\
	@:debug  &  Forces debug information to be generated into the Swf even without \expr{-debug}   &  flash \\
	@:decl   &     &  cpp \\
	@:defParam  &    &  all \\
	@:delegate  &  Automatically added by \expr{-net-lib} on delegates   &  cs \\
	@:depend  &     &  cpp \\
	@:deprecated   &  Automatically added by \expr{-java-lib} on class fields annotated with \expr{@Deprecated} annotation. Has no effect on types compiled by Haxe  &  java \\
	@:event  &  Automatically added by \expr{-net-lib} on events. Has no effect on types compiled by Haxe   &  cs \\
	@:enum  &  Defines finite value sets to abstract definitions. See \tref{enum abstracts}{types-abstract-enum}  &  all \\
	@:expose \_(?Name=Class path)\_  &  Makes the class available on the \expr{window} object or \expr{exports} for node.js. See \tref{exposing Haxe classes for JavaScript}{target-javascript-expose} &  js \\
	@:extern  &  Marks the field as extern so it is not generated  &  all \\
	@:fakeEnum \_(Type name)\_  &  Treat enum as collection of values of the specified type  &  all \\
	@:file(File path)  &  Includes a given binary file into the target Swf and associates it with the class (must extend \expr{flash.utils.ByteArray})  &  flash \\
	@:final  &  Prevents a class from being extended  &  all \\
	@:font \_(TTF path Range String)\_  &  Embeds the given TrueType font into the class (must extend \expr{flash.text.Font})  &  flash \\
	@:forward \_(List of field names)\_  &  \tref{Forwards field access}{types-abstract-forward} to underlying type  &  all \\
	@:from   &  Specifies that the field of the abstract is a cast operation from the type identified in the function. See \tref{Implicit Casts}{types-abstract-implicit-casts}  &  all \\
	@:functionCode  &     &  cpp \\
	@:functionTailCode  &    &  cpp \\
	@:generic &  Marks a class or class field as \tref{generic}{type-system-generic} so each type parameter combination generates its own type/field  &  all \\
	@:genericBuild  &  Builds instances of a type using the specified macro   &  all \\
	@:getter \_(Class field name)\_  &  Generates a native getter function on the given field   &  flash \\
	@:hack   &  Allows extending classes marked as \expr{@:final}  &  all \\
	@:headerClassCode  &  Code to be injected into the generated class, in the header  &  cpp \\
	@:headerCode   &  Code to be injected into the generated header file  &  cpp \\
	@:headerNamespaceCode  &    &  cpp \\
	@:hxGen  &  Annotates that an extern class was generated by Haxe  &  cs  java \\
	@:ifFeature \_(Feature name)\_  &  Causes a field to be kept by \tref{DCE}{cr-dce} if the given feature is part of the compilation  &  all \\
	@:include &     &  cpp \\
	@:initPackage  &    &  all \\
	@:internal  &  Generates the annotated field/class with \expr{internal} access  &  cs  java \\
	@:isVar  &  Forces a physical field to be generated for properties that otherwise would not require one  &  all \\
	@:javaCanonical \_(Output type package,Output type name)\_ &  Used by the Java target to annotate the canonical path of the type  &  java \\
	@:jsRequire  &  Generate javascript module require expression for given extern  &  js \\
	@:keep   &  Causes a field or type to be kept by \tref{DCE}{cr-dce}  &  all \\
	@:keepInit  &  Causes a class to be kept by \tref{DCE}{cr-dce} even if all its field are removed  &  all \\
	@:keepSub &  Extends \expr{@:keep} metadata to all implementing and extending classes  &  all \\
	@:macro  &  \_(deprecated)\_  &  all \\
	@:mergeBlock  &  Merge the annotated block into the current scope  &  all \\
	@:meta   &  Internally used to mark a class field as being the metadata field  &  all \\
	@:multiType \_(Relevant type parameters)\_  &  Specifies that an abstract chooses its this-type from its \expr{@:to} functions  &  all \\
	@:native \_(Output type path)\_  &  Rewrites the path of a class or enum during generation  &  all \\
	@:nativeChildren  &  Annotates that all children from a type should be treated as if it were an extern definition - platform native  &  cs java \\
	@:nativeGen  &  Annotates that a type should be treated as if it were an extern definition - platform native  &  cs  java \\
	@:nativeProperty  &  Use native properties which will execute even with dynamic usage  &  cpp \\
	@:noCompletion  &  Prevents the compiler from suggesting \tref{completion}{cr-completion} on this field  &  all \\
	@:noDebug &  Does not generate debug information into the Swf even if \expr{-debug} is set   &  flash \\
	@:noDoc  &  Prevents a type from being included in documentation generation  &  all \\
	@:noImportGlobal  &  Prevents a static field from being imported with \expr{import Class.*}  &  all \\
	@:noPrivateAccess  &  Disallow private access to anything for the annotated expression  &  all \\
	@:noStack &     &  cpp \\
	@:noUsing &  Prevents a field from being used with \expr{using}  &  all \\
	@:nonVirtual &  Declares function to be non-virtual  &  cpp \\
	@:notNull &  Declares an abstract type as not accepting \tref{\expr{null} values}{types-nullability}  &  all \\
	@:ns  &  Internally used by the Swf generator to handle namespaces   &  flash \\
	@:op \_(The operation)\_  &   Declares an abstract field as being an \tref{operator overload}{types-abstract-operator-overloading}  &  all \\
	@:optional  &  Marks the field of a structure as optional. See \tref{Optional Arguments}{types-nullability-optional-arguments}  &  all \\
	@:overload \_(Function specification)\_  &  Allows the field to be called with different argument types. Function specification cannot be an expression  &  all \\
	@:privateAccess  &  Allow private access to anything for the annotated expression  &  all \\
	@:property  &  Marks a property field to be compiled as a native C\# property   &  cs \\
	@:protected  &  Marks a class field as being protected  &  all \\
	@:public  &  Marks a class field as being public  &  all \\
	@:publicFields  &  Forces all class fields of inheriting classes to be public  &  all \\
	@:pythonImport  &  Generates python import statement for extern classes  &  python \\
	@:readOnly  &  Generates a field with the \expr{readonly} native keyword   &  cs \\
	@:remove  &  Causes an interface to be removed from all implementing classes before generation  &  all \\
	@:require \_(Compiler flag to check)\_  &  Allows access to a field only if the specified \tref{compiler flag}{lf-condition-compilation} is set  &  all \\
	@:rtti   &  Adds runtime type informations. See \tref{RTTI}{cr-rtti}  &  all \\
	@:runtime  &    &  all \\
	@:runtimeValue  &  Marks an abstract as being a runtime value  &  all \\
	@:selfCall  &  Translates method calls into calling object directly  &  js \\
	@:setter \_(Class field name)\_  &  Generates a native setter function on the given field   &  flash \\
	@:sound \_(File path)\_  &  Includes a given \_.wav\_ or \_.mp3\_ file into the target Swf and associates it with the class (must extend \expr{flash.media.Sound})  &  flash \\
	@:sourceFile  &  Source code filename for external class  &  cpp \\
	@:strict  &  Used to declare a native C\# attribute or a native Java metadata. Is type checked  &  cs java \\
	@:struct  &  Marks a class definition as a struct   &  cs \\
	@:structAccess  &  Marks an extern class as using struct access('.') not pointer('->')  &  cpp \\
	@:suppressWarnings  &  Adds a SuppressWarnings annotation for the generated Java class  &  java \\
	@:throws \_(Type as String)\_  &  Adds a \expr{throws} declaration to the generated function   &  java \\
	@:to  &  Specifies that the field of the abstract is a cast operation to the type identified in the function. See \tref{Implicit Casts}{types-abstract-implicit-casts} & all \\
	@:transient  &  Adds the \expr{transient} flag to the class field  &  java \\
	@:unbound  &  Compiler internal to denote unbounded global variable  &  all \\
	@:unifyMinDynamic  &  Allows a collection of types to unify to Dynamic  &  all \\
	@:unreflective  &    &  cpp \\
	@:unsafe  &  Declares a class  or a method with the C\#'s \expr{unsafe} flag   &  cs \\
	@:usage  &    &  all \\
	@:value  &  Used to store default values for fields and function arguments  &  all \\
	@:void  &  Use Cpp native 'void' return type  &  cpp \\
	@:volatile  &    &  cs  java \\
\end{tabular}
\end{center}

\section{Dead Code Elimination}
\label{cr-dce}

Dead Code Elimination or \emph{DCE} is a compiler feature which removes unused code from the output. After typing, the compiler evaluates the DCE entry-points (usually the main-method) and recursively determines which fields and types are used. Used fields are marked accordingly and unmarked fields are then removed from their classes.

DCE has three modes which are set when invoking the command line:

\begin{description}
	\item[-dce std:] Only classes in the Haxe Standard Library are affected by DCE. This is the default setting on all targets.
	\item[-dce no:] No DCE is performed.
	\item[-dce full:] All classes are affected by DCE.
\end{description}
The DCE-algorithm works well with typed code, but may fail when \tref{dynamic}{types-dynamic} or \tref{reflection}{std-reflection} is involved. This may require explicit marking of fields or classes as being used by attributing the following metadata:

\begin{description}
	\item[\expr{@:keep}:] If used on a class, the class along with all fields is unaffected by DCE. If used on a field, that field is unaffected by DCE.
	\item[\expr{@:keepSub}:] If used on a class, it works like \expr{@:keep} on the annotated class as well as all subclasses.
	\item[\expr{@:keepInit}:] Usually, a class which had all fields removed by DCE (or is empty to begin with) is removed from the output. By using this metadata, empty classes are kept.
\end{description}

If a class needs to be marked with \expr{@:keep} from the command line instead of editing its source code, there is a compiler macro available for doing so: \expr{--macro keep('type dot path')} See the \href{http://api.haxe.org/haxe/macro/Compiler.html#keep}{haxe.macro.Compiler.keep API} for details of this macro. It will mark package, module or sub-type to be kept by DCE and includes them for compilation.
 
The compiler automatically defines the flag \expr{dce} with a value of either \expr{"std"}, \expr{"no"} or \expr{"full"} depending on the active mode. This can be used in \tref{conditional compilation}{lf-condition-compilation}.

\trivia{DCE-rewrite}{DCE was originally implemented in Haxe 2.07. This implementation considered a function to be used when it was explicitly typed. The problem with that was that several features, most importantly interfaces, would cause all class fields to be typed in order to verify type-safety. This effectively subverted DCE completely, prompting the rewrite for Haxe 2.10.}

\trivia{DCE and try.haxe.org}{DCE for the \type{JavaScript} target saw vast improvements when the website \url{http://try.haxe.org} was published. Initial reception of the generated \target{JavaScript} code was mixed, leading to a more fine-grained selection of which code to eliminate.}




\section{Completion}
\label{cr-completion}
\state{NoContent}

\subsection{Overview}
\label{cr-completion-overview}

The rich \tref{type system}{type-system} of the Haxe Compiler makes it difficult for IDEs and editors to provide accurate completion information. Between \tref{type inference}{type-system-type-inference} and \tref{macros}{macro}, it would require a substantial amount of work to replicate the required processing. This is why the Haxe Compiler comes with a built-in completion mode for third-party software to use.

All completion is triggered using the \ic{--display file@position[@mode]} compiler argument. The required arguments are:

\begin{description}
	\item[file:] The file to check for completion. This must be an absolute or relative path to a .hx file. It does not respect any class paths or libraries.
	\item[position:] The byte position (not character position) of where to check for completion in the given file.
	\item[mode:] The completion mode to use (see below).
\end{description}

We will look into the following completion modes in detail:

\begin{description}
	\item[\tref{Field access}{cr-completion-field-access}:] Provides a list of fields that can be accessed on a given type.
	\item[\tref{Call argument}{cr-completion-call-argument}:] Reports the type of the function which is currently being called.
	\item[\tref{Type path}{cr-completion-type-path}:] Lists sub-packages, sub-types and static fields.
	\item[\tref{Usage}{cr-completion-usage}:] Lists all occurrences of a given type, field or variable in all compiled files. (mode: \ic{"usage"})
	\item[\tref{Position}{cr-completion-position}:] Reports the position of where a given type, field or variable is defined. (mode: \ic{"position"})
	\item[\tref{Top-level}{cr-completion-top-level}:] Lists all identifiers which are available at a given position. (mode: \ic{"toplevel"})
\end{description}

Due to Haxe being a very fast compiler, it is often sufficient to rely on the normal compiler invocation for completion. For bigger projects Haxe provides a \tref{server mode}{cr-completion-server} which ensures that only those files are re-compiled that actually changed or had any of their dependencies changes.

\paragraph{General notes on the interface}
\label{cr-completion-interface-notes}

\begin{itemize}
	\item The position-argument can be set to 0 if the file in question contains a pipeline \ic{|} character at the position of interest. This is very useful for demonstration and testing as it allows us to ignore the byte-counting process a real IDE would have to do. The examples in this section are making use of this feature. Note that this only works in places where \ic{|} is not valid syntax otherwise, e.g. after dots (\ic{.|}) and opening parentheses (\ic{(|}).
	\item The output is HTML-escaped so that \ic{\&}, \ic{<} and \ic{>} become \ic{\&amp;}, \ic{\&lt;} and \ic{\&gt;} respectively.
	\item Otherwise any documentation output is preserved which means longer documentation might include new-line and tab-characters as it does in the source files.
	\item When run in completion mode, the compiler does not display errors but instead tries to ignore them or recover from them.  If a critical error occurs while getting completion, the Haxe Compiler prints the error message instead of the completion output. Any non-XML output can be treated as a critical error message.
\end{itemize}

\subsection{Field access completion}
\label{cr-completion-field-access}

Field completion is triggered after a dot \ic{.} character to list the fields that are available on the given type. The compiler parses and types everything up to the point of completion and then outputs the relevant information to stderr:

\begin{lstlisting}
class Main {
  public static function main() {
    trace("Hello".|
  }
}
\end{lstlisting}

If this file is saved to Main.hx, the completion can be invoked using the command \ic{haxe --display Main.hx@0}. The output looks similar to this (we omit several fields for brevity and improve the formatting for readability):

\begin{lstlisting}
<list>
<i n="length">
  <t>Int</t>
  <d>
    The number of characters in `this` String.
  </d>
</i>
<i n="charAt">
  <t>index : Int -&gt; String</t>
  <d>
    Returns the character at position `index` of `this` String.
    If `index` is negative or exceeds `this.length`, the empty String
    "" is returned.
  </d>
</i>
<i n="charCodeAt">
  <t>index : Int -&gt; Null&lt;Int&gt;</t>
  <d>
    Returns the character code at position `index` of `this` String.
    If `index` is negative or exceeds `this.length`, null is returned.
    To obtain the character code of a single character, "x".code can
    be used instead to inline the character code at compile time.
    Note that this only works on String literals of length 1.
  </d>
</i>
</list>
\end{lstlisting}

The XML structure follows:

\begin{itemize}
	\item The document node \ic{list} encloses several nodes \ic{i}, each representing a field.
	\item The \ic{n} attribute contains the name of the field.
	\item The \ic{t} node contains the type of the field.
	\item the \ic{d} node contains the documentation of the field.
\end{itemize}

\since{3.2.0}

When compiling with \ic{-D display-details}, each field additionally has a \ic{k} attribute which can either be \ic{var} or \ic{method}. This allows distinguishing method fields from variable fields that have a function type.



\subsection{Call argument completion}
\label{cr-completion-call-argument}

Call argument completion is triggered after an opening parenthesis character \ic{(} to show the type of the function that is currently being called. It works for any function call as well as constructor calls:

\begin{lstlisting}
class Main {
  public static function main() {
    trace("Hello".split(|
  }
}
\end{lstlisting}

If this file is saved to Main.hx, the completion can be invoked using the command \ic{haxe --display Main.hx@0}. The output looks like this:

\begin{lstlisting}
<type>
delimiter : String -&gt; Array&lt;String&gt;
</type>
\end{lstlisting}

IDEs can parse this to recognize that the called function requires one argument named \ic{delimiter} of type \type{String} and returns an \type{Array<String>}.

\trivia{Problems with the output structure}{We acknowledge that the current format requires a bit of manual parsing which can be annoying. In the future we might look into providing a more structured output, especially for functions.}

\subsection{Type path completion}
\label{cr-completion-type-path}

Type path completion can occur in \tref{import declarations}{type-system-import}, \tref{using declarations}{lf-static-extension} or any place a type is referenced. We can identify three different cases:

\paragraph{package completion}

This lists all sub-packages of the haxe package as well as all modules in that package:

\begin{lstlisting}
import haxe.|
\end{lstlisting}

\begin{lstlisting}
<list>
<i n="CallStack"><t></t><d></d></i>
<i n="Constraints"><t></t><d></d></i>
<i n="DynamicAccess"><t></t><d></d></i>
<i n="EnumFlags"><t></t><d></d></i>
<i n="EnumTools"><t></t><d></d></i>
<i n="Http"><t></t><d></d></i>
<i n="Int32"><t></t><d></d></i>
<i n="Int64"><t></t><d></d></i>
<i n="Json"><t></t><d></d></i>
<i n="Log"><t></t><d></d></i>
<i n="PosInfos"><t></t><d></d></i>
<i n="Resource"><t></t><d></d></i>
<i n="Serializer"><t></t><d></d></i>
<i n="Template"><t></t><d></d></i>
<i n="Timer"><t></t><d></d></i>
<i n="Ucs2"><t></t><d></d></i>
<i n="Unserializer"><t></t><d></d></i>
<i n="Utf8"><t></t><d></d></i>
<i n="crypto"><t></t><d></d></i>
<i n="ds"><t></t><d></d></i>
<i n="extern"><t></t><d></d></i>
<i n="format"><t></t><d></d></i>
<i n="io"><t></t><d></d></i>
<i n="macro"><t></t><d></d></i>
<i n="remoting"><t></t><d></d></i>
<i n="rtti"><t></t><d></d></i>
<i n="unit"><t></t><d></d></i>
<i n="web"><t></t><d></d></i>
<i n="xml"><t></t><d></d></i>
<i n="zip"><t></t><d></d></i>
</list>
\end{lstlisting}


\paragraph{import module completion}

This lists all \tref{sub-types}{type-system-module-sub-types} of the module \type{haxe.Unserializer} as well as all its public static fields (because these can be imported too):

\begin{lstlisting}
import haxe.Unserializer.|
\end{lstlisting}

\begin{lstlisting}
<list>
<i n="DEFAULT_RESOLVER">
  <t>haxe.TypeResolver</t>
  <d>
    This value can be set to use custom type resolvers.

    A type resolver finds a Class or Enum instance from a given String.
    By default, the haxe Type Api is used.

    A type resolver must provide two methods:

    1. resolveClass(name:String):Class&lt;Dynamic&gt; is called to
      determine a Class from a class name
    2. resolveEnum(name:String):Enum&lt;Dynamic&gt; is called to
      determine an Enum from an enum name

    This value is applied when a new Unserializer instance is created.
    Changing it afterwards has no effect on previously created
    instances.
  </d>
</i>
<i n="run">
  <t>v : String -&gt; Dynamic</t>
  <d>
    Unserializes `v` and returns the according value.

    This is a convenience function for creating a new instance of
    Unserializer with `v` as buffer and calling its unserialize()
    method once.
  </d>
</i>
<i n="TypeResolver"><t></t><d></d></i>
<i n="Unserializer"><t></t><d></d></i>
</list>
\end{lstlisting}


\begin{lstlisting}
using haxe.Unserializer.|
\end{lstlisting}


\paragraph{other module completion}

This lists all \tref{sub-types}{type-system-module-sub-types} of the module \type{haxe.Unserializer}:

\begin{lstlisting}
using haxe.Unserializer.|
\end{lstlisting}

\begin{lstlisting}
class Main {
  static public function main() {
    var x:haxe.Unserializer.|
  }
}
\end{lstlisting}

\begin{lstlisting}
<list>
<i n="TypeResolver"><t></t><d></d></i>
<i n="Unserializer"><t></t><d></d></i>
</list>
\end{lstlisting}


\subsection{Usage completion}
\label{cr-completion-usage}
\since{3.2.0}

Usage completion is enabled by using the \ic{"usage"} mode argument (see \Fullref{cr-completion-overview}). We demonstrate it here using a local variable. Note that it would work with fields and types the same way:

\begin{lstlisting}
class Main {
  public static function main() {
    var a = 1;
    var b = a + 1;
    trace(a);
    a.|
  }
}
\end{lstlisting}

If this file is saved to Main.hx, the completion can be invoked using the command \ic{haxe --display Main.hx@0@usage}. The output looks like this:

\begin{lstlisting}
<list>
<pos>main.hx:4: characters 9-10</pos>
<pos>main.hx:5: characters 7-8</pos>
<pos>main.hx:6: characters 1-2</pos>
</list>
\end{lstlisting}



\subsection{Position completion}
\label{cr-completion-position}
\since{3.2.0}

Position completion is enabled by using the \ic{"position"} mode argument (see \Fullref{cr-completion-overview}). We demonstrate it using a field. Note that it would work with local variables and types the same way:

\begin{lstlisting}
class Main {
  static public function main() {
    "foo".split.|
}
\end{lstlisting}

If this file is saved to Main.hx, the completion can be invoked using the command \ic{haxe --display Main.hx@0@position}. The output looks like this:

\begin{lstlisting}
<list>
<pos>std/string.hx:124: characters 1-54</pos>
</list>
\end{lstlisting}

\trivia{Effects of omitting a target specifier}{In this example the compiler reports the standard String.hx definition which does not actually have an implementation. This happens because we did not specify any target, which is allowed in completion-mode. If the command line arguments included, say, \ic{-neko neko.n}, the reported position would instead be \ic{std/neko/_std/string.hx:84: lines 84-98}.}


\subsection{Top-level completion}
\label{cr-completion-top-level}
\since{3.2.0}

Top-level completion displays all identifiers which the Haxe Compiler knows about at a given compilation position. This is the only completion method for which we need a real position argument in order to demonstrate its effect:

\begin{lstlisting}
class Main {
  static public function main() {
    var a = 1;
  }
}

enum MyEnum {
  MyConstructor1;
  MyConstructor2(s:String);
}
\end{lstlisting}

If this file is saved to Main.hx, the completion can be invoked using the command \ic{haxe --display Main.hx@63@toplevel}. The output looks similar to this (we omit several entries for brevity):

\begin{lstlisting}
<il>
<i k="local" t="Int">a</i>
<i k="static" t="Void -&gt; Unknown&lt;0&gt;">main</i>
<i k="enum" t="MyEnum">MyConstructor1</i>
<i k="enum" t="s : String -&gt; MyEnum">MyConstructor2</i>
<i k="package">sys</i>
<i k="package">haxe</i>
<i k="type" p="Int">Int</i>
<i k="type" p="Float">Float</i>
<i k="type" p="MyEnum">MyEnum</i>
<i k="type" p="Main">Main</i>
</il>
\end{lstlisting}

The structure of the XML depends on the \ic{k} attribute of each entry. In all cases the node value of the \ic{i} nodes contains the relevant name.

\begin{description}
	\item[\ic{local}, \ic{member}, \ic{static}, \ic{enum}, \ic{global}:] The \ic{t} attribute holds the type of the variable or field.
	\item[\ic{global}, \ic{type}:] The \ic{p} attribute holds the path of the module which contains the type or field.
\end{description}



\subsection{Completion server}
\label{cr-completion-server}

To get the best speed for both compilation and completion, you can use the \ic{--wait} commandline parameter to start a Haxe compilation server. You can also use \ic{-v} to have the server print the log. Here's an example:

\begin{lstlisting}
haxe -v --wait 6000
\end{lstlisting}

You can then connect to the Haxe server, send commandline parameters followed by a 0 byte and, then, read the response (either completion result or errors).

Use the \ic{--connect} commandline parameter to have Haxe send its compilation commands to the server instead of executing them directly :

\begin{lstlisting}
haxe --connect 6000 myproject.hxml
\end{lstlisting}

Please note that you can use the parameter \ic{--cwd} at the first sent command line to change the Haxe server's current working directory. Usually class paths and other files are relative to your project.

\paragraph{How it works}
The compilation server will cache the following things:

\begin{description}
	\item[parsed files] the files will only get parsed again if they are modified or if there was a parse error
	\item[haxelib calls] the previous results of haxelib calls will be reused (only for completion : they are ignored when doing a compilation)
	\item[typed modules] compilation modules will be cached after a successful compilation and can be reused in later compilation/completions if none of their dependencies have been modified
\end{description}

You can get precise reading of the times spent by the compiler and how using the compilation server affects them by adding \ic{--times} to the command line.

\paragraph{Protocol}
As the following Haxe/Neko example shows, you can simply connect on the server port and send all commands (one per line) ending with a 0 binary char. You can, then, read the results.

Macros and other commands can log events which are not errors. From the command line, we can see the difference between what is printed on \ic{stdout} and what is print on \ic{stderr}. This is not the case in socket mode. In order to differentiate between the two, log messages (not errors) are prefixed with a \ic{\\x01} character and all newline-characters in the message are replaced by the same \ic{\\x01} character.

Warnings and other messages can also be considered errors, but are not fatal ones. If a fatal error occurred, it will send a single \ic{\\x02} message-line.

Here's some code that will treat connection to the server and handle the protocol details:

\begin{lstlisting}
class Test {
    static function main() {
		var newline = "\textbackslash\ n";
        var s = new neko.net.Socket();
        s.connect(new neko.net.Host("127.0.0.1"),6000);
        s.write("--cwd /my/project" + newline);
        s.write("myproject.hxml" + newline);
        s.write("\textbackslash\ 000");
		
        var hasError = false;
        for (line in s.read().split(newline))
		{
            switch (line.charCodeAt(0)) {
				case 0x01: 
					neko.Lib.print(line.substr(1).split("\textbackslash\ x01").join(newline));
				case 0x02: 
					hasError = true;
				default: 
					neko.io.File.stderr().writeString(line + newline);
            }
		}
        if (hasError) neko.Sys.exit(1);
    }
}
\end{lstlisting}

\paragraph{Effect on macros}
The compilation server can have some side effects on \tref{macro execution}{macro}.



\section{Resources}
\label{cr-resources}
\flag{fold}{true}

Haxe provides a simple resource embedding system that can be used for embedding files directly into the compiled application.

While it may be not optimal to embed large assets such as images or music in the application file, it comes in very handy for embedding smaller resources like configuration or XML data.

\subsection{Embedding resources}
\label{cr-resources-embed}

External files are embedded using the \emph{-resource} compiler argument:

\todo{what to use for listing of non-haxe code like hxml?}
\begin{lstlisting}
-resource hello_message.txt@welcome
\end{lstlisting}

The string after the \emph{@} symbol is the \emph{resource identifier} which is used in the code for retrieving the resource. If it is omitted (together with the \emph{@} symbol) then the file name will become the resource identifier.

\subsection{Retrieving text resources}
\label{cr-resources-getString}

To retrieve the content of an embedded resource we use the static method \emph{getString} of \type{haxe.Resource}, passing a \emph{resource identifier} to it:

\haxe{assets/ResourceGetString.hx}

The code above will display the content of the \emph{hello_message.txt} file that we included earlier using \emph{welcome} as the identifier.

\subsection{Retrieving binary resources}
\label{cr-resources-getBytes}

While it's not recommended to embed large binary files in the application, it still may be useful to embed binary data. The binary representation of an embedded resource can be accessed using the static method \emph{getBytes} of \type{haxe.Resource}:

\haxe{assets/ResourceGetBytes.hx}

The return type of \emph{getBytes} method is \type{haxe.io.Bytes}, which is an object providing access to individual bytes of the data.

\subsection{Implementation details}
\label{cr-resources-impl}

Haxe uses the target platform's native resource embedding if there is one, otherwise it provides its own implementation.

\begin{itemize}
\item \emph{Flash} resources are embedded as ByteArray definitions
\item \emph{C\#} resources are included in the compiled assembly
\item \emph{Java} resources are packed in the resulting JAR file
\item \emph{C++} resources are stored in global byte array constants.
\item \emph{JavaScript} resources are serialized in Haxe serialization format and stored in a static field of \type{haxe.Resource} class.
\item \emph{Neko} resources are stored as strings in a static field of \type{haxe.Resource} class.
\end{itemize}



\section{Runtime Type Information}
\label{cr-rtti}

The Haxe compiler generates runtime type information (RTTI) for classes that are annotated or extend classes that are annotated with the \expr{:rtti} metadata. This information is stored as a XML string in a static field \expr{__rtti} and can be processed through \type{haxe.rtti.XmlParser}. The resulting structure is described in \Fullref{cr-rtti-structure}.

\since{3.2.0}

The type \type{haxe.rtti.Rtti} has been introduced in order to simplify working with RTTI. Retrieving this information is now very easy:

\haxe{assets/RttiUsage.hx}

\subsection{RTTI structure}
\label{cr-rtti-structure}

\paragraph{General type information}

\begin{description}
	\item[path:] The \tref{type path}{define-type-path} of the type.
	\item[module:] The type path of the \tref{module}{define-module} containing the type.
	\item[file:] The full slash path of the .hx file containing the type. This might be \expr{null} in case there is no such file, e.g. if the type is defined through a \tref{macro}{macro}.
	\item[params:] An array of strings representing the names of the \tref{type parameters}{type-system-type-parameters} the type has. As of Haxe 3.2.0, this does not include the \tref{constraints}{type-system-type-parameter-constraints}.
	\item[doc:] The documentation of the type. This information is only available if the \tref{compiler flag}{define-compiler-flag} \expr{-D use_rtti_doc} was in place. Otherwise, or if the type has no documentation, the value is \expr{null}.
	\item[isPrivate:] Whether or not the type is \tref{private}{define-private-type}.
	\item[platforms:] A list of strings representing the targets where the type is available.
	\item[meta:] The meta data the type was annotated with.
\end{description}
	
\paragraph{Class type information}
\label{cr-rtti-class-type-information}

\begin{description}
	\item[isExtern:] Whether or not the class is \tref{extern}{lf-externs}.
	\item[isInterface:] Whether or not the class is actually an \tref{interface}{types-interfaces}.
	\item[superClass:] The class' parent class defined by its type path and list of type parameters.
	\item[interfaces:] The list of interfaces defined by their type path and list of type parameters.
	\item[fields:] The list of member \tref{class fields}{class-field}, described in \Fullref{cr-rtti-class-field-information}.
	\item[statics:] The list of static class fields, described in \Fullref{cr-rtti-class-field-information}.
	\item[tdynamic:] The type which is \tref{dynamically implemented}{types-dynamic-implemented} by the class or \expr{null} if no such type exists.
\end{description}

\paragraph{Enum type information}

\begin{description}
	\item[isExtern:] Whether or not the enum is \tref{extern}{lf-externs}.
	\item[constructors:] The list of enum constructors.
\end{description}

\paragraph{Abstract type information}

\begin{description}
	\item[to:] An array containing the defined \tref{implicit to casts}{types-abstract-implicit-casts}.
	\item[from:] An array containing the defined \tref{implicit from casts}{types-abstract-implicit-casts}.
	\item[impl:] The \tref{class type information}{cr-rtti-class-type-information} of the implementing class.
	\item[athis:] The \tref{underlying type}{define-underlying-type} of the abstract.
\end{description}
	
	
\paragraph{Class field information}
\label{cr-rtti-class-field-information}

\begin{description}
	\item[name:] The name of the field.
	\item[type:] The type of the field.
	\item[isPublic:] Whether or not the field is \tref{public}{class-field-visibility}.
	\item[isOverride:] Whether or not the field \tref{overrides}{class-field-override} another field.
	\item[doc:] The documentation of the field. This information is only available if the \tref{compiler flag}{define-compiler-flag} \expr{-D use_rtti_doc} was in place. Otherwise, or if the field has no documentation, the value is \expr{null}.
	\item[get:] The \tref{read access behavior}{define-read-access} of the field.
	\item[set:] The \tref{write access behavior}{define-write-access} of the field.
	\item[params:] An array of strings representing the names of the \tref{type parameters}{type-system-type-parameters} the field has. As of Haxe 3.2.0, this does not include the \tref{constraints}{type-system-type-parameter-constraints}.
	\item[platforms:] A list of strings representing the targets where the field is available.
	\item[meta:] The meta data the field was annotated with.
	\item[line:] The line number where the field is defined. This information is only available if the field has an expression. Otherwise the value is \expr{null}.
	\item[overloads:] The list of available overloads for the fields or \expr{null} if no overloads exists.
\end{description}

\paragraph{Enum constructor information}
\label{cr-rtti-enum-constructor-information}

\begin{description}
	\item[name:] The name of the constructor.
	\item[args:] The list of arguments the constructor has or \expr{null} if no arguments are available.
	\item[doc:] The documentation of the constructor. This information is only available if the \tref{compiler flag}{define-compiler-flag} \expr{-D use_rtti_doc} was in place. Otherwise, or if the constructor has no documentation, the value is \expr{null}.
	\item[platforms:] A list of strings representing the targets where the constructor is available.
	\item[meta:] The meta data the constructor was annotated with.
\end{description}

\chapter{Macros}
\label{macro}

Macros are without a doubt the most advanced feature in Haxe. They are often perceived as dark magic that only a select few are capable of mastering, yet there is nothing magical (and certainly nothing dark) about them.

\define{Abstract Syntax Tree (AST)}{define-ast}{The AST is the result of \emph{parsing} Haxe code into a typed structure. This structure is exposed to macros through the types defined in the file \expr{haxe/macro/Expr.hx} of the Haxe Standard Library.}

\begin{flowchart}{macro-compilation-role}{The role of macros during compilation.}

\tikzstyle{macro} = [ fill = orange!40 ]
\tikzstyle{macroEdge} = [ dashed, color = orange!70!black ]
\tikzstyle{edge} = [ midway, auto = left, outer sep = 0.2cm ]

\node (src) [process] {Source code};
\node (lexpar) [process, right = of src] {Lexer / Parser};
\node (ast1) [process, right = of lexpar] {Abstract Syntax Tree (AST)};
\node (ast1t) [above = of ast1, text width = 4cm] {
	\begin{itemize}
		\itemsep-0.2em
		\item Expression
		\item Complex Type
		\item haxe.macro.Expr
	\end{itemize}
};
\node (macro) [process, right = of ast1, macro] {Macro processor};
\node (ast2) [process, below = 4cm of macro, macro] {Abstract Syntax Tree (AST)};
\node (typer) [process, left = of ast2] {Typer};
\node (ast3) [process, left = of typer] {Typed AST};
\node (ast3t) [above = of ast3, xshift = 0.5cm, text width =4cm] {
	\begin{itemize}
		\itemsep-0.2em
		\item Typed Expression
		\item Type
		\item haxe.macro.Type
	\end{itemize}
};
\node (gen) [process, left = of ast3] {Generator};
\node (out) [process, above = of gen] {Output};

\draw [flowchartArrow] (src) -- (lexpar);
\draw [flowchartArrow] (lexpar) -- (ast1) node[edge] {parse};
\draw [dashed] (ast1t.-144) -- (ast1t.144 |- ast1.north);
\draw [flowchartArrow, macroEdge] (ast1) -- (macro);
\draw [flowchartArrow, macroEdge] (macro) -- (ast2) node[edge] {transform};
\draw [flowchartArrow, macroEdge] (ast2) -- (typer);
\draw [flowchartArrow] (typer |- ast1.south) -- (typer);
\draw [flowchartArrow] (typer) -- (ast3) node[edge] {type};
\draw [dashed] (ast3t.-144) -- (ast3t.-144 |- ast3.north);
\draw [flowchartArrow] (ast3) -- (gen);
\draw [flowchartArrow] (gen) -- (out) node[edge] {generate};

\end{flowchart}

A basic macro is a \emph{syntax-transformation}. It receives zero or more \tref{expressions}{expression} and also returns an expression. If a macro is called, it effectively inserts code at the place it was called from. In that respect, it could be compared to a preprocessor like \expr{\#define} in C++, but a Haxe macro is not a textual replacement tool.

We can identify different kinds of macros, which are run at specific compilation stages:

\begin{description}
	\item[Initialization Macros:] These are provided by command line using the \ic{--macro} compiler parameter. They are executed after the compiler arguments were processed and the \emph{typer context} has been created, but before any typing was done (see \Fullref{macro-initialization}).
	\item[Build Macros:] These are defined for classes, enums and abstracts through the \expr{@:build} or \expr{@:autoBuild} \tref{metadata}{lf-metadata}. They are executed per-type, after the type has been set up (including its relation to other types, such as inheritance for classes) but before its fields are typed (see \Fullref{macro-type-building}).
	\item[Expression Macros:] These are normal functions which are executed as soon as they are typed.
\end{description}
	
\section{Macro Context}
\label{macro-context}

\define{Macro Context}{define-macro-context}{The macro context is the environment in which the macro is executed. Depending on the macro type, it can be considered to be a class being built or a function being typed. Contextual information can be obtained through the \ic{haxe.macro.Context} API.}

Haxe macros have access to different contextual information depending on the macro type. Other than querying such information, the context also allows some modifications such as defining a new type or registering certain callbacks. It is important to understand that not all information is available for all macro kinds, as the following examples demonstrate:

\begin{itemize}
	\item Initialization macros will find that the \expr{Context.getLocal*()} methods return \expr{null}. There is no local type or method in the context of an initialization macro.
	\item Only build macros get a proper return value from \expr{Context.getBuildFields()}. There are no fields being built for the other macro kinds.
	\item Build macros have a local type (if incomplete), but no local method, so \expr{Context.getLocalMethod()} returns \expr{null}.
\end{itemize}

The context API is complemented by the \expr{haxe.macro.Compiler} API detailed in \Fullref{macro-initialization}. While this API is available to all macro kinds, care has to be taken for any modification outside of initialization macros. This stems from the natural limitation of undefined \tref{build order}{macro-limitations-build-order}, which could cause e.g. a flag definition through \expr{Compiler.define()} to take effect before or after a \tref{conditional compilation}{lf-condition-compilation} check against that flag.

\section{Arguments}
\label{macro-arguments}

Most of the time, arguments to macros are expressions represented as an instance of enum \type{Expr}. As such, they are parsed but not typed, meaning they can be anything conforming to Haxe's syntax rules. The macro can then inspect their structure, or (try to) get their type using \expr{haxe.macro.Context.typeof()}.

It is important to understand that arguments to macros are not guaranteed to be evaluated, so any intended side-effect is not guaranteed to occur. On the other hand, it is also important to understand that an argument expression may be duplicated by a macro and used multiple times in the returned expression:

\haxe{assets/MacroArguments.hx}

The macro \expr{add} is called with \expr{x++} as argument and thus returns \expr{x++ + x++} using \tref{expression reification}{macro-reification-expression}, causing \expr{x} to be incremented twice.

\subsection{ExprOf}
\label{macro-ExprOf}

Since \type{Expr} is compatible with any possible input, Haxe provides the type \type{haxe.macro.ExprOf<T>}. For the most part, this type is identical to \type{Expr}, but it allows constraining the type of accepted expressions. This is useful when combining macros with \tref{static extensions}{lf-static-extension}:

\haxe{assets/ExprOf.hx}

The two direct calls to \expr{identity} are accepted, even though the argument is declared as \expr{ExprOf<String>}. It might come as a surprise that the \type{Int} \expr{1} is accepted, but it is a logical consequence of what was explained about \tref{macro arguments}{macro-arguments}: The argument expressions are never typed, so it is not possible for the compiler to check their compatibility by \tref{unifying}{type-system-unification}.

This is different for the next two lines which are using static extensions (note the \expr{using Main}): For these it is mandatory to type the left side (\expr{"foo"} and \expr{1}) first in order to make sense of the \expr{identity} field access. This makes it possible to check the types against the argument types, which causes \expr{1.identity()} to not consider \expr{Main.identity()} as a suitable field.

\subsection{Constant Expressions}
\label{macro-constant-arguments}

A macro can be declared to expect \tref{constant}{expression-constants} arguments:

\haxe{assets/MacroArgumentsConst.hx}

With these it is not necessary to detour over expressions as the compiler can use the provided constants directly.

\subsection{Rest Argument}
\label{macro-rest-argument}

If the final argument of a macro is of type \type{Array<Expr>}, the macro accepts an arbitrary number of extra arguments which are available from that array:

\haxe{assets/MacroArgumentsRest.hx}




\section{Reification}
\label{macro-reification}

The Haxe Compiler allows \emph{reification} of expressions, types and classes to simplify working with macros. The syntax for reification is \expr{macro expr}, where \expr{expr} is any valid Haxe expression.

\subsection{Expression Reification}
\label{macro-reification-expression}

Expression reification is used to create instances of \type{haxe.macro.Expr} in a convenient way. The Haxe Compiler accepts the usual Haxe syntax and translates it to an expression object. It supports several escaping mechanisms, all of which are triggered by the \expr{\$} character:

\begin{description}
	\item[\expr{\$\{\}} and \expr{\$e\{\}}:] \type{Expr -> Expr} This can be used to compose expressions. The expression within the delimiting \expr{\{ \}} is executed, with its value being used in place.
	\item[\expr{\$a\{\}}:] \type{Expr -> Array<Expr>} If used in a place where an \type{Array<Expr>} is expected (e.g. call arguments, block elements), \expr{\$a\{\}} treats its value as that array. Otherwise it generates an array declaration.
	\item[\expr{\$b\{\}}:] \type{Array<Expr> -> Expr} Generates a block expression from the given expression array.
	\item[\expr{\$i\{\}}:] \type{String -> Expr} Generates an identifier from the given string.
	\item[\expr{\$p\{\}}:] \type{Array<String> -> Expr} Generates a field expression from the given string array.
	\item[\expr{\$v\{\}}:] \type{Dynamic -> Expr} Generates an expression depending on the type of its argument. This is only guaranteed to work for \tref{basic types}{types-basic-types} and \tref{enum instances}{types-enum-instance}.
\end{description}

Additionally the \tref{metadata}{lf-metadata} \expr{@:pos(p)} can be used to map the position of the annotated expression to \expr{p} instead of the place it is reified at.

This kind of reification only works in places where the internal structure expects an expression. This disallows \expr{object.\$\{fieldName\}}, but \expr{object.\$fieldName} works. This is true for all places where the internal structure expects a string:

\begin{itemize}
	\item field access \expr{object.\$name}
	\item variable name \expr{var \$name = 1;}
\end{itemize}
\since{3.1.0}
\begin{itemize}
	\item field name \expr{\{ \$name: 1\} }
	\item function name \expr{function \$name() \{ \}}
	\item catch variable name \expr{try e() catch(\$name:Dynamic) \{\}}
\end{itemize}


\subsection{Type Reification}
\label{macro-reification-type}

Type reification is used to create instances of \type{haxe.macro.Expr.ComplexType} in a convenient way. It is identified by a \expr{macro : Type}, where \expr{Type} can be any valid type path expression. This is similar to explicit type hints in normal code, e.g. for variables in the form of \expr{var x:Type}.

Each constructor of \type{ComplexType} has a distinct syntax:

\begin{description}
	\item[\expr{TPath}:] \expr{macro : pack.Type}
	\item[\expr{TFunction}:] \expr{macro : Arg1 -> Arg2 -> Return}
	\item[\expr{TAnonymous}:] \expr{macro : \{ field: Type \}}
	\item[\expr{TParent}:] \expr{macro : (Type)}
	\item[\expr{TExtend}:] \expr{macro : \{> Type, field: Type \}}
	\item[\expr{TOptional}:] \expr{macro : ?Type}
\end{description}

\subsection{Class Reification}
\label{macro-reification-class}

It is also possible to use reification to obtain an instance of \type{haxe.macro.Expr.TypeDefinition}. This is indicated by the \expr{macro class} syntax as shown here:

\haxe{assets/ClassReification.hx}

The generated \type{TypeDefinition} instance is typically passed to \expr{haxe.macro.Context.defineType} in order to add a new type to the calling context (not the macro context itself).

This kind of reification can also be useful to obtain instances of \expr{haxe.macro.Expr.Field}, which are available from the \expr{fields} array of the generated \type{TypeDefinition}. 

\section{Tools}
\label{macro-tools}

The Haxe Standard Library comes with a set of tool-classes to simplify working with macros. These classes work best as \tref{static extensions}{lf-static-extension} and can be brought into context either individually or as a whole through \expr{using haxe.macro.Tools}. These classes are:

\begin{description}
	\item[\type{ComplexTypeTools}:] Allows printing \type{ComplexType} instances in a human-readable way. Also allows determining the \type{Type} corresponding to a \type{ComplexType}.
	\item[\type{ExprTools}:] Allows printing \type{Expr} instances in a human-readable way. Also allows iterating and mapping expressions.
	\item[\type{MacroStringTools}:] Offers useful operations on strings and string expressions in macro context.
	\item[\type{TypeTools}:] Allows printing \type{Type} instances in a human-readable way. Also offers several useful operations on types, such as \tref{unifying}{type-system-unification} them or getting their corresponding \type{ComplexType}.
\end{description}

Furthermore the \type{haxe.macro.Printer} class has public methods for printing various types as a human-readable format. This can be helpful when debugging macros.

\trivia{The tinkerbell library and why Tools.hx works}{We learned about static extensions that using a \emph{module} implies that all its types are brought into static extension context. As it turns out, such a type can also be a \tref{typedef}{type-system-typedef} to another type. The compiler then considers this type part of the module, and extends static extension accordingly.\\
This ``trick'' was first used in Juraj Kirchheim's \emph{tinkerbell}\footnote{https://github.com/back2dos/tinkerbell} library for exactly the same purpose. Tinkerbell provided many useful macro tools long before they made it into the Haxe Compiler and Haxe Standard Library. It remains the primary library for additional macro tools and offers other useful functionality as well.} 



\section{Type Building}
\label{macro-type-building}

Type-building macros are different from expression macros in several ways:

\begin{itemize}
	\item They do not return expressions, but an array of class fields. Their return type must be set explicitly to \type{Array<haxe.macro.Expr.Field>}.
	\item Their \tref{context}{macro-context} has no local method and no local variables.
	\item Their context does have build fields, available from \expr{haxe.macro.Context.getBuildFields()}.
	\item They are not called directly, but are argument to a \expr{@:build} or \expr{@:autoBuild} \tref{metadata}{lf-metadata} on a \tref{class}{types-class-instance} or \tref{enum}{types-enum-instance} declaration.
\end{itemize}

The following example demonstrates type building. Note that it is split up into two files for a reason: If a module contains a \expr{macro} function, it has to be typed into macro context as well. This is often a problem for type-building macros because the type to be built could only be loaded in its incomplete state, before the building macro has run. We recommend to always define type-building macros in their own module.

\haxe{assets/TypeBuildingMacro.hx}
\haxe{assets/TypeBuilding.hx}

The \expr{build} method of \type{TypeBuildingMacro} performs three steps:

\begin{enumerate}
	\item It obtains the build fields using \expr{Context.getBuildFields()}.
	\item It declares a new \type{haxe.macro.expr.Field} field using the \expr{funcName} macro argument as field name. This field is a \type{String} \tref{variable}{class-field-variable} with a default value \expr{"my default"} (from the \expr{kind} field) and is public and static (from the \expr{access} field).
	\item It adds the new field to the build field array and returns it.
\end{enumerate}

This macro is argument to the \expr{@:build} metadata on the \type{Main} class. As soon as this type is required, the compiler does the following:

\begin{enumerate}
	\item It parses the module file, including the class fields.
	\item It sets up the type, including its relation to other types through \tref{inheritance}{types-class-inheritance} and \tref{interfaces}{types-interfaces}.
	\item It executes the type-building macro according to the \expr{@:build} metadata.
	\item It continues typing the class normally with the fields returned by the type-building macro.
\end{enumerate}

This allows adding and modifying class fields at will in a type-building macro. In our example, the macro is called with a \expr{"myFunc"} argument, making \expr{Main.myFunc} a valid field access.

If a type-building macro should not modify anything, the macro can return \expr{null}. This indicates to the compiler that no changes are intended and is preferable to returning \expr{Context.getBuildFields()}.



\subsection{Enum building}
\label{macro-enum-building}

Building \tref{enums}{types-enum-instance} is analogous to building classes with a simple mapping:

\begin{itemize}
	\item Enum constructors without arguments are variable fields \expr{FVar}.
	\item Enum constructors with arguments are method fields \expr{FFun}.
\end{itemize}

\todo{Check if we can build GADTs this way.}

\haxe{assets/EnumBuildingMacro.hx}
\haxe{assets/EnumBuilding.hx}

Because enum \type{E} is annotated with a \expr{:build} metadata, the called macro builds two constructors \expr{A} and \expr{B} ``into'' it. The former is added with the kind being \expr{FVar(null, null)}, meaning it is a constructor without argument. For the latter, we use \tref{reification}{macro-reification-expression} to obtain an instance of \type{haxe.macro.Expr.Function} with a singular \type{Int} argument.

The \expr{main} method proves the structure of our generated enum by \tref{matching}{lf-pattern-matching} it. We can see that the generated type is equivalent to this:

\begin{lstlisting}
enum E {
	A;
	B(value:Int);
}
\end{lstlisting}


\subsection{@:autoBuild}
\label{macro-auto-build}

If a class has the \expr{:autoBuild} metadata, the compiler generates \expr{:build} metadata on all extending classes. If an interface has the \expr{:autoBuild} metadata, the compiler generates \expr{:build} metadata on all implementing classes and all extending interfaces. Note that \expr{:autoBuild} does not imply \expr{:build} on the class/interface itself.

\haxe{assets/AutoBuildingMacro.hx}
\haxe{assets/AutoBuilding.hx}

This outputs during compilation:

\begin{lstlisting}
AutoBuildingMacro.hx:6:
  fromInterface: TInst(I2,[])
AutoBuildingMacro.hx:6:
  fromInterface: TInst(Main,[])
AutoBuildingMacro.hx:11:
  fromBaseClass: TInst(Main,[])
\end{lstlisting}

It is important to keep in mind that the order of these macro executions is undefined, which is detailed in \Fullref{macro-limitations-build-order}.



\subsection{@:genericBuild}
\label{macro-generic-build}
\since{3.1.0}

Normal \tref{build-macros}{macro-type-building} are run per-type and are already very powerful. In some cases it is useful to run a build macro per type \emph{usage} instead, i.e. whenever it actually appears in the code. Among other things this allows accessing the concrete type parameters in the macro.

\expr{@:genericBuild} is used just like \expr{@:build} by adding it to a type with the argument being a macro call:

\haxe{assets/GenericBuildMacro1.hx}

\haxe{assets/GenericBuild1.hx}

When running this example the compiler outputs \ic{TAbstract(Int,[])} and \ic{TInst(String,[])}, indicating that it is indeed aware of the concrete type parameters of \type{MyType}. The macro logic could use this information to generate a custom type (using \expr{haxe.macro.Context.defineType}) or refer to an existing one. For brevity we return \expr{null} here which asks the compiler to \tref{infer}{type-system-type-inference} the type.

In Haxe 3.1 the return type of a \expr{@:genericBuild} macro has to be a \type{haxe.macro.Type}. Haxe 3.2 allows (and prefers) returning a \type{haxe.macro.ComplexType} instead, which is the syntactic representation of a type. This is easier to work with in many cases because types can simply be referenced by their paths.

\paragraph{Const type parameter}

Haxe allows passing \tref{constant expression}{expression-constants} as a type parameter if the type parameter name is \expr{Const}. This can be utilized in the context of \expr{@:genericBuild} macros to pass information from the syntax directly to the macro:

\haxe{assets/GenericBuildMacro2.hx}

\haxe{assets/GenericBuild2.hx}

Here the macro logic could load a file and use its contents to generate a custom type.



\section{Limitations}
\label{macro-limitations}
\state{NoContent}

\subsection{Macro-in-Macro}
\label{macro-limitations-macro-in-macro}

\subsection{Static extension}
\label{macro-limitations-static-extension}

The concepts or \tref{static extensions}{lf-static-extension} and macros are somewhat conflicting: While the former requires a known type in order to determine used functions, macros execute before typing on plain syntax. It is thus not surprising that combining these two features can lead to issues. Haxe 3.0 would try to convert the typed expression back to a syntax expression, which is not always possible and may lose important information. We recommend that it is used with caution.

\since{3.1.0}

The combination of static extensions and macros was reworked for the 3.1.0 release. The Haxe Compiler does not even try to find the original expression for the macro argument and instead passes a special \expr{@:this this} expression. While the structure of this expression conveys no information, the expression can still be typed correctly:

\haxe{assets/MacroStaticExtension.hx}



\subsection{Build Order}
\label{macro-limitations-build-order}

The build order of types is unspecified and this extends to the execution order of \tref{build-macros}{macro-type-building}. While certain rules can be determined, we strongly recommend to not rely on the execution order of build-macros. If type building requires multiple passes, this should be reflected directly in the macro code. In order to avoid multiple build-macro execution on the same type, state can be stored in static variables or added as \tref{metadata}{lf-metadata} to the type in question:

\haxe{assets/MacroBuildOrder.hx}

With both interfaces \type{I1} and \type{I2} having \expr{:autoBuild} metadata, the build macro is executed twice for class \type{C}. We guard against duplicate processing by adding a custom \expr{:processed} metadata to the class, which can be checked during the second macro execution.


\subsection{Type Parameters}
\label{macro-limitations-type-parameters}


\section{Initialization macros}
\label{macro-initialization}

Initialization macros are invoked from command line by using the \expr{--macro callExpr(args)} command. This registers a callback which the compiler invokes after creating its context, but before typing what was argument to \expr{-main}. This then allows configuring the compiler in some ways.

If the argument to \expr{--macro} is a call to a plain identifier, that identifier is looked up in the class \type{haxe.macro.Compiler} which is part of the Haxe Standard Library. It comes with several useful initialization macros which are detailed in its \href{http://api.haxe.org//haxe/macro/Compiler.html}{API}.

As an example, the \expr{include} macro allows inclusion of an entire package for compilation, recursively if necessary. The command line argument for this would then be \expr{--macro include('some.pack', true)}.

Of course it is also possible to define custom initialization macros to perform various tasks before the real compilation. A macro like this would then be invoked via \expr{--macro some.Class.theMacro(args)}. For instance, as all macros share the same \tref{context}{macro-context}, an initialization macro could set the value of a static field which other macros use as configuration.


\part{標準ライブラリ}
\chapter{Standard Library}
\label{std}
\state{NoContent}

Standard library

\section{String}
\label{std-String}

\define[Type]{String}{define-string}{A String is a sequence of characters.}

%TODO: utf8 crap %

\section{Data Structures}
\label{std-ds}
\state{NoContent}

\subsection{Array}
\label{std-Array}

An \type{Array} is a collection of elements. It has one \tref{type parameter}{type-system-type-parameters} which corresponds to the type of these elements. Arrays can be created in three ways:

\begin{enumerate}
	\item By using their constructor: \expr{new Array()}
	\item By using \tref{array declaration syntax}{expression-array-declaration}: \expr{[1, 2, 3]}
	\item By using \tref{array comprehension}{lf-array-comprehension}: \expr{[for (i in 0...10) if (i \% 2 == 0) i]}
\end{enumerate}

Arrays come with an \href{http://api.haxe.org/Array.html}{API} to cover most use-cases. Additionally they allow read and write \tref{array access}{expression-array-access}:

\haxe{assets/ArrayAccess.hx}

Since array access in Haxe is unbounded, i.e. it is guaranteed to not throw an exception, this requires further discussion:

\begin{itemize}
	\item If a read access is made on a non-existing index, a target-dependent value is returned.
	\item If a write access is made with a positive index which is out of bounds, \expr{null} (or the \tref{default value}{define-default-value} for \tref{basic types}{types-basic-types} on \tref{static targets}{define-static-target}) is inserted at all positions between the last defined index and the newly written one.
	\item If a write access is made with a negative index, the result is unspecified.
\end{itemize}

Arrays define an \tref{iterator}{lf-iterators} over their elements. This iteration is typically optimized by the compiler to use a \tref{\expr{while} loop}{expression-while} with array index:

\haxe{assets/ArrayIterator.hx}

Haxe generates this optimized \target{Javascript} output:

\begin{lstlisting}
Main.main = function() {
	var scores = [110,170,35];
	var sum = 0;
	var _g = 0;
	while(_g < scores.length) {
		var score = scores[_g];
		++_g;
		sum += score;
	}
	console.log(sum);
};
\end{lstlisting}

Haxe does not allow arrays of mixed types unless the parameter type is forced to \tref{\type{Dynamic}}{types-dynamic}:

\haxe{assets/ArrayDynamic.hx}

\trivia{Dynamic Arrays}{In Haxe 2, mixed type array declarations were allowed. In Haxe 3, arrays can have mixed types only if they are explicitly declared as \expr{Array<Dynamic>}.}


\subsection{Vector}
\label{std-vector}

A \type{Vector} is an optimized fixed-length \emph{collection} of elements. Much like \tref{Array}{std-Array}, it has one \tref{type parameter}{type-system-type-parameters} and all elements of a vector must be of the specified type, it can be \emph{iterated over} using a \tref{for loop}{expression-for} and accessed using \tref{array access syntax}{types-abstract-array-access}. However, unlike \type{Array} and \type{List}, vector length is specified on creation and cannot be changed later.

\haxe{assets/Vector.hx}

\type{haxe.ds.Vector} is implemented as an abstract type (\ref{types-abstract}) over a native array implementation for given target and can be faster for fixed-size collections, because the memory for storing its elements is pre-allocated.

\subsection{List}
\label{std-List}
A \type{List} is a \emph{collection} for storing elements.  On the surface, a list is similar to an \Fullref{std-Array}.  However, the underlying implementation is very different.  This results in several functional differences:

\begin{enumerate}
	\item A list can not be indexed using square brackets, i.e. \expr{[0]}.
	\item A list can not be initialized.
	\item There are no list comprehensions.
	\item A list can freely modify/add/remove elements while iterating over them.
\end{enumerate}

See the \href{http://api.haxe.org/List.html}{List API} for details about the list methods.  A simple example for working with lists:
\haxe{assets/ListExample.hx}

\subsection{GenericStack}
\label{std-GenericStack}
A \type{GenericStack}, like \type{Array} and \type{List} is a container for storing elements.  It has one \tref{type parameter}{type-system-type-parameters} and all elements of the stack must be of the specified type. See the \href{http://api.haxe.org/haxe/ds/GenericStack.html}{GenericStack API} for details about its methods.  Here is a small example program for initializing and working with a \type{GenericStack}.
\haxe{assets/GenericStackExample.hx}
\trivia{FastList}{In Haxe 2, the GenericStack class was known as FastList.  Since its behavior more closely resembled a typical stack, the name was changed for Haxe 3.}
The \emph{Generic} in \type{GenericStack} is literal.  It is attributed with the \expr{:generic} metadata.  Depending on the target, this can lead to improved performance on static targets.  See \Fullref{type-system-generic} for more details.



\subsection{Map}
\label{std-Map}

A \type{Map} is a container composed of \emph{key}, \emph{value} pairs.  A \type{Map} is also commonly referred to as an associative array, dictionary, or symbol table. The following code gives a short example of working with maps:

\haxe{assets/MapExample.hx}

See the \href{http://api.haxe.org/Map.html}{Map API} for details of its methods.

Under the hood, a \type{Map} is an \tref{abstract}{types-abstract} type. At compile time, it gets converted to one of several specialized types depending on the \emph{key} type:
\begin{itemize}
	\item \type{String}: \type{haxe.ds.StringMap}
	\item \type{Int}: \type{haxe.ds.IntMap}
	\item \type{EnumValue}: \type{haxe.ds.EnumValueMap}
	\item \type{\{\}}: \type{haxe.ds.ObjectMap}
\end{itemize}

The \type{Map} type does not exist at runtime and has been replaced with one of the above objects. 

Map defines \tref{array access}{types-abstract-array-access} using its key type.


\subsection{Option}
\label{std-Option}

An option is an \tref{enum}{types-enum-instance} in the Haxe Standard Library which is defined like so:

\begin{lstlisting}
enum Option<T> {
	Some(v:T);
	None;
}
\end{lstlisting}

It can be used in various situations, such as communicating whether or not a method had a valid return and if so, what value it returned:

\haxe{assets/OptionUsage.hx}



\section{Regular Expressions}
\label{std-regex}

Haxe has built-in support for \emph{regular expressions}\footnote{http://en.wikipedia.org/wiki/Regular_expression}. They can be used to verify the format of a string, transform a string or extract some regular data from a given text.

Haxe has special syntax for creating regular expressions. We can create a regular expression object by typing it between the \expr{\textasciitilde/} combination and a single \expr{/} character:

\begin{lstlisting}
var r = ~/haxe/i;
\end{lstlisting}

Alternatively, we can create regular expression with regular syntax:

\begin{lstlisting}
var r = new EReg("haxe", "i");
\end{lstlisting}

First argument is a string with regular expression pattern, second one is a string with \emph{flags} (see below).

We can use standard regular expression patterns such as:
\begin{itemize}
	\item \expr{.} any character
	\item \expr{*} repeat zero-or-more
	\item \expr{+} repeat one-or-more
	\item \expr{?} optional zero-or-one
	\item \expr{[A-Z0-9]} character ranges
	\item \expr{[\textasciicircum\textbackslash r\textbackslash n\textbackslash t]} character not-in-range
	\item \expr{(...)} parenthesis to match groups of characters
	\item \expr{\textasciicircum} beginning of the string (beginning of a line in multiline matching mode)
	\item \expr{\$} end of the string (end of a line in multiline matching mode)
	\item \expr{|} "OR" statement.
\end{itemize}

For example, the following regular expression matches valid email addresses:
\begin{lstlisting}
~/[A-Z0-9._\%-]+@[A-Z0-9.-]+\.[A-Z][A-Z][A-Z]?/i;
\end{lstlisting}

Please notice that the \expr{i} at the end of the regular expression is a \emph{flag} that enables case-insensitive matching.

The possible flags are the following:
\begin{itemize}
	\item \expr{i} case insensitive matching
	\item \expr{g} global replace or split, see below
	\item \expr{m} multiline matching, \expr{\textasciicircum} and \expr{\$} represent the beginning and end of a line
	\item \expr{s} the dot \expr{.} will also match newlines \emph{(Neko, C++, PHP and Java targets only)}
	\item \expr{u} use UTF-8 matching \emph{(Neko and C++ targets only)}
\end{itemize}

\subsection{Matching}
\label{std-regex-match}

Probably one of the most common uses for regular expressions is checking whether a string matches the specific pattern. The \expr{match} method of a regular expression object can be used to do that:
\haxe{assets/ERegMatch.hx}

\subsection{Groups}
\label{std-regex-groups}

Specific information can be extracted from a matched string by using \emph{groups}. If \expr{match()} returns true, we can get groups using the \expr{matched(X)} method, where X is the number of a group defined by regular expression pattern:

\haxe{assets/ERegGroups.hx}

Note that group numbers start with 1 and \expr{r.matched(0)} will always return the whole matched substring.

The \expr{r.matchedPos()} will return the position of this substring in the original string:

\haxe{assets/ERegMatchPos.hx}

Additionally, \expr{r.matchedLeft()} and \expr{r.matchedRight()} can be used to get substrings to the left and to the right of the matched substring:

\haxe{assets/ERegMatchLeftRight.hx}

\subsection{Replace}
\label{std-regex-replace}

A regular expression can also be used to replace a part of the string:

\haxe{assets/ERegReplace.hx}

We can use \expr{\$X} to reuse a matched group in the replacement:

\haxe{assets/ERegReplaceGroups.hx}

\subsection{Split}
\label{std-regex-split}

A regular expression can also be used to split a string into several substrings:

\haxe{assets/ERegSplit.hx}

\subsection{Map}
\label{std-regex-map}

The \expr{map} method of a regular expression object can be used to replace matched substrings using a custom function. This function takes a regular expression object as its first argument so we may use it to get additional information about the match being done and do conditional replacement. For example:

\haxe{assets/ERegMap.hx}


\subsection{Implementation Details}
\label{std-regex-implementation-details}

Regular Expressions are implemented:

\begin{itemize}
	\item in JavaScript, the runtime is providing the implementation with the object RegExp.
	\item in Neko and C++, the PCRE library is used
	\item in Flash, PHP, C\# and Java, native implementations are used
	\item in Flash 6/8, the implementation is not available
\end{itemize}


\section{Math}
\label{std-math}

Haxe includes a floating point math library for some common mathematical operations. Most of the functions operate on and return \type{floats}. However, an \type{Int} can be used where a \type{Float} is expected, and Haxe also converts \type{Int} to \type{Float} during most numeric operations  (see \Fullref{types-numeric-operators} for more details).

Here are some example uses of the math library.  See the \href{http://api.haxe.org/Math.html}{Math API} for all available functions.

\haxe{assets/MathExample.hx}

\subsection{Special Numbers}
\label{std-math-special-numbers}

The math library has definitions for several special numbers:

\begin{itemize}
	\item NaN (Not a Number): returned when a mathmatically incorrect operation is executed, e.g. Math.sqrt(-1)
	\item POSITIVE_INFINITY: e.g. divide a positive number by zero
	\item NEGATIVE_INFINITY: e.g. divide a negative number by zero
	\item PI : 3.1415...
\end{itemize}

\subsection{Mathematical Errors}
\label{std-math-mathematical-errors}
Although neko can fluidly handle mathematical errors, like division by zero, this is not true for all targets.  Depending on the target, mathematical errors may produce exceptions and ultimately errors.

\subsection{Integer Math}
\label{std-math-integer-math}

If you are targeting a platform that can utilize integer operations, e.g. integer division, it should be wrapped in \emph{Std.int()} for improved performance.  The Haxe Compiler can then optimize for integer operations.  An example:

\begin{lstlisting}
	var intDivision = Std.int(6.2/4.7);
\end{lstlisting}

\todo{I think C++ can use integer operatins, but I don't know about any other targets. Only saw this mentioned in an old discussion thread, still true?}

\subsection{Extensions}
\label{std-math-extensions}
It is common to see \Fullref{lf-static-extension} used with the math library.  This code shows a simple example:  
\haxe{assets/MathStaticExtension.hx}
\haxe{assets/MathExtensionUsage.hx}


\section{Lambda}
\label{std-Lambda}

\define{Lambda}{define-lambda}{Lambda is a functional language concept within Haxe that allows you to apply a function to a list or \tref{iterators}{lf-iterators}. The Lambda class is a collection of functional methods in order to use functional-style programming with Haxe.}

It is ideally used with \expr{using Lambda} (see \tref{Static Extension}{lf-static-extension}) and then acts as an extension to \type{Iterable} types. 

On static platforms, working with the \type{Iterable} structure might be slower than performing the operations directly on known types, such as \type{Array} and \type{List}.

\paragraph{Lambda Functions}
The Lambda class allows us to operate on an entire \type{Iterable} at once.
This is often preferable to looping routines since it is less error prone and easier to read. 
For convenience, the \type{Array} and \type{List} class contains some of the frequently used methods from the Lambda class.

It is helpful to look at an example. The exists function is specified as:

\begin{lstlisting}
static function exists<A>( it : Iterable<A>, f : A -> Bool ) : Bool
\end{lstlisting}

Most Lambda functions are called in similar ways. The first argument for all of the Lambda functions is the \type{Iterable} on which to operate. Many also take a function as an argument.

\begin{description}
	\item[\expr{Lambda.array}, \expr{Lambda.list}] Convert Iterable to \type{Array} or \type{List}. It always returns a new instance.
	\item[\expr{Lambda.count}] Count the number of elements.  If the Iterable is a \type{Array} or \type{List} it is faster to use its length property.
	\item[\expr{Lambda.empty}] Determine if the Iterable is empty. For all Iterables it is best to use the this function; it's also faster than compare the length (or result of Lambda.count) to zero.
	\item[\expr{Lambda.has}] Determine if the specified element is in the Iterable.
	\item[\expr{Lambda.exists}] Determine if criteria is satisfied by an element.
	\item[\expr{Lambda.indexOf}] Find out the index of the specified element.
	\item[\expr{Lambda.find}] Find first element of given search function.
	\item[\expr{Lambda.foreach}] Determine if every element satisfies a criteria.
	\item[\expr{Lambda.iter}] Call a function for each element.
	\item[\expr{Lambda.concat}] Merge two Iterables, returning a new List.
	\item[\expr{Lambda.filter}] Find the elements that satisfy a criteria, returning a new List.
	\item[\expr{Lambda.map}, \expr{Lambda.mapi}] Apply a conversion to each element, returning a new List.
	\item[\expr{Lambda.fold}] Functional fold, which is also known as reduce, accumulate, compress or inject.
\end{description}

This example demonstrates the Lambda filter and map on a set of strings:

\begin{lstlisting}
using Lambda;
class Main {
    static function main() {
        var words = ['car', 'boat', 'cat', 'frog'];

		var isThreeLetters = function(word) return word.length == 3;
		var capitalize = function(word) return word.toUpperCase();
		
		// Three letter words and capitalized. 
		trace(words.filter(isThreeLetters).map(capitalize)); // [CAR,CAT]
    }
}
\end{lstlisting} 

This example demonstrates the Lambda count, has, foreach and fold function on a set of ints.

\begin{lstlisting}
using Lambda;
class Main {
    static function main() {
        var numbers = [1, 3, 5, 6, 7, 8];
		
		trace(numbers.count()); // 6
		trace(numbers.has(4)); // false
		
        // test if all numbers are greater/smaller than 20
		trace(numbers.foreach(function(v) return v < 20)); // true
        trace(numbers.foreach(function(v) return v > 20)); // false
		
        // sum all the numbers
		var sum = function(num, total) return total += num;
		trace(numbers.fold(sum, 0)); // 30
    }
}
\end{lstlisting} 

\section{Template}
\label{std-template}

Haxe comes with a standard template system with an easy to use syntax which is interpreted by a lightweight class called \type{haxe.Template}.

A template is a string or a file that is used to produce any kind of string output depending on the input. Here is a small template example:

\haxe{assets/Template.hx}

The console will trace \ic{My name is Mark, 30 years old}.

\paragraph{Expressions}
An expression can be put between the \ic{::}, the syntax allows the current possibilities:

\begin{description}
	\item[\ic{::name::}] the variable name
	\item[\ic{::expr.field::}] field access
	\item[\ic{::(expr)::}] the expression expr is evaluated
	\item[\ic{::(e1 op e2)::}] the operation op is applied to e1 and e2
	\item[\ic{::(135)::}] the integer 135. Float constants are not allowed
\end{description}

\paragraph{Conditions}
It is possible to test conditions using \ic{::if flag1::}. Optionally, the condition may be followed by \ic{::elseif flag2::} or \ic{::else::}. Close the condition with \ic{::end::}.

\begin{lstlisting} 
::if isValid:: valid ::else:: invalid ::end::
\end{lstlisting} 

Operators can be used but they don't deal with operator precedence. Therefore it is required to enclose each operation in parentheses \ic{()}. Currently, the following operators are allowed: \ic{+}, \ic{-}, \ic{*}, \ic{/}, \ic{>}, \ic{<},  \ic{>=}, \ic{<=}, \ic{==}, \ic{!=}, \ic{\&\&} and \ic{||}.

For example \ic{::((1 + 3) == (2 + 2))::} will display true. 

\begin{lstlisting} 
::if (points == 10):: Great! ::end::
\end{lstlisting} 

To compare to a string, use double quotes \ic{"} in the template.
\begin{lstlisting} 
::if (name == "Mark"):: Hi Mark ::end::
\end{lstlisting} 

\paragraph{Iterating}
Iterate on a structure by using \ic{::foreach::}. End the loop with \ic{::end::}.
\begin{lstlisting} 
<table>
	<tr>
		<th>Name</th>
		<th>Age</th>
	</tr>
	::foreach users::
		<tr>
			<td>::name::</td>
			<td>::age::</td>
		</tr>
	::end::
</table>
\end{lstlisting} 

\paragraph{Sub-templates}
To include templates in other templates, pass the sub-template result string as a parameter.
\begin{lstlisting} 
var users = [{name:"Mark", age:30}, {name:"John", age:45}];

var userTemplate = new haxe.Template("::foreach users:: ::name::(::age::) ::end::");
var userOutput = userTemplate.execute({users: users});

var template = new haxe.Template("The users are ::users::");
var output = template.execute({users: userOutput});
trace(output);
\end{lstlisting} 
The console will trace \ic{The users are Mark(30) John(45)}.

\paragraph{Template macros}
To call custom functions while parts of the template are being rendered, provide a \expr{macros} object to the argument of \expr{Template.execute}. The key will act as the template variable name, the value refers to a callback function that should return a \type{String}. The first argument of this macro function is always a \expr{resolve()} method, followed by the given arguments. The resolve function can be called to retrieve values from the template context. If \expr{macros} has no such field, the result is unspecified.

The following example passes itself as macro function context and executes \ic{display} from the template.
\haxe{assets/TemplateMacros.hx}
The console will trace \ic{The results: Mark ran 3.5 kilometers in 15 minutes}.

\paragraph{Globals}
Use the \expr{Template.globals} object to store values that should be applied across all \type{haxe.Template} instances. This has lower priority than the context argument of \expr{Template.execute}.

\paragraph{Using resources}

To separate the content from the code, consider using the \tref{resource embedding system}{cr-resources}. 
Place the template-content in a new file called \ic{sample.mtt}, add \ic{-resource sample.mtt@my_sample} to the compiler arguments and retrieve the content using \expr{haxe.Resource.getString}.
\haxe{assets/TemplateResource.hx}

When running the template system on the server side, you can simply use \expr{neko.Lib.print} or \expr{php.Lib.print} instead of trace to display the HTML template to the user.


\section{Reflection}
\label{std-reflection}

Haxe supports runtime reflection of types and fields. Special care has to be taken here because runtime representation generally varies between targets. In order to use reflection correctly it is necessary to understand what kind of operations are supported and what is not. Given the dynamic nature of reflection, this can not always be determined at compile-time.

The reflection API consists of two classes:

\begin{description}
	\item[Reflect:] A lightweight API which work best on \tref{anonymous structures}{types-anonymous-structure}, with limited support for \tref{classes}{types-class-instance}. 
	\item[Type:] A more robust API for working with classes and \tref{enums}{types-enum-instance}.
\end{description}

The available methods are detailed in the API for \href{http://api.haxe.org//Reflect.html}{Reflect} and \href{http://api.haxe.org//Type.html}{Type}.

Reflection can be a powerful tool, but it is important to understand why it can also cause problems. As an example, several functions expect a \tref{String}{std-String} argument and try to resolve it to a type or field. This is vulnerable to typing errors:

\haxe{assets/ReflectionTypo.hx}

However, even if there are no typing errors it is easy to come across unexpected behavior:

\haxe{assets/ReflectionMissingType.hx}

The problem here is that the compiler never actually ``sees'' the type \type{haxe.Template}, so it does not compile it into the output. Furthermore, even if it were to see the type there could be issues arising from \tref{dead code elimitation}{cr-dce} eliminating types or fields which are only used via reflection.

Another set of problems comes from the fact that, by design, several reflection functions expect arguments of type \tref{Dynamic}{types-dynamic}, meaning the compiler cannot check if the passed in arguments are correct. The following example demonstrates a common mistake when working with \expr{callMethod}:

\haxe{assets/ReflectionWrongUsage.hx}

The commented out call would be accepted by the compiler because it assigns the string \expr{"f"} to the function argument \expr{func} which is specified to be \expr{Dynamic}.

A good advice when working with reflection is to wrap it in a few functions within an application or API which are called by otherwise type-safe code. An example could look like this:

\haxe{assets/ReflectionWrap.hx}

While the method \expr{reflective} could interally work with reflection (and \type{Dynamic} for that matter) a lot, its return value is a typed structure which the callers can use in a type-safe manner.


\section{Serialization}
\label{std-serialization}

Many runtime values can be serialized and deserialized using the \type{haxe.Serializer} and \type{haxe.Unserializer} classes. Both support two usages:

\begin{enumerate}
	\item Create an instance and continuously call the \expr{serialize}/\expr{unserialize} method to handle multiple values.
	\item Call their static \expr{run} method to serialize/deserialize a single value.
\end{enumerate}

The following example demonstrates the first usage:

\haxe{assets/SerializationExample.hx}

The result of the serialization (here stored in local variable \expr{s}) is a \tref{String}{std-String} and can be passed around at will, even remotely. Its format is described in \Fullref{std-serialization-format}.

\paragraph{Supported values}

\begin{itemize}
	\item \expr{null}
	\item \type{Bool}, \type{Int} and \type{Float} (including infinities and \expr{NaN})
	\item \type{String}
	\item \type{Date}
	\item \type{haxe.io.Bytes} (encoded as base64)
	\item \tref{\type{Array}}{std-Array} and \tref{\type{List}}{std-List}
	\item \type{haxe.ds.StringMap}, \type{haxe.ds.IntMap} and \type{haxe.ds.ObjectMap}
	\item \tref{anonymous structures}{types-anonymous-structure}
	\item Haxe \tref{class instances}{types-class-instance} (not native ones)
	\item \tref{enum instances}{types-enum-instance}
\end{itemize}

\paragraph{Serialization configuration}

Serialization can be configured in two ways. For both a static variable can be set to influence all \type{haxe.Serializer} instances, and a member variable can be set to only influence a specific instance:

\begin{description}
	\item[\expr{USE_CACHE}, \expr{useCache}:] If true, repeated structures or class\slash enum instances are serialized by reference. This can avoid infinite loops for recursive data at the expense of longer serialization time. By default, object caching is disabled; strings however are always cached.
	\item[\expr{USE_ENUM_INDEX}, \expr{useEnumIndex}:] If true, enum constructors are serialized by their index instead of their name. This can make the resulting string shorter, but breaks if enum constructors are inserted into the type before deserialization. This behavior is disabled by default.
\end{description}

\paragraph{Deserialization behavior}

If the serialization result is stored and later used for deserialization, care has to be taken to maintain compatibility when working with class and enum instances. It is then important to understand exactly how unserialization is implemented.

\begin{itemize}
	\item The type has to be available in the runtime where the deserialization is made. If \tref{dead code elimination}{cr-dce} is active, a type which is used only through serialization might be removed.
	\item Each \type{Unserializer} has a member variable \expr{resolver} which is used to resolve classes and enums by name. Upon creation of the \type{Unserializer} this is set to \expr{Unserializer.DEFAULT_RESOLVER}. Both that and the instance member can be set to a custom resolver.
	\item Classes are resolved by name using \expr{resolver.resolveClass(name)}. The instance is then created using \expr{Type.createEmptyInstance}, which means that the class constructor is not called. Finally, the instance fields are set according to the serialized value.
	\item Enums are resolved by name using \expr{resolver.resolveEnum(name)}. The enum instance is then created using \expr{Type.createEnum}, using the serialized argument values if available. If the constructor arguments were changed since serialization, the result is unspecified.
\end{itemize}

\paragraph{Custom (de)serialization}

If a class defines the member method \expr{hxSerialize}, that method is called by the serializer and allows custom serialization of the class. Likewise, if a class defines the member method \expr{hxUnserialize} it is called by the deserializer:

\haxe{assets/SerializationCustom.hx}

In this example we decide that we want to ignore the value of member variable \expr{y} and do not serialize it. Instead we default it to \expr{-1} in \expr{hxUnserialize}. Both methods are annotated with the \expr{:keep} metadata to prevent \tref{dead code elimination}{cr-dce} from removing them as they are never properly referenced in the code.


\subsection{Serialization format}
\label{std-serialization-format}

Each supported value is translated to a distinct prefix character, followed by the necessary data.

\begin{description}
	\item[\expr{null}:] \expr{n}
	\item[\type{Int}:] \expr{z} for zero, or \expr{i} followed by the integer display (e.g. \expr{i456})
	\item[\type{Float}:] \mbox{}
		\begin{description}
			\item[\expr{NaN}:] \expr{k}
			\item[negative infinity:] \expr{m}
			\item[positive infinity:] \expr{p}
			\item[finite floats:] \expr{d} followed by the float display (e.g. \expr{d1.45e-8})
		\end{description}
	\item[\type{Bool}:] \expr{t} for \expr{true}, \expr{f} for \expr{false}
	\item[\type{String}:] \expr{y} followed by the url encoded string length, then \expr{:} and the url encoded string (e.g. \expr{y10:hi\%20there for "hi there".}
	\item[name-value pairs:] a serialized string representing the name followed by the serialized value
	\item[structure:] \expr{o} followed by the list of name-value pairs and terminated by \expr{g} (e.g. \expr{oy1:xi2y1:kng} for \expr{\{x:2, k:null\}})
	\item[\type{List}:] \expr{l} followed by the list of serialized items, followed by \expr{h} (e.g. \expr{lnnh} for a list of two \expr{null} values)
	\item[\type{Array}:] \expr{a} followed by the list of serialized items, followed by \expr{h}. For multiple consecutive \expr{null} values, \expr{u} followed by the number of \expr{null} values is used (e.g. \expr{ai1i2u4i7ni9h for [1,2,null,null,null,null,7,null,9]})
	\item[\type{Date}:] \expr{v} followed by the date itself (e.g. \expr{v2010-01-01 12:45:10})
	\item[\type{haxe.ds.StringMap}:] \expr{b} followed by the name-value pairs, followed by \expr{h} (e.g. \expr{by1:xi2y1:knh} for \expr{\{"x" => 2, "k" => null\}})
	\item[\type{haxe.ds.IntMap}:] \expr{q} followed by the key-value pairs, followed by \expr{h}. Each key is represented as \expr{:<int>} (e.g. \expr{q:4n:5i45:6i7h} for \expr{\{4 => null, 5 => 45, 6 => 7\}})
	\item[\type{haxe.ds.ObjectMap}:] \expr{M} followed by serialized value pairs representing the key and value, followed by \expr{h}
	\item[\type{haxe.io.Bytes}:] \expr{s} followed by the length of the base64 encoded bytes, then \expr{:} and the byte representation using the codes \expr{A-Za-z0-9\%} (e.g. \expr{s3:AAA} for 2 bytes equal to \expr{0}, and \expr{s10:SGVsbG8gIQ} for \expr{haxe.io.Bytes.ofString("Hello !")})
	\item[exception:] \expr{x} followed by the exception value
	\item[class instance:] \expr{c} followed by the serialized class name, followed by the name-value pairs of the fields, followed by \expr{g} (e.g. \expr{cy5:Pointy1:xzy1:yzg} for \expr{new Point(0, 0)} (having two integer fields \expr{x} and \expr{y})
        \item[enum instance (by name):] \expr{w} followed by the serialized enum name, followed by the serialized constructor name, followed by \expr{:}, followed by the number of arguments, followed by the argument values (e.g. \expr{wy3:Fooy1:A:0} for \expr{Foo.A} (with no arguments), \expr{wy3:Fooy1:B:2i4n} for \expr{Foo.B(4,null)})
	\item[enum instance (by index):] \expr{j} followed by the serialized enum name, followed by \expr{:}, followed by the constructor index (starting from 0), followed by \expr{:}, followed by the number of arguments, followed by the argument values (e.g. \expr{wy3:Foo:0:0} for \expr{Foo.A} (with no arguments), \expr{wy3:Foo:1:2i4n} for \expr{Foo.B(4,null)})
	\item[cache references:] \mbox{}
		\begin{description}
			\item[\type{String}:] \expr{R} followed by the corresponding index in the string cache (e.g. \expr{R456})
			\item[class, enum or structure] \expr{r} followed by the corresponding index in the object cache (e.g. \expr{r42})
		\end{description}
	\item[custom:] \expr{C} followed by the class name, followed by the custom serialized data, followed by \expr{g}
\end{description}

\noindent Cached elements and enum constructors are indexed from zero.

\section{Json}
\label{std-Json}

Haxe provides built-in support for (de-)serializing \emph{JSON}\footnote{http://en.wikipedia.org/wiki/JSON} data via the \type{haxe.Json} class.

\subsection{Parsing JSON}
\label{std-Json-parsing}

Use the \expr{haxe.Json.parse} static method to parse \emph{JSON} data and obtain a Haxe value from it:
\haxe{assets/JsonParse.hx}

Note that the type of the object returned by \expr{haxe.Json.parse} is \expr{Dynamic}, so if the structure of our data is well-known, we may want to specify a type using \tref{anonymous structures}{types-anonymous-structure}. This way we provide compile-time checks for accessing our data and most likely more optimal code generation, because compiler knows about types in a structure:
\haxe{assets/JsonParseTyped.hx}

\subsection{Encoding JSON}
\label{std-Json-encoding}

Use the \expr{haxe.Json.stringify} static method to encode a Haxe value into a \emph{JSON} string:
\haxe{assets/JsonStringify.hx}

\subsection{Implementation details}
\label{std-Json-implementation-details}

The \type{haxe.Json} API automatically uses native implementation on targets where it is available, i.e. \emph{JavaScript}, \emph{Flash} and \emph{PHP} and provides its own implementation for other targets.

Usage of Haxe own implementation can be forced with \expr{-D haxeJSON} compiler argument. This will also provide serialization of \tref{enums}{types-enum-instance} by their index, \tref{maps}{std-Map} with string keys and class instances.

Older browsers (Internet Explorer 7, for instance) may not have native \emph{JSON} implementation. In case it's required to support them, we can include one of the JSON implementations available on the internet in the HTML page. Alternatively, a \expr{-D old_browser} compiler argument that will make \type{haxe.Json} try to use native JSON and, in case it's not available, fallback to its own implementation.

\section{Xml}
\label{std-Xml}

\section{Input/Output}
\label{std-input-output}

\section{Sys/sys}
\label{std-sys}

\section{Remoting}
\label{std-remoting}

Haxe remoting is a way to communicate between different platforms. With Haxe remoting, applications can transmit data transparently, send data and call methods between server and client side.

\subsection{Remoting Connection}
\label{std-remoting-connection}

In order to use remoting, there must be a connection established. There are two kinds of Haxe Remoting connections: 
\begin{description}
	\item[\expr{haxe.remoting.Connection}] is used for \emph{synchronous connections}, where the results can be directly obtained when calling a method. 
	\item[\expr{haxe.remoting.AsyncConnection}] is used for \emph{asynchronous connections}, where the results are events that will happen later in the execution process.
\end{description}

\paragraph{Start a connection}
There are some target-specific constructors with different purposes that can be used to set up a connection:

\begin{description}
	\item[All targets:]
		\begin{description}
			\item[\expr{HttpAsyncConnection.urlConnect(url:String)}]  
				Returns an asynchronous connection to the given URL which should link to a Haxe server application. 
		\end{description}
		
	\item[Flash:]
		\begin{description}
			\item[\expr{ExternalConnection.jsConnect(name:String, ctx:Context)}]  
				Allows a connection to the local JavaScript Haxe code. The JS Haxe code must be compiled with the class ExternalConnection included. This only works with Flash Player 8 and higher.
			\item[\expr{AMFConnection.urlConnect(url:String)} and \expr{AMFConnection.connect( cnx : NetConnection )}]  
				Allows a connection to an \href{http://en.wikipedia.org/wiki/Action_Message_Format}{AMF Remoting server} such as \href{http://www.adobe.com/products/adobe-media-server-family.html}{Flash Media Server} or \href{http://www.silexlabs.org/amfphp/}{AMFPHP}.
			\item[\expr{SocketConnection.create(sock:flash.XMLSocket)}]  
				Allows remoting communications over an \type{XMLSocket}
			\item[\expr{LocalConnection.connect(name:String)}]  
				Allows remoting communications over a \href{http://api.haxe.org/haxe/remoting/LocalConnection.html}{Flash LocalConnection}
		\end{description}
		
	\item[Javascript:]
		\begin{description}
			\item[\expr{ExternalConnection.flashConnect(name:String, obj:String, ctx:Context)}]  
				Allows a connection to a given Flash Object. The Haxe Flash content must be loaded and it must include the \expr{haxe.remoting.Connection} class. This only works with Flash 8 and higher. 
		\end{description}
		
	\item[Neko:]
		\begin{description}
			\item[\expr{HttpConnection.urlConnect(url:String)}]  
				Will work like the asynchronous version but in synchronous mode.
			\item[\expr{SocketConnection.create(...)}]  
				Allows real-time communications with a Flash client which is using an \type{XMLSocket} to connect to the server.
		\end{description}
\end{description}

\paragraph{Remoting context}

Before communicating between platforms, a remoting context has to be defined. This is a shared API that can be called on the connection at the client code.

This server code example creates and shares an API:
\begin{lstlisting}
class Server {
	function new() { }
	function foo(x, y) { return x + y; }

	static function main() {
		var ctx = new haxe.remoting.Context();
		ctx.addObject("Server", new Server());
		
		if(haxe.remoting.HttpConnection.handleRequest(ctx))
		{
			return;
		}
		
		// handle normal request
		trace("This is a remoting server !");
	} 
}
\end{lstlisting}

\paragraph{Using the connection}

Using a connection is pretty convenient. Once the connection is obtained, use classic dot-access to evaluate a path and then use \expr{call()} to call the method in the remoting context and get the result.
The asynchronous connection takes an additional function parameter that will be called when the result is available.

This client code example connects to the server remoting context and calls a function \expr{foo()} on its API.
\begin{lstlisting}
class Client {
  static function main() {
    var cnx = haxe.remoting.HttpAsyncConnection.urlConnect("http://localhost/");
    cnx.setErrorHandler( function(err) trace('Error: $err'); } );
    cnx.Server.foo.call([1,2], function(data) trace('Result: $data'););
  }
}
\end{lstlisting}

To make this work for the Neko target, setup a Neko Web Server, point the url in the Client to \ic{"http://localhost2000/remoting.n"} and compile the Server using \ic{-main Server -neko remoting.n}.

\paragraph{Error handling}

\begin{itemize}
	\item When an error occurs in a asynchronous call, the error handler is called as seen in the example above.
	\item When an error occurs in a synchronous call, an exception is raised on the caller-side as if we were calling a local method.
\end{itemize}

\paragraph{Data serialization}

Haxe Remoting can send a lot of different kinds of data. See \tref{Serialization}{std-serialization}.

\subsection{Implementation details}
\label{std-remoting-implementation-details}

\paragraph{Javascript security specifics}

The html-page wrapping the js client must be served from the same domain as the one where the server is running. The same-origin policy restricts how a document or script loaded from one origin can interact with a resource from another origin. The same-origin policy is used as a means to prevent some of the cross-site request forgery attacks.

To use the remoting across domain boundaries, CORS (cross-origin resource sharing) needs to be enabled by defining the header \ic{X-Haxe-Remoting} in the \ic{.htaccess}:

\begin{lstlisting} 
# Enable CORS
Header set Access-Control-Allow-Origin "*"
Header set Access-Control-Allow-Methods: "GET,POST,OPTIONS,DELETE,PUT"
Header set Access-Control-Allow-Headers: X-Haxe-Remoting
\end{lstlisting} 

See \href{http://en.wikipedia.org/wiki/Same-origin_policy}{same-origin policy} for more information on this topic.

Also note that this means that the page can't be served directly from the file system \ic{"file:///C:/example/path/index.html"}.

\paragraph{Flash security specifics}

When Flash accesses a server from a different domain, set up a \ic{crossdomain.xml} file on the server, enabling the \ic{X-Haxe} headers.

\begin{lstlisting} 
<cross-domain-policy>
	<allow-access-from domain="*"/> <!-- or the appropriate domains -->
	<allow-http-request-headers-from domain="*" headers="X-Haxe*"/>
</cross-domain-policy>
\end{lstlisting} 

\paragraph{Arguments types are not ensured}

There is no guarantee of any kind that the arguments types will be respected when a method is called using remoting. 
That means even if the arguments of function \expr{foo} are typed to \type{Int}, the client will still be able to use strings while calling the method. 
This can lead to security issues in some cases. When in doubt, check the argument type when the function is called by using the \expr{Std.is} method.

\section{Unit testing}
\label{std-unit-testing}

The Haxe Standard Library provides basic unit testing classes from the \type{haxe.unit} package. 

\paragraph{Creating new test cases}

First, create a new class extending \type{haxe.unit.TestCase} and add own test methods. Every test method name must start with "\ic{test}".

\haxe{assets/UnitTestCase.hx}

\paragraph{Running unit tests}
To run the test, an instance of \type{haxe.unit.TestRunner} has to be created. Add the \expr{TestCase} using the \expr{add} method and call \expr{run} to start the test.

\haxe{assets/UnitTestRunner.hx}

The result of the test looks like this:
\begin{lstlisting} 
Class: MyTestCase
.
OK 1 tests, 0 failed, 1 success
\end{lstlisting} 

\paragraph{Test functions}
The \type{haxe.unit.TestCase} class comes with three test functions.

\begin{description}
	\item[\expr{assertEquals(a, b)}] Succeeds if \ic{a} and \ic{b} are equal, where \ic{a} is value tested and \ic{b} is the expected value.
	\item[\expr{assertTrue(a)}] Succeeds if \ic{a} is \expr{true}
	\item[\expr{assertFalse(a)}] Succeeds if \ic{a} is \expr{false}
\end{description}

\paragraph{Setup and tear down}

To run code before or after the test, override the functions \expr{setup} and \expr{tearDown} in the \expr{TestCase}. 

\begin{description}
	\item[\expr{setup}] is called before each test runs.
	\item[\expr{tearDown}] is called once after all tests are run.
\end{description}

\haxe{assets/UnitTestSetup.hx}

\paragraph{Comparing Complex Objects}

With complex objects it can be difficult to generate expected values to compare to the actual ones. It can also be a problem that \expr{assertEquals} doesn't do a deep comparison. One way around these issues is to use a string as the expected value and compare it to the actual value converted to a string using \expr{Std.string}. Below is a trivial example using an array.

\begin{lstlisting} 
public function testArray() {
  var actual = [1,2,3];
  assertEquals("[1, 2, 3]", Std.string(actual));
}
\end{lstlisting} 


\part{Miscellaneous}
\chapter{Haxelib}
\label{haxelib}

Haxelib is the library manager that comes with any Haxe distribution. Connected to a central repository, it allows submitting and retrieving libraries and has multiple features beyond that. Available libraries can be found at \url{http://lib.haxe.org}.

A basic Haxe library is a collection of \ic{.hx} files. That is, libraries are distributed by source code by default, making it easy to inspect and modify their behavior. Each library is identified by a unique name, which is utilized when telling the Haxe Compiler which libraries to use for a given compilation.

\section{Using a Haxe library with the Haxe Compiler}
\label{haxelib-using-haxe}

Any installed Haxe library can be made available to the compiler through the \ic{-lib <library-name>} argument. This is very similiar to the \ic{-cp <path>} argument, but expects a library name instead of a directory path. These commands are explained thoroughly in \Fullref{compiler-usage}.

For our exemplary usage we chose a very simple Haxe library called ``random''. It provides a set of static convenience methods to achieve various random effects, such as picking a random element from an array.

\haxe{assets/HaxelibRandom.hx}

Compiling this without any \ic{-lib} argument causes an error message along the lines of \ic{Unknown identifier : Random}. This shows that installed Haxe libraries are not available to the compiler by default unless they are explicitly added. A working command line for above program is \ic{haxe -lib random -main Main --interp}.

If the compiler emits an error \ic{Error: Library random is not installed : run 'haxelib install random'} the library has to be installed via the \ic{haxelib} command first. As the error message suggests, this is achieved through \ic{haxelib install random}. We will learn more about the \ic{haxelib} command in \Fullref{haxelib-using}.



\section{haxelib.json}
\label{haxelib-json}

Each Haxe library requires a \ic{haxelib.json} file in which the following attributes are defined:

\begin{description}
	\item[name:] The name of the library. It must contain at least 3 characters among the following: \ic{\[A-Za-z0-9_-.\]}. In particular, no spaces are allowed.
	\item[url:] The URL of the library, i.e. where more information can be found.
	\item[license:] The license under which the library is released. Can be \ic{GPL}, \ic{LGPL}, \ic{BSD}, \ic{Public} (for Public Domain) or \ic{MIT}.
	\item[tags:] An array of tag-strings which are used on the repository website to sort libraries.
	\item[description:] The description of what the library is doing.
	\item[version:] The version string of the library. This is detailed in \Fullref{haxelib-json-versioning}.
	\item[classPath:] The path string to the source files.
	\item[releasenote:] The release notes of the current version.
	\item[contributors:] An array of user names which identify contributors to the library. 
	\item[dependencies:] An object describing the dependencies of the library. This is detailed in \Fullref{haxelib-json-dependencies}.
\end{description}

The following JSON is a simple example of a haxelib.json:

\begin{lstlisting}
{
  "name": "useless_lib",
  "url" : "https://github.com/jasononeil/useless/",
  "license": "MIT",
  "tags": ["cross", "useless"],
  "description": "This library is useless in the same way on every platform.",
  "version": "1.0.0",
  "releasenote": "Initial release, everything is working correctly.",
  "contributors": ["Juraj","Jason","Nicolas"],
  "dependencies": {
    "tink_macro": "",
    "nme": "3.5.5"
  }
}
\end{lstlisting}

\subsection{Versioning}
\label{haxelib-json-versioning}

Haxelib uses a simplified version of \href{http://semver.org/}{SemVer}. The basic format is this:

\begin{lstlisting}
major.minor.patch
\end{lstlisting}

These are the basic rules:

\begin{itemize}
	\item Major versions are incremented when you break backwards compatibility - so old code will not work with the new version of the library.
	\item Minor versions are incremented when new features are added.
	\item Patch versions are for small fixes that do not change the public API, so no existing code should break.
	\item When a minor version increments, the patch number is reset to 0. When a major version increments, both the minor and patch are reset to 0.
\end{itemize}

Examples:

\begin{description}
\item["0.0.1":] A first release.  Anything with a "0" for the major version is subject to change in the next release - no promises about API stability!
\item["0.1.0":] Added a new feature!   Increment the minor version, reset the patch version
\item["0.1.1":] Realised the new feature was broken.  Fixed it now, so increment the patch version
\item["1.0.0":] New major version, so increment the major version, reset the minor and patch versions.   You promise your users not to break this API until you bump to 2.0.0
\item["1.0.1":] A minor fix
\item["1.1.0":] A new feature
\item["1.2.0":] Another new feature
\item["2.0.0":] A new version, which might break compatibility with 1.0.  Users are to upgrade cautiously.
\end{description}

If this release is a preview (Alpha, Beta or Release Candidate), you can also include that, with an optional release number:

\begin{lstlisting}
major.minor.patch-(alpha/beta/rc).release
\end{lstlisting}

Examples:

\begin{description}
\item["1.0.0-alpha":] The alpha of 1.0.0 - use with care, things are changing!
\item["1.0.0-alpha.2":] The 2nd alpha
\item["1.0.0-beta":] Beta - things are settling down, but still subject to change.
\item["1.0.0-rc.1":] The 1st release candidate for 1.0.0 - you shouldn't be adding any more features now
\item["1.0.0-rc.2":] The 2nd release candidate for 1.0.0
\item["1.0.0":] The final release!  
\end{description}


\subsection{Dependencies}
\label{haxelib-json-dependencies}

As of Haxe 3.1.0, haxelib supports only exact version matching for dependencies. Dependencies are defined as part of the \tref{haxelib.json}{haxelib-json}, with the library name serving as key and the expected version (if required) as value in the format described in \Fullref{haxelib-json-versioning}.

We have seen an example of this when introducing haxelib.json:

\begin{lstlisting}
"dependencies": {
  "tink_macros": "",
  "nme": "3.5.5"
}
\end{lstlisting}

This adds two dependencies to the given Haxe library:

\begin{enumerate}
	\item The library ``tink_macros'' can be used in any version. Haxelib will then always try to use the latest version.
	\item The library ``nme'' is required in version ``3.5.5''. Haxelib will make sure that this exact version is used, avoiding potential breaking changes with future versions.
\end{enumerate}


\section{extraParams.hxml}
\label{haxelib-extraParams}

If you add a file named \ic{extraParams.hxml} to your library root (at the same level as \ic{haxelib.json}), these parameters will be automatically added to the compilation parameters when someone use your library with \ic{-lib}.


\section{Using Haxelib}
\label{haxelib-using}

If the \ic{haxelib} command is executed without any arguments, it prints an exhaustive list of all available arguments. Access the \url{http://lib.haxe.org} website to view all the libraries available. 

The following commands are available:

\begin{description}
	\item[Basic]
		\begin{description}
			\item[\ic{haxelib install [project-name] [version]}] installs the given project. You can optionally specify a specific version to be installed. By default, latest released version will be installed.
			\item[\ic{haxelib update [project-name]}] updates a single library to their latest version. 
			\item[\ic{haxelib upgrade}] upgrades all the installed projects to their latest version. This command prompts a confirmation for each upgradeable project.
			\item[\ic{haxelib remove project-name [version]}] removes complete project or only a specified version if specified.
			\item[\ic{haxelib list}] lists all the installed projects and their versions. For each project, the version surrounded by brackets is the current one.
			\item[\ic{haxelib set [project-name] [version]}] changes the current version for a given project. The version must be already installed.
		\end{description}
		
	\item[Information]
		\begin{description}
			\item[\ic{haxelib search [word]}] lists the projects which have either a name or description matching specified word.
			\item[\ic{haxelib info [project-name]}] gives you information about a given project.
			\item[\ic{haxelib user [user-name]}] lists information on a given Haxelib user.
			\item[\ic{haxelib config}] prints the Haxelib repository path. This is where Haxelib get installed by default.
			\item[\ic{haxelib path [project-name]}] prints paths to libraries and its dependencies (defined in \ic{haxelib.xml}).
		\end{description}
		
	\item[Development]
		\begin{description}
			\item[\ic{haxelib submit [project.zip]}] submits a package to Haxelib. If the user name is unknown, you'll be first asked to register an account. If the user already exists, you will be prompted for your password. If the project does not exist yet, it will be created, but no version will be added. You will have to submit it a second time to add the first released version. If you want to modify the project url or description, simply modify your \ic{haxelib.xml} (keeping version information unchanged) and submit it again.
			\item[\ic{haxelib register [project-name]}] submits or update a library package.
			\item[\ic{haxelib local [project-name]}] tests the library package. Make sure everything (both installation and usage) is working correctly before submitting, since once submitted, a given version cannot be updated.
			\item[\ic{haxelib dev [project-name] [directory]}] sets a development directory for the given project. To set project directory back to global location, run command and omit directory.
			\item[\ic{haxelib git [project-name] [git-clone-path] [branch] [subdirectory]}] uses git repository as library. This is useful for using a more up-to-date development version, a fork of the original project, or for having a private library that you do not wish to post to Haxelib. When you use \ic{haxelib upgrade} any libraries that are installed using GIT will automatically pull the latest version.
		\end{description}
		
	\item[Miscellaneous]
		\begin{description}
			\item[\ic{haxelib setup}] sets the Haxelib repository path. To print current path use \ic{haxelib config}.
			\item[\ic{haxelib selfupdate}] updates Haxelib itself. It will ask to run \ic{haxe update.hxml} after this update.
			\item[\ic{haxelib convertxml}] converts \ic{haxelib.xml} file to \ic{haxelib.json}.
			\item[\ic{haxelib run [project-name] [parameters]}] runs the specified library with parameters. Requires  a precompiled Haxe/Neko \ic{run.n} file in the library package. This is useful if you want users to be able to do some post-install script that will configure some additional things on the system. Be careful to trust the project you are running since the script can damage your system.
			\item[\ic{haxelib proxy}] setup the Http proxy.
		\end{description}
\end{description}

\chapter{Target Details}
\label{target-details}
\state{NoContent}

\section{JavaScript}
\label{target-javascript}
\state{NoContent}

\subsection{Getting started with Haxe/JavaScript}
\label{target-javascript-getting-started}

Haxe can be a powerful tool for developing JavaScript applications. Let's look at our first sample.
This is a very simple example showing the toolchain. 

Create a new folder and save this class as \ic{Main.hx}.

\begin{lstlisting}
import js.Lib;
import js.Browser;
class Main {
    static function main() {
        var button = Browser.document.createButtonElement();
        button.textContent = "Click me!";
        button.onclick = function(event) {
            Lib.alert("Haxe is great");
        }
        Browser.document.body.appendChild(button);
    }
}
\end{lstlisting}

To compile, either run the following from the command line:

\begin{lstlisting}
haxe -js main-javascript.js -main Main -D js-flatten -dce full
\end{lstlisting}

Another possibility is to create and run (double-click) a file called \ic{compile.hxml}. In this example the hxml-file should be in the same directory as the example class.

\begin{lstlisting}
-js main-javascript.js
-main Main
-D js-flatten
-dce full
\end{lstlisting}

The output will be a main-javascript.js, which creates and adds a clickable button to the document body.

\paragraph{Run the JavaScript}

To display the output in a browser, create an HTML-document called \ic{index.html} and open it.

\begin{lstlisting}
<!DOCTYPE html>
<html>
	<body>
		<script src="main-javascript.js"></script>
	</body>
</html>
\end{lstlisting}

\paragraph{More information}

\begin{itemize}
	\item \href{http://api.haxe.org/js/}{Haxe JavaScript API docs}
	\item \href{https://developer.mozilla.org/en-US/docs/Web/JavaScript/Reference}{MDN JavaScript Reference}
\end{itemize}

\subsection{Using external JavaScript libraries}
\label{target-javascript-external-libraries}

The \tref{externs mechanism}{lf-externs} provides access to the native APIs in a type-safe manner. It assumes that the defined types exist at run-time but assumes nothing about how and where those types are defined. 

An example of an extern class is the \href{https://github.com/HaxeFoundation/haxe/blob/development/std/js/jquery/JQuery.hx}{jQuery class} of the Haxe Standard Library. 
To illustrate, here is a simplified version of this extern class:

\begin{lstlisting}
package js.jquery;
@:native("$") extern class JQuery {
	/**
		Creates DOM elements on the fly from the provided string of raw HTML.
		OR
		Accepts a string containing a CSS selector which is then used to match a set of elements.
		OR
		Binds a function to be executed when the DOM has finished loading.
	**/
	@:selfCall
	@:overload(function(element:js.html.Element):Void { })
	@:overload(function(selection:js.jquery.JQuery):Void { })
	@:overload(function(callback:haxe.Constraints.Function):Void { })
	@:overload(function(selector:String, ?context:haxe.extern.EitherType<js.html.Element, js.jquery.JQuery>):Void { })
	public function new():Void;

	/**
		Adds the specified class(es) to each element in the set of matched elements.
	**/
	@:overload(function(_function:Int -> String -> String):js.jquery.JQuery { })
	public function addClass(className:String):js.jquery.JQuery;

	/**
		Get the HTML contents of the first element in the set of matched elements.
		OR
		Set the HTML contents of each element in the set of matched elements.
	**/
	@:overload(function(htmlString:String):js.jquery.JQuery { })
	@:overload(function(_function:Int -> String -> String):js.jquery.JQuery { })
	public function html():String;
}
\end{lstlisting}

Note that functions can be overloaded to accept different types of arguments and return values, using the \expr{@:overload} metadata. Function overloading works only in externs.

Using this extern, we can use jQuery like this:

\begin{lstlisting}
import js.jquery.*;
..
new JQuery("#my-div").addClass("brand-success").html("haxe is great!");
..
\end{lstlisting}

The package and class name of the extern class should be the same as defined in the external library. If that is not the case, rewrite the path of a class using \expr{@:native}.

\begin{lstlisting}
package my.application.media;

@:native('external.library.media.video')
extern class Video {
..
\end{lstlisting}

Some JavaScript libraries favor instantiating classes without using the \expr{new} keyword. To prevent the Haxe compiler outputting the \expr{new} keyword when using a class, we can attach a \expr{@:selfCall} metadata to its constructor. For example, when we instantiate the jQuery extern class above, \expr{new JQuery()} will be outputted as \expr{\$()} instead of \expr{new \$()}. The \expr{@:selfCall} metadata can also be attached to a method. In this case, the method will be interpreted as a direct call to the object, illustrated as follows:

\begin{lstlisting}
extern class Functor {
	public function new():Void;
	@:selfCall function call():Void;
}

class Test {
	static function main() {
		var f = new Functor();
		f.call(); // will be outputted as `f();`
	}
}
\end{lstlisting}

Beside externs, \tref{Typedefs}{type-system-typedef} can be another great way to name (or alias) a JavaScript type. The major difference between typedefs and externs is that, typedefs are duck-typed but externs are not. Typedefs are suitable for common data structures, e.g. point (\expr{\{x:Float, y:Float\}}). Use of a point structure typedef for function arguments allows external JavaScript functions to accept point class instances from Haxe or from another JavaScript library. It is also useful for typing JSON objects.

The Haxe Standard Library comes with externs of \href{https://jquery.com/}{jQuery} and \href{http://blog.deconcept.com/swfobject/}{SWFObject}. Their version compatibility is summarized as follows:

\begin{center}
\begin{tabular}{| l | l | l |}
	\hline
	Haxe version & Library               & Externs location \\ \hline
	3.3          & jQuery 1.11.3 / 2.1.4 & <code>js.jquery.*</code> \\
	3.2-         & jQuery 1.6.4          & <code>js.JQuery</code> \\
	3.3          & SWFObject 2.3         & <code>js.swfobject.*</code> \\
	3.2-         & SWFObject 1.5         & <code>js.SWFObject</code> \\ \hline
\end{tabular}
\end{center}

There are many externs for other popular native libraries available on \tref{Haxelib library}{haxelib}. To view a list of them, check out the \href{http://lib.haxe.org/t/extern/}{extern tag}.

\subsection{Inject raw JavaScript}
\label{target-javascript-injection}

In Haxe, it is possible to call an exposed function thanks to the \expr{untyped} keyword. This can be useful in some cases if we don't want to write externs. Anything untyped that is valid syntax will be generated as it is.

\begin{lstlisting}
untyped window.trackEvent("page1");  
\end{lstlisting}


\subsection{JavaScript untyped functions}
\label{target-javascript-untyped}

These functions allow to access specific JavaScript platform features. It works only when the Haxe compiler is targeting JavaScript and should always be prefixed with \expr{untyped}. 

\emph{Important note:} Before using these functions, make sure there is no alternative available in the Haxe Standard Library. The resulting syntax can not be validated by the Haxe compiler, which may result in invalid or error-prone code in the output.

\paragraph{\expr{untyped __js__(expr, params)}}
\tref{Injects raw JavaScript expressions}{target-javascript-injection}. It's allowed to use \ic{\{0\}}, \ic{\{1\}}, \ic{\{2\}} etc in the expression and use the rest arguments to feed Haxe fields. The Haxe compiler will take care of the surrounding quotes if needed. The function can also return values.

\begin{lstlisting}
untyped __js__('alert("Haxe is great!")');
// output: alert("Haxe is great!");

var myMessage = "Haxe is great!";
untyped __js__('alert({0})', myMessage);
// output: 
//	var myMessage = "Haxe is great!";
//	alert(myMessage);

var myVar:Bool = untyped __js__('confirm({0})', "Are you sure?");
// output: var myVar = confirm("Are you sure?");

var hexString:String = untyped __js__('({0}).toString({1})', 255, 16);
// output: var hexString = (255).toString(16);
\end{lstlisting}

\paragraph{\expr{untyped __instanceof__(o,cl)}} 
Same as \ic{o instanceof cl} in JavaScript.

\begin{lstlisting}
var myString = new String("Haxe is great");
var isString = untyped __instanceof__(myString, String);
output: var isString = (myString instanceof String);
\end{lstlisting}

\paragraph{\expr{untyped __typeof__(o)}} 
Same as \ic{typeof o} in JavaScript.

\begin{lstlisting}
var isNodeJS = untyped __typeof__(window) == null;
output: var isNodeJS = typeof(window) == null;
\end{lstlisting}

\paragraph{\expr{untyped __strict_eq__(a,b)}} 
Same as \ic{a === b}  in JavaScript, tests on \href{https://developer.mozilla.org/en-US/docs/Web/JavaScript/Equality_comparisons_and_sameness}{strict equality} (or "triple equals" or "identity").

\begin{lstlisting}
var a = "0";
var b = 0;
var isEqual = untyped __strict_eq__(a, b);
output: var isEqual = ((a) === b);
\end{lstlisting}

\paragraph{\expr{untyped __strict_neq__(a,b)}} 
Same as \ic{a !== b}  in JavaScript, tests on negative strict equality.

\begin{lstlisting}
var a = "0";
var b = 0;
var isntEqual = untyped __strict_neq__(a, b);
output: var isntEqual = ((a) !== b);
\end{lstlisting}

\paragraph{Expression injection} 

In some cases it may be needed to inject raw JavaScript code into Haxe-generated code. With the \expr{__js__} function we can inject pure JavaScript code fragments into the output. This code is always untyped and can not be validated, so it accepts invalid code in the output, which is error-prone.
This could, for example, write a JavaScript comment in the output.

\begin{lstlisting}
untyped __js__('// haxe is great!');
\end{lstlisting}

A more useful demonstration would be to call a function and pass  arguments using the \expr{__js__} function. This example illustrates how to call this function and how to pass parameters. Note that the \emph{code interpolation} will wrap the quotes around strings in the generated output.

\begin{lstlisting}
// Haxe code:
var myVar = untyped __js__('myObject.myJavaScriptFunction({0}, {1})', "Mark", 31);
\end{lstlisting}

This will generate the following JavaScript code:
\begin{lstlisting}
// JavaScript Code
var myVar = myObject.myJavaScriptFunction("Mark", 31);
\end{lstlisting}

\subsection{Debugging JavaScript}
\label{target-javascript-debugging}

Haxe is able to generate \href{http://www.html5rocks.com/en/tutorials/developertools/sourcemaps/}{source maps}, allowing Javascript debuggers to map from generated JavaScript back to the original Haxe source. This makes reading error stack traces, debugging with breakpoints, and profiling much easier.

Compiling with the \ic{-debug} flag will create a .map alongside the .js file. Enable it in Chrome by clicking on the cog settings button in the bottom right of the Developer Tools window, and checking "Enable source maps". The pause button on the bottom left can be toggled to pause on uncaught exceptions.

\subsection{JavaScript target Metadata}
\label{target-javascript-metadata}

This is the list of JavaScript specific metadata. For more information, see also the complete list of all \tref{Haxe built-in metadata}{cr-metadata}.

\begin{center}
\begin{tabular}{| l | l |}
	\hline
	\multicolumn{2}{|c|}{JavaScript metadata} \\ \hline
	Metadata &  Description \\ \hline
	@:expose \_(?Name=Class path)\_  &  Makes the class available on the \expr{window} object or \expr{exports} for node.js  \\
	@:jsRequire  &  Generate javascript module require expression for given extern \\
	@:selfCall  &  Translates method calls into calling object directly \\
\end{tabular}
\end{center}

\subsection{Exposing Haxe classes for JavaScript}
\label{target-javascript-expose}

It is possible to make Haxe classes or static fields available for usage in plain JavaScript. 
To expose, add the \expr{@:expose} metadata to the desired class or static fields.

This example exposes the Haxe class \ic{MyClass}.

\haxe{assets/ClassExpose.hx}

It generates the following JavaScript output:

\begin{lstlisting}
(function ($hx_exports) { "use strict";
var MyClass = $hx_exports.MyClass = function(name) {
	this.name = name;
};
MyClass.prototype = {
	foo: function() {
		return "Greetings from " + this.name + "!";
	}
};
})(typeof window != "undefined" ? window : exports);
\end{lstlisting}

By passing globals (like \ic{window} or \ic{exports}) as parameters to our anonymous function in the JavaScript module, it becomes available which allows to expose the Haxe generated module.

In plain JavaScript it is now possible to create an instance of the class and call its public functions.

\begin{lstlisting}
// JavaScript code
var instance = new MyClass('Mark');
console.log(instance.foo()); // logs a message in the console
\end{lstlisting}

The package path of the Haxe class will be completely exposed. To rename the class or define a different package for the exposed class, use \expr{@:expose("my.package.MyExternalClass")}

\paragraph{Shallow expose}

When the code generated by Haxe is part of a larger JavaScript project and wrapped in a large closure it is not always necessary to expose the Haxe types to global variables.
Compiling the project using \ic{-D shallow-expose} allows the types or static fields to be available for the surrounding scope of the generated closure only.

When the code is compiled using \ic{-D shallow-expose}, the generated output will look like this:

\begin{lstlisting}
var $hx_exports = $hx_exports || {};
(function () { "use strict";
var MyClass = $hx_exports.MyClass = function(name) {
	this.name = name;
};
MyClass.prototype = {
	foo: function() {
		return "Greetings from " + this.name + "!";
	}
};
})();
var MyClass = $hx_exports.MyClass;
\end{lstlisting}

In this pattern, a var statement is used to expose the module; it doesn't write to the \ic{window} or \ic{exports} object. 

\subsection{Loading extern classes using "require" function}
\label{target-javascript-require}
\since{3.2.0}

Modern \target{JavaScript} platforms, such as Node.js provide a way of loading objects
from external modules using the "require" function. Haxe supports automatic generation
of "require" statements for \expr{extern} classes.

This feature can be enabled by specifying \expr{@:jsRequire} metadata for the extern class. If our \expr{extern} class represents a whole module, we pass a single argument to the \expr{@:jsRequire} metadata specifying the name of the module to load:

\haxe{assets/JSRequireModule.hx}

In case our \expr{extern} class represents an object within a module, second \expr{@:jsRequire} argument specifies an object to load from a module:

\haxe{assets/JSRequireObject.hx}

The second argument is a dotted-path, so we can load sub-objects in any hierarchy.

If we need to load custom JavaScript objects in runtime, a \expr{js.Lib.require} function can be used. It takes \expr{String} as its only argument and returns \expr{Dynamic}, so it is advised to be assigned to a strictly typed variable.

\section{Flash}
\label{target-flash}
\state{NoContent}

\subsection{Getting started with Haxe/Flash}
\label{target-flash-getting-started}

Developing Flash applications is really easy with Haxe. Let's look at our first code sample.
This is a basic example showing most of the toolchain. 

Create a new folder and save this class as \ic{Main.hx}.

\begin{lstlisting}
import flash.Lib;
import flash.display.Shape;
class Main {
    static function main() {
        var stage = Lib.current.stage;
        
        // create a center aligned rounded gray square
        var shape = new Shape();
        shape.graphics.beginFill(0x333333);
		shape.graphics.drawRoundRect(0, 0, 100, 100, 10);
		shape.x = (stage.stageWidth - 100) / 2;
		shape.y = (stage.stageHeight - 100) / 2;
		
		stage.addChild(shape);
    }    
}
\end{lstlisting}

To compile this, either run the following from the command line:

\begin{lstlisting}
haxe -swf main-flash.swf -main Main -swf-version 15 -swf-header 960:640:60:f68712
\end{lstlisting}

Another possibility is to create and run (double-click) a file called \ic{compile.hxml}. In this example the hxml-file should be in the same directory as the example class.

\begin{lstlisting}
-swf main-flash.swf
-main Main
-swf-version 15
-swf-header 960:640:60:f68712
\end{lstlisting}

The output will be a main-flash.swf with size 960x640 pixels at 60 FPS with an orange background color and a gray square in the center.

\paragraph{Display the Flash}

Run the SWF standalone using the \href{https://www.adobe.com/support/flashplayer/downloads.html}{Standalone Debugger FlashPlayer}. 

To display the output in a browser using the Flash-plugin, create an HTML-document called \ic{index.html} and open it.

\begin{lstlisting}
<!DOCTYPE html>
<html>
	<body>
		<embed src="main-flash.swf" width="960" height="640">
	</body>
</html>
\end{lstlisting}

\paragraph{More information}

\begin{itemize}
	\item \href{http://api.haxe.org/flash/}{Haxe Flash API docs}
	\item \href{http://help.adobe.com/en_US/FlashPlatform/reference/actionscript/3/}{Adobe Livedocs}
\end{itemize}

\subsection{Embedding resources}
\label{target-flash-resources}

Embedding resources is different in Haxe compared to ActionScript 3. Instead of using \ic{\[embed\]} (AS3-metadata) use \tref{Flash specific compiler metadata}{target-flash-metadata} like \ic{@:bitmap}, \ic{@:font}, \ic{@:sound} or \ic{@:file}.

\begin{lstlisting}
import flash.Lib;
import flash.display.BitmapData;
import flash.display.Bitmap;

class Main {
  public static function main() {
    var img = new Bitmap( new MyBitmapData(0, 0) );
    Lib.current.addChild(img);
  }
}

@:bitmap("relative/path/to/myfile.png") 
class MyBitmapData extends BitmapData { }
\end{lstlisting}

\subsection{Using external Flash libraries}
\label{target-flash-external-libraries}

To embed external \ic{.swf} or \ic{.swc} libraries, use the following \href{http://haxe.org/documentation/introduction/compiler-usage.html}{compilation options}:

\begin{description}
	\item[\expr{-swf-lib <file>}] Embeds the SWF library in the compiled SWF.
	\item[\expr{-swf-lib-extern <file>}] Adds the SWF library for type checking but doesn't include it in the compiled SWF.
\end{description}

The standard compilation options also provide more Haxe sources to be added to the project:

\begin{itemize}
	\item To add another class path use \expr{-cp <directory>}.
	\item To add a \tref{Haxelib library}{haxelib} use \expr{-lib <library-name>}.
	\item To force a whole package to be included in the project, use \expr{--macro include('mypackage')} which will include all the classes declared in the given package and subpackages. 
\end{itemize}

\subsection{Flash target Metadata}
\label{target-flash-metadata}

This is the list of Flash specific metadata. For a complete list see \tref{Haxe built-in metadata}{cr-metadata}.

\begin{center}
\begin{tabular}{| l | l |}
	\hline
	\multicolumn{2}{|c|}{Flash metadata} \\ \hline
	Metadata &  Description  \\ \hline
	@:bind  &  Override Swf class declaration \\
	@:bitmap \_(Bitmap file path)\_  &  \_Embeds given bitmap data into the class (must extend \expr{flash.display.BitmapData}) \\
	@:debug  &  Forces debug information to be generated into the Swf even without \expr{-debug} \\
	@:file(File path)  &  Includes a given binary file into the target Swf and associates it with the class (must extend \expr{flash.utils.ByteArray}) \\
	@:font \_(TTF path Range String)\_  &  Embeds the given TrueType font into the class (must extend \expr{flash.text.Font}) \\
	@:getter \_(Class field name)\_  &  Generates a native getter function on the given field  \\
	@:noDebug &  Does not generate debug information into the Swf even if \expr{-debug} is set \\
	@:ns  &  Internally used by the Swf generator to handle namespaces \\
	@:setter \_(Class field name)\_  &  Generates a native setter function on the given field \\
	@:sound \_(File path)\_  &  Includes a given \_.wav\_ or \_.mp3\_ file into the target Swf and associates it with the class (must extend \expr{flash.media.Sound}) \\
\end{tabular}
\end{center}

\section{Neko}
\label{target-neko}

\section{PHP}
\label{target-php}
\state{NoContent}

\subsection{Getting started with Haxe/PHP}
\label{target-php-getting-started}

To get started with Haxe/PHP, create a new folder and save this class as \ic{Main.hx}.

\haxe{assets/HelloPHP.hx}

To compile, either run the following from the command line:

\begin{lstlisting}
haxe -php bin -main Main
\end{lstlisting}

Another possibility is to create and run (double-click) a file called \ic{compile.hxml}. In this example the hxml-file should be in the same directory as the example class.

\begin{lstlisting}
-php bin
-main Main
\end{lstlisting}

The compiler outputs in the given \emph{bin}-folder, which contains the generated PHP classes that prints the traced message when you run it. The generated PHP-code runs for version 5.1.0 and later.

\paragraph{More information}

\begin{itemize}
	\item \href{http://api.haxe.org/php/}{Haxe PHP API docs}
	\item \href{http://php.net/docs.php}{PHP.net Documentation}
	\item \href{http://phptohaxe.haqteam.com/code.php}{PHP to Haxe tool}
\end{itemize}


\subsection{PHP untyped functions}
\label{target-php-untyped}

These functions allow to access specific PHP platform features. It works only when the Haxe compiler is targeting PHP and should always be prefixed with \expr{untyped}. 

\emph{Important note:} Before using these functions, make sure there is no alternative available in the Haxe Standard Library. The resulting syntax can not be validated by the Haxe compiler, which may result in invalid or error-prone code in the output.

\paragraph{\expr{untyped __php__(expr)}}
Injects raw PHP code expressions. It's possible to pass fields from Haxe source code using \tref{String Interpolation}{lf-string-interpolation}.

\begin{lstlisting}
var value:String = "test";
untyped __php__("echo '<pre>'; print_r($value); echo '</pre>';");
// output: echo '<pre>'; print_r('test'); echo '</pre>';
\end{lstlisting}

\paragraph{\expr{untyped __call__(function, arg, arg, arg...)}}
Calls a PHP function with the desired number of arguments and returns what the PHP function returns.

\begin{lstlisting}
var value = untyped __call__("array", 1,2,3); 
// output returns a NativeArray with values [1,2,3]
\end{lstlisting}

\paragraph{\expr{untyped __var__(global, paramName)}}
Get the values from global vars. Note that the dollar sign in the Haxe code is omitted.

\begin{lstlisting}
var value : String = untyped __var__('_SERVER', 'REQUEST_METHOD')  
// output: $value = $_SERVER['REQUEST_METHOD']
\end{lstlisting}

\paragraph{\expr{untyped __physeq__(val1, val2)}}
Strict equals test between the two values. Returns a \type{Bool}.

\begin{lstlisting}
var isFalse = untyped __physeq__(false, value);
// output: $isFalse = false === $value;
\end{lstlisting}


\section{C++}
\label{target-cpp}
\state{NoContent}

\subsection{Using C++ Defines}
\label{target-cpp-defines}
\begin{itemize}
	\item ANDROID_HOST
	\item ANDROID_NDK_DIR
	\item ANDROID_NDK_ROOT
	\item BINDIR
	\item DEVELOPER_DIR
	\item HXCPP
	\item HXCPP_32
	\item HXCPP_COMPILE_CACHE
	\item HXCPP_COMPILE_THREADS
	\item HXCPP_CONFIG
	\item HXCPP_CYGWIN
	\item HXCPP_DEPENDS_OK
	\item HXCPP_EXIT_ON_ERROR
	\item HXCPP_FORCE_PDB_SERVER
	\item HXCPP_M32
	\item HXCPP_M64
	\item HXCPP_MINGW
	\item HXCPP_MSVC
	\item HXCPP_MSVC_CUSTOM
	\item HXCPP_MSVC_VER
	\item HXCPP_NO_COLOR
	\item HXCPP_NO_COLOUR
	\item HXCPP_VERBOSE
	\item HXCPP_WINXP_COMPAT
	\item IPHONE_VER
	\item LEGACY_MACOSX_SDK
	\item LEGACY_XCODE_LOCATION
	\item MACOSX_VER
	\item MSVC_VER
	\item NDKV
	\item NO_AUTO_MSVC
	\item PLATFORM
	\item QNX_HOST
	\item QNX_TARGET
	\item TOOLCHAIN_VERSION
	\item USE_GCC_FILETYPES
	\item USE_PRECOMPILED_HEADERS
	\item android
	\item apple
	\item blackberry
	\item cygwin
	\item dll_import
	\item dll_import_include
	\item dll_import_link
	\item emcc
	\item emscripten
	\item gph
	\item hardfp
	\item haxe_ver
	\item ios
	\item iphone
	\item iphoneos
	\item iphonesim
	\item linux
	\item linux_host
	\item mac_host
	\item macos
	\item mingw
	\item rpi
	\item simulator
	\item tizen
	\item toolchain
	\item webos
	\item windows
	\item windows_host
	\item winrt
	\item xcompile
\end{itemize}

\subsection{Using C++ Pointers}
\label{target-cpp-pointers}

\section{Java}
\label{target-java}

\section{C\#}
\label{target-cs}

\section{Python}
\label{target-python}
\state{NoContent}


\end{document}
