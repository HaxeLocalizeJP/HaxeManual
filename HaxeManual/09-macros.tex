\chapter{マクロ}
\label{macro}

マクロは疑いようもなくHaxeの最も高度な機能です。マクロは少数の精鋭にとってのみこれをマスターする価値があるので黒魔術と呼ばれることがありますが、実際には魔法のようなものは何もありません(もちろん闇も)。

\define{抽象構文木(AST:Abstract Syntax Tree)}{define-ast}{抽象構文木はHaxeのコードを構文解析して型付けされた構造へと変換した結果です。この構造はHaxe標準ライブラリの\expr{haxe/macro/Expr.hx}ファイルで定義されている型をつかってマクロから利用可能です。}

\begin{flowchart}{macro-compilation-role}{The role of macros during compilation.}

\tikzstyle{macro} = [ fill = orange!40 ]
\tikzstyle{macroEdge} = [ dashed, color = orange!70!black ]
\tikzstyle{edge} = [ midway, auto = left, outer sep = 0.2cm ]

\node (src) [process] {Source code};
\node (lexpar) [process, right = of src] {Lexer / Parser};
\node (ast1) [process, right = of lexpar] {Abstract Syntax Tree (AST)};
\node (ast1t) [above = of ast1, text width = 4cm] {
	\begin{itemize}
		\itemsep-0.2em
		\item Expression
		\item Complex Type
		\item haxe.macro.Expr
	\end{itemize}
};
\node (macro) [process, right = of ast1, macro] {Macro processor};
\node (ast2) [process, below = 4cm of macro, macro] {Abstract Syntax Tree (AST)};
\node (typer) [process, left = of ast2] {Typer};
\node (ast3) [process, left = of typer] {Typed AST};
\node (ast3t) [above = of ast3, xshift = 0.5cm, text width =4cm] {
	\begin{itemize}
		\itemsep-0.2em
		\item Typed Expression
		\item Type
		\item haxe.macro.Type
	\end{itemize}
};
\node (gen) [process, left = of ast3] {Generator};
\node (out) [process, above = of gen] {Output};

\draw [flowchartArrow] (src) -- (lexpar);
\draw [flowchartArrow] (lexpar) -- (ast1) node[edge] {parse};
\draw [dashed] (ast1t.-144) -- (ast1t.144 |- ast1.north);
\draw [flowchartArrow, macroEdge] (ast1) -- (macro);
\draw [flowchartArrow, macroEdge] (macro) -- (ast2) node[edge] {transform};
\draw [flowchartArrow, macroEdge] (ast2) -- (typer);
\draw [flowchartArrow] (typer |- ast1.south) -- (typer);
\draw [flowchartArrow] (typer) -- (ast3) node[edge] {type};
\draw [dashed] (ast3t.-144) -- (ast3t.-144 |- ast3.north);
\draw [flowchartArrow] (ast3) -- (gen);
\draw [flowchartArrow] (gen) -- (out) node[edge] {generate};

\end{flowchart}

基本的なマクロの1つは\emph{構文変形}です。これは0個以上の式を受け取り、1つの式を返します。マクロが呼び出されると、その結果としてマクロを呼び出した位置にコードが挿入されます。この点はプリプロセッサに似ていますが、Haxeのマクロはテキストの置換ツールではありません。

マクロには種類があり、それぞれ異なるコンパイルの段階で動作します。

\begin{description}
	\item[初期化マクロ:] コマンドラインから\ic{--macro}コンパイラパラメータを使うことで使用します。コンパイラ引数が処理されて、\emph{型付けコンテクスト}が作成されたあとに実行されます。ただし、これはどの型付けが実行されるよりも前です(詳しくは\Fullref{macro-initialization})。
	\item[ビルドマクロ:] クラス、列挙型、抽象型を\expr{@:build}または\expr{@:autoBuild}\tref{メタデータ}{lf-metadata}を使って定義します。その型の生成(クラスの継承関係などの他の型との依存関係の解決も含めて)後に型ごとに実行されます。ただし、これはフィールドが型付けされるよりは前です(詳しくは\Fullref{macro-type-building})。
	\item[式マクロ:] 型付けされると同時に実行される普通の関数です。
\end{description}

\section{マクロコンテクスト}
\label{macro-context}

\define{マクロコンテクスト}{define-macro-context}{マクロコンテクストとはマクロが実行される環境です。マクロの種別によって、クラスのビルドや、関数の型付けなどを行います。コンテクストについての情報は\ic{haxe.macro.Context} APIを通して入手できます。}

Haxeマクロはマクロの種類に応じて異なる、コンテクストの情報にアクセスできます。\type{Context}のAPIでは情報の問い合わせをするだけでなく、新しい型を定義したり、特定のコールバックを登録したりといった編集も行えます。以下で説明している通り、すべてのマクロの種類ですべての情報が利用可能なわけではないことに注意してください。

\begin{itemize}
	\item 初期化マクロでは\expr{Context.getLocal*()}メソッドは\expr{null}を返します。初期化マクロのコンテクストでは、ローカルの型やメソッドはありません。
	\item ビルドマクロのみで\expr{Context.getBuildFields()}からしかるべき値が返ってきます。その他のマクロでは、ビルドされるフィールドはありません。
	\item ビルドマクロでは、ローカルの型があります(ただし、不完全)が、ローカルのメソッドはありません。\expr{Context.getLocalMethod()}は、\expr{null}を返します。
\end{itemize}

\type{Context}のAPIは、\expr{haxe.macro.Compiler}のAPIと合わせて完全なものになります。\expr{haxe.macro.Compiler}について、詳しくは\Fullref{macro-initialization}を参照してください。こちらのAPIはすべての種類のマクロで使用可能で、初期化マクロ以外からでもあらゆる編集が可能であることに注意しなくてはいけません。定義されていない\tref{ビルド順序}{macro-limitations-build-order}の自然な制限に逆らいます。つまり、例えば\expr{Compiler.define()}でのフラグの定義は、そのフラグに対する\tref{条件付きコンパイル}{lf-condition-compilation}のチェックの後でも先でも影響を与えます。

\section{引数}
\label{macro-arguments}

ほとんどの場合、マクロへの引数は式を\type{Expr}列挙型インスタンスとして表現したものです。これらは構文解析されていますが、型付けはされていません。つまり、Haxeの構文ルールに従うものであればどのようなものもあり得ます。マクロではその構造を解析したり、\expr{haxe.macro.Context.typeof()}を使ってその構造を調べたりできます。

マクロへの引数は評価できるとは限らないため、意図する副作用が起きる保証がないことに気を付けてください。また、引数の式はマクロで複製されて戻り値の式で複数使うことができるというのも重要です。

\haxe{assets/MacroArguments.hx}

\expr{add}マクロは、\expr{x++}を引数としてよびだされており、\tref{式の実体化}{macro-reification-expression}を使って\expr{x++ + x++}を返しており、このため2度インクリメントがされています。

\subsection{ExprOf}
\label{macro-ExprOf}

\type{Expr}はあらゆる入力と一致するため、Haxeでは\type{haxe.macro.ExprOf<T>}型を提供しています。ほとんどの面では\type{Expr}と同じですが、受け入れる式の型を強制することができます。これは\tref{静的拡張}{lf-static-extension}と合わせてマクロを使うときに便利です。

\haxe{assets/ExprOf.hx}

上2種類の\expr{identity}の呼び出しは両方とも問題ありません。たとえ引数が\expr{ExprOf<String>}で宣言されていてもです。\type{Int}型の\expr{1}が許容されることに驚くかもしれませんが、\tref{マクロの引数}{macro-arguments}での説明からの論理的な必然性があります。つまり引数の式は型付けされないので、コンパイラは\tref{単一化}{type-system-unification}の一致チェックができないというわけです。

次の2つの行の呼び出しは、静的拡張を使っている点で状況が異なります(\expr{using Main}に注目してください)。静的拡張では左側(\expr{"foo"}や\expr{1})の型によって、\expr{identity}のフィールドアクセスが意味を持ちます。これにより引数の型に対しての型チェックが可能になり、\expr{1.identity()}が\expr{Main.identity()}のフィールドが合わないという結果になっています。

\subsection{定数の式}
\label{macro-constant-arguments}

マクロは\tref{定数}{expression-constants}の引数を要求するように宣言することができます。

\haxe{assets/MacroArgumentsConst.hx}

これによりわざわざ式を経由するなく、コンパイラはその定数を直接使うことができます。

\subsection{残りの引数}
\label{macro-rest-argument}

マクロの引数の最後が\type{Array<Expr>}型だった場合、任意の個数の追加の引数を渡して、それを配列として利用できます。

\haxe{assets/MacroArgumentsRest.hx}

\section{実体化(レイフィケーション)}
\label{macro-reification}

Haxeコンパイラではマクロの活用を簡単にするために式、型、クラスの\emph{実体化(レイフィケーション)}が可能です。実体化の構文は\expr{macro expr}であり、\expr{expr}は正当なHaxe式であれば何でもかまいません。

\subsection{式の実体化}
\label{macro-reification-expression}

式の実体化を使うと、手軽に\type{haxe.macro.Expr}のインスタンスを作成できます。Haxeのコンパイラは通常のHaxeの構文を式のオブジェクトへと変換します。これにはエスケープの仕組みがあり、それらはすべて\expr{\$}の文字からはじまります。

\begin{description}
	\item[\expr{\$\{\}}または\expr{\$e\{\}}:] \type{Expr -> Expr} これは式の構築に使います。\expr{\{ \}}の中の式が評価されてその値がその位置に配置されます。
	\item[\expr{\$a\{\}}:] \type{Expr -> Array<Expr>} \type{Array<Expr>}が要求される場所(例えば、呼び出し引数や、ブロックの要素)で使用すると、\expr{\$a\{\}}の値を配列にします。そのほかの場合は、配列の宣言を生成します。
	\item[\expr{\$b\{\}}:] \type{Array<Expr> -> Expr} 与えられた配列からブロック式を生成します。
	\item[\expr{\$i\{\}}:] \type{String -> Expr} 与えられた文字列の識別子を生成します。
	\item[\expr{\$p\{\}}:] \type{Array<String> -> Expr} 文字列の配列から、フィールドアクセス式を生成します。
	\item[\expr{\$v\{\}}:] \type{Dynamic -> Expr} その引数の型にあわせて式を作ります。これは\tref{基本型}{types-basic-types}と\tref{列挙型インスタンス}{types-enum-instance}でのみ動作することが保証されています。
\end{description}

加えて\expr{@:pos(p)}\tref{メタデータ}{lf-metadata}を使って、実体化の場所の代わりに\expr{p}に式の位置を対応させられます。

この種類の実体化は式が期待されている場所でのみ動作します。また、\expr{object.\$\{fieldName\}}は動作しませんが、\expr{object.\$fieldName}は動作します。これはすべての文字列を期待する場所で同じです。

\begin{itemize}
	\item フィールドアクセス \expr{object.\$name}
	\item 変数名 \expr{var \$name = 1;}
\end{itemize}

\since{3.1.0}
\begin{itemize}
	\item フィールド名 \expr{\{ \$name: 1\} }
	\item 関数名 \expr{function \$name() \{ \}}
	\item キャッチの変数名 \expr{try e() catch(\$name:Dynamic) \{\}}
\end{itemize}

\subsection{型の実体化}
\label{macro-reification-type}

型の実体化を使うと、手軽に\type{haxe.macro.Expr.ComplexType}のインスタンスを生成できます。構文は\expr{macro : Type}で、\expr{Type}は正当な型のパスの式であれば何でもかまいません。この構文は通常の明示的な型注釈のコードに似ています(例えば、\expr{var x:Type}の変数宣言)。

\type{ComplexType}のコンストラクタごとに、以下の別々の構文があります。

\begin{description}
	\item[\expr{TPath}:] \expr{macro : pack.Type}
	\item[\expr{TFunction}:] \expr{macro : Arg1 -> Arg2 -> Return}
	\item[\expr{TAnonymous}:] \expr{macro : \{ field: Type \}}
	\item[\expr{TParent}:] \expr{macro : (Type)}
	\item[\expr{TExtend}:] \expr{macro : \{> Type, field: Type \}}
	\item[\expr{TOptional}:] \expr{macro : ?Type}
\end{description}

\subsection{クラスの実体化}
\label{macro-reification-class}

\type{haxe.macro.Expr.TypeDefinition}のインスタンスを取得するためにも、実体化は使えます。これには以下のような、\expr{macro class}の構文を使います。

\haxe{assets/ClassReification.hx}

生成された\type{TypeDefinition}のインスタンスは、多くの場合は\expr{haxe.macro.Context.defineType}に渡すことで、呼び出し対象のコンテクストに(マクロコンテクスト自体にではありません)新しい型を追加して使います。

この種類の実体化は\type{TypeDefinition}の\expr{fields}の配列から\expr{haxe.macro.Expr.Field}のインスタンスの取得するのにも便利です。

\section{Tools}
\label{macro-tools}

Haxe標準ライブラリには、マクロの活用を簡単にするツールクラス一式も用意されています。これらのクラスは\tref{静的拡張}{lf-static-extension}で使うのが最適で、\expr{using haxe.macro.Tools}で各コンテクストそれぞれやすべてに持ち込むことができます。

\begin{description}
	\item[\type{ComplexTypeTools}:] \type{ComplexType}のインスタンスを人間が読める形に出力したり、\type{ComplexType}対応する\type{Type}を見つけたりできます。
	\item[\type{ExprTools}:] \type{Expr}のインスタンスを人間が読める形で出力したり、式の繰り返しやマッピングの処理を行ったりできます。
	\item[\type{MacroStringTools}:] マクロコンテクストで有用な文字列と文字列式を扱う処理を提供します。
	\item[\type{TypeTools}:] \type{Type}インスタンスを人間が読める形で出力したり、\tref{単一化}{type-system-unification}や対応する\type{ComplexType}の取得といった型を扱うのに便利な機能を提供します。
\end{description}

さらに\type{haxe.macro.Printer}クラスが、様々な型を人間の読めるフォーマットで出力する\expr{public}メソッドを提供しています。これはマクロのデバッグをするのに便利です。

\trivia{ティンカーベルライブラリとなぜTools.hxが動作するのか}{
モジュールを\expr{using}することでそのすべての型が静的拡張のコンテクストに取り込まれることはこれまでに学んできました。つまるところ、その型というのは他の型指定する\tref{typedef}{type-system-typedef}でも良いわけです。コンパイラはよその型をモジュールの一部と認識して、それが静的拡張にも引き継がれるわけです。\\
このテクニックはJuraj Kirchheimの\emph{tinkerbell}\footnote{https://github.com/back2dos/tinkerbell}で同じ目的で初めて使われました。tinkerbellではHaxeコンパイラと標準ライブラリが提供するよりもずっと先に便利なマクロツールを提供していました。このライブラリは今でも追加のマクロのツールを得るためのライブラリとして第一候補でありつづけており、さらにその他の便利機能も提供しています。}


\section{型ビルド}
\label{macro-type-building}

型ビルドのマクロはいくつかの点で式マクロとは使い方が違います。

\begin{itemize}
	\item 式は返しません。その代わりクラスフィールドの配列を返します。戻り値の型は明示的に\type{Array<haxe.macro.Expr.Field>}を指定しないといけません。
	\item \tref{コンテクスト}{macro-context}にローカルメソッドとローカル変数が含まれません。
	\item コンテクストにビルドフィールドが含まれ、\expr{haxe.macro.Context.getBuildFields()}で使用可能です。
	\item 直接呼び出すのではなく、\tref{class}{types-class-instance}または\tref{enum}{types-enum-instance}の宣言に対する\expr{@:build}または\expr{@:autoBuild}\tref{メタデータ}{lf-metadata}の引数として指定します。
\end{itemize}

以下の例で型ビルドを実演しています。モジュールが\expr{macro}関数を含むとそのモジュールがマクロコンテクストで型付けされてしまうため、2つのファイルに分割していることに気を付けてください。ビルドされる型はビルドマクロが走る前は不完全な状態でしか読み込みがされないので、このことがよく問題になります。型ビルドのマクロは常にそれ用のモジュールに分けて定義することをオススメします。

\haxe{assets/TypeBuildingMacro.hx}
\haxe{assets/TypeBuilding.hx}

\type{TypeBuildingMacro}の\expr{build}メソッドは次の3つのステップを経て動作します。

\begin{enumerate}
	\item \expr{Context.getBuildFields()}を使ってビルドフィールドを取得する。
	\item \expr{funcName}マクロ引数をフィールド名として使って、新しい\type{haxe.macro.expr.Field}を宣言する。このフィールドは\type{String}\tref{変数}{class-field-variable}でデフォルト値は\expr{"my default"}(\expr{kind}フィールドより)で\expr{public static}です(\expr{access}フィールドより)。
	\item 新しいフィールドをビルドフィールドに追加してそれを返す。
\end{enumerate}

このマクロは\type{Main}クラスに対する\expr{@:build}メタデータの引数です。この型が必要になるとコンパイラは以下を行います。

\begin{enumerate}
	\item クラスフィールドも含めて、このモジュールを構文解析する。
	\item \tref{インターフェース}{types-interfaces}や\tref{継承}{types-class-inheritance}などの他の型との関係も含めて、型の設定をする。
	\item \expr{@:build}メタデータに従って、型ビルドのマクロを実行する。
	\item 型ビルドの返したフィールドに従って、クラスの型付けを通常通り続行する。
\end{enumerate}

こうして型ビルドマクロによって思いのままにクラスのフィールドを追加したり、編集したりができます。上の例では、マクロは\expr{"myFunc"}の引数で呼び出されて、\expr{Main.myFunc}を正当なフィールドアクセスにしています。

型ビルドのマクロで何も編集したくない場合、マクロで\expr{null}を返してもかまいません。これでコンパイラに何の変更もしないことが伝わります。\expr{Context.getBuildFields()}を返すよりも好ましいです。



\subsection{列挙型ビルド}
\label{macro-enum-building}

\tref{列挙型}{types-enum-instance}のビルドは、クラスのビルドと類似しており簡単な対応関係があります。

\begin{itemize}
	\item 引数を持たない列挙型は変数フィールド\expr{FVar}です。
	\item 引数を持つ列挙型はメソッドフィールド\expr{FFun}です。
\end{itemize}

\todo{Check if we can build GADTs this way.}

\haxe{assets/EnumBuildingMacro.hx}
\haxe{assets/EnumBuilding.hx}

列挙型\type{E}は\expr{:build}メタデータの修飾されており、呼び出されたマクロが2つのコンストラクタ\expr{A}と\expr{B}を追加しています。\expr{A}は\expr{FVar(null, null)}、つまり引数の無いコンストラクタとして追加されています。\expr{B}は\tref{実体化}{macro-reification-expression}を使って、\type{Int}引数1つを持つ\type{haxe.macro.Expr.Function}を取得しています。

\expr{main}メソッドは\tref{マッチング}{lf-pattern-matching}によって生成された列挙型の構造を証明しています。これで、生成された型が以下と同じだということがわかります。

\begin{lstlisting}
enum E {
	A;
	B(value:Int);
}
\end{lstlisting}


\subsection{@:autoBuild}
\label{macro-auto-build}

クラスが\expr{:autoBuild}メタデータを持つ場合、それを継承するすべてのクラスに\expr{:build}メタデータを生成します。インターフェースが\expr{:autoBuild}メタデータを持つ場合、それを継承するすべてのインターフェースとすべての実装クラスに\expr{:build}メタデータを生成します。\expr{:autoBuild}はそのクラスやインターフェース自身には\expr{:build}を適用しないことに気をつけてください。

\haxe{assets/AutoBuildingMacro.hx}
\haxe{assets/AutoBuilding.hx}

コンパイル中に以下の出力がされます。

\begin{lstlisting}
AutoBuildingMacro.hx:6:
  fromInterface: TInst(I2,[])
AutoBuildingMacro.hx:6:
  fromInterface: TInst(Main,[])
AutoBuildingMacro.hx:11:
  fromBaseClass: TInst(Main,[])
\end{lstlisting}

これらのマクロ実行順序は不定であることを留意しておきましょう、詳しくは\Fullref{macro-limitations-build-order}で説明されています。


\subsection{@:genericBuild}
\label{macro-generic-build}
\since{3.1.0}

通常の\tref{ビルドマクロ}{macro-type-building}は型ごとに実行するもので、これでも十分に強力です。いくつかの用途では、型の\emph{使用}ごと、つまりコードに出現するごとにビルドマクロが走ることが役立つものもあります。何より、これにより具体的な型パラメータにもアクセスできるようになります。

\expr{@:genericBuild}は\expr{@:build}と全くおなじように型に引数付きのマクロ呼び出しを追加することで使用します。

\haxe{assets/GenericBuildMacro1.hx}
\haxe{assets/GenericBuild1.hx}

この例を実行するとコンパイラは\ic{TAbstract(Int,[])}と\ic{TInst(String,[])}を出力することから、\type{MyType}の具体的な型が認識されたことがわかります。このマクロの処理では、この情報をカスタムの型の生成もできます(\expr{haxe.macro.Context.defineType}を使うことで)し、すでに存在する型の参照もできます。簡潔さのためにここでは\expr{null}を返して、コンパイラに型を\tref{推論}{type-system-type-inference}させています。

Haxe 3.1では\expr{@:genericBuild}マクロの戻り値は\type{haxe.macro.Type}でなくてはいけませんでした。Haxe 3.2では、
\type{haxe.macro.ComplexType}を返すことが許可(そして推奨)されています。多くの場合は、型はただパスで参照するだけで動作するのでこのほうが簡単です。

\paragraph{定数型パラメータ}

Haxeでは型パラメータ名が\expr{Const}の場合、\tref{定数値の式}{expression-constants}を型パラメータとして渡すことができます。\expr{@:genericBuild}マクロのコンテクストでマクロに直接情報を渡すのに役立ちます。

\haxe{assets/GenericBuildMacro2.hx}
\haxe{assets/GenericBuild2.hx}

このマクロの処理ではファイルを読み込んで、カスタムの型を生成することができます。

\section{制限}
\label{macro-limitations}
\state{NoContent}

\subsection{マクロ内のマクロ}
\label{macro-limitations-macro-in-macro}

\subsection{静的拡張}
\label{macro-limitations-static-extension}

マクロと\tref{静的拡張}{lf-static-extension}の概念には多少の衝突があります。静的拡張は使用される関数を決定するために既知の型を要求しますが、構文が型付けされる前に実行されます。ですからこの2つの機能を合わせて使うと問題が生じるというのは驚くことではありません。Haxe 3.0では型付けされた式を元の構文の式に戻す変換を試みます。これは必ず成功するわけではなく、重要な情報が失われることもあります。これらについては十分に気を付けたうえで使用することを推奨します。

\since{3.1.0}

静的拡張とマクロの合わせた使用について3.1.0のリリースで修正がされました。Haxeコンパイラはマクロの引数から元の式の復元を試行しなくなった代わりに、特殊な\expr{@:this this}の式を渡すようになりました。式の構造については何の情報も提供しない代わりに正しく型付けができます。

\haxe{assets/MacroStaticExtension.hx}

\subsection{ビルド順序}
\label{macro-limitations-build-order}

型のビルド順序は未定義であり、それは\tref{ビルドマクロ}{macro-type-building}の実行順序についても同じです。いくつかのルールは決まってはいますが、ビルドマクロは実行順序に依存しないようにすることを強く推奨します。もし型ビルドを複数回実行する必要があるなら、マクロのコードで直接解決してください。ビルドマクロが同じ型に対して複数回実行されるのを避けるためには、状態を静的変数にいれておくか、型に\tref{メタデータ}{lf-metadata}を追加するのが有効です。

\haxe{assets/MacroBuildOrder.hx}

\type{I1}と、\type{I2}の両方のインターフェースが\expr{:autoBuild}を持っており、\type{C}クラスに対して2度ビルドマクロが実行されます。ここではクラスに\expr{:processed}メタデータを足して、2度目の実行でそれを確認することで重複した処理を回避しています。


\subsection{型パラメータ}
\label{macro-limitations-type-parameters}


\section{初期化マクロ}
\label{macro-initialization}

初期化マクロはコマンドラインから\expr{--macro callExpr(args)}コマンドを使って呼び出します。これにより、コンテクストを生成した後の、\expr{-main}の引数が型付けされるより前に呼び出されるコールバックを登録します。これにより様々な方法でコンパイラの設定ができます。

\expr{--macro}の引数が単なる識別子の呼び出しだった場合、その識別子はHaxe標準ライブラリの\type{haxe.macro.Compiler}内から検索されます。このクラスには便利な初期化マクロがいくつもあります。詳しくは\href{http://api.haxe.org//haxe/macro/Compiler.html}{API}を記載されています。

例えば、\expr{include}マクロではパッケージをまるまる、必要であれば再帰的にコンパイルに含めることができます。その場合のコマンドライン引数は\expr{--macro include('some.pack', true)}といった形になります。

もちろん、カスタムの初期化マクロを定義して実際のコンパイルの前に様々な作業をさせることもできます。そういったマクロは\expr{--macro some.Class.theMacro(args)}の形で呼び出します。例えば、すべてのマクロに共通の\tref{コンテクスト}{macro-context}が使われるので、初期化マクロで他のマクロの設定のための静的フィールドに値を設定することができます。