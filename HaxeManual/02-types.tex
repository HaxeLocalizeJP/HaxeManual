\chapter{型}
\label{types}

Haxeコンパイラは豊かな型システムを持っており、これがコンパイル時に型エラーを検出することを手助けします。型エラーとは、文字列による割り算や、整数のフィールドへのアクセス、不十分な(あるいは多すぎる)引数での関数呼び出し、といった型が不正である演算のことです。

いくつかの言語では、この安全性を得るためには各構文での明示的な型の宣言が強いられるので、コストがかかります。

\begin{lstlisting}
var myButton:MySpecialButton = new MySpecialButton(); // AS3
MySpecialButton* myButton = new MySpecialButton(); // C++ 
\end{lstlisting}

一方、Haxeではコンパイラが型を\emph{推論}できるため、この明示的な型注釈は必要ではありません。

\begin{lstlisting}
var myButton = new MySpecialButton(); // Haxe
\end{lstlisting}

型推論の詳細については\Fullref{type-system-type-inference}で説明します。今のところは、上のコードの変数\expr{myButton}は\type{MySpecialButton}の\emph{クラスインスタンス}とわかると言っておけば十分でしょう。

Haxeの型システムは、以下の7つの型を認識します。

\begin{description}
 \item[\emph{クラスインスタンス}:] クラスかインスタンスのオブジェクト
 \item[\emph{列挙型インスタンス}:] Haxeの列挙型(enum)の値
 \item[\emph{構造体}:] 匿名の構造体。例えば、連想配列。
 \item[\emph{関数}:] 引数と戻り値1つの型の複合型。
 \item[\emph{ダイナミック}:] あらゆる型に一致する、なんでも型。
 \item[\emph{抽象(abstract)}:] 実行時には別の型となる、コンパイル時の型。
 \item[\emph{単相}:] 後で別の型が付けられる未知(Unknown)の型。
\end{description}

ここからの節で、それぞれの型のグループとこれらがどうかかわっているのかについて解説していきます。

\define{複合型(Compound Type)}{define-compound-type}{
複合型というのは、従属する型を持つ型です。\tref{型パラメータ}{type-system-type-parameters}を持つ型や、\tref{関数}{types-function}型がこれに当たります。
}

\section{基本型}
\label{types-basic-types}

\emph{基本型}は\type{Bool}と\type{Float}と\type{Int}です。文法上、これらの値は以下のように簡単に識別可能です。

\begin{itemize}
	\item \expr{true}と\expr{false}は\type{Bool}。
	\item \expr{1}、\expr{0}、\expr{-1}、\expr{0xFF0000}は\type{Int}。
	\item \expr{1.0}、\expr{0.0}、\expr{-1.0}、\expr{1e10}は\type{Float}。
\end{itemize}

Haxeでは基本型は\tref{クラス}{types-class-instance}ではありません。これらは\tref{抽象型}{types-abstract}として実装されており、以降の項で解説するコンパイラ内部の演算処理に結び付けられています。

\subsection{数値型}
\label{types-numeric-types}

\define[Type]{Float}{define-float}{IEEEの64bit倍精度浮動小数点数を表します。}

\define[Type]{Int}{define-int}{整数を表します。}

\type{Int}は\type{Float}が期待されるすべての場所で使用することが可能です (IntはFloatへの代入が可能で、Floatとして表現可能です)。ですが、逆はできません。 \type{Float}から\type{Int}への代入は精度を失ってしまう場合があり、信頼できません。

\subsection{オーバーフロー}
\label{types-overflow}

パフォーマンスのためにHaxeコンパイラはオーバーフローに対する挙動を矯正しません。オーバーフローに対する挙動は、ターゲットのプラットフォームが責任を持ちます。各プラットフォームごとのオーバーフローの挙動を以下にまとめています。

\begin{description}
	\item[C++, Java, C\#, Neko, Flash:] 一般的な挙動をもつ32Bit符号付き整数。
	\item[PHP, JS, Flash 8:] ネイティブの\emph{Int}型を持たない。Floatの上限(2\textsuperscript{52})を超えた場合に精度を失う。
\end{description}

代替手段として、プラットフォームごとの追加の計算を行う代わりに、正しいオーバーフローの挙動を持つ\emph{haxe.Int32}と\emph{haxe.Int64}クラスが用意されています。

\subsection{数値の演算子}
\label{types-numeric-operators}

\todo{make sure the types are right for inc, dec, negate, and bitwise negate}
\todo{While introducing the different operations, we should include that information as well, including how they differ with the "C" standard, see http://haxe.org/manual/operators}
This the list of numeric operators in Haxe, grouped by descending priority.

\begin{center}
\begin{tabular}{| l | l | l | l | l |}
	\hline
	\multicolumn{5}{|c|}{算術演算} \\ \hline
	演算子 & 演算 & 引数1 & 引数2 & 戻り値 \\ \hline
	\expr{++} & 1増加 & \type{Int} & なし & \type{Int}\\
	& & \type{Float} & なし & \type{Float}\\
	\expr{--} & 1減少 & \type{Int} & なし & \type{Int}\\
	& & \type{Float} & なし & \type{Float}\\
	\expr{+} & 加算 & \type{Float} & \type{Float} & \type{Float} \\
	& & \type{Float} & \type{Int} & \type{Float} \\
	& & \type{Int} & \type{Float} & \type{Float} \\
	& & \type{Int} & \type{Int} & \type{Int} \\
	\expr{-} & 減算 & \type{Float} & \type{Float} & \type{Float} \\
	& & \type{Float} & \type{Int} & \type{Float} \\
	& & \type{Int} & \type{Float} & \type{Float} \\
	& & \type{Int} & \type{Int} & \type{Int} \\
	\expr{*} & 乗算 & \type{Float} & \type{Float} & \type{Float} \\
	& & \type{Float} & \type{Int} & \type{Float} \\
	& & \type{Int} & \type{Float} & \type{Float} \\
	& & \type{Int} & \type{Int} & \type{Int} \\	
	\expr{/} & 除算 & \type{Float} & \type{Float} & \type{Float} \\
	& & \type{Float} & \type{Int} & \type{Float} \\
	& & \type{Int} & \type{Float} & \type{Float} \\
	& & \type{Int} & \type{Int} & \type{Float} \\
	\expr{\%} & 剰余 & \type{Float} & \type{Float} & \type{Float} \\
	& & \type{Float} & \type{Int} & \type{Float} \\
	& & \type{Int} & \type{Float} & \type{Float} \\
	& & \type{Int} & \type{Int} & \type{Int} \\	 \hline
	\multicolumn{5}{|c|}{比較演算} \\ \hline
	演算子 & 演算 & 引数1 & 引数2 & 戻り値 \\ \hline
	\expr{==} & 等価 & \type{Float/Int} & \type{Float/Int} & \type{Bool} \\
	\expr{!=} & 不等価 & \type{Float/Int} & \type{Float/Int} & \type{Bool} \\
	\expr{<} & より小さい & \type{Float/Int} & \type{Float/Int} & \type{Bool} \\
	\expr{<=} & より小さいか等しい & \type{Float/Int} & \type{Float/Int} & \type{Bool} \\
	\expr{>} & より大きい & \type{Float/Int} & \type{Float/Int} & \type{Bool} \\
	\expr{>=} & より大きいか等しい & \type{Float/Int} & \type{Float/Int} & \type{Bool} \\ \hline
	\multicolumn{5}{|c|}{ビット演算} \\ \hline
	演算子 & 演算 & 引数1 & 引数2 & 戻り値 \\ \hline
	\expr{\textasciitilde} & ビット単位の否定(NOT) & \type{Int} & なし & \type{Int} \\	
	\expr{\&} & ビット単位の論理積(AND) & \type{Int} & \type{Int} & \type{Int} \\	
	\expr{|} & ビット単位の論理和(OR) & \type{Int} & \type{Int} & \type{Int} \\	
	\expr{\^} & ビット単位の排他的論理和(XOR) & \type{Int} & \type{Int} & \type{Int} \\	
	\expr{<<} & 左シフト & \type{Int} & \type{Int} & \type{Int} \\
	\expr{>>} & 右シフト & \type{Int} & \type{Int} & \type{Int} \\
	\expr{>>>} & 符号なしの右シフト & \type{Int} & \type{Int} & \type{Int} \\ \hline
	\multicolumn{5}{|c|}{Comparison} \\ \hline
\end{tabular}
\end{center}

\subsection{Bool(真偽値)}
\label{types-bool}

\define[Type]{Bool}{define-bool}{真(\emph{true})または、偽(\emph{false})のどちらかになる値を表す。}

\type{Bool}型の値は、\tref{\expr{if}}{expression-if}や\tref{\expr{while}}{expression-while}のような\emph{条件文}によく表れます。以下の演算子は、\type{Bool}値を受け取って\type{Bool}値を返します。

\begin{itemize}
	\item \expr{\&\&} (and)
	\item \expr{||} (or)
	\item \expr{!} (not)
\end{itemize}

Haxeは、Bool値の2項演算は、実行時に左から右へ必要な分だけ評価することを保証します。例えば、\expr{A \&\& B}という式は、まず\expr{A}を評価して\expr{A}が\expr{true}だった場合のみ\expr{B}が評価されます。同じように、\expr{A || B}では\expr{A}が\expr{true}だった場合は、\expr{B}の値は意味を持たないので評価されません。

これは、以下のような場合に重要です。

\begin{lstlisting}
if (object != null && object.field == 1) { }
\end{lstlisting}

\expr{object}が\expr{null}の場合に\expr{object.field}にアクセスするとランタイムエラーになりますが、事前に\expr{object != null}のチェックをすることでエラーから守ることができます。

\subsection{Void}
\label{types-void}

\define[Type]{Void}{define-void}{Voidは型が存在しないことを表します。特定の場面(主に関数)で値を持たないことを表現するのに使います。}

Voidは型システムにおける特殊な場合です。Voidは実際には型ではありません。Voidは特に関数の引数と戻り値で型が存在しないことを表現するのに使います。私たちはすでに最初の``Hello World''の例でVoidを使用しています。
\todo{please review, doubled content}

\haxe{assets/HelloWorld.hx}

関数型について詳しくは\Fullref{types-function}で解説しますが、ここで軽く予習をしておきましょう。上の例の\expr{main}関数は\type{Void->Void}型です。これは``引数は無く、戻り値も無い''という意味です。

Haxeでは、フィールドや変数に対してVoidを指定することはできません。以下のように書こうとするとエラーが発生します。
\todo{review please, sounds weird}

\begin{lstlisting}
// Arguments and variables of type Void are not allowed
var x:Void;
\end{lstlisting}



\section{Nullable(null許容型)}
\label{types-nullability}

\define{Nullable}{define-nullable}{Haxeでは、ある型が値として\expr{null}をとる場合に\emph{Nullable}(null許容型)であるとみなす。}

プログラミング言語は、Nullableについてなにか1つ明確な定義を持つのが一般的です。ですが、Haxeではターゲットとなる言語のもともとの挙動に従うことで妥協しています。ターゲット言語のうちのいくつかは全てがデフォルト値として\expr{null}をとり、その他は特定の型では\expr{null}を許容しません。つまり、以下の2種類の言語を区別しなくてはいけません。

\define{静的ターゲット}{define-static-target}{静的ターゲットでは、その言語自体が基本型が\expr{null}を許容しないような型システムを持っています。この性質は\target{Flash}、\target{C++}、\target{Java}、\target{C\#}ターゲットに当てはまります。}
\define{動的ターゲット}{define-dynamic-target}{動的ターゲットは型に関して寛容で、基本型が\expr{null}を許容します。これは\target{JavaScript}と\target{PHP}、\target{Neko}、\target{Flash 6-8}ターゲットが当てはまります。}
\todo{for starters...please review}

\define{デフォルト値}{define-default-value}{
  基本型は、静的ターゲットではデフォルト値は以下になります。
  \begin{description}
		\item[\type{Int}:] \expr{0}。
		\item[\type{Float}:] \target{Flash}では\expr{NaN}。その他の静的ターゲットでは\expr{0.0}。
		\item[\type{Bool}:] \expr{false}。
	\end{description}
}

その結果、Haxeコンパイラは静的ターゲットでは基本型に対する\expr{null}を代入することはできません。\expr{null}を代入するためには、以下のように基本型を\type{Null$<$T$>$}で囲う必要があります。

\begin{lstlisting}
// error on static platforms
var a:Int = null;
var b:Null<Int> = null; // allowed
\end{lstlisting}

同じように、基本型は\type{Null$<$T$>$}で囲わなければ\expr{null}と比較することはできません。

\begin{lstlisting}
var a : Int = 0;
// error on static platforms
if( a == null ) { ... }
var b : Null<Int> = 0;
if( b != null ) { ... } // allowed
\end{lstlisting}

この制限は\tref{unification}{type-system-unification}が動作するすべての状況でかかります。

\define[Type]{\expr{Null<T>}}{define-null-t}{静的ターゲットでは、\type{Null<Int>}、\type{Null<Float>}、\type{Null<Bool>}の型で\expr{null}を許容することが可能になります。動的ターゲットでは\expr{Null<T>}に効果はありません。また、\expr{Null<T>}はその型が\expr{null}を持つことを表すドキュメントとしても使うことができます。}

nullの値は隠匿されます。つまり、\type{Null$<$T$>$}や\type{Dynamic}のnullの値を基本型に代入した場合には、デフォルト値が使用されます。

\begin{lstlisting}
var n : Null<Int> = null;
var a : Int = n;
trace(a); // 0 on static platforms
\end{lstlisting}



\subsection{オプション引数とnull許容}
\label{types-nullability-optional-arguments}

null許容について考える場合、オプション引数についても考慮しなくてはいけません。

特に、null許容ではない\emph{ネイティブ}のオプション引数と、それとは異なる、null許容であるHaxe特有のオプション引数があることです。この違いは以下のように、オプション引数にクエスチョンマークを付けることで作ります。

\begin{lstlisting}
// x is a native Int (not nullable)
function foo(x : Int = 0) {}
// y is Null<Int> (nullable)
function bar( ?y : Int) {}
// z is also Null<Int>
function opt( ?z : Int = -1) {}
\end{lstlisting}
\todo{Is there a difference between \type{?y : Int} and \type{y : Null$<$Int$>$} or can you even do the latter? Some more explanation and examples with native optional and Haxe optional arguments and how they relate to nullability would be nice.}

\trivia{アーギュメント(Argument)とパラメータ(Parameter)}{他のプログラミング言語では、よく\emph{アーギュメント}と\emph{パラメータ}は同様の意味として使われます。Haxeでは、関数に関連する場合に\emph{アーギュメント}を、\Fullref{type-system-type-parameters}と関連する場合に\emph{パラメータ}を使います。}

\section{クラスインスタンス}
\label{types-class-instance}


多くのオブジェクト指向言語と同じように、Haxeでも大抵のプログラムではクラスが最も重要なデータ構造です。Haxeのすべてのクラスは、明示された名前と、クラスの配置されたパスと、0個以上のクラスフィールドを持ちます。ここではクラスの一般的な構造とその関わり合いについて焦点を当てていきます。クラスフィールドの詳細については後で\Fullref{class-field}の章で解説をします。
\todo{please review future tense}

以下のサンプルコードが、この節で学ぶ基本になります。

\haxe{assets/Point.hx}

意味的にはこれは不連続の2次元空間上の点を表現するものですが、このことはあまり重要ではありません。代わりにその構造に注目しましょう。

\begin{itemize}
	\item \expr{class}のキーワードは、クラスを宣言していることを示すものです。
	\item \type{Point}はクラス名です。\tref{型の識別子のルール}{define-identifier}に従っているものが使用できます。
	\item クラスフィールドは\expr{$\left\{\right\}$}で囲われます。
	\item \type{Int}型の\expr{x}と\expr{y}の2つの\emph{変数}フィールドと、
	\item クラスの\emph{コンストラクタ}となる特殊な\emph{関数}フィールド\expr{new}と、
	\item 通常の関数\expr{toString}でクラスフィールドが構成されています。
\end{itemize}

また、Haxeにはすべてのクラスと一致する特殊な型があります。

\define[Type]{\expr{Class$<$T$>$}}{define-class-t}{
この型はすべてのクラスの型と一致します。つまり、すべてのクラス(インスタンスではなくクラス)をこれに代入することができます。コンパイル時に、\type{Class<T>}は全てのクラスの型の共通の親の型となります。しかし、この関係性は生成されたコードに影響を与えません。

この型は、任意のクラスを要求するようなAPIで役立ちます。例えば、\tref{HaxeリフレクションAPI}{std-reflection}のいくつかのメソッドがこれに当てはまります。
}

\subsection{クラスのコンストラクタ}
\label{types-class-constructor}

クラスのインスタンスは、クラスのコンストラクタを呼び出すことで生成されます。この過程は一般的に\emph{インスタンス化}と呼ばれます。クラスインスタンスは、別名として\emph{オブジェクト}と呼ぶこともあります。ですが、クラス/クラスインスタンスと、列挙型/列挙型インスタンスという似た概念を区別するために、クラスインスタンスと呼ぶことが好まれます。

\begin{lstlisting}
var p = new Point(-1, 65);
\end{lstlisting}

この例で、変数\expr{p}に代入されたのが\type{Point}クラスのインスタンスです。\type{Point}のコンストラクタは\expr{-1}と\expr{65}の2つの引数を受け取り、これらをそれぞれインスタンスの\expr{x}と\expr{y}の変数に代入しています(\Fullref{types-class-instance}で、定義を確認してください)。\expr{new}の正確な意味については、後の\ref{expression-new}の節で再習します。現時点では、\expr{new}はクラスのコンストラクタを呼び、適切なオブジェクトを返すものと考えておきましょう。


\subsection{継承}
\label{types-class-inheritance}

クラスは他のクラスから継承ができます。Haxeでは、継承は\expr{extends}キーワードを使って行います。

\haxe{assets/Point3.hx}

この関係は、よく"BはAである(is-a)"の関係とよく言われます。つまり、すべての\type{Point3}クラスのインスタンスは、同時に\type{Point}のインスタンスである、ということです。\type{Point}は\type{Point3}の\emph{親クラス}であると言い、\type{Point3}は\type{Point}の\emph{子クラス}であると言います。1つのクラスはたくさんの子クラスを持つことができますが、親クラスは1つしか持つことができません。ただし、``クラスXの親クラス''というのは、直接の親クラスだけでなく、親クラスの親クラスや、そのまた親、また親のクラスなどを指すこともよくあります。

上記のクラスは\type{Point}コンストラクタによく似ていますが、2つの新しい構文が登場しています。

\begin{itemize}
	\item \expr{extends Point} は\type{Point}からの継承を意味します。
	\item \expr{super(x, y)} は親クラスのコンストラクタを呼び出します。この場合は\expr{Point.new}です。
\end{itemize}

上の例ではコンストラクタを定義していますが、子クラスで自分自身のコンストラクタを定義する必要はありません。ただし、コンストラクタを定義する場合\expr{super()}の呼び出しが必須になります。他のよくあるオブジェクト指向言語とは異なり、\expr{super()}はコンストラクタの最初である必要はなく、どこで呼び出しても構いません。

また、クラスはその親クラスの\tref{メソッド}{class-field-method}を\expr{override}キーワードを明示して記述することで上書きすることができます。その効果と制限について詳しくは\Fullref{class-field-overriding}であつかいます。


\subsection{インターフェース}
\label{types-interfaces}

インターフェースはクラスのパブリックフィールドを記述するもので、クラスの署名ともいうべきものです。インターフェースは実装を持たず、構造に関する情報のみを与えます。

\begin{lstlisting}
interface Printable {
	public function toString():String;
}
\end{lstlisting}
この構文は以下の点をのぞいて、クラスによく似ています。

\begin{itemize}
	\item \expr{interface}キーワードを\expr{class}キーワードの代わりに使う。
	\item 関数が\tref{式}{expression}を持たない。
	\item すべてのフィールドが型を明示する必要がある。
\end{itemize}

インタフェースは、\tref{構造的部分型}{type-system-structural-subtyping}とは異なり、クラス間の\emph{静的な関係性}について記述します。以下のように明示的に宣言した場合にのみ、クラスはインターフェースと一致します。

\begin{lstlisting}
class Point implements Printable { }
\end{lstlisting}

\expr{implements}キーワードの記述により、"\type{Point}は\type{Printable}である(is-a)"の関係性が生まれます。つまり、すべての\type{Point}のインスタンスは、\type{Printable}のインスタンスでもあります。クラスは親のクラスを1つしか持てませんが、以下のように複数の\expr{implements}キーワードを使用することで複数のインターフェースを実装(implements)することが可能です。

\begin{lstlisting}
class Point implements Printable implements Serializable
\end{lstlisting}

コンパイラは実装が条件を満たしているかの確認を行います。つまり、クラスが実際にインターフェースで要求されるフィールドを実装しているかを確めます。フィールドの実装は、そのクラス自体と、その親となるいずれかのクラスの実装が考慮されます。

インターフェースのフィールドは、変数とプロパティのどちらであるかに対する制限は与えません:

\haxe{assets/InterfaceWithVariables.hx}

インターフェースは\expr{extends}キーワードで複数のインタフェースを継承することができます。

\begin{lstlisting}
interface Debuggable extends Printable extends Serializable
\end{lstlisting}

\trivia{Implementsの構文}{Haxeの3.0よりも前のバージョンでは、\expr{implements}キーワードはカンマで区切られていました。Javaのデファクトスタンダードに合わせるため、私たちはカンマを取り除くことに決定しました。これが、Haxe2と3の間の破壊的な変更の1つです。}

\section{列挙型インスタンス}
\label{types-enum-instance}

Haxeには強力な列挙型(enum)をもっています。この列挙型は実際には\emph{代数的データ型} (ADT)\footnote{\url{http://en.wikipedia.org/wiki/Algebraic_data_type}}に当たります。列挙型は\tref{式}{expression}を持つことはできませんが、データ構造を表現するのに非常に役に立ちます。

\haxe{assets/Color.hx}

このコードでは、enumは、赤、緑、青のいずれかか、またはRGB値で表現した色、を書き表しています。この文法の構造は以下の通りです。

\begin{itemize}
	\item \expr{enum}キーワードが、列挙型について定義することを宣言しています。
	\item \type{Color}が列挙型の名前です。\tref{型の識別子のルール}{define-identifier}に従うすべてのものが使用できます。
	\item 中カッコ \expr{$\left\{\right\}$} で囲んだ中に\emph{列挙型のコンストラクタ}を記述します。
	\item \expr{Red}と\expr{Green}と\expr{Blue}には引数がありません。
	\item \expr{Rgb}は、\expr{r}、\expr{g}、\expr{b}の3つの\type{Int}型の引数を持ちます。
\end{itemize}

Haxの型システムには、すべての列挙型を統合する型があります。

\define[Type]{\expr{Enum$<$T$>$}}{define-enum-t}{すべての列挙型と一致する型です。コンパイル時に、\type{Enum<T>}は全ての列挙型の共通の親の型となります。しかし、この関係性は生成されたコードに影響を与えません。} 
\todo{Same as in 2.2, what is \type{Enum$<$T$>$} syntax?}

\subsection{列挙型のコンストラクタ}
\label{types-enum-constructor}

クラスと同じように、列挙型もそのコンストラクタを使うことでインスタンス化を行います。しかし、クラスとは異なり列挙型は複数のコンストラクタを持ち、以下のようにコンストラクタの名前を使って呼び出します。

\begin{lstlisting}
var a = Red;
var b = Green;
var c = Rgb(255, 255, 0);
\end{lstlisting}
このコードでは変数\expr{a}、\expr{b}、\expr{c}の型は\type{Color}です。変数\expr{c}は\expr{Rgb}コンストラクタと引数を使って初期化されています。
\todo{list arguments}

すべての列挙型のインスタンスは\type{EnumValue}という特別な型に対して代入が可能です。

\define[Type]{EnumValue}{define-enumvalue}{EnumValueはすべての列挙型のインスタンスと一致する特別な型です。この型はHaxeの標準ライブラリでは、すべての列挙型に対して可能な操作を提供するのに使われます。またユーザーのコードでは、特定の列挙型ではなく任意の列挙型のインスタンスを要求するAPIで利用できます。}

以下の例からわかるように、列挙型とそのインスタンスを区別することは大切です。

\haxe{assets/EnumUnification.hx}

もし、上でコメント化されている行のコメント化が解除された場合、このコードはコンパイルできなくなります。これは、列挙型のインスタンスである\expr{Red}は、列挙型である\type{Enum<Color>}型の変数には代入できないためです。

この関係性は、クラスとそのインスタンスの関係性に似ています。

\trivia{\type{Enum$<$T$>$の型パラメータを具体化する}}{このマニュアルのレビューアの一人は上のサンプルコードの\type{Color}と\type{Enum<Color>}の違いについて困惑しました。実際、型パラメータの具体化は意味のないもので、デモンストレーションのためのものでしかありませんでした。私たちはよく型を書くのを省いて、型についてあつかうのを\tref{型推論}{type-system-type-inference}にまかせてしまいます。

しかし、型推論では\type{Enum<Color>}ではないものが推論されます。コンパイラは、列挙型のコンストラクタをフィールドとしてみなした、仮の型を推論します。現在のHaxe3.2.0では、この仮の型について表現することは不可能であり、また表現する必要もありません。}


\subsection{列挙型を使う}
\label{types-enum-using}

列挙型は、有限の種類の値のセットが許されることを表現するだけでも有用です。それぞれのコンストラクタについて多様性が示されるので、コンパイラはありうる全ての値が考慮されていることをチェックすることが可能です。これは、例えば以下のような場合です。

\haxe{assets/Color2.hx}

\expr{getColor()}の戻り値を\expr{color}に代入し、その値で\tref{\expr{switch}式}{expression-switch}の分岐を行います。

初めの\expr{Red}、\expr{Green}、\expr{Blue}の3ケースについては些細な内容で、ただColorの引数無しのコンストラクタとの一致するか調べています。最後の\expr{Rgb(r, g, b)}のケースでは、コンストラクタの引数の値をどうやって利用するのかがわかります。引数の値はケースの式の中で出てきたローカル変数として、\tref{\expr{var}の式}{expression-var}を使った場合と同じように、利用可能です。

\expr{switch}の使い方について、より高度な情報は後の\tref{パターンマッチング}{lf-pattern-matching}の節でお話します。


\section{匿名の構造体}
\label{types-anonymous-structure}

匿名の構造体は、型を明示せずに利用できるデータの集まりです。以下の例では、\expr{x}と\expr{name}の2つのフィールドを持つ構造体を生成して、それぞれを\expr{12}と\expr{"foo"}の値で初期化しています。

\haxe{assets/Structure.hx}

構文のルールは以下の通りです :

\begin{enumerate}
	\item 構造体は中カッコ \expr{$\left\{\right\}$} で囲う。
	\item \emph{カンマで区切られた} キーと値のペアのリストを持つ。
	\item \tref{識別子}{define-identifier}の条件を満たすカギと、値が\emph{コロン}で区切られる。
	\item\label{valueanytype} 値には、Haxeのあらゆる式が当てはまる。
\end{enumerate}
ルール\ref{valueanytype}は複雑にネストした構造体を含みます。例えば、以下のような。
\todo{please reformat}

\begin{lstlisting}
var user = {
  name : "Nicolas",
  age : 32,
  pos : [
    { x : 0, y : 0 },
    { x : 1, y : -1 }
  ],
};
\end{lstlisting}
構造体のフィールドは、クラスと同じように、\emph{ドット}(\expr{.})を使ってアクセスします。

\begin{lstlisting}
// get value of name, which is "Nicolas"
user.name;
// set value of age to 33
user.age = 33;
\end{lstlisting}
特筆すべきは、匿名の構造体の使用は型システムを崩壊させないことです。コンパイラは実際に利用可能なフィールドにしかアクセスを許しません。つまり、以下のようなコードはコンパイルできません。

\begin{lstlisting}
class Test {
  static public function main() {
    var point = { x: 0.0, y: 12.0 };
    // { y : Float, x : Float } has no field z
    point.z;
  }
}
\end{lstlisting}
このエラーメッセージはコンパイラが\expr{point}の型を知っていることを表します。この\expr{point}の型は、\expr{x}と\expr{y}の\type{Float}型のフィールドを持つ構造体であり、\expr{z}というフィールドは持たないのでアクセスに失敗しました。
この\expr{point}の型は\tref{型推論}{type-system-type-inference}により識別され、そのおかげでローカル変数では型を明示しなくて済みます。ただし、\expr{point}が、クラスやインスタンスのフィールドだった場合、以下のように型の明示が必要になります。

\begin{lstlisting}
class Path {
    var start : { x : Int, y : Int };
    var target : { x : Int, y : Int };
    var current : { x : Int, y : Int };
}
\end{lstlisting}

このような冗長な型の宣言をさけるため、特にもっと複雑な構造体の場合、以下のように\tref{typedef}{type-system-typedef}を使うことをお勧めします。

\begin{lstlisting}
typedef Point = { x : Int, y : Int }

class Path {
    var start : Point;
    var target : Point;
    var current : Point;
}
\end{lstlisting}


\subsection{JSONで構造体を書く}
\label{types-structure-json}

以下のように、\emph{文字列の定数値}をキーに使う\emph{JavaScript Object Notation(JSON)}の構文を構造体に使うこともできます。

\begin{lstlisting}
var point = { "x" : 1, "y" : -5 };
\end{lstlisting}

キーには\emph{文字列の定数値}すべてが使えますが、フィールドが\tref{Haxeの識別子}{define-identifier}として有効である場合のみ型の一部として認識されます。そして、Haxeの構文では識別子として無効なフィールドにはアクセスできないため、\tref{リフレクション}{std-reflection}の\expr{Reflect.field}と\expr{Reflect.setField}を使ってアクセスしなくてはいけません。

\subsection{構造体の型のクラス記法}
\label{types-structure-class-notation}

構造体の型を書く場合に、Haxeでは\Fullref{class-field}を書くときと同じ構文が使用できます。以下の\tref{typedef}{type-system-typedef}では、\type{Int}型の\expr{x}の\expr{y}変数フィールドを持つ\type{Point}型を定義しています。

\begin{lstlisting}
typedef Point = {
    var x : Int;
    var y : Int;
}
\end{lstlisting}

\subsection{オプションのフィールド}
\label{types-structure-optional-fields}

\todo{I don't really know how these work yet.}

\subsection{パフォーマンスへの影響}
\label{types-structure-performance}

構造体をつかって、さらに\tref{構造的部分型付け}{type-system-structural-subtyping}を使った場合、\tref{動的ターゲット}{define-dynamic-target}ではパフォーマンスに影響はありません。しかし、\tref{静的ターゲット}{define-static-target}では、動的な検査が発生するので通常は静的なフィールドアクセスよりも遅くなります。

\section{関数}
\label{types-function}

\todo{It seems a bit convoluted explanations. Should we maybe start by "decoding" the meaning of  Void -> Void, then Int -> Bool -> Float, then maybe have samples using \$type}

関数の型は、\tref{単相}{types-monomorph}と共に、Haxeのユーザーからよく隠れている型の1つです。コンパイル時に式の型を出力させる\expr{\$type}という特殊な識別子を使えば、この型を以下のように浮かび上がらせることが可能です。

\haxe{assets/FunctionType.hx}

初めの\expr{\$type}の出力は、test関数の定義と強い類似性があります。では、その相違点を見てみます。

\begin{itemize}
	\item \emph{関数の引数}は、カンマではなく\expr{->}で区切られる。
	\item \emph{引数の戻り値}の型は、もう一つ\expr{->}を付けた後に書かれる。
\end{itemize}

どちらの表記でも、\expr{test}関数が1つ目の引数として\type{Int}を受け取り、2つ目の引数として\type{String型}を受け取り、\type{Bool型}の値を返すことはよくわかります。2つ目の\expr{\$type}式の\expr{test(1, "foo")}のようにこの関数を呼び出すと、Haxeの型検査は\expr{1}が\type{Int}に代入可能か、\expr{"foo"}が\type{String}に代入可能かをチェックします。そして、その呼び出し後の型は、\expr{test}の戻り値の型の\type{Bool}となります。

もし、ある関数の型が、別の関数の型を引数か戻り値に含む場合、丸かっこをグループ化に使うことができます。例えば、\type{Int -> (Int -> Void) -> Void}は初めの引数の型が\type{Int}、2番目の引数が\type{Int -> Void}で、戻り値が\type{Void}の関数を表します。

\subsection{オプション引数}
\label{types-function-optional-arguments}

オプション引数は、引数の識別子の直前にクエスチョンマーク(\expr{?})を付けることで表現できます。

\haxe[label=assets/OptionalArguments.hx]{assets/OptionalArguments.hx}

\expr{test}関数は、2つのオプション引数を持ちます。\type{Int}型の\expr{i}と\type{String}型の\expr{s}です。これは3行目の関数型の出力に直接反映されています。

この例では、関数を4回呼び出しその結果を出力しています。

\begin{enumerate}
	\item 初めの呼び出しは引数無し。
	\item 2番目の呼び出しは\expr{1}のみの引数。
	\item 3番目の呼び出しは\expr{1}と\expr{"foo"}の2つの引数。
	\item 4番目の呼び出しは\expr{"foo"}のみの引数。
\end{enumerate}

この出力を見ると、オプション引数が呼び出し時に省略されると\expr{null}になることがわかります。つまり、これらの引数は\expr{null}が入る型でなくてはいけないことになり、ここで\tref{null許容}{types-nullability}に関する疑問が浮かび上がります。Haxeのコンパイラは\tref{静的ターゲット}{define-static-target}に出力する場合に、オプションの基本型の引数の型を\type{Null<T>}であると推論することで、オプション引数の型がnull許容であることを保証してます。

初めの3つの呼び出しは直観的なものですが、4つ目の呼び出しには驚くかもしれません。後の引数に代入可能な値が渡されたため、オプション引数はスキップされています。

\subsection{デフォルト値}
\label{types-function-default-values}

Haxeでは、引数のデフォルト値として定数値を割り当てることが可能です。

\haxe{assets/DefaultValues.hx}
この例は、\Fullref{types-function-optional-arguments}のものとよく似ています。違いは、関数の引数の\expr{i}と\expr{s}それぞれに\expr{12}と\expr{"bar"}を代入していることだけです。これにより、引数が省略された場合に\expr{null}ではなく、このデフォルト値が使われるようになります。

%TODO: Default values do not imply nullability, even if the value is \expr{null}. 

Haxeでのデフォルト値は、型の一部では無いので、出力時に呼び出し元で置き換えられるわけではありません(ただし、特有の動作を行う\tref{インライン}{class-field-inline}の関数を除く)。いくつかのターゲットでは、無視された引数に対してやはり\expr{null}を渡して、以下の関数と同じようなコードを生成します。

\begin{lstlisting}
	static function test(i = 12, s = "bar") {
		if (i == null) i = 12;
		if (s == null) s = "bar";
		return "i: " +i + ", s: " +s;
	}
\end{lstlisting}
つまり、パフォーマンスが要求されるコードでは、デフォルト値を使わない書き方をすることが重要だと考えてください。




\section{ダイナミック}
\label{types-dynamic}

Haxeは静的な型システムを持っていますが、この型システムは\type{Dynamic}型を使うことで事実上オフにすることが可能です。\emph{Dynamicな値}は、あらゆるものに割り当て可能です。逆に、\type{Dynamic}に対してはあらゆる値を割り当て可能です。これにはいくつかの弱点があります。

\begin{itemize}
	\item 代入、関数呼び出しなど、特定の型を要求される場面でコンパイラが型チェックをしなくなります。
	\item 特定の最適化が、特に静的ターゲットにコンパイルする場合に、効かなくなります。
	\item よくある間違い(フィールド名のタイポなど)がコンパイル時に検出できなくなって、実行時のエラーが起きやすくなります。
	\item \Fullref{cr-dce}は、\type{Dynamic}を通じて使用しているフィールドを検出できません。
\end{itemize}

\type{Dynamic}が実行時に問題を起こすような例を考えるのはとても簡単です。以下の2行を静的ターゲットへコンパイルすることを考えてください。

\begin{lstlisting}
var d:Dynamic = 1;
d.foo;
\end{lstlisting}

これをコンパイルしたプログラムを、Flash Playerで実行した場合、\texttt{Number にプロパティ foo が見つからず、デフォルト値もありません。}というエラーが発生します。\type{Dynamic}を使わなければ、このエラーはコンパイル時に検出できます。

\trivia{Haxe3より前のDynamicの推論}{Haxe3のコンパイラは型を\type{Dynamic}として推論することはないので、\type{Dynamic}を使いたい場合はそのことを明示しなければ行きません。以前のHaxeのバージョンでは、混ざった型のArrayを\type{Array<Dynamic>}として推論してました(例えば、\expr{[1, true, "foo"]})。私たちはこの挙動はたくさんの型の問題を生み出すことに気づき、この仕様をHaxe3で取り除きました。}

実際のところ\type{Dynamic}は使ってしまいますが、多くの場面では他のもっと良い選択肢があるので\type{Dynamic}の使用は最低限にすべきです。例えば、Haxeの\Fullref{std-reflection}APIは、コンパイル時には構造のわからないカスタムのデータ構造をあつかう際に最も良い選択肢になりえます。

\type{Dynamic}は、\tref{単相(monomorph)}{types-monomorph}を\tref{単一化}{type-system-unification}する場合に、特殊な挙動をします。以下のような場合に、とんでもない結果を生んでしまうので、単相が\type{Dynamic}に拘束されることはありません。

\haxe{assets/DynamicInferenceIssue.hx}

\expr{Json.parse}の戻り値は\type{Dynamic}ですが、ローカル変数のjsonの型は\type{Dynamic}に拘束されません。単相のままです。そして、\expr{json.length}のフィールドにアクセスした時に\tref{匿名の構造体}{types-anonymous-structure}として推論されて、それにより\expr{json[0]}の配列アクセスでエラーになっています。これは、\expr{json}に対して、\expr{var json:Dynamic}というように明示的に\type{Dynamic}の型付けをすることで避けることができます。

\trivia{標準ライブラリでのDynamic}{DynamicはHaxe3より前の標準ライブラリではかなり頻繁に表れていましたが、Haxe3までの継続的な型システムの改善によってDynamicの出現頻度を減らすことができました。}

\subsection{型パラメータ付きのダイナミック}
\label{types-dynamic-with-type-parameter}

\type{Dynamic}は、\tref{型パラメータ}{type-system-type-parameters}を付けても付けなくても良いという点でも特殊な型です。型パラメータを付けた場合、\Fullref{types-dynamic}のすべてのフィールドがパラメータの型であることが強制されます。

\begin{lstlisting}
var att : Dynamic<String> = xml.attributes;
// valid, value is a String
att.name = "Nicolas";
// dito (this documentation is quite old)
att.age = "26";
// error, value is not a String
att.income = 0;
\end{lstlisting}


\subsection{ダイナミックを実装(implements)する}
\label{types-dynamic-implemented}

クラスは\type{Dynamic}と\type{Dynamic$<$T$>$}を\tref{実装}{types-interfaces}することができます。
これにより任意のフィールドへのアクセスが可能になります。\type{Dynamic}の場合、フィールドはあらゆる型になる可能性があり、\type{Dynamic$<$T$>$}の場合、フィールドはパラメータの型と矛盾しない型のみに強制されます。

\haxe{assets/ImplementsDynamic.hx}

\type{Dynamic}を実装しても、 他のインターフェースが要求する実装を満たすことにはなりません。明示的な実装が必要です。

型パラメータなしの\type{Dynamic}を実装したクラスでは、特殊なメソッド\expr{resolve}を利用することができます。\tref{読み込みアクセス}{define-read-access}がありフィールドが存在しなかった場合、\expr{resolve}メソッドが以下のように呼び出されます。

\haxe{assets/DynamicResolve.hx}



\section{抽象型(abstract)}
\label{types-abstract}

抽象(abstract)型は、実行時には別の型になる型です。抽象型は挙動を編集したり強化したりするために、具体型(=抽象型でない型)を``おおう''型を定義するコンパイル時の機能です。

\haxe[firstline=1,lastline=5]{assets/MyAbstract.hx}

上記のコードからは以下を学ぶことができます。

\begin{itemize}
	\item \expr{abstract}キーワードは、抽象型を定義することを宣言している。
	\item \type{AbstractInt}は抽象型の名前であり、型の識別子のルールを満たすものなら何でも使える。
	\item 丸かっこ\expr{()}の中は、その\emph{基底型}の\type{Int}である。
	\item 中カッコ\expr{$\left\{\right\}$}の中はフィールドで、
	\item \type{Int}型の\expr{i}のみを引数とするコンストラクタの\expr{new}関数がある。
\end{itemize}

\define{基底型}{define-underlying-type}{
抽象型の基底型は、実行時にその抽象型を表すために使われる型です。基底型はたいていの場合は具体型ですが、別の抽象型である場合もあります。
}

構文はクラスを連想させるもので、意味合いもよく似ています。実際、抽象型のボディ部分(中カッコの開始以降)は、クラスフィールドとして構文解析することが可能です。抽象型は\tref{メソッド}{class-field-method}と、\tref{実体}{define-physical-field}の無い\tref{プロパティ}{class-field-property}フィールドを持つことが可能です。

さらに、抽象型は以下のように、クラスと同じようにインスタンス化して使用することができます

\haxe[firstline=7,lastline=12]{assets/MyAbstract.hx}

はじめに書いたとおり、抽象型はコンパイル時の機能ですから、見るべきは上記のコードの実際の出力です。この出力例としては、簡潔なコードが出力される\target{JavaScript}が良いでしょう。上記のコードを\texttt{haxe -main MyAbstract -js myabstract.js}でコンパイルすると以下のような\target{JavaScript}が出力されます。

\begin{lstlisting}
var a = 12;
console.log(a);
\end{lstlisting}

抽象型の\type{AbstractInt}は出力から完全に消えてしまい、その基底型の\type{Int}の値のみが残っています。これは、\type{AbstractInt}のコンストラクタがインライン化されて、そのインラインの式が値を\expr{this}に代入します(インライン化については後の\Fullref{class-field-inline}で学びます)。これは、クラスのように考えていた場合、驚くべきことかもしれません。しかし、これこそが抽象型を使って表現したいことそのものです。
抽象型のすべての\emph{インラインのメンバメソッド}では\expr{this}への代入が可能で、これにより``内部の値''が編集できます。

``もしメンバ関数でinlineが宣言されていなかった場合、何が起こるのか?''というのは良い疑問です。そのようなコードははっきりと成立します。その場合、Haxeは実装クラスと呼ばれるprivateのクラスを生成します。この実装クラスは抽象型のメンバ関数を、最初の引数としてその基底型の\expr{this}を加えた静的な(static)関数で持ちます。さらに実装の詳細の話をすると、この実装クラスは\tref{選択的関数}{types-abstract-selective-functions}でも使われます。

\trivia{基本型と抽象型}{抽象型が生まれる前には、基本型はexternクラスと列挙型で実装されていました。\type{Int}型を\type{Float}型の``子クラス''としてあつかうなどのいくつかの面では便利でしたが、一方で問題も引き起こしました。例えば、\type{Float}がexternクラスなので、実際のオブジェクトしか受け入れないはずの空の構造体の型\expr{\{\}}として単一化できました。}



\subsection{暗黙のキャスト}
\label{types-abstract-implicit-casts}

クラスとは異なり抽象型は暗黙のキャストを許します。抽象型には2種類の暗黙のキャストがあります。

\begin{description}
	\item[直接:] 他の型から抽象型への直接のキャストを許します。これは\expr{to}と\expr{from}のルールを抽象型に設定することでできます。 これは、その抽象型の基底型に単一化可能な型のみで利用可能です。
	\item[クラスフィールド:] 特殊なキャスト関数を呼び出すことによるキャストを許します。この関数は\expr{@:to}と\expr{@:from}のメタデータを使って定義されます。この種類のキャストは全ての型で利用可能です。
\end{description}

下のコードは、直接キャストの例です。

\haxe{assets/ImplicitCastDirect.hx}

\expr{from Int}かつ\expr{to Int}の\type{MyAbstract}を定義しました。これは\type{Int}を代入することが可能で、かつ\type{Int}に代入することが可能だという意味です。このことは、9、10行目に表れています。まず、\type{Int}の12を\type{MyAbstract}型の変数\expr{a}に代入しています(これは\expr{from Int}の宣言により可能になります)。そして次に、\type{Int}型の変数\expr{b}に、抽象型のインスタンスを代入しています(これは\expr{to Int}の宣言により可能になります)。

クラスフィールドのキャストも同じ意味を持ちますが、定義の仕方はまったく異なります。

\haxe{assets/ImplicitCastField.hx}

静的な関数に\expr{@:from}を付けることで、その引数の型からその抽象型への暗黙のキャストを行う関数として判断されます。この関数はその抽象型の値を返す必要があります。\expr{static}を宣言する必要もあります。

同じように関数に\expr{@:to}を付けることで、その抽象型からその戻り値の型への暗黙のキャストを行う関数として判断されます。この関数は普通はメンバ関数ですが、\expr{static}でも構いません。そして、これは\tref{選択的関数}{types-abstract-selective-functions}として働きます。

上の例では、\expr{fromString}メソッドが\expr{"3"}の値を\type{MyAbstract}型の変数\expr{a}への代入を可能にし、
\expr{toArray}メソッドがその抽象型インスタンスを\type{Array<Int>}型の変数\expr{b}への代入を可能にします。

この種類のキャストを使った場合、必要な場所でキャスト関数の呼び出しが発生します。このことは\target{JavaScript}出力を見ると明らかです。

\begin{lstlisting}
var a = _ImplicitCastField.MyAbstract_Impl_.fromString("3");
var b = _ImplicitCastField.MyAbstract_Impl_.toArray(a);
\end{lstlisting}

これは2つのキャスト関数で\tref{インライン化}{class-field-inline}を行うことでさらなる最適化を行うことができます。これにより出力は以下のように変わります。

\begin{lstlisting}
var a = Std.parseInt("3");
var b = [a];
\end{lstlisting}


型\expr{A}から時の型\expr{B}への代入の時にどちらかまたは両方が抽象型である場合に使われるキャストの\emph{選択アルゴリズム}は簡単です。

\begin{enumerate}
	\item \expr{A}が抽象型でない場合は3へ。
	\item \expr{A}が、\expr{B}\emph{への}変換を持っている場合、これを適用して6へ。
	\item \expr{B}が抽象型でない場合は5へ。
	\item \expr{B}が、\expr{A}\emph{からの}変換を持っている場合、これを適用して6へ。
	\item 単一化失敗で、終了。
	\item 単一化成功で、終了。
\end{enumerate}

\begin{flowchart}{types-abstract-implicit-casts-selection-algorithm}{選択アルゴリズムのフローチャート}

\tikzset {
	level distance = 1.8cm
}

\tikzstyle{edgeBelow} = [ auto = left, outer sep = 0.2cm ]

\Tree
[.\node [decisionc] (dec1) {\expr{A}は抽象型};
\edge [edgeBelow] node {はい};
[.\node [decisionc] (dec2) {\expr{A}に\expr{B}への\expr{to}変換がある};
\edge [edgeBelow] node {いいえ};
[.\node [decisionc] (dec3) {\expr{B}は抽象型};
\edge [edgeBelow] node {はい};
[.\node [decisionc] (dec4) {\expr{B}に\expr{A}からの\expr{from}変換がある};
\edge [edgeBelow] node {いいえ};
[.\node [startstop, fill = red!70] (fail) {単一化失敗};
]]]]]


\node [startstop, fill = green!70, xshift = 3cm] (success) at (fail.east) {単一化成功};

\tikzstyle{altNode} = [above right, at start]
\tikzstyle{skipNode} = [above left, at start]
\tikzstyle{skipArrow} = [out = 195, in = 165, looseness = 1.6]
\coordinate (altAnchor) at (success.north);

\draw [flowchartArrow] (dec1.west) [skipArrow] to node [skipNode] {いいえ} (dec3.north west);
\draw [flowchartArrow] (dec3.south west) [skipArrow] to node [skipNode] {いいえ} (fail.west);
\draw [flowchartArrow] (dec2) -| (altAnchor) node [altNode] {はい};
\draw [flowchartArrow] (dec4) -| (altAnchor) node [altNode] {はい};

\end{flowchart}

意図的に暗黙のキャストは連鎖的ではありません。これは以下の例でわかります。

\haxe{assets/ImplicitTransitiveCast.hx}

\type{A}から\type{B}、\type{B}から\type{C}への個々のキャストは可能ですが、\type{A}から\type{C}への連鎖的なキャストはできません。これは、キャスト方法が複数生まれてしまうことは避けて、選択アルゴリズムの簡潔さを保つためです。




\subsection{演算子オーバーロード}
\label{types-abstract-operator-overloading}

抽象型ではクラスフィールドに\expr{@:op}メタデータを付けることで、単項演算子と2項演算子のオーバーロードが可能です。

\haxe{assets/AbstractOperatorOverload.hx}

\expr{@:op(A * B)}を宣言することで、\expr{repeat}関数は、左辺が\type{MyAbstract}で右辺が\type{Int}の場合の\expr{*}演算子による乗算の関数として利用されます。これは18行目で利用されています。この部分は\target{JavaScript}にコンパイルすると以下のようになります。

\begin{lstlisting}
console.log(_AbstractOperatorOverload.
  MyAbstract_Impl_.repeat(a,3));
\end{lstlisting}


\tref{クラスフィールドによる暗黙の型変換}{types-abstract-implicit-casts}と同様に、オーバーロードメソッドも要求された場所で呼び出しが発生します。上記の例の\expr{repeat}関数は可換ではありません。\expr{MyAbstract * Int}は動作しますが、\expr{Int * MyAbstract}では動作しません。\expr{Int * MyAbstract}でも動作させたい場合は\expr{@:commutative}のメタデータが使えます。逆に、\expr{MyAbstract * Int}ではなく\expr{Int * MyAbstract}でのみ動作させてたい場合、1つ目の引数で\type{Int}型、2つ目の引数で\type{MyAbstract}型を受け取る静的な関数をオーバーロードメソッドにすることができます。

単項演算子の場合もこれによく似ています。

\haxe{assets/AbstractUnopOverload.hx}

2項演算子と単項演算子の両方とも、戻り値の型は何でも構いません。

\paragraph{基底型の演算を公開する}

基底型が抽象型でそこで許容されている演算子でかつ戻り値を元の抽象型に代入可能なものについては、\expr{@:op}関数のボディを省略することが可能です。

\haxe{assets/AbstractExposeTypeOperations.hx}

\todo{please review for correctness}

\subsection{配列アクセス}
\label{types-abstract-array-access}

配列アクセスは、配列の特定の位置の値にアクセスするのに伝統的に使われている特殊な構文です。これは大抵の場合、\type{Int}のみを引数としますが、抽象型の場合はカスタムの配列アクセスを定義することが可能です。\tref{Haxeの標準ライブラリ}{std}では、これを\type{Map}型に使っており、これには以下の2つのメソッドがあります。
\todo{You have marked ``Map'' for some reason}

\begin{lstlisting}
@:arrayAccess
public inline function get(key:K) {
  return this.get(key);
}
@:arrayAccess
public inline function arrayWrite(k:K, v:V):V {
	this.set(k, v);
	return v;
}
\end{lstlisting}

配列アクセスのメソッドは以下の2種類があります。

\begin{itemize}
	\item \expr{@:arrayAccess}メソッドが1つの引数を受け取る場合、それは読み取り用です。
	\item \expr{@:arrayAccess}メソッドが2つの引数を受け取る場合、それは書き込み用です。
\end{itemize}

上記のコードの\expr{get}メソッドと\expr{arrayWrite}メソッドは、以下のように使われます。

\haxe{assets/AbstractArrayAccess.hx}

ここでは以下のように出力に配列アクセスのフィールドの呼び出しが入ることになりますが、驚かないでください。

\begin{lstlisting}
map.set("foo",1);
console.log(map.get("foo")); // 1
\end{lstlisting}

\paragraph{配列アクセスの解決順序}
\label{types-abstract-array-access-order}

Haxe3.2以前では、バグのため\expr{:arrayAccess}のフィールドがチェックされる順序は定義されていませんでした。これは、Haxe 3.2では修正されて一貫して上から下へと確認が行われるようになりました。

\haxe{assets/AbstractArrayAccessOrder.hx}

\expr{a[0]}の配列アクセスは、\expr{getInt1}のフィールドとして解決されて、小文字の\expr{f}が返ります。バージョン3.2以前のHaxeでは、結果が異なる場合があります。

\tref{暗黙のキャスト}{types-abstract-implicit-casts}が必要な場合であっても、先に定義されている方が優先されます。

\subsection{選択的関数}
\label{types-abstract-selective-functions}

コンパイラは抽象型のメンバ関数を静的な(static)関数へと変化させるので、手で静的な関数を記述してそれを抽象型のインスタンスで使うことができます。この意味は、関数の最初の引数の型で、その関数が使えるようになる\tref{静的拡張}{lf-static-extension}に似ています。

\haxe{assets/SelectiveFunction.hx}

抽象型の\type{MyAbstract}の\expr{getString}のメソッドは、最初の引数として\type{MyAbstract$<$String$>$}を受け取ります。これにより、14行目の変数\expr{a}の関数呼び出しが可能になります(\expr{a}の型が\type{MyAbstract$<$String$>$}なので)。しかし、\type{MyAbstract$<$Int$>$}の変数\expr{b}では使えません。

\trivia{偶然の機能}{
実際のところ選択的関数は意図して作られたというよりも、発見された機能です。この機能について初めて言及されてから実際に動作せせるまでに必要だったのは軽微な修正のみでした。この発見が、Mapのような複数の型の抽象型にもつながっています。}


\subsection{抽象型列挙体}
\label{types-abstract-enum}
\since{3.1.0}

抽象型の宣言に\expr{@:enum}のメタデータを追加することで、その値を有限の値のセットを定義して使うことができます。

\haxe{assets/AbstractEnum.hx}

以下のJavaScriptへの出力を見ても明らかなように、Haxeコンパイラは抽象型\type{HttpStatus}の全てのフィールドへのアクセスをその値に変換します。

\begin{lstlisting}
Main.main = function() {
	var status = 404;
	var msg = Main.printStatus(status);
};
Main.printStatus = function(status) {
	switch(status) {
	case 404:
		return "Not found";
	case 405:
		return "Method not allowed";
	}
};
\end{lstlisting}

これは\tref{インライン変数}{class-field-inline}によく似ていますが、いくつかの利点があります。

\begin{itemize}
	\item コンパイラがそのセットのすべての値が正しく型付けされていることを保証できます。
	\item パターンマッチで、抽象型列挙体への\tref{マッチング}{lf-pattern-matching}を行う場合に\tref{網羅性}{lf-pattern-matching-exhaustiveness}がチェックされます。
	\item 少ない構文でフィールドを定義できます。
\end{itemize}


\subsection{抽象型フィールドの繰り上げ}
\label{types-abstract-forward}
\since{3.1.0}

基底型をラップした場合、その機能性のを``保ちたい''場合があります。繰り上がりの関数を手で書くのは面倒なので、Haxeでは\expr{@:forward}メタデータを利用できるようにしています。

\haxe{assets/AbstractExpose.hx}

この例では、抽象型の\type{MyArray}が\type{Array}をラップしています。この\expr{@:forward}メタデータは、基底型から繰り上がらせるフィールド2つを引数として与えられています。上記の例の\expr{main}関数は、\type{MyArray}をインスタンス化して、その\expr{push}と\expr{pop}のメソッドにアクセスしています。コメント化されている行は、\expr{length}フィールドは利用できないことを実演するものです。

ではどのようなコードが出力されるのか、いつものように\target{JavaScript}への出力を見てみましょう。

\begin{lstlisting}
Main.main = function() {
	var myArray = [];
	myArray.push(12);
	myArray.pop();
};
\end{lstlisting}

全てのフィールドを繰り上げる場合は、引数なしの\expr{@:forward}を利用できます。もちろんこの場合でも、Haxeコンパイラは基底型にそのフィールドが存在していることを保証します。

\trivia{マクロとして実装}{\expr{@:enum}と\expr{@:forward}の両機能は、もともとは\tref{ビルドマクロ}{macro-type-building}を利用して実装していました。この実装はマクロなしのコードから使う場合はうまく動作していましたが、マクロからこれらの機能を使った場合に問題を起こしました。このため、これらの機能はコンパイラへと移されました。}


\subsection{コアタイプの抽象型}
\label{types-abstract-core-type}

Haxeの標準ライブラリは、基本型のセットをコアタイプの抽象型として定義しています。これらは\expr{@:coreType}メタデータを付けることで識別されて、基底型の定義を欠きます。これらの抽象型もまた異なる型の表現として考えることができます。
そして、その型はHaxeのターゲットのネイティブの型です。

カスタムのコアタイプの抽象型の導入は、Haxeのターゲットにその意味を理解させる必要があり、ほとんどのユーザーのコードで必要ないでしょう。ですが、マクロを使いたい人や、新しいHaxeのターゲットを作りたい人にとっては興味深い利用例があります

コアタイプの抽象型は、不透過の抽象型(他の型をラップする抽象型のこと)とは異なる以下の性質をもちます。

\begin{itemize}
	\item 基底型を持たない。
	\item \expr{@:notNull}メタデータの注釈を付けない限り、null許容としてあつかわれる。
	\item 式の無い\tref{配列アクセス}{types-abstract-array-access}関数を定義できる。
	\item Haxeの制限から離れた、式を持たない\tref{演算子オーバーロードのフィールド}{types-abstract-operator-overloading}が可能。
\end{itemize}



\section{単相(モノモーフ)}
\label{types-monomorph}

単相は、\tref{単一化}{type-system-unification}の過程で、他の異なる型へと形を変える型です。これについて詳しくは\tref{型推論}{type-system-type-inference}の節で話します。
