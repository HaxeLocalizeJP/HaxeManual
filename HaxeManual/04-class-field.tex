\chapter{クラスフィールド}
\label{class-field}

\define{クラスフィールド}{define-class-field}{クラスフィールドはクラスに属する変数、プロパティまたはメソッドです。これは静的、または非静的になることができます。\emph{静的メソッド}と\emph{メンバ変数}といった名前を使うのと同じように、非静的フィールドについては\emph{メンバ}フィールドと呼びます。}

ここまで、一般的なHaxeのプログラムと型がどのように構成されているのかを見てきました。
このクラスフィールドに関する章では、構成に関する話題をまとめて、Haxeの動作に関する話題へのかけ橋とします。これはクラスフィールドが\tref{式}{expression}を持つ場所だからです。

クラスフィールドには3種類あります。

\begin{description}
	\item[変数:] \tref{変数}{class-field-variable}クラスフィールドにはある型の値が入っていて、参照、または代入することができます。
	\item[プロパティ:] \tref{プロパティ}{class-field-property}クラスフィールドはアクセスされた時のカスタムの動作を定義します。クラスの外からは変数フィールドのように見えます。
	\item[メソッド:] \tref{メソッド}{class-field-method}はコードを実行するために呼び出すことのできる関数です。
\end{description}

厳密に言うと、変数は特定のアクセス方法を持つプロパティであると見なせます。実際に、Haxeコンパイラは変数とプロパティを型付けの段階では区別していません。しかし、構文レベルでは区別されます。

用語について補足すると、メソッドは(静的また非静的の)クラスに属する関数であり、式の中で現れる\tref{ローカル関数}{expression-function}のようなその他の関数はメソッドではありません。

\section{変数}
\label{class-field-variable}

変数フィールドについては、すでに前章でいくつかのサンプルコードで見てきました。変数フィールドは値を保持するもので、その性質はほとんどプロパティと共通しています(すべてでは無い)。

\haxe{assets/VariableField.hx}

ここから変数が以下のようなものだとわかります。

\begin{enumerate}
	\item 名前を持つ(ここでは\expr{member}),
	\item 型を持つ(ここでは\type{String}),
	\item 一定の初期値を持つ場合がある(ここでは\expr{"bar"}) and
	\item \tref{アクセス修飾子}{class-field-access-modifier}を持つ場合がある(ここでは\expr{static})
\end{enumerate}

上の例は最初に\expr{member}の初期値を出力した後、\expr{"foo"}を割り当ててから新しい値を出力しています。アクセス修飾子の効果は3種類のクラスフィールドで共通しており、その内容については後の節で説明します。

変数フィールドが初期値をもつ場合には、型の明示は不要になります。この場合、コンパイラが\tref{推論}{type-system-type-inference}を行います。

\begin{flowchart}{class-field-variable-init-values}{変数フィールドの値の初期化}
\Tree[.\node [decision] {\expr{inline}};
	\edge node[auto=right] {はい};
	[.\node [decision] {\expr{static}};
		\edge node[auto=right] {いいえ};
		\node [startstop, valueNone] {不正};
		\edge node[auto=left] {はい};
		\node [startstop, valueSome] {定数のみ};
	]
	\edge node[auto=left] {いいえ};
	[.\node [decision] {\expr{extern}};
		\edge node[auto=right] {いいえ};
		[.\node [decision] {\expr{static}};
			\edge node[auto=right] {いいえ};
			\node [startstop, valueSome] {'this'を使えない};
			\edge node[auto=left] {はい};
			\node [startstop, valueAll] {何でも};
		]
		\edge node[auto=left] {はい};
		\node [startstop, valueNone] {なし};
	]
]
\end{flowchart}


\section{プロパティ}
\label{class-field-property}

\tref{変数}{class-field-variable}に続き、プロパティがクラスにデータ持つ2番目の方法になります。変数とは異なり、プロパティはどのようなアクセスが許可されるかと、どのように生成されるかのより細かい制御が要求されます。よくある使い方は、例えば以下のようなものです。

\begin{itemize}
	\item どこからでも読み込み可能だが、書き込みは定義しているクラスからのみのフィールドを作る
	\item 読み込みアクセスがされたときに\emph{ゲッター}メソッドが実行されるフィールドを作る。
	\item 書き込みアクセスがされたときに\emph{セッター}メソッドが実行されるフィールドを作る。
\end{itemize}

プロパティをあつかう場合、2種類のアクセスについて理解することが重要です。

\define{読み込みアクセス}{define-read-access}{読み込みアクセスは右辺側で\tref{フィールドアクセス式}{expression-field-access}が使われると発生します。これには\expr{obj.field()}の形の関数呼び出しもふくまれるため、この\expr{field}も読み込みアクセスがされます。}

\define{書き込みアクセス}{define-write-access}{フィールドへの書き込みアクセスは、\tref{フィールドアクセス式}{expression-field-access}に\expr{obj.field = value}の形式で値の代入することで発生します。また、\expr{obj.field += value}の式\expr{+=}のような特殊な代入演算子を使うと、書き込みアクセスと\tref{読み込みアクセス}{define-read-access}の両方が発生します。} 

読み込みアクセスと書き込みアクセスを以下の構文を使って直接指定します。

\haxe{assets/Property.hx}

大部分は変数の構文と同じで、同じルールが適用されます。プロパティは以下の点で異なります。

\begin{itemize}
	\item フィールド名の後から小かっこが始まります(\expr{(})。
	\item 次に、特殊な\emph{アクセス識別子}が来ます(ここでは\expr{default})。
	\item カンマ(\expr{,})で区切ります。
	\item もう一つ特殊なアクセス識別子が続きます(ここでは\expr{null})。
	\item 小かっこを閉じます(\expr{)})。
\end{itemize}

1つ目のアクセス識別子はフィールドの読み込み、2つ目は書き込み時の挙動を決定します。アクセス識別子には以下の値が使用できます。

\begin{description}
	\item[\expr{default}:] フィールドの可視性が\expr{public}の場合、通常のフィールドと同じです。その他の場合、\expr{null}アクセスと同じです。
	\item[\expr{null}:] 定義したクラスのみからアクセスできます。
	\item[\expr{get}/\expr{set}:] アクセス時に\emph{アクセサメソッド}を呼び出します。コンパイラが使用可能なアクセサの存在を確認します。
	\item[\expr{dynamic}:] \expr{get}/\expr{set}アクセスに似ていますが、アクセサフィールドの存在を確認しません。
	\item[\expr{never}:] いかなるアクセスも許可しません。
\end{description}

\define{アクセサメソッド}{define-accessor-method}{型が\type{T}でフィールド名が\expr{field}のフィールドに対する\emph{アクセサメソッド}は、\type{Void->T}型のフィールド名\expr{get_field}の\emph{ゲッター}または\type{T->T}型のフィールド名\expr{set_field}の\emph{セッター}です。アクセサメソッドは略して\emph{アクセサ}とも呼びます。}

\trivia{アクセサ名}{Haxe 2では、アクセス識別子に自由な識別子を使うことが可能で、その場合はそれがカスタムのアクセサメソッド名となっていました。しかし、これにより実装は変則的なものになっていました。例えば、\expr{Reflect.getProperty()}と\expr{Reflect.setProperty()}はどのような名前が名前が使われていたとしても対応する必要がありました。そのため、ターゲット出力時に参照のためのメタ情報を生成する必要がありました。\\
この識別子の名前を\expr{get_}、\expr{set_}から始まるもののみに制限することで、実装を大きく簡略化することに成功しました。これがHaxe 2と3の間の破壊的な変更の1つです。}

\subsection{よくあるアクセス識別子の組み合わせ}
\label{class-field-property-common-combinations}

次の例はプロパティのよくあるアクセス識別子の組み合わせです。

\haxe{assets/Property2.hx}

\expr{main}メソッドの\target{JavaScript}へのコンパイル結果は、フィールドアクセスがどのようなものなのか理解する助けになるでしょう。

\begin{lstlisting}
var Main = function() {
	var v = this.get_x();
	this.set_x(2);
	var _g = this;
	_g.set_x(_g.get_x() + 1);
};
\end{lstlisting}

このとおり、読み込みアクセスは\expr{get_x()}の呼び出しとなり、書き込みアクセスは\expr{x}への\expr{2}の代入が\expr{set_x(2)}の呼び出しになりました。\expr{+=}の場合の出力は最初は少し不思議に見えるかもしれませんが、次の例で簡単にわかるはずです。

\haxe{assets/Property3.hx}

\expr{main}メソッドの\expr{x}のフィールドアクセスについて、ここで起きる事象は複雑です。まずこの場合は、\type{Main}のインスタンス化という副作用があります。そのため、コンパイラは\expr{new Main().x = new Main().x + 1}という出力を行わないように、複雑な式をローカル変数にキャッシュします。

\begin{lstlisting}
Main.main = function() {
	var _g = new Main();
	_g.set_x(_g.get_x() + 1);
}
\end{lstlisting}

\subsection{型システムへの影響}
\label{class-field-property-type-system-impact}

プロパティの存在は型システム対して、いくつかの重要な影響をもたらします。もっとも重要なのはプロパティはコンパイル時の機能であり、\emph{型がわかっている}必要があるということです。クラスインスタンスを\type{Dynamic}に代入すると、フィールドアクセスはアクセサメソッドを参照\emph{しません}。同じようにアクセス制限も働かなくなり、すべてのアクセスは\expr{public}と同じになります。

\expr{get}または\expr{set}のアクセス識別子を使うと、コンパイラはゲッターとセッターが本当に存在するかを確認します。以下はコンパイルできません。

\haxe{assets/Property4.hx}

\expr{get_x}メソッドを忘れていますが、親クラスでそれが定義されていた場合は今のクラスでそれを定義する必要はなくなります。

\haxe{assets/Property5.hx}

\expr{dynamic}アクセス識別子は\expr{get}や\expr{set}と同じように動作しますが、この存在チェックは行われません。

\subsection{ゲッターとセッターのルール}
\label{class-field-property-rules}

アクセサメソッドの可視性は、プロパティの可視性に影響を与えません。つまり、プロパティが\expr{public}であってもそのゲッターは\expr{private}でも構わないということです。

ゲッターとセッターは、その物理的フィールドにアクセスしてデータを使用する場合があります。アクセサメソッド自身からそのフィールドへのアクセスが行われた場合、コンパイラはこれをアクセサメソッド経由しないアクセスと見なします。これにより無限ループが回避されます。

\haxe{assets/GetterSetter.hx}

しかし、フィールドが少なくとも1つ、\expr{default}または\expr{null}のアクセス識別子を持つ時のみ、コンパイラはその物理的フィールドが存在していると考えます。

\define{物理的フィールド}{define-physical-field}{以下のいずれかの場合にフィールドが\emph{物理的}であると考えられます
	\begin{itemize}
		\item \tref{変数}{class-field-variable}
		\item 読み込みアクセスか書き込みアクセスのアクセス識別子が\expr{default}または\expr{null}である\tref{プロパティ}{class-field-property}
		\item \expr{:isVar}\tref{メタデータ}{lf-metadata}がつけられた\tref{プロパティ}{class-field-property}
	\end{itemize}
}

これらのケースに含まれない場合、アクセサメソッド内での自身のフィールドへのアクセスはコンパイルエラーを起こします。

\haxe{assets/GetterSetter2.hx}

物理的フィールドが必要であれば、\expr{:isVar}\tref{メタデータ}{lf-metadata}をフィールドつけることでそれを強制できます。

\haxe{assets/GetterSetter3.hx}

\trivia{プロパティのセッターの型}{新しいHaxeのユーザーにとって、セッターの型が\type{T->Void}ではなくて\type{T->T}でなくてはいけないというのはなじみがなく、驚かれるかもしれません。ではなぜ\emph{setter}が値を返す必要があるのでしょうか?\\
それはセッターを使ったフィールドへの代入を右辺の式として使いたいからです。\expr{x = y = 1}のような連結された式は、\expr{x = (y = 1)}として評価されます。\expr{x}に\expr{y = 1}の結果を代入するためには、\expr{y = 1}が値を持たなければなりません。\expr{y}のセッターの戻り値が\type{Void}であれば、それは不可能です。}

\section{メソッド}
\label{class-field-method}

\tref{変数}{class-field-variable}がデータを保持する一方で、メソッドは\tref{式}{expression}をもってプログラムの動作を定義します。このマニュアルのさまざまなサンプルコード中で、メソッドフィールドを見てきました。最初の\tref{Hello World}{introduction-hello-world}の例ですら、\expr{main}メソッドとして現れています。

\haxe{assets/HelloWorld.hx}

メソッドは\expr{function}キーワードから始まることで識別されます。そして以下の要素を持ちます。

\begin{enumerate}
	\item 名前を持つ(ここでは\expr{main})。
	\item 引数のリストを持つ(ここでは空の\expr{()})。
	\item 戻り値を持つ(ここでは\type{Void})。
	\item \tref{アクセス修飾子}{class-field-access-modifier}を持つ場合がある(ここでは\expr{static}と\expr{public})
	\item 式を持つ場合がある(ここでは\expr{\{trace("Hello World");\}})。
\end{enumerate}

引数と戻り値の型について学ぶために次の例を見てみましょう。

\haxe{assets/MethodField.hx}

引数はフィールド名の後に、小かっこ(\expr{(})を続け、引数の詳細のリストをカンマ(\expr{,})区切りで並べて、小かっこを閉じる(\expr{)})ことで記述します。引数の詳細についての情報は\Fullref{types-function}で説明されています。

この例からは\tref{型推論}{type-system-type-inference}が引数と戻り値についてどのように動作するのかもわかります。\expr{myFunc}は2つの引数を持ちますが、最初の引数の\expr{f}のみで\type{String}の型が明示されていて、2つ目の引数の\expr{i}には型注釈がありません。コンパイラがこのメソッドの呼び出しから推論を行うように残してあります。同じように、メソッドの戻り値の型も\expr{return true}から推論されて\type{Bool}になります。

\subsection{メソッドのオーバーライド(override)}
\label{class-field-overriding}

フィールドのオーバーライドは、クラスの階層構造を作る助けになります。オーバーライドはさまざまなデザインパターンで活用されますが、ここでは基本的な機能のみを説明します。クラスでオーバーライドを使うためには、\tref{親クラス}{types-class-inheritance}を持つ必要があります。次の例を見てみましょう。

\haxe{assets/Override.hx}

ここで重要なのは以下の要素です。

\begin{itemize}
	\item \type{Base}クラスは\expr{myMethod}メソッドとコンストラクタを持つ。
	\item \type{Child}は\type{Base}を継承しており、\expr{override}を宣言した\expr{myMethod}を持つ。
	\item \type{Main}クラスは\expr{main}メソッドで\expr{Child}をインスタンス化して、\type{Base}型を明示した\expr{child}変数に代入して、その\expr{myMethod()}を呼び出している。
\end{itemize}

\expr{child}変数の\type{Base}型を明示することで、コンパイル時には\type{Base}型であっても、実行時には\type{Child}型の\expr{myMethod}メソッドが実行されるという重要なことを強調しました。これはフィールドのアクセスが実行時に動的に解決されるからです。

\type{Child}クラスでは\expr{super.methodName()}を呼び出して、オーバーライドされたメソッドにアクセスすることができます。

\haxe{assets/OverrideCallParent.hx}

\expr{new}コンストラクタ内での\expr{super()}の使用については、\Fullref{types-class-inheritance}の節で説明してあります。

\subsection{変性とアクセス修飾子の影響}
\label{class-field-override-effects}

オーバーライドは\tref{変性}{type-system-variance}のルールと深い関わりがあります。というのは、引数の型が\emph{反変性} (より一般的な型)を許容し、戻り値の型は\emph{共変性}(より具体的な型)を許容するからです。

\haxe{assets/OverrideVariance.hx}

直観的には、この挙動は引数は関数へ「書き込み」であり戻り値は「読み込み」であるという事実から来ています。

この例から、\tref{可視性}{class-field-visibility}が変更できるということもわかります。オーバーライドされる側のフィールドが\expr{private}の場合は、\expr{public}のフィールドでオーバーライドすることができます。ただし、そのほかの場合は、可視性の変更はできません。

\tref{\expr{inline}}{class-field-inline}の宣言をされたフィールドもオーバーライドできません。これはインライン化がコンパイル時に関数の中身で書き換えを行う一方で、オーバーライドのフィールドは実行時に解決される、という衝突を避けるためです。	
	
\section{アクセス修飾子}
\label{class-field-access-modifier}
\state{NoContent}

\subsection{可視性}
\label{class-field-visibility}

フィールドの可視性はデフォルトでは\emph{private}です。つまり、そのクラス自身とその子クラスからしかアクセスできません。\expr{public}アクセス修飾子を使うことでどこからでもアクセスができるようにフィールドを公開できます。

\haxe{assets/Visibility.hx}

\type{Main}から\type{MyClass}の\expr{available}フィールドへのアクセスは、フィールドが\expr{public}なので許可されます。しかし\expr{unavailable}については、\type{MyClass}からのアクセスは許可されますが\type{Main}からは許可されません。これはフィールドが\expr{private}だからです(ここでは無くてもいい明示的宣言を行っています)。

この例では\emph{static}フィールドを使って可視性の実演をしていますが、メンバフィールドでもこのルールは同じです。次の例は\tref{継承}{types-class-inheritance}がある場合の可視性について実演しています。

\haxe{assets/Visibility2.hx}

\type{Child2}からの、\type{Child1}という異なる型の\expr{child1.baseField()}へのアクセスが許可されていることがわかります。これはこのフィールドが共通の親クラスの\type{Base}で定義されているからです。反対に\expr{child1Field}については、\type{Child2}からはアクセスできません。

可視性の修飾子の省略はデフォルトでは\expr{private}になることが多いですが、以下の場合は例外的に\expr{public}になります。

\begin{enumerate}
	\item クラスが\expr{extern}として宣言されている。
	\item \tref{インターフェース}{types-interfaces}で宣言しているフィールドである。
	\item \expr{public}フィールドを\tref{オーバーライド}{class-field-overriding}している。
\end{enumerate}

\trivia{protected}{HaxeにはJavaやC++やその他のオブジェクト指向言語で知られる\expr{protected}キーワードはありません。しかし、\expr{private}の挙動がこれらの言語の\expr{protected}の挙動に当たります。つまり、Haxeにはこれらの言語の\expr{private}に当たる挙動がありません。}

\subsection{inline(インライン化)}
\label{class-field-inline}

\expr{inline}キーワードはその関数の式を関数を呼び出した位置に直接挿し込みできるようにします。これは強力な最適化手段ですが、すべての関数にインライン化の挙動を持つ資格があるわけでありません。基本的な使い方は以下の通りです。

\haxe{assets/Inline.hx}

\target{JavaScript}出力を見るとインライン化の効果がわかります。

\begin{lstlisting}
(function () { "use strict";
var Main = function() { }
Main.main = function() {
	var a = 1;
	var b = 2;
	var c = (a + b) / 2;
}
Main.main();
})();
\end{lstlisting}

見てのとおり\expr{mid}フィールドの関数本体の\expr{(s1 + s2) / 2}が、\expr{mid(a, b)}の位置で\expr{s1}を\expr{a}に\expr{s2}を\expr{b}に置き換えられて出力されています。ターゲットによっては消えない場合もありますが、関数呼び出しが消滅しており、これが大きなパフォーマンスの改善になりえます。

インライン化するべきかの判断は簡単ではありません。書き込み処理の無い短い関数(\expr{=}の代入のみといった)は、たいていインライン化すると良いですし、より複雑な関数であってもインライン化する候補となりえます。一方で、インライン化がパフォーマンスに悪影響を与える場合もあります(複雑な式では、コンパイラが一時変数を作るなどのため)。

\expr{inline}キーワードは、インライン化されることを保証しません。コンパイラはさまざまな理由でインライン化をキャンセルします。例えば、コマンドラインの引数で\ic{--no-inline}が与えられた場合です。例外としてクラスが\tref{extern}{lf-externs}か、フィールドが\expr{:extern}\tref{メタデータ}{lf-metadata}を付けられていた場合、インライン化が強制されます。インライン化ができない場合、コンパイラはエラーを出力します。

これはインライン化を使う上で重要なので覚えておきましょう。

\haxe{assets/InlineRelying.hx}

\expr{error}の呼び出しがインライン化できれば、制御フローのチェッカーはインライン化された\tref{throw}{expression-throw}に満足してプログラムは正しくコンパイルされます。インライン化されなければ、コンパイラは関数呼び出しのみを見て、\expr{A return is missing here}(ここにリターンが足りません)というエラーを出力します。

\subsection{dynamic}
\label{class-field-dynamic}

メソッドは\expr{dynamic}キーワードをつけることで、束縛のしなおしをできるようにします。

\haxe{assets/DynamicFunction.hx}

最初の\expr{test()}の呼び出しではもともとの関数を実行して\expr{"original"}の文字列を返します。つぎの行で、\expr{test()}に新しい関数が代入されます。これが\expr{dynamic}が可能にする関数の再束縛です。その結果として、次の\expr{test()}の呼び出しでは\expr{"new"}の文字列が返っています。

\expr{dynamic}フィールドは\expr{inline}フィールドにできません。その理由は明らかです。インライン化はコンパイル時に行われますが、\expr{dynamic}な関数は実行時に解決されます。

%TODO: performance estimation %

\subsection{override}
\label{class-field-override}

\expr{override}アクセス修飾子は\tref{親クラス}{types-class-inheritance}ですでに存在するフィールドを定義するときに必要です。これはクラスの継承関係が大きい場合でも、書き手がオーバーライドに気づくようにするためです。同様に、実際には何もオーバーライドしないフィールドに\expr{override}しようとするとエラーになります(例えば、スペルミスの場合)。

オーバーライドの効果については、\Fullref{class-field-overriding}で詳しく説明しています。この修飾子は\tref{メソッド}{class-field-method}フィールドのみに使用可能です。
