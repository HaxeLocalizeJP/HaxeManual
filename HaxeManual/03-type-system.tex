\chapter{型システム}
\label{type-system}

私たちは\Fullref{types}の章でさまざまな種類の型について学んできました。ここからはそれらがお互いにどう関連しあっているかを見ていきます。まず、複雑な型に対して名前(別名)を与える仕組みである\tref{Typedef}{type-system-typedef}の紹介から簡単に始めます。typedefは特に、\tref{型パラメータ}{type-system-type-parameters}を持つ型で役に立ちます。

<<<<<<< HEAD:03-type-system.tex
任意の2つの型について、その上位の型のグループが矛盾しないかをチェックすることで多くの型安全性が得られます。これがコンパイラが試みる\emph{単一化}であり、\Fullref{type-system-unification}の節で詳しく説明します。
=======
A lot of type-safety is achieved by checking if two given types of the type groups above are compatible. Meaning, the compiler tries to perform \emph{unification} between them as detailed in \Fullref{type-system-unification}.
>>>>>>> english/master:HaxeManual/03-type-system.tex

すべての型は\emph{モジュール}に所属し、\emph{パス}を通して呼び出されます。\Fullref{type-system-modules-and-paths}では、これらに関連した仕組みについて詳しい説明を行います。

\section{typedef}
\label{type-system-typedef}

<<<<<<< HEAD:03-type-system.tex
typedefは\tref{匿名構造体}{types-anonymous-structure}の節で、すでに登場しています。そこでは複雑な構造体の型について名前を与えて簡潔にあつかう方法を見ています。この利用法はtypedefが一体なにに良いのかを的確に表しています。構造体の型に対して名前を与えるのは、typedefの主たる用途かもしれません。実際のところ、この用途が一般的すぎて、多くのHaxeユーザーがtypdefを構造体のためのものだと思ってしまっています。
=======
We briefly looked at typedefs while talking about \tref{anonymous structures}{types-anonymous-structure} and saw how we could shorten a complex \tref{structure type}{types-anonymous-structure} by giving it a name. This is precisely what typedefs are good for. Giving names to structure types might even be considered their primary use. In fact, it is so common that the distinction appears somewhat blurry and many Haxe users consider typedefs to actually \emph{be} the structure.
>>>>>>> english/master:HaxeManual/03-type-system.tex

typedefは他のあらゆる型に対して名前を与えることが可能です。

\begin{lstlisting}
typedef IA = Array<Int>;
\end{lstlisting}
<<<<<<< HEAD:03-type-system.tex

これにより\expr{Array$<$Int$>$}が使われる場所で、代わりに\expr{IA}を使うことが可能になります。この場合、はほんの数回のタイプ数しか減らせませんが、より複雑な複合型の場合は違います。これこそが、typedefと構造体が強く結びついて見える理由です。
=======
This enables us to use \expr{IA} in places where we would normally use \expr{Array$<$Int$>$}. While this saves only a few keystrokes in this particular case, it can make a much bigger difference for more complex, compound types. Again, this is why typedef and structures seem so connected:
>>>>>>> english/master:HaxeManual/03-type-system.tex

\begin{lstlisting}
typedef User = {
    var age : Int;
    var name : String;
}
\end{lstlisting}
<<<<<<< HEAD:03-type-system.tex

typedefはテキスト上の置き換えではなく、実は本物の型です。Haxe標準ライブラリの\type{Iterable}のように\tref{型パラメータ}{type-system-type-parameters}を持つことができます。
=======
A typedef is not a textual replacement but actually a real type. It can even have \tref{type parameters}{type-system-type-parameters} as the \type{Iterable} type from the Haxe Standard Library demonstrates:
>>>>>>> english/master:HaxeManual/03-type-system.tex

\begin{lstlisting}
typedef Iterable<T> = {
	function iterator() : Iterator<T>;
}
\end{lstlisting}

\subsection{拡張}
\label{type-system-extensions}

% TODO: move to structures? %
<<<<<<< HEAD:03-type-system.tex

拡張は、構造体が与えられた型のフィールドすべてと、加えていくつかのフィールドを持っていることを表すために使われます。
=======
Extensions are used to express that a structure has all the fields of a given type as well as some additional fields of its own:
>>>>>>> english/master:HaxeManual/03-type-system.tex

\haxe{assets/Extension.hx}
大なりの演算子を使うことで、追加のクラスフィールドを持つ\type{Iterable$<$T$>$}の拡張が作成されました。このケースでは、読み込み専用の\tref{プロパティ}{class-field-property} である\type{Int}型の\expr{length}が要求されます。 

\type{IterableWithLength$<$T$>$}に適合するためには、\type{Iterable$<$T$>$}にも適合してさらに読み込み専用の\type{Int}型のプロパティ\expr{length}を持ってなきゃいけません。例では、Arrayが割り当てられており、これはこれらの条件をすべて満たしています。

\since{3.1.0}
複数の構造体を拡張することもできます。

\haxe{assets/Extension2.hx}

\section{型パラメータ}
\label{type-system-type-parameters}

\tref{クラスフィールド}{class-field}や\tref{列挙型コンストラクタ}{types-enum-constructor}のように、Haxeではいくつかの型についてパラメータ化を行うことができます。型パラメータは山カッコ\expr{$<>$}内にカンマ区切りで記述することで、定義することができます。シンプルな例は、Haxe標準ライブラリの\type{Array}です。

\begin{lstlisting}
class Array<T> {
	function push(x : T) : Int;
}
\end{lstlisting}
\type{Array}のインスタンスが作られると、型パラメータ\type{T}は\tref{単相}{types-monomorph}となります。つまり、1度に1つの型であれば、あらゆる型を適用することができます。この適用は以下のどちらか方法で行います

\begin{description}
	\item[明示的に、]\expr{new Array$<$String$>$()}のように型を記述してコンストラクタを呼び出して適用する。
	\item[暗黙に]、\tref{型推論}{type-system-type-inference}で適用する。例えば、\expr{arrayInstance.push("foo")}を呼び出す。
\end{description}

型パラメータが付くクラスの定義の内部では、その型パラメータは不定の型です。\tref{制約}{type-system-type-parameter-constraints}が追加されない限り、コンパイラはその型パラメータはあらゆる型になりうるものと決めつけることになります。その結果、型パラメータの\tref{cast}{expression-cast}を使わなければ、その型のフィールドにアクセスできなくなります。また、\tref{ジェネリック}{type-system-generic}にして適切な制約をつけない限り、その型パラメータの型のインスタンスを新しく生成することもできません。

以下は、型パラメータが使用できる場所についての表です。

\begin{center}
\begin{tabular}{| l | l | l |}
	\hline
	パラメータが付く場所 & 型を適用する場所 & 備考 \\ \hline
	Class & インスタンス作成時 & メンバフィールドにアクセスする際に型を適用することもできる \\
	Enum & インスタンス作成時 & \\
	Enumコンストラクタ & インスタンス作成時 & \\
	関数 & 呼び出し時 & メソッドと名前付きのローカル関数で利用可能	\\
	構造体 & インスタンス作成時 & \\ \hline
\end{tabular}
\end{center}
<<<<<<< HEAD:03-type-system.tex

関数の型パラメータは呼び出し時に適用される、この型パラメータは(制約をつけない限り)あらゆる型を許容します。しかし、一回の呼び出しにつき適用は1つの型のみ可能です。このことは関数が複数の引数を持つ場合に役立ちます。

\haxe{assets/FunctionTypeParameter.hx}

\expr{equals}関数の\expr{expected}と\expr{actual}の引数両方が、\type{T}型になっています。これは\expr{equals}の呼び出しで2つの引数の型が同じでなければならないことを表しています。コンパイラは最初(両方の引数が\type{Int}型)と2つめ(両方の引数が\type{String}型)の呼び出しは認めていますが、3つ目の呼び出しはコンパイルエラーにします。

\trivia{式の構文内での型パラメータ}{なぜ、\expr{method<String>(x)}のようにメソッドに型パラメータをつけた呼び出しができないのか?という質問をよくいただきます。このときのエラーメッセージはあまり参考になりませんが、これには単純な理由があります。それは、このコードでは、\expr{<}と\expr{>}の両方が2項演算子として構文解析されて、\expr{(method < String) > (x)}と見なされるからです。}
=======
With function type parameters being bound upon invocation, such a type parameter (if unconstrained) accepts any type. However, only one type per invocation is accepted. This can be utilized if a function has multiple arguments:

\haxe{assets/FunctionTypeParameter.hx}

Both arguments \expr{expected} and \expr{actual} of the \expr{equals} function have type \type{T}. This implies that for each invocation of \expr{equals} the two arguments must be of the same type. The compiler admits the first call (both arguments being of \type{Int}) and the second call (both arguments being of \type{String}) but the third attempt causes a compiler error.

\trivia{Type parameters in expression syntax}{We often get the question why a method with type parameters cannot be called as \expr{method<String>(x)}. The error messages the compiler gives are not very helpful. However, there is a simple reason for that: The above code is parsed as if both \expr{<} and \expr{>} were binary operators, yielding \expr{(method < String) > (x)}.}
>>>>>>> english/master:HaxeManual/03-type-system.tex

\subsection{制約}
\label{type-system-type-parameter-constraints}

型パラメータは複数の型で制約を与えることができます。

\haxe{assets/Constraints.hx}

\expr{test}メソッドの型パラメータ\type{T}は、\type{Iterable$<$String$>$}と\type{Measurable}の型に制約されます。\type{Measurable}の方は、便宜上\tref{typedef}{type-system-typedef}を使って、\type{Int}型の読み込み専用\tref{プロパティ}{class-field-property}\expr{length}を要求しています。つまり、以下の条件を満たせば、これらの制約と矛盾しません。

\begin{itemize}
	\item \type{Iterable$<$String$>$}である
	\item かつ、\type{Int}型の\expr{length}を持つ
\end{itemize}
<<<<<<< HEAD:03-type-system.tex
=======
We can see that invoking \expr{test} with an empty array in line 7 and an \type{Array$<$String$>$} in line 8 works fine. This is because \type{Array} has both a \expr{length}-property and an \expr{iterator}-method. However, passing a \type{String} as argument in line 9 fails the constraint check because \type{String} is not compatible with \type{Iterable$<$T$>$}. 
>>>>>>> english/master:HaxeManual/03-type-system.tex

7行目では空の配列で、8行目では\type{Array$<$String$>$}で\expr{test}関数を呼び出すことができることを確認しました。しかし、10行目の\type{String}の引数では制約チェックで失敗しています。これは、\type{String}は\type{Iterable$<$T$>$}と矛盾するからです。

\section{ジェネリック}
\label{type-system-generic}

<<<<<<< HEAD:03-type-system.tex
大抵の場合、Haxeコンパイラは型パラメータが付けられていた場合でも、1つのクラスや関数を生成します。これにより自然な抽象化が行われて、ターゲット言語のコードジェネレータは出力先の型パラメータはあらゆる型になりえると思い込むことになります。つまり、生成されたコードで型チェックが働き、動作が邪魔されることがあります。
=======
Usually, the Haxe Compiler generates only a single class or function even if it has type parameters. This results in a natural abstraction where the code generator for the target language has to assume that a type parameter could be of any type. The generated code then might have to perform some type checks which can be detrimental to performance.
>>>>>>> english/master:HaxeManual/03-type-system.tex

クラスや関数は、\expr{:generic} \tref{メタデータ}{lf-metadata}で\emph{ジェネリック}属性をつけることで一般化することができます。これにより、コンパイラは型パラメータの組み合わせごとのクラスまたは関数を修飾された名前で書き出します。このような設計により\tref{静的ターゲット}{define-static-target}のパフォーマンスに直結するコード部位では、出力サイズの巨大化と引き換えに、速度を得られます。

\haxe{assets/GenericClass.hx}

<<<<<<< HEAD:03-type-system.tex
あまり使わない明示的な\type{MyArray<String>}の型宣言があり、よく使う\tref{型推論}{type-system-type-inference}であつかっていますが、これが重要です。コンパイラは、コンストラクタの呼び出し時にジェネリッククラスの正確な型な型を知っている必要があります。この\target{JavaScript}出力は以下のような結果になります。
=======
It seems unusual to see the explicit type \type{MyArray<String>} here as we usually let \tref{type inference}{type-system-type-inference} deal with this. Nonetheless, it is indeed required in this case. The compiler has to know the exact type of a generic class upon construction. The \target{Javascript} output shows the result:
>>>>>>> english/master:HaxeManual/03-type-system.tex

\begin{lstlisting}
(function () { "use strict";
var Test = function() { };
Test.main = function() {
	var a = new MyValue_String("Hello");
	var b = new MyValue_Int(5);
};
var MyValue_Int = function(value) {
	this.value = value;
};
var MyValue_String = function(value) {
	this.value = value;
};
Test.main();
})();
\end{lstlisting}

\type{MyArray<String>}と\type{MyArray<Int>}は、それぞれ\type{MyArray_String}と\type{MyArray_Int}になっています。これはジェネリック関数でも同じです。

\haxe{assets/GenericFunction.hx}

\target{JavaScript}出力を見れば明白です。

\begin{lstlisting}
(function () { "use strict";
var Main = function() { }
Main.method_Int = function(t) {
}
Main.method_String = function(t) {
}
Main.main = function() {
	Main.method_String("foo");
	Main.method_Int(1);
}
Main.main();
})();
\end{lstlisting}


\subsection{ジェネリック型パラメータのコンストラクト}
\label{type-system-generic-type-parameter-construction}

\define{ジェネリック型パラメータ}{define-generic-type-parameter}{型パラメータを持っているクラスまたはメソッドがジェネリックであるとき、その型パラメータもジェネリックであるという。}

<<<<<<< HEAD:03-type-system.tex
普通の型パラメータでは、\expr{new T()}のようにその型をコンストラクトすることはできません。これは、Haxeが1つの関数を生成するために、そのコンストラクトが意味をなさないからです。しかし、型パラメータがジェネリックの場合は違います。これは、コンパイラはすべての型パラメータの組み合わせに対して別々の関数を生成しています。このため\expr{new T()}の\type{T}を実際の型に置き換えることができます。

\haxe{assets/GenericTypeParameter.hx}

ここでは、\type{T}の実際の型の決定は、\tref{トップダウンの推論}{type-system-top-down-inference}で行われることに注意してください。この方法での型パラメータのコンストラクトを行うには2つの必須事項があります。
=======
It is not possible to construct normal type parameters, e.g. \expr{new T()} is a compiler error. The reason for this is that Haxe generates only a single function and the construct makes no sense in that case. This is different when the type parameter is generic: Since we know that the compiler will generate a distinct function for each type parameter combination, it is possible to replace the \type{T} \expr{new T()} with the real type.

\haxe{assets/GenericTypeParameter.hx}

It should be noted that \tref{top-down inference}{type-system-top-down-inference} is used here to determine the actual type of \type{T}. There are two requirements for this kind of type parameter construction to work: The constructed type parameter must be
>>>>>>> english/master:HaxeManual/03-type-system.tex

\begin{enumerate}
	\item ジェネリックであること
	\item 明示的に、\tref{コンストラクタ}{types-class-constructor}を持つように\tref{制約}{type-system-type-parameter-constraints}されていること
\end{enumerate}

先ほどの例は、1つ目は\expr{make}が\expr{@:generic}メタデータを持っており、2つ目\type{T}が\type{Constructible}に制約されています。\type{String}と\type{haxe.Template}の両方とも1つ\type{String}の引数のコンストラクタを持つのでこの制約に当てはまります。確かにJavascript出力は予測通りのものになっています。

\begin{lstlisting}
var Main = function() { }
Main.__name__ = true;
Main.make_haxe_Template = function() {
	return new haxe.Template("foo");
}
Main.make_String = function() {
	return new String("foo");
}
Main.main = function() {
	var s = Main.make_String();
	var t = Main.make_haxe_Template();
}
\end{lstlisting}

\section{変性(バリアンス)}
\label{type-system-variance}

<<<<<<< HEAD:03-type-system.tex
変性とは他のものとの関連を表すもので、特に型パラメータに関するものが連想されます。そして、この文脈では驚くようなことがよく起こります。変性のエラーを起こすことはとても簡単です。
=======
While variance is also relevant in other places, it occurs particularly often with type parameters and comes as a surprise in this context. Additionally, it is very easy to trigger variance errors:
>>>>>>> english/master:HaxeManual/03-type-system.tex

\haxe{assets/Variance.hx}

見てわかるとおり、\type{Child}は\type{Base}に代入できるにもかかわらず、\type{Array<Child>}を\type{Array<Base>}に代入することはできません。この理由は少々予想外のものかもしれません。それはこの配列への書き込みが可能だからです。例えば、\expr{push()}メソッドです。この変性のエラーを無視してしまうことは簡単です。

\haxe{assets/Variance2.hx}

<<<<<<< HEAD:03-type-system.tex
ここでは\tref{cast}{expression-cast}を使って型チェッカーを破壊して、12行目の代入を可能にしてしまっています。\expr{bases}は元々の配列への参照を持っており、\type{Array<Base>}の型付けをされています。このため、\type{Base}に適合する別の型の\type{OtherChild}を配列に追加できます。しかし、元々の\expr{children}の参照は\type{Array<Child>}のままです。そのため良くないことに繰り返し処理の中で\type{OtherChild}のインスタンスに出くわします。

もし\type{Array}が\expr{push()}メソッドを持っておらず、他の編集方法も無かったならば、適合しない型を追加することができなくなるのでこの代入は安全になります。Haxeでは\tref{構造的部分型付け}{type-system-structural-subtyping}を使って型を適切に制限することでこれを実現できます。

\haxe{assets/Variance3.hx}

\expr{b}は\type{MyArray<Base>}として型付けされており、\type{MyArray}は\expr{pop()}メソッドしか持たないため、安全に代入することができます。\type{MyArray}には適合しない型を追加できるメソッドを持っておらず、このことは\emph{共変性}と呼ばれます。
=======
Here we subvert the type checker by using a \tref{cast}{expression-cast}, thus allowing the assignment after the commented line. With that we hold a reference \expr{bases} to the original array, typed as \type{Array<Base>}. This allows pushing another type compatible with \type{Base} (\type{OtherChild}) onto that array. However, our original reference \expr{children} is still of type \type{Array<Child>} and things go bad when we encounter the \type{OtherChild} instance in one of its elements while iterating.

If \type{Array} had no \expr{push()} method and no other means of modification, the assignment would be safe because no incompatible type could be added to it. In Haxe, we can achieve this by restricting the type accordingly using \tref{structural subtyping}{type-system-structural-subtyping}:

\haxe{assets/Variance3.hx}

We can safely assign with \expr{b} being typed as \type{MyArray<Base>} and \type{MyArray} only having a \expr{pop()} method. There is no method defined on \type{MyArray} which could be used to add incompatible types, it is thus said to be \emph{covariant}.

\define{Covariance}{define-covariance}{A \tref{compound type}{define-compound-type} is considered covariant if its component types can be assigned to less specific components, i.e. if they are only read, but never written.}

\define{Contravariance}{define-contravariance}{A \tref{compound type}{define-compound-type} is considered contravariant if its component types can be assigned to less generic components, i.e. if they are only written, but never read.}
>>>>>>> english/master:HaxeManual/03-type-system.tex

\define{共変性}{define-covariance}{\tref{複合型}{define-compound-type}がそれを構成する型よりも一般な型で構成される複合型に代入できる場合に、共変であるという。 つまり、読み込みのみが許されて書き込みができない場合です。}

\define{反変性}{define-contravariance}{\tref{複合型}{define-compound-type}がそれを構成する型よりも特殊な型で構成される複合型に代入できる場合に、反変であるという。 つまり、書き込みのみが許されて読み込みができない場合です。}

\section{単一化(ユニフィケーション)}
\label{type-system-unification}

\todo{Mention toString()/String conversion somewhere in this chapter.}

単一化は型システムの要であり、Haxeの堅牢さに大きく貢献しています。この節ではある型が他の型と適合するかどうかをチェックする過程を説明していきます。

\define{単一化}{define-unification}{型Aの型Bでの単一化というのは、AがBに代入可能かを調べる指向性を持つプロセスです。型が\tref{単相}{types-monomorph}の場合または単相を含む場合は、それを変化させることができます。}

単一化のエラーは簡単に起こすことができます。

\begin{lstlisting}
class Main {
	static public function main() {
    // Int should be String
		var s:String = 1;
	}
}
\end{lstlisting}

\type{Int}型の値を\type{String}型の変数に代入しようとしたので、コンパイラは\emph{IntをStringで単一化}しようと試みます。これはもちろん許可されておらず、コンパイラは\expr{"Int should be String"}というエラーを出力します。

このケースでは単一化は\emph{代入}によって引き起こされており、この文脈は「代入可能」という定義に対して直感的です。ただ、これは単一化が働くケースのうちの1つでしかありません。

\begin{description}
<<<<<<< HEAD:03-type-system.tex
	\item[代入:] \expr{a}が\expr{b}に代入された場合、\expr{a}の型は\expr{b}で単一化されます。
	\item[関数呼び出し:] \tref{関数}{types-function}の型の紹介ですでに触れています。一般的に言うと、コンパイラは渡された最初の引数の型を要求される最初の引数の型で単一化し、渡された二番目の引数の型を要求される二番目の引数の型で単一化するということを、すべての引数で行います。
	\item[関数のreturn:] 関数が\expr{return e}の式をもつ場合は常に\expr{e}の型は関数の戻り値の型で単一化されます。もし関数の戻り値の型が明示されていなければ、\expr{e}の型に型推論されて、それ以降の\expr{return}式は\expr{e}の型に推論されます。
	\item[配列の宣言:] コンパイラは、配列の宣言では与えられたすべての型に共通する最小の型を見つけようとします。詳細は\Fullref{type-system-unification-common-base-type}を参照してください。
	\item[オブジェクトの宣言:] オブジェクトを指定された型に対して宣言する場合、コンパイラは与えられたフィールドすべての型を要求されるフィールドの型で単一化します。
	\item[演算子の単一化:] 特定の型を要求する特定の演算子は、与えられた型をその型で単一化します。例えば、\expr{a \&\& b}という式は\expr{a}と\expr{b}両方を\type{Bool}で単一化します。式\expr{a == b}は\expr{a}を\expr{b}で単一化します。
=======
	\item[Assignment:] If \expr{a} is assigned to \expr{b}, the type of \expr{a} is unified with the type of \expr{b}.
	\item[Function call:] We have briefly seen this one while introducing the \tref{function}{types-function} type. In general, the compiler tries to unify the first given argument type with the first expected argument type, the second given argument type with the second expected argument type and so on until all argument types are handled.
	\item[Function return:] Whenever a function has a \expr{return e} expression, the type of \expr{e} is unified with the function return type. If the function has no explicit return type, it is inferred to the type of \expr{e} and subsequent \expr{return} expressions are inferred against it.
	\item[Array declaration:] The compiler tries to find a minimal type between all given types in an array declaration. Refer to \Fullref{type-system-unification-common-base-type} for details.
	\item[Object declaration:] If an object is declared ``against'' a given type, the compiler unifies each given field type with each expected field type.
	\item[Operator unification:] Certain operators expect certain types which the given types are unified against. For instance, the expression \expr{a \&\& b} unifies both \expr{a} and \expr{b} with \type{Bool} and the expression \expr{a == b} unifies \expr{a} with \expr{b}.
>>>>>>> english/master:HaxeManual/03-type-system.tex
\end{description}


\subsection{クラスとインターフェースの単一化}
\label{type-system-unification-between-classes-and-interfaces}

<<<<<<< HEAD:03-type-system.tex
クラスの間の単一化について定義を行う場合、単一化が指向性を持つことを心に留めておくべきです。より特殊なクラス(例えば、子クラス)を一般的なクラス(例えば、親クラス)に対して代入することはできますが、逆はできません。
=======
When defining unification behavior between classes, it is important to remember that unification is directional: We can assign a more specialized class (e.g. a child class) to a generic class (e.g. a parent class) but the reverse is not valid.
>>>>>>> english/master:HaxeManual/03-type-system.tex

以下のような、代入が許可されます。

\begin{itemize}
	\item 子クラスの親クラスへの代入
	\item クラスの実装済みのインターフェースへの代入
	\item インターフェースの親インターフェースへの代入
\end{itemize}

これらのルールは連結可能です。つまり、子クラスをその親クラスの親クラスへ代入可能であり、さらに親クラスが実装しているインターフェースへ代入可能であり、そのインターフェースの親インターフェースへ代入可能であるということです。

\todo{''parent class'' should probably be used here, but I have no idea what it means, so I will refrain from changing it myself.}

\subsection{構造的部分型付け}
\label{type-system-structural-subtyping}

\define{構造的部分型付け}{define-structural-subtyping}{構造的部分型付けは、同じ構造を持つ型の暗黙の関係を示します。}

<<<<<<< HEAD:03-type-system.tex
Haxeでは、構造的部分型付けはクラスインスタンスを構造体に代入するときのみ可能です。以下に、\tref{Haxe標準ライブラリ}{std}の\type{Lambda}の一部の例を挙げます。
=======
Structural sub-typing in Haxe is allowed when unifying

\begin{itemize}
	\item a \tref{class}{types-class-instance} with a \tref{structure}{types-anonymous-structure} and
	\item a structure with another structure.
\end{itemize}

The following example is part of the \type{Lambda} class of the \tref{Haxe Standard Library}{std}:
>>>>>>> english/master:HaxeManual/03-type-system.tex

\begin{lstlisting}
public static function empty<T>(it : Iterable<T>):Bool {
	return !it.iterator().hasNext();
}
\end{lstlisting}
<<<<<<< HEAD:03-type-system.tex

\expr{empty}メソッドは、\type{Iterable}が要素を持つかをチェックします。この目的では、引数の型について、それが列挙可能(Iterable)であること以外に何も知る必要がありません。Haxe標準ライブラリにはたくさんある\type{Iterable$<$T$>$}で単一化できる型すべてで、これを呼び出すことができるわけです。この種の型付けは非常に便利ですが、静的ターゲットでは大量に使うとパフォーマンスの低下を招くことがあります。詳しくは\Fullref{types-structure-performance}に書かれています。
=======
The \expr{empty}-method checks if an \type{Iterable} has an element. For this purpose, it is not necessary to know anything about the argument type other than the fact that it is considered an iterable. This allows calling the \expr{empty}-method with any type that unifies with \type{Iterable$<$T$>$} which applies to a lot of types in the Haxe Standard Library.

This kind of typing can be very convenient but extensive use may be detrimental to performance on static targets, which  is detailed in \Fullref{types-structure-performance}.
>>>>>>> english/master:HaxeManual/03-type-system.tex


\subsection{単相}
\label{type-system-monomorphs}

\tref{単相}{types-monomorph}である、あるいは単相を含む型についての単一化は\Fullref{type-system-type-inference}で詳しくあつかいます。

\subsection{関数の戻り値}
\label{type-system-unification-function-return}

<<<<<<< HEAD:03-type-system.tex
関数の戻り値の型の単一化では\tref{\type{Void}型}{types-void}も関連しており、\type{Void}型での単一化のはっきりとした定義が必要です。\type{Void}型は型の不在を表し、あらゆる型が代入できません。\type{Dynamic}でさえも代入できません。つまり、関数が明示的に\type{Dynamic}を返すと定義されている場合、\type{Void}を返してはいけません。
=======
Unification of function return types may involve the \tref{\type{Void}-type}{types-void} and requires a clear definition of what unifies with \type{Void}. With \type{Void} describing the absence of a type, it is not assignable to any other type, not even \type{Dynamic}. This means that if a function is explicitly declared as returning \type{Dynamic}, it cannot return \type{Void}.
>>>>>>> english/master:HaxeManual/03-type-system.tex

その逆も同じです。関数の戻り値が\type{Void}であると宣言しているなら、\type{Dynamic}やその他すべての型は返すことができません。しかし、関数の型を代入する時のこの方向の単一化は許可されています。

\begin{lstlisting}
var func:Void->Void = function() return "foo";
\end{lstlisting}
<<<<<<< HEAD:03-type-system.tex
=======

The right-hand function clearly is of type \type{Void->String}, yet we can assign it to the variable \expr{func} of type \type{Void->Void}. This is because the compiler can safely assume that the return type is irrelevant, given that it could not be assigned to any non-\type{Void} type.
>>>>>>> english/master:HaxeManual/03-type-system.tex

右辺の関数ははっきりと\type{Void->String}型ですが、これを\type{Void->Void}型の\expr{func}変数に代入することができます。これはコンパイラが戻り値は無関係だと安全に判断できるからで、その結果\type{Void}ではないあらゆる型を代入できるようになります。

\subsection{共通の基底型}
\label{type-system-unification-common-base-type}

複数の型の組み合わせが与えられたとき、そのすべての型が\emph{共通の基底型}で単一化されます。

\haxe{assets/UnifyMin.hx}

\type{Base}とは書かれていないにも関わらず、Haxeコンパイラは\type{Child1}と\type{Child2}の共通の型として\type{Base}を推論しています。Haxeコンパイラはこの方法の単一化を以下の場面で採用しています。

\begin{itemize}
	\item 配列の宣言
	\item \expr{if}/\expr{else}
	\item \expr{switch}のケース
\end{itemize}


\section{型推論}
\label{type-system-type-inference}

型推論はこのドキュメントで何度も出てきており、これ以降でも重要です。型推論の動作の簡単なサンプルをお見せします。

\haxe{assets/TypeInference.hx}
<<<<<<< HEAD:03-type-system.tex

この特殊な構文\expr{\$type}は、\Fullref{types-function}の型の説明をわかりやすくするためにも使っていました。それではここで公式な説明をしましょう。
=======
The special construct \expr{\$type} was previously mentioned in order to simplify the explanation of the \Fullref{types-function} type, so let us now introduce it officially:
>>>>>>> english/master:HaxeManual/03-type-system.tex

%TODO: $type
\define[Construct]{\expr{\$type}}{define-dollar-type}{\expr{\$type}は関数のように呼び出せるコンパイル時の仕組みで、一つの引数を持ちます。コンパイラは引数の式を評価し、そしてその式の型を出力します。}

上記の例では、最初の\expr{\$type}では\expr{Unknown<0>}が表示されます。これは\tref{単相}{types-monomorph}で、未知の型です。次の行の\expr{x = "foo"}で定数値の\type{String}を\expr{x}に代入しており、\type{String}の単相での\tref{単一化}{type-system-unification}が起こります。そして、\expr{x}がこのとき\type{String}に変わったことがわかります。

\Fullref{types-dynamic}以外の型が、単相での単一化を行うと単相はその型になります(その型に変形(\emph{morph})します)。このため、この型はもう別の型には変形できません。これが単相(monomorph)の\emph{mono}の部分です。

以下が単一化のルールです。型推論は複合型でも起こります。

\haxe{assets/TypeInference2.hx}

変数\expr{x}は初め空の\type{Array}で初期化されています。この時点で\expr{x}の型は配列であると言えますが、配列の要素の型については未知です。その結果\expr{x}の型は、\type{Array<Unknown<0>>}となります。この配列が\type{Array<String>}だとわかるには、\type{String}をこの配列にプッシュするだけで十分です。

\subsection{トップダウンの推論}
\label{type-system-top-down-inference}

多くの場合、ある型はその型で要求される型を推論します。しかし一部では、要求される型で型を推論します。これを\emph{トップダウンの推論}と呼びます。

\define{要求される型}{define-expected-type}{要求される型は、式の型が式が型付けされるより前にわかっている場合に現れます。例えば、式が関数の呼び出しの引数の場合です。この場合、\tref{トップダウンの推論}{type-system-top-down-inference}と呼ばれる方法で、式に型が伝搬します。}

良い例は型の混ざった配列です。\Fullref{types-dynamic}で書いた通り、\expr{[1, "foo"]}は要素の型を決定できないので、コンパイラはこれを拒絶します。これはトップダウンの推論を使えば解決します。

\haxe{assets/TopDownInference.hx}

ここでは、\expr{[1, "foo"]}に型付けするとき、要求される型が\type{Array<Dynamic>}であり、その要素は\type{Dynamic}であるとわかります。コンパイラが\tref{共通の基底型}{type-system-unification-common-base-type}を探す(そして失敗する)通常の単一化の挙動の代わりに、個々の要素が\type{Dynamic}で単一化され、型付けされます。

\tref{ジェネリック型パラメータのコンスラクト}{type-system-generic-type-parameter-construction}の紹介で、もう一つトップダウンの推論の面白い利用例を見ています。

\haxe{assets/GenericTypeParameter.hx}

\type{String}と\type{haxe.Template}の明示された型が、\expr{make}の戻り値の型の決定に使われています。これは、\expr{make()}の戻り値が変数へ代入されるのがわかっているので動作します。この方法を使うと、未知の型\type{T}をそれぞれ\type{String}と\type{haxe.Template}に紐づけることが可能です。

% this is not really top down inference
%Top-down inference is also utilized when dealing with \tref{enum constructors}{types-enum-constructor}:

%\haxe{assets/TopDownInference2.hx}

%The constructors \expr{TObject} and \expr{TFunction} of type \expr{ValueType} are recognized even though their containing module \type{Type} is not \tref{imported}{Import}. This is possible because the return type of \expr{Type.typeof("foo")} is known to be \expr{ValueType}.


\subsection{制限}
\label{type-system-inference-limitations}

ローカル変数をあつかう場合、型推論のおかげで多くの手動の型付けを省略できますが、型システムが助けを必要とする場面もあります。実際、\tref{変数フィールド}{class-field-variable}や\tref{プロパティ}{class-field-property}では、直接の初期化をしていない限りは型推論されません。

<<<<<<< HEAD:03-type-system.tex
また、再帰的な関数呼び出しでも型推論が制限される場面があります。型がまだ(完全に)わかっていない場合、型推論が間違って特殊すぎる型を推論する場合があります。

\section{モジュールとパス}
=======
A different kind of limitation involves the readability of code. If type inference is overused it might be difficult to understand parts of a program due to the lack of visible types. This is particularly true for method signatures. It is recommended to find a good balance between type inference and explicit type hints.


\section{Modules and Paths}
>>>>>>> english/master:HaxeManual/03-type-system.tex
\label{type-system-modules-and-paths}

\define{モジュール}{define-module}{
すべてのHaxeのコードはモジュールに属しており、パスを使って指定されます。要するに、.hxファイルそれぞれが一つのモジュールを表し、その中にいくつか型を置くことができます。型は\expr{private}にすることが可能で、その場合はその型の属するモジュールからしかアクセスできません。}

モジュールとそれに含まれる型との区別は意図的に不明瞭です。実際、\expr{haxe.ds.StringMap<Int>}の指定は、\expr{haxe.ds.StringMap.StringMap<Int>}の省略形とも考えられます。後者は4つ部位で構成されています。

\begin{enumerate}
	\item パッケージ \expr{haxe.ds}
	\item モジュール名 \expr{StringMap}
	\item 型名 \type{StringMap}
	\item 型パラメータ \type{Int}
\end{enumerate}

モジュールと型の名前が同じの場合、重複を取り除くことが可能で、これで\expr{haxe.ds.StringMap<Int>}という省略形が使えます。しかし、長い記述について知っていれば、\tref{モジュールの従属型}{type-system-module-sub-types}の指定の仕方の理解しやすくなります。

パスは、\tref{import}{type-system-import}を使ってパッケージの部分を省略することで、さらに短くすることができます。importの利用は不適切な識別子を作ってしまう場合があるので、\tref{解決順序}{type-system-resolution-order}についての理解が必要です。

\define{型のパス}{define-type-path}{(ドット区切りの)型のパスはパッケージ、モジュール名、型名から成ります。この一般的な形は\expr{pack1.pack2.packN.ModuleName.TypeName}です。} 


\subsection{モジュールの従属型}
\label{type-system-module-sub-types}

モジュール従属型とは、その型を定義しているモジュールの名前と異なる名前の型です。これにより、一つの.hxファイルに複数の型の定義が可能になり、これらの型はモジュール内では無条件でアクセス可能で、ほかのモジュールからも\expr{package.Module.Type}の形式でアクセスできます。

\begin{lstlisting}
var e:haxe.macro.Expr.ExprDef;
\end{lstlisting}

ここでは\expr{haxe.macro.Expr}の従属型\type{ExprDef}にアクセスしています。

<<<<<<< HEAD:03-type-system.tex
従属型の関係は、実行時には影響を与えません。publicの従属型はそのパッケージのメンバーになります。このため、同じパッケージの別々のモジュールで同じ従属型を定義した場合に衝突を起こします。
The sub-type relation is not reflected at runtime. That is, public sub-types become a member of their containing package, which could lead to conflicts if two modules within the same package try to define the same sub-type. Naturally the Haxe compiler detects these cases and reports them accordingly. In the example above, \type{ExprDef} is generated as \type{haxe.macro.ExprDef}.
=======
The sub-type relation is not reflected at run-time. That is, public sub-types become a member of their containing package, which could lead to conflicts if two modules within the same package tried to define the same sub-type. Naturally, the Haxe compiler detects these cases and reports them accordingly. In the example above \type{ExprDef} is generated as \type{haxe.macro.ExprDef}.
>>>>>>> english/master:HaxeManual/03-type-system.tex

Sub-types can also be made private:

\begin{lstlisting}
private class C { ... }
private enum E { ... }
private typedef T { ... }
private abstract A { ... }
\end{lstlisting}

\define{Private type}{define-private-type}{A type can be made private by using the \expr{private} modifier. As a result, the type can only be directly accessed from within the \tref{module}{define-module} it is defined in.

Private types, unlike public ones, do not become a member of their containing package.}

The accessibility of types can be controlled more fine-grained by using \tref{access control}{lf-access-control}.



\subsection{Import}
\label{type-system-import}

If a type path is used multiple times in a .hx file, it might make sense to use an \expr{import} to shorten it. This allows omitting the package when using the type:

\haxe{assets/Import.hx}

With \expr{haxe.ds.StringMap} being imported in the first line, the compiler is able to resolve the unqualified identifier \expr{StringMap} in the \expr{main} function to this package. The module \type{StringMap} is said to be \emph{imported} into the current file.

In this example, we are actually importing a \emph{module}, not just a specific type within that module. This means that all types defined within the imported module are available:

\haxe{assets/Import2.hx}

The type \type{Binop} is an \tref{enum}{types-enum-instance} declared in the module \type{haxe.macro.Expr}, and thus available after the import of said module. If we were to import only a specific type of that module, e.g. \expr{import haxe.macro.Expr.ExprDef}, the program would fail to compile with \expr{Class not found : Binop}.

There are several aspects worth knowing about importing:

\begin{itemize}
	\item The bottommost import takes priority (detailed in \Fullref{type-system-resolution-order}).
	\item The \tref{static extension}{lf-static-extension} keyword \expr{using} implies the effect of \expr{import}.
	\item If an enum is imported (directly or as part of a module import), all its \tref{enum constructors}{types-enum-constructor} are also imported (this is what allows the \expr{OpAdd} usage in the above example).
\end{itemize}

Furthermore, it is also possible to import \tref{static fields}{class-field} of a class and use them unqualified:

\haxe{assets/Import3.hx}

Special care has to be taken with field names or local variable names that conflict with a package name: Since they take priority over packages, a local variable named \expr{haxe} blocks off usage the entire \expr{haxe} package.

\paragraph{Wildcard import}

Haxe allows using \expr{.*} to allow import of all modules in a package, all types in a module or all static fields in a type. It is important to understand that this kind of import only crosses a single level as we can see in the following example:

\haxe{assets/ImportWildcard.hx}

Using the wildcard import on \expr{haxe.macro} allows accessing \type{Expr} which is a module in this package, but it does not allow accessing \type{ExprDef} which is a sub-type of the \type{Expr} module. This rule extends to static fields when a module is imported.

When using wildcard imports on a package the compiler does not eagerly process all modules in that package. This means that these modules are never actually seen by the compiler unless used explicitly and are then not part of the generated output.

\paragraph{Import with alias}

If a type or static field is used a lot in an importing module it might help to alias it to a shorter name. This can also be used to disambiguate conflicting names by giving them a unique identifier.

\haxe{assets/ImportAlias.hx}

Here we import \expr{String.fromCharCode} as \expr{f} which allows us to use \expr{f(65)} and \expr{f(66)}. While the same could be achieved with a local variable, this method is compile-time exclusive and guaranteed to have no run-time overhead.

\since{3.2.0}

Haxe also allows the more natural \expr{as} in place of \expr{in}.


\subsection{Resolution Order}
\label{type-system-resolution-order}

Resolution order comes into play as soon as unqualified identifiers are involved. These are \tref{expressions}{expression} in the form of \expr{foo()}, \expr{foo = 1} and \expr{foo.field}. The last one in particular includes module paths such as \expr{haxe.ds.StringMap}, where \expr{haxe} is an unqualified identifier.  

We describe the resolution order algorithm here, which depends on the following state:

\begin{itemize}
	\item the declared \tref{local variables}{expression-var} (including function arguments)
	\item the \tref{imported}{type-system-import} modules, types and statics
	\item the available \tref{static extensions}{lf-static-extension}
	\item the kind (static or member) of the current field
	\item the declared member fields on the current class and its parent classes
	\item the declared static fields on the current class
	\item the \tref{expected type}{define-expected-type}
	\item the expression being \expr{untyped} or not
\end{itemize}

\todo{proper label and caption + code/identifier styling for diagram}

\begin{flowchart}{type-system-resolution-order-diagram}{識別子`i'の解決順序}

\tikzset {
	level distance = 1.4cm,
	scale = 1
}
\tikzset{multiline/.style={align=center}}

\tikzstyle{noEdge} = [ auto = left, outer sep = 0.2cm ]

% Compact decision shape (cut off rectangle corners if you know how)
\tikzstyle{decisionc} = [
	decision,
	minimum height = 0.8cm,
	rectangle
]

\Tree
[.\node [decisionc] (dec1) {'i' == 'true'または'false'、'this'、'super'、'null'};
\edge [noEdge] node {いいえ};
[.\node [decisionc] (dec2) {ローカル変数'i'が存在する};
\edge [noEdge] node {いいえ};
[.\node [decisionc] (dec3) {現在のフィールドが静的フィールドである};
\edge [noEdge] node {いいえ};
[.\node [decisionc] (dec4) {現在のクラス、親クラスのいずれかにフィールド'i'が存在する};
\edge [noEdge] node {いいえ};
[.\node [decisionc] (dec5) {'this'の型の静的拡張があるか};
\edge [noEdge] node {いいえ};
[.\node [decisionc] (dec6) {現在のクラスに静的フィールド'i'があるか};
\edge [noEdge] node {いいえ};
[.\node [decisionc] (dec7) {インポートされたenumにコンストラクタ`i'があるか};
\edge [noEdge] node {いいえ};
[.\node [decisionc] (dec8) {静的フィールド`i'がインポートされているか};
\edge [noEdge] node {いいえ};
[.\node [decisionc] (dec9) {`i'が小文字から始まるか};
\edge [noEdge] node {いいえ};
[.\node [decisionc] (dec10) {型`i'がインポートされているか};
\edge [noEdge] node {いいえ};
[.\node [decisionc] (dec11) {現在のパッケージが型`i'をふくむモジュール`i'を持つか};
\edge [noEdge] node {いいえ};
[.\node [decisionc] (dec12) {トップレベルの型`i'が存在するか};
\edge [noEdge] node {いいえ};
[.\node [decisionc] (dec13) {untypedモードか};
\edge [noEdge] node {はい};
[.\node [decisionc] (dec14) {`i' == `__this__'};
\edge [noEdge] node {いいえ};
[.\node [decisionc] (dec15) {ローカル変数`i'を生成する};
]]]]]]]]]]]]]]]


\node [startstop, fill = green!70, xshift = 5cm] (resolve) at (dec15.east) {解決};

\tikzstyle{yesNode} = [above right, at start]

\coordinate (yesAnchor) at (resolve.north);

\draw [flowchartArrow] (dec1) -| (yesAnchor) node [yesNode] {はい};
\draw [flowchartArrow] (dec2) -| (yesAnchor) node [yesNode] {はい};
\draw [flowchartArrow] (dec4) -| (yesAnchor) node [yesNode] {はい};
\draw [flowchartArrow] (dec5) -| (yesAnchor) node [yesNode] {はい};
\draw [flowchartArrow] (dec6) -| (yesAnchor) node [yesNode] {はい};
\draw [flowchartArrow] (dec7) -| (yesAnchor) node [yesNode] {はい};
\draw [flowchartArrow] (dec8) -| (yesAnchor) node [yesNode] {はい};
\draw [flowchartArrow] (dec10) -| (yesAnchor) node [yesNode] {はい};
\draw [flowchartArrow] (dec11) -| (yesAnchor) node [yesNode] {はい};
\draw [flowchartArrow] (dec12) -| (yesAnchor) node [yesNode] {はい};
\draw [flowchartArrow] (dec14) -| (yesAnchor) node [yesNode] {はい};
\draw [flowchartArrow] (dec15) -- (resolve.west);

\draw [flowchartArrow] (dec3) to [out = 180, in = 180, distance = 4cm] (dec6);
\draw (dec3.west) node [above left] {はい};

\draw [flowchartArrow] (dec9) to [out = 180, in = 180, distance = 4cm] (dec13);
\draw (dec9.west) node [above left] {はい};

\node [startstop, fill = red!70, xshift = 3cm] (fail) at (dec13.east) {失敗};
\draw [flowchartArrow] (dec13.east) -- (fail.west) node [above right, at start] {いいえ};


\end{flowchart}

Given an identifier \expr{i}, the algorithm is as follows:

\begin{enumerate}
	\item If i is \expr{true}, \expr{false}, \expr{this}, \expr{super} or \expr{null}, resolve to the matching constant and halt.
	\item If a local variable named \expr{i} is accessible, resolve to it and halt.
	\item If the current field is static, go to \ref{resolution:static-lookup}.
	\item If the current class or any of its parent classes has a field named \expr{i}, resolve to it and halt.
	\item\label{resolution:static-extension} If a static extension with a first argument of the type of the current class is available, resolve to it and halt.
	\item\label{resolution:static-lookup} If the current class has a static field named \expr{i}, resolve to it and halt.
	\item\label{resolution:enum-ctor} If an enum constructor named \expr{i} is declared on an imported enum, resolve to it and halt.
	\item If a static named \expr{i} is explicitly imported, resolve to it and halt.
	\item If \expr{i} starts with a lower-case character, go to \ref{resolution:untyped}.
	\item\label{resolution:type} If a type named \expr{i} is available, resolve to it and halt.
	\item\label{resolution:untyped} If the expression is not in untyped mode, go to \ref{resolution:failure}
	\item If \expr{i} equals \expr{__this__}, resolve to the \expr{this} constant and halt.
	\item Generate a local variable named \expr{i}, resolve to it and halt.
	\item\label{resolution:failure} Fail
\end{enumerate}

For step \ref{resolution:type}, it is also necessary to define the resolution order of types:

\begin{enumerate}
	\item\label{resolution:import} If a type named \expr{i} is imported (directly or as part of a module), resolve to it and halt.
	\item If the current package contains a module named \expr{i} with a type named \expr{i}, resolve to it and halt.
	\item If a type named \expr{i} is available at top-level, resolve to it and halt.
	\item Fail
\end{enumerate}

For step \ref{resolution:import} of this algorithm as well as steps \ref{resolution:static-extension} and \ref{resolution:enum-ctor} of the previous one, the order of import resolution is important:

\begin{itemize}
	\item Imported modules and static extensions are checked from bottom to top with the first match being picked.
	\item Within a given module, types are checked from top to bottom.
	\item For imports, a match is made if the name equals.
	\item For \tref{static extensions}{lf-static-extension}, a match is made if the name equals and the first argument \tref{unifies}{type-system-unification}. Within a given type being used as static extension, the fields are checked from top to bottom.
\end{itemize}
